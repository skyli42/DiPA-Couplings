\documentclass{article}
\usepackage[utf8]{inputenc}

% Standard packages that I use for math documents.
\usepackage[margin=0.75in]{geometry}
\usepackage{enumerate}
\usepackage{amsmath}
\usepackage{amsfonts} 
\usepackage{amssymb}
\usepackage{amsthm}
\usepackage{mathtools}
\usepackage{float}
\usepackage{array}
\usepackage{makecell}
\usepackage{commath}
\usepackage{easybmat} % for block matrices
\usepackage{parskip}
% Document-specific packages
\usepackage{xcolor, soul} % for highlights
\sethlcolor{yellow}

% Standard settings I use for nice formatting.
\setlength\parindent{0pt}
\DeclarePairedDelimiter{\ceil}{\lceil}{\rceil}
\renewcommand\theadalign{bc}
\renewcommand\theadfont{\bfseries}
\renewcommand\theadgape{\Gape[4pt]}
\renewcommand\cellgape{\Gape[4pt]}

% Standard definitions of macros I use regularly.
\newcommand{\N}{\mathbb{N}}
\newcommand{\Z}{\mathbb{Z}}
\newcommand{\Q}{\mathbb{Q}}
\newcommand{\C}{\mathbb{C}}
\newcommand{\R}{\mathbb{R}}
\newcommand{\F}{\mathbb{F}}
\newtheorem{theorem}{Theorem}
\newtheorem{proposition}{Proposition}[section]
\newtheorem{hypothesis}{Hypothesis}[section]
\newtheorem{corollary}{Corollary}[theorem]
\newtheorem{definition}{Definition}[section]
\newtheorem{lemma}[theorem]{Lemma}
\newtheorem*{remark}{Remark}
\newcommand{\cdotscalar}{\;\widetilde{\cdot}\;}
\newcommand{\vectorplus}{\;\widetilde{+}\;}
\newcommand{\Span}{\text{Span}}
\newcommand{\Null}{\text{Null}}
\newcommand{\Range}{\text{Range}}
\newcommand{\D}{\frac{d}{\dif x}}
\renewcommand{\epsilon}{\varepsilon}
\newcommand{\Or}{\mbox{ OR }}
\renewcommand{\And}{\mbox{ AND }}
\newcommand{\Not}{\mbox{NOT }}
\newcommand{\Iff}{\mbox{ IFF }}
\newcommand{\Width}{\textup{width}}
\newcommand{\Mesh}{\textup{mesh}}
\newcommand{\Int}{\textup{Int}}
\newcommand{\Ext}{\textup{Ext}}
\newcommand{\Bd}{\textup{Bd}}
\newcommand{\Supp}{\textup{Support}}
\newcommand{\sgn}{\textup{sgn}}
\newcommand{\loss}{\textup{loss}}
\newcommand{\insamplegeqx}{\textup{\texttt{insample$\geq$x}}}
\newcommand{\insampleltx}{\textup{\texttt{insample$<$x}}}
\newcommand{\insample}{\textup{\texttt{insample}}}
\newcommand\widebar[1]{\mathop{\overline{#1}}}
\newcommand*\closure[1]{\widebar{#1}}
\newcommand\Ball[2]{U(#1; #2)}

\newcommand{\1}{\langle 1 \rangle}
\newcommand{\2}{\langle 2 \rangle}

% End of preamble.  

\begin{document}

\tableofcontents

\section{Some notes from which to expand this document}

\begin{enumerate}
    \item An overview of coupling proofs of privacy
    \item A tight shift-coupling proof of privacy (segment free)
    \begin{enumerate}
        \item What are the connecting constraints, and why are they there?
        \item Proofs have to depend on $\Delta$!
        \item Proofs have to depend on sequences of segments! 
    \end{enumerate}
    \item Simplifying the problem above in various ways: 
    \begin{enumerate}
        \item Separability and the introduction of segments
        \item Only inter-segment transitions matter! 
        \begin{itemize}
            \item Given $\Delta$, the $\gamma$ values on inter-segment transitions are easily determined. 
            \item The $\Delta$ values on the inter-segment transitions can be determined. 
        \end{itemize}
    \end{enumerate}
    \item Solving the problem.
    \begin{enumerate}
        \item Hardness (incomplete)!!!
        \item Solving the easier version, where proofs don't depend on $\Delta$.
        \item Showing that they are bounded within $n$ of each other. 
        \begin{align*}
            \exists \text{ finite DP bound } &\iff \text{ hard system admits a feasible solution }
        \end{align*}
    \end{enumerate}
\end{enumerate}

\section{Definitions}

% \subsection{DiPA}

% Insert definition for DiPA here. 

% \subsection{Probability}

% \begin{definition}
%     The probability $\Pr(\rho | X)$ of a path $\rho = q_0 \to q_1 \to \dots \to q_m$ given an input $X = \langle a_1 , \dots, a_m \rangle$ is defined recursively as the probability that all transitions in $\rho$ are traversed in sequence given the input $X$ starting at state $q_0$.
% \end{definition}

% \begin{definition}
%     Let $\mathcal{A}$ be a DiPA, and $s \in seg(\mathcal{A})$ be a segment. The \textbf{privacy loss} $\loss(s)$ of a segment $s \in seg(\mathcal{A})$ is defined as 

%     \[\loss(s) = \sup_{\rho \in segF(s)}\sup_{X' \sim X} \left(\frac{Pr(\rho | X)}{P(\rho | X')} \right) \]

%     where $X$ and $X'$ vary over all pairs of neighbouring datasets. 
% \end{definition}

% \begin{definition}
%     Let $f_\epsilon(x)$ be the probability density function of a random variable $X$ with $X \sim Lap(0, 1/\epsilon)$. 

%     \[f_\epsilon(x) = \frac{\epsilon}{2} \exp(-\epsilon |x|)\]
% \end{definition}

% \begin{definition}
%     Let $F_\epsilon(x)$ be the cumulative distribution function of a random variable $X$ with $X \sim Lap(0, 1/\epsilon)$.

%     \[F_\epsilon(x) = P(X \leq x) = \begin{cases}
%         \frac{1}{2} \exp(\epsilon x) & x < 0 \\
%         1 - \frac{1}{2} \exp(-\epsilon x) & x \geq 0
%     \end{cases}\]
% \end{definition}

\section{Coupling proofs of privacy}

A coupling proof of privacy 

\section{Shift-coupling proofs of privacy}

Consider a path $\rho = q_0 \to q_1 \to \dots \to q_m$. Consider inputs $X \langle 1 \rangle = \langle a_1 \langle 1 \rangle, \dots, a_m \langle 1 \rangle \rangle$ and $X \langle 2 \rangle = \langle a_1 \langle 2 \rangle, \dots, a_m \langle 2 \rangle \rangle$ such that $X \langle 1 \rangle \sim X \langle 2 \rangle$. We wish to show that there exists $\epsilon \in (0, \infty)$ such that

\begin{align*}
    \Pr\left[\rho | X \langle 1 \rangle \right] \leq \exp(\epsilon) \cdot \Pr\left[\rho | X \langle 2 \rangle\right]
\end{align*}

\hl{TODO}: Write all of this later after consulting with Sky! ALSO, distinguish between path equivalence and output equivalence.

\begin{definition}
    Let  $X = \langle a_1, \dots, a_m \rangle$ be an input, and $\rho = q_0 \to q_1 \to \dots \to q_m$ be a path taken by DiPA $\mathcal{A}$ on $X$. Define $\tilde{a_i}$ to be the value of $\insample$ on the $i$th transition in $\rho$ on input $X$.
\end{definition}

\begin{itemize}
    \item The above is true if and only if $path_A(X) \Psi_\rho^{(\epsilon, 0)} path_A(X')$, where

    \[\Psi_\rho = \{(\rho_1, \rho_2) \in P \times P : \rho_1 = \rho \implies \rho_2 = \rho \} \]
    
    Shift-couplings are a technique to show that $path_A(X) \Psi_\rho^{(\epsilon, 0)} path_A(X')$ by constructing the couplings
    
    \[\tilde{a_i} \1 + \gamma_i (=)^{\#(\epsilon_i, 0)} \tilde{a_i} \2 \qquad \forall i \in \{0, \dots, m - 1\}\]

    and showing that 

    \[(\tilde{a} \1 + \gamma = \tilde{a} \2) \implies path_A(X \1) \Psi_\rho path_A(X \2)\]

    thus showing that $path_A(X) \Psi_\rho^{(\epsilon, 0)} path_A(X')$ for \[\epsilon = \sum_{i = 0}^{m - 1} \epsilon_i\]
    
    % where \[\psi_{\rho, i} = \{(\rho_1, \rho_2) \in P \times P : \rho_1[i] = \rho[i] \implies \rho_2[i] = \rho[i]\}\]
    
    % \[A(X) \Psi_{o, i} A(X') \qquad \forall i \in \{1, \dots, m\}\] where 
    \item A discussion that ends in choosing $\gamma$ shifts for each segment. 
    \item Maybe: A discussion of the cost of each shift.    
\end{itemize}


\subsection{Constraints} 

\begin{definition}
    
Given a fixed path, we say that an assignment of shifts $\{\gamma_i\}$ is \textbf{path-valid} if

\[(\tilde{a} \1 + \gamma = \tilde{a} \2) \implies path_A(X \1) \Psi_\rho path_A(X \2)\]

\end{definition}

\begin{definition}
    Let $\rho = q_0 \to q_1 \to \dots \to q_m$ be a path, and let $i \in \{1, \dots, m\}$. Define $at(i)$ to be the largest index $a(i) < i$ such that $\rho[a(i)] \to \rho[a(i) + 1]$ is an assignment transition.  
\end{definition}

\begin{definition}
    Let $\rho = q_0 \to q_1 \to \dots \to q_m$ be a path. Define $t_i$ to be the transition $q_i \to q_{i + 1}$.
\end{definition}

\begin{definition}
    Let $X = \langle a_1, \dots, a_m \rangle$ be an input, and let $\rho = q_0 \to q_1 \to \dots \to q_m$ be a path. Define $X[i]$ to be the value of $a_i$.
\end{definition}

Note that such an index must exist due to the initialization condition on DiPA. 

\begin{proposition}
    Given $\rho = q_0 \to q_1 \to \dots \to q_m$, an assignment of shifts $\{\gamma_i\}$ is valid if and only if it satisfies the following constraints for all $i \in \{0, \dots, m - 1\}$:

    \begin{align*}
        \gamma_i \leq \gamma_{at(i)} \qquad &\text{if } t_i \text{ has guard } < \\
        \gamma_i \geq \gamma_{at(i)} \qquad &\text{if } t_i \text{ has guard } \geq \\
    \end{align*}
\end{proposition}

\begin{proof}
    (Constraints $\implies$ valid) Suppose that the above constraints hold. We will show that $\{\gamma_i\}$ is valid using induction on $m = |\rho|$. Construct the couplings $\tilde{a_i} \1 + \gamma_i (=)^{(\epsilon_i, 0)} \tilde{a_i} \2$ for all $i \in \{0, \dots, m - 1\}$. 

    For a base case, assume $m = 1$. Then $\rho$ consists of an assignment transition $t_0$ with \texttt{true} guard (initialization condition). The constraints are trivially satisfied, and we have that $path_A(X \1 ) \Psi_\rho^{(0, 0)} path_A(X \2)$.

    Assume that the constraints hold for all paths of length $m$. Let $\rho = q_0 \to q_1 \to \dots \to q_{m} \to q_{m + 1}$. We will show that $\{\gamma_i\}$ is valid for $\rho$. First, by the validity of $\{\gamma_i\}_{i = 0}^{m - 1}$ for $\rho[0:m]$ by the inductive hypothesis, we have that \[path_A(X \1) \Psi_{\rho[0:m]} path_A(X \2)\] 
    
    by the inductive hypothesis. Now, assume $path_A(X \1) = \rho$. We have $path_A(X \2)[0:m] = \rho[0:m]$. Consider the last transition $t_{m}$ in $\rho$. Since $path_A(X \1) = \rho$, we know that $t_m$ is traversed by $A$ on $X \1$.

    \begin{itemize}
        \item If $t_{m}$ has guard \texttt{true}, then we trivially have that $path_A(X \2) = \rho$.
        \item If $t_{m}$ has guard $<$, we have from the constraints that $\gamma_{m} \leq \gamma_{at(m)}$. The value of the state variable $x \langle 1 \rangle$ is 
        
        \[x \1 = \tilde{a}_{at(m)} \1 \]

        and since $t_m$ is traversed by $A$ on $X \1$, we have 

        \begin{align*}
            \tilde{a}_{m} \1 &< \tilde{a}_{at(m)} \1\\
            \tilde{a}_{m} \2 - \gamma_m &< \tilde{a}_{at(m)} \2 - \gamma_{at(m)}\\
            \tilde{a}_{m} \2 &< \tilde{a}_{at(m)} \2 - (\gamma_{at(m)} - \gamma_{m}) < \tilde{a}_{at(m)} \2\\
        \end{align*}

        showing that $\tilde{a}_{m} \2$ satisfies the guard of $t_m$. Thus, $path_A(X \2) = \rho$.
        \item If $t_{m}$ has guard $\geq$, a similar argument as above shows that $path_A(X \2) = \rho$.
    \end{itemize}

    Thus, assuming that $a \1 + \gamma = a \2$, we have shown that $path_A(X \1) = \rho \implies path_A(X \2) = \rho$, which shows $path_A(X \1) \Psi_\rho path_A(X \2)$, and so $\{\gamma_i\}$ is valid.

    (Valid $\implies$ constraints) Suppose that $\{\gamma_i\}$ is valid. Let $i \in \{0, \dots, m - 1\}$. We will show that the constraints hold for $i$. 

    We will run the argument above in reverse. Again, we use induction on the length $m = |\rho|$. For a base case, assume $m = 1$, and so $\rho$ consists of an assignment transition $t_0$ with \texttt{true} guard. The constraints are trivially satisfied, since there are none. 

    Assume that the constraints hold for all valid shift assignments on paths of length $m$, and let $\rho = q_0 \to q_1 \to \dots \to q_{m} \to q_{m + 1}$. Since $\{\gamma_i\}$ is valid for $\rho$, we have that $a \1 + \gamma = a \2 \implies path_A(X \1) \Psi_\rho path_A(X \2)$. Also, we have that $\{\gamma_i\}_{i = 0}^{m - 1}$ is valid for $\rho[0:m]$, and that constraints on transitions $t_i \in \rho[0:m]$ hold. 

    We will now show that the constraints on $t_{m}$ hold by cases on the guard of $t_{m}$.

    \begin{itemize}
        \item If $t_{m}$ has guard \texttt{true}, then there is no constraint on $\gamma_{m}$, and so the constraints hold.
        \item If $t_{m}$ has guard $<$, the constraint to be shown is $\gamma_m \leq \gamma_{at(m)}$. Recall that we have $a \1 + \gamma = a \2 \implies path_A(X \1) \Psi_\rho path_A(X \2)$, showing that $t_m$ being traversed by $A$ on $X \1$ leads $t_m$ to be traversed by $A$ on $X \2$. Thus, we have 
        
        \begin{align*}
            \tilde{a}_m \1 < \tilde{a}_{at(m)} \1 &\implies \tilde{a}_m \2 < \tilde{a}_{at(m)} \2\\
             &\iff \tilde{a}_m \1 + \gamma_m < \tilde{a}_{at(m)} \1 + \gamma_{at(m)}\\
             &\iff \tilde{a}_m \1 < \tilde{a}_{at(m)} \1 + (\gamma_{at(m)} - \gamma_m)\\
        \end{align*}

        which is true if and only if $\gamma_m \leq \gamma_{at(m)}$. Thus, the constraint holds.
        \item A symmetric argument shows that the constraint holds if $t_{m}$ has guard $\geq$.
    \end{itemize}

    Thus, the given constraints on $\gamma$ hold if and only if it is valid for $\rho$.
\end{proof}

We can now reduce checking path-validity to checking the above constraints. 

\begin{definition}
    We say that $\gamma$ is output-valid for $\rho$ and $\Delta$ if we have 

    \[(\tilde{a} \1 + \gamma = \tilde{a} \2) \implies output_A(X \1) \Psi_o output_A(X \2)\]

    where \[\Psi_o = \{(o_1, o_2) \in \mathcal{O}_\rho : o_1 = o \implies o_2 = o\}\]

    where $\mathcal{O}_\rho$ is the set of all outputs that can be produced by the path $\rho$.
\end{definition}

\begin{definition}
    Define $\mathcal{P}$ to be the set of all paths in $\mathcal{A}$.
\end{definition}

Note: we show that this is actually a proof after the next section. 

\begin{definition}
    Given a DiPA $\mathcal{A}$, a \textbf{shift-coupling proof of privacy} for $\mathcal{A}$ is a map 

    \begin{align*}
        \Gamma: \mathcal{P} &\to ([-1, 1]^{\rho} \to [-1, 1]^{|\rho|})\\
        \rho &\mapsto (\Delta \mapsto \{\gamma_i\})
    \end{align*}

    such that for all $\rho \in \mathcal{P}$ and all $\Delta \in [-1, 1]^{\rho}$, we have that $\{\gamma_i\}$ is valid for $\rho$ and $\Delta$, and satisfies the output constraints.
\end{definition}

\subsection{The cost of a shift-coupling}

\begin{proposition}
    Consider a transition $t_i = q_{i} \to q_{i + 1}$ which is traversed independently by $A$ on input $a_i \1$ and $a_i \2$. Let $\Delta_i = a_i \2 - a_i \1$. Let $q_i$ draw from the distribution $Lap(0, \epsilon_i)$ to noise \texttt{insample}. The $\epsilon$-cost of the coupling 

    \[\tilde{a_i} \1 + \gamma_i (=)^{(c_i, 0)} \tilde{a_i} \2\]

    is given by \[c_i = |\Delta_i - \gamma_i| \epsilon_i \]
\end{proposition} 

\begin{proof}
    TODO, but easy to see from coupling construction rules.
\end{proof}

\begin{definition}
    Given a path $\rho$ and input differences $\Delta$, we define the $\rho$-$\Delta$-\textbf{cost} of the shifts $\{\gamma_i\}$ to be

    \[cost_{\rho, \Delta} (\{\gamma_i\}) = \sum_{i = 0}^{|\rho| - 1} |\Delta_i - \gamma_i| \epsilon_i\]
\end{definition}

\begin{definition}
    Given a shift-coupling proof of privacy $\Gamma$, we define the privacy cost of $\Gamma$ to be

    \[cost(\Gamma) = \sup_{\rho \in \mathcal{P}} \sup_{\Delta \in [-1, 1]^{\rho}} cost_{\rho, \Delta}(\Gamma(\rho, \Delta))\]
\end{definition}

\subsection{Privacy}

\begin{theorem}
    Let $\mathcal{A}$ be a DiPA, and $\Gamma$ be a shift-coupling proof of privacy for $\mathcal{A}$ with finite cost $\epsilon = cost(\Gamma)$. Then, $\mathcal{A}$ is $(\epsilon, 0)$-differentially private.
\end{theorem}

\begin{proof}
    This is a direct consequence of output validity. 
\end{proof}

\subsection{Why the above had to be the way it is.}

Since the total validity constraints on $\{\gamma_i\}$ does not depend on $X \1$ and $X \2$, one might be tempted to produce a proof of privacy by choosing $\gamma_i$ to be the same for all $X \1$ and $X \2$, given a path $\rho$. Although this is possible, this does not in general produce a tight proof of privacy.

\begin{proposition}
    \textbf{(A tight proof must regard input differences)} There exists a family of DiPA $\mathcal{F}$ and for $\mathcal{A} \in \mathcal{F}$, a shift-coupling proof $\Gamma^*: \mathcal{P} \to ([-1, 1]^{\rho} \to [-1, 1]^{|\rho|})$ such that for all assignments $\Pi: \mathcal{P} \to [-1, 1]^{\rho}$ of paths to shifts, we have 

    \[cost(\Gamma^*) < cost(\Pi)\]
\end{proposition}

\begin{proof}
    The construction is a DiPA with a one-segment path with same number of $<$ and $\geq$ transitions.
\end{proof}

When constructing a shift-coupling proof of privacy, we are actually choosing a shift for each transition. Is it reasonable to ignore paths, and just choose a shift for each transition? The answer is no, as the following proposition shows.

\begin{proposition}
    \textbf{(A tight proof must regard paths)} There exists a family of DiPA $\mathcal{F}$ and for $\mathcal{A} \in \mathcal{F}$, a shift-coupling proof $\Gamma^*: \mathcal{P} \to ([-1, 1]^{\rho} \to [-1, 1]^{|\rho|})$ such that for all assignments $\Pi: E \to ([-1, 1]^{\rho} \to [-1, 1]^{\rho})$ of transitions and differences to shifts, we have

    \[cost(\Gamma^*) < cost(\Pi)\]
\end{proposition}

\begin{proof}
    
\end{proof}

\section{The search for a tight proof as an optimization problem}

\subsection{Stating the problem abstractly}

Now that we have a characterization of shift-coupling proofs of privacy, the problem of finding a tight proof of privacy can be formulated as finding, given $\rho$ and $\Delta$,

\begin{align*}
    \inf_{\Gamma} cost(\Gamma) &= \inf_{\Gamma} \sup_{\rho} \sup_{\Delta} cost_{\rho, \Delta}(\Gamma(\rho, \Delta)) \\
\end{align*}

which is characterized by $\Gamma^*$ such that for any shift-coupling proof $\Gamma$, we have

\[\sup_{\rho} \sup_{\Delta} cost(\Gamma^*(\rho, \Delta)) \leq \sup_{\rho} \sup_{\Delta} cost(\Gamma(\rho, \Delta))\]


One such $\Gamma^*$ is the shift-coupling proof that chooses \[\Gamma^*(\rho, \Delta) = \inf_{\gamma \in [-1, 1]^{|\rho|}} cost_{\rho, \Delta} (\gamma)\]

which is the shift-coupling proof that chooses the optimal shift for each input difference and path independently. We will now direct our focus to computing $\Gamma^*$ given an automaton $\mathcal{A}$.

\subsection{Simplifying the problem with fixed $\rho, \Delta$}

\begin{proposition}
    \label{prop:gammas_non_assignment}
    Let $\rho, \Delta$ be given, and let 
    
    \[\gamma_i^* = \arg \inf_{\gamma \in [-1, 1]^{|\rho|}} cost_{\rho, \Delta} (\gamma)\]
    
    For non-assignment transitions $t_i \in \rho$, we have that 

    \begin{align*}
        \gamma_i^* &= \min(\Delta_i, \gamma_{at(i)}) \qquad \text{if } t_i \text{ has guard } < \\
        \gamma_i^* &= \max(\Delta_i, \gamma_{at(i)}) \qquad \text{if } t_i \text{ has guard } \geq \\
    \end{align*}
\end{proposition}

\begin{proof}
    TODO
\end{proof}

% \begin{corollary}
%     Let $\rho$ be given, and let $t_i$ and $t_j$ be two instances in which the same non-assignment transition is traversed in $\rho$. If $at(i) = at(j)$, then $\gamma_i = \gamma_j$. 
% \end{corollary}

% \begin{proposition}
%     If there exists a path $\rho$ which has the same assignment transition traversed twice, then $\mathcal{A}$ is not $\epsilon$-differentially private for any $\epsilon$.
% \end{proposition}

% \begin{corollary}
%     Let $\rho$ be given with each assignment transition traversed at most once. Then, if two instances $t_i$, $t_j$ of the same non-assignment transition are traversed in $\rho$, they must have $at(i) = at(j)$.
% \end{corollary}

% The above corollaries are important as they show that we can roll-up traversals of a non-assignment transition on a cycle into a single traversal, and so only need to consider one shift.

% \begin{definition}
%     Let $\rho$ be a path. Define the function $acyclic: P \to P$ such that $acyclic(\rho)$ is the path obtained by removing all cycles in $\rho$. (TODO: Perhaps elaborate using Sky's definition of acyclic)
% \end{definition}

% \begin{definition}
%     Let $\mathcal{P}_{eq}$ be the set of equivalence classes of paths, where $\rho, \rho' \in \mathcal{P}$ are equivalent if $acyclic(\rho) = acyclic(\rho')$.
% \end{definition}

\subsection{Simplifying the problem with fixed $\rho$}

\begin{proposition}
    \label{prop:deltas_non_assignment}
    Let $\rho$ be given. Let \[\Delta^* = \arg \sup_{\Delta \in [-1, 1]^{|\rho|}} \inf_{\gamma \in [-1, 1]^{|\rho|}} cost_{\rho, \Delta} (\gamma)\]
    
    For non-assignment transitions $t_i \in \rho$, we have that 

    \begin{align*}
        \Delta_i^* = \begin{cases}
            1 & \text{if } t_i \text{ has guard } < \\
            -1 & \text{if } t_i \text{ has guard } \geq \\
        \end{cases}
    \end{align*}
\end{proposition}

\begin{proof}
    TODO
\end{proof}

\begin{corollary}
    \label{cor:non_assignment_solved}
    Let $\rho$ be given. Let \[\Delta^* = \arg \sup_{\Delta \in [-1, 1]^{|\rho|}} \inf_{\gamma \in [-1, 1]^{|\rho|}} cost_{\rho, \Delta} (\gamma)\]
    \[\gamma^* = \arg \inf_{\gamma \in [-1, 1]^{|\rho|}} cost_{\rho, \Delta}(\gamma)\]
    As a consequence of Propositions \ref{prop:gammas_non_assignment} and \ref{prop:deltas_non_assignment}, if $t_i \in \rho$ is a non-assignment transition, then \[\gamma_i^* = 
    \gamma^*_{at(i)}\].

    % Proof: 
    % \begin{cases}
    %     \min(\gamma^*_{at(i)}, \Delta^*_i) & \text{if } t_i \text{ has guard } < \\
    %     \max(\gamma^*_{at(i)}, \Delta^*_i) & \text{if } t_i \text{ has guard } \geq \\
    %     \end{cases} = \begin{cases}
    %         \min(\gamma^*_{at(i)}, 1) & \text{if } t_i \text{ has guard } <  \\
    %         \max(\gamma^*_{at(i)}, -1) & \text{if } t_i \text{ has guard } \geq \\
    %         \end{cases}
\end{corollary}

The corollary above is important: it reveals that the only transitions that matter are assignment transitions!

\subsubsection{Identifying segments}

Corollary \ref{cor:non_assignment_solved} allows us to formulate the problem of finding cost-minimal shifts $\gamma$ over maximal input differences $\Delta \in [-1, 1]^{|\rho|}$ given $\rho$ to the problem of finding $\gamma$ and $\Delta$ for only the assignment transitions in $\rho$. 

\begin{definition}
    Consider a DiPA $\mathcal{A}$. Let $q_i, q_j \in Q$ be such that there is a path $\rho = a_1 \to \dots \to a_m$ such that:

    \begin{itemize}
        \item $a_1 = q_i$ and $a_m = q_j$
        \item $a_1 \to a_2$ is the only assignment transition in $\rho$
        \item There exists an assignment transition out of $q_j$ or it is a terminal state
    \end{itemize}

    Then we define $seg(q_i, q_j)$ to be the set of all paths from $q_i$ to $q_j$ that are acyclic with their first transition being their only assignment transition. We call such a path $s \in seg(q_i, q_j)$ a \textbf{segment}. 
\end{definition}

Consider a path $\rho$ with assignment transitions from states $a_0, a_2, \dots, a_{n - 1}$, and terminal state $a_n$. Further, associate $\rho$ with the sequence of segments $s_1 \hookrightarrow s_2 \hookrightarrow \dots \hookrightarrow s_n$ where $s_i \in seg(a_{i - 1}, a_i)$.

\begin{definition}
    Given a path $\rho$ associated with the sequence of segments $s_1 \hookrightarrow s_2 \hookrightarrow \dots \hookrightarrow s_n$, we define the \textbf{segment cost} of $s_i$ given $\rho$ to be 

    \[segcost_{s_i, \rho, \Delta}(\gamma) = \sum_{t_j \in \rho, t_j \in s} cost(\gamma_j) \]

    TODO: make this pathless
\end{definition}

Since we know from Corollary \ref{cor:non_assignment_solved} that shifts on a segment are equal to the shift on the assignment transition of the segment, we can find

\begin{align*}
    \sup_{\Delta \in [-1, 1]^{|\rho|}} \inf_{\gamma \in [-1, 1]^{|\rho|}} cost_{\rho, \Delta} (\gamma) &= \max_{\Delta \in [-1, 1]^{n}} \min_{\gamma \in [-1, 1]^{n}} seg-cost()\\
\end{align*}

\section{Deciding Privacy in Linear Time}

\begin{proposition}
    
\end{proposition}

\section{Do shift-coupling proofs of privacy have matching lower bounds?}
\textbf{Last Updated: Wednesday, June 28th, 2023}

The relevant definitions and lemmata for proofs in this section are in the appendix. It is also assumed, for now, that all transition outputs are in the output alphabet. 

\subsection{$S^L$ is tight when there is an $L$-cycle}

\begin{theorem} ($S^L$ is tight for segments with $L$-cycles)
    Consider a segment $s \in seg(\mathcal{A})$ corresponding to the sequence of states $q_0 \to q_1 \to \dots \to q_m$. If $s$ contains an $L$-cycle, then the L-cost of the segment gives a tight upper bound on the privacy loss of the segment. That is, \[\loss(s) =  \exp\left(2 \epsilon_0 + \sum_{i > 0: guard(a_i) = \insamplegeqx} 2\epsilon_i \right)\]

    given that state $q_i$ draws from the distribution $Lap(0, 1/\epsilon_i)$ to noise \insample.
\end{theorem}

\begin{proof}
    We will prove the result for when $\epsilon_i = \epsilon$ for all $i \geq 0$. The proof for the general case goes through in the same fashion. Let $f, F$ be the probability density function and cumulative distribution function of a random variable $X$ with $X \sim Lap(0, 1/\epsilon)$ as defined in the appendix. 

    Since $s$ has an L-cycle, there exists a sequence of paths $\rho_i$ for $i \in \N$ each with $l_i$ number of L-transitions such that $\lim_{i \to \infty} l_i = \infty$. Let $m$ be the number of $G$-transitions in $\rho_i$. We will assume that this number is the same across all $\rho_i$.\footnote{Otherwise, $s$ has a G-cycle, and $\mathcal{A}$ is not differentially private. The privacy loss through $s$ is $\infty$, which matches the $L$-cost.}
    
    For each $\rho_i$, construct the adjacent pair of inputs $X_i, X_i'$ as follows. Let $X_i[j] = 0$ for all $j \in \{1, \dots, |\rho_i|\}$, where $|\rho_i|$ is the number of transitions in $\rho$. Define $X_i[j]$ as follows:

    \[X_i[j] = \begin{cases}
        1 & \text{if } \rho_i[j] \to \rho_i[j + 1] \text{ is an assignment transition or has guard \insamplegeqx} \\
        -1 & \text{otherwise, in which case } \rho_i[j] \to \rho_i[j + 1] \text{ has guard \insampleltx}\\
    \end{cases}\]

    Let $\tilde{a_j}$ be the random variable representing the value of $\insample$ before the $j$th transition in $\rho$ on input $X_i$. Let $\tilde{b_j}$ be the random variable representing the value of $\insample$ before the $j$th transition in $\rho$ on input $X_i'$. Further, let $\Gamma_L = \{j : \rho_i[j] \to \rho_i[j + 1] \text{ has guard } \insampleltx\}$, and $\Gamma_G = \{j : \rho_i[j] \to \rho_i[j + 1] \text{ has guard } \insamplegeqx\}$. 
    
    Notice that $\tilde{a_j} = \tilde{b_j} + 1$ for $j \in \Gamma_L$, and $\tilde{a_j} + 1 = \tilde{b_j}$ for $j \in \{0\} \cup \Gamma_G$. Since $\tilde{a_j}$ is distributed as $Lap(X_i[j], 1/\epsilon)$, we can write its probability density function as $f(x - X_i[j])$, and its cumulative distribution function as $F(x - X_i[j])$. A similar statement holds for $\tilde{b_j}$.
    
    We may now compute and compare $Pr(\rho_i | X_i')$ and $\Pr(\rho_i | X_i)$ as follows.

    \begin{align*}
        \Pr(\rho_i | X_i') &= \int_{-\infty}^\infty \Pr(\tilde{b_0} = x) \prod_{j \in \Gamma_L} \Pr(\tilde{b_j} < x) \prod_{j \in \Gamma_G} \Pr(\tilde{b_j} \geq x) \dif x\\
        &= \int_{-\infty}^\infty \Pr(\tilde{b_0} = x) \prod_{j \in \Gamma_L} \Pr(\tilde{b_j} < x) \prod_{j \in \Gamma_G} \Pr(\tilde{b_j} \geq x) \dif x\\
        &= \int_{-\infty}^\infty f_\epsilon(x - X_i[0]) \prod_{j \in \Gamma_L} F_\epsilon(x - X_i[j]) \prod_{j \in \Gamma_G} (1 - F_\epsilon(x - X_i[j])) \dif x\\
        &= \int_{-\infty}^\infty f(x - 1) F(x + 1)^{\ell_i}  (1 - F(x - 1))^m \\
        &= \int_{-\infty}^\infty f(x) F(x + 2)^{\ell_i}  (1 - F(x))^m \\
        &= \exp(2\epsilon (m + 1) )\left(\int_{(-\infty, -2) \cup (2, \infty)} f(x) F(x)^{\ell_i}  (1 - F(x))^m \dif x + g(\ell_i) \int_{-2}^2 f(x) F(x + 2)^{\ell_i}  (1 - F(x))^m\right)
    \end{align*}

    with $g(\ell_i) \to 1$ as $\ell_i \to \infty$. As we take $\ell_i \to \infty$, we see that 

    \begin{align*}
        h(\ell_i) := \frac{\left(\int_{(-\infty, -2) \cup (2, \infty)} f(x) F(x)^{\ell_i}  (1 - F(x))^m \dif x + g(\ell_i) \int_{-2}^2 f(x) F(x + 2)^{\ell_i}  (1 - F(x))^m\right)}{\Pr(\rho_i | X_i)} \to 1
    \end{align*}

    and so as we take the supremum over $\rho_i$ below, we get: 

    \begin{align*}
        \loss(s) \geq \sup_{\rho_i} \frac{\Pr(\rho_i | X_i')}{\Pr(\rho_i | X_i)} &= \exp(2\epsilon(m + 1)) \sup_{\rho_i} \left\{ h(l_i) \right\}\\
        &= \exp(2\epsilon(m + 1))
    \end{align*}

    We know that $S^L$ is tight, and gives the bound $\exp(2\epsilon(m + 1))$. Thus, we have shown that $\loss(s) = \exp(2\epsilon(m + 1))$, as desired.
\end{proof}

\subsection{An alternative coupling strategy: $S^J$}

\begin{definition}
    $S^J$ is a coupling strategy in which we do not couple the noised threshold, but couple the results of all other transitions with twice the cost. [TODO: Describe in more detail]
\end{definition}

\begin{theorem}
    Let $s = q_0 \to \dots \to q_m$ be a segment with only L-transitions. If $S^J$ is the least-cost coupling strategy on $s$, then it provides a tight bound on $\loss(s)$ given by \[\loss(s) = \sum_{i = 1}^{m} 2\epsilon_i\]
\end{theorem}

\begin{proof}
    I have a proof for this, but I will add it into this document soon. [TODO]

    \begin{figure}[H]
        \centering
        \includegraphics[width=0.9\textwidth]{figures/only_l_transitions.png}
        \caption{A segment $s$ with only L-transitions.}
        \label{fig:segment_j}
    \end{figure}

    \begin{figure}[H]
        \centering
        \includegraphics[width=0.5\textwidth]{figures/only_l_transitions_plot.png}
        \caption{}
        \label{fig:segment_j_coupling}
    \end{figure}
\end{proof}

\begin{hypothesis}
    For segments which contain only $L$-transitions and for which the $J$-cost exceeds the $L$-cost, $S^L$ is tight.
\end{hypothesis}

\begin{proof}
    I think this is true from the graph above, but I need to prove it.

    Note June 28 2023: I think this is not true for segments that contain both $L$-transitions and $G$-transitions.
\end{proof}

\appendix

\section{Lemmata}

\subsection{Properties of $f_\epsilon$ and $F_\epsilon$}

\begin{lemma} 
    \label{lemma:F_equality_neg}
    For $x \leq 0$, we have 
    \[F_\epsilon(x) = \exp(2\epsilon)F_\epsilon(x - 2)\]
    and equivalently for $x \leq -2$, we have 
    \[F_\epsilon(x + 2) = \exp(2\epsilon)F_\epsilon(x)\]
\end{lemma}

\begin{lemma}
    \label{lemma:F_equality_pos}
    For $x \geq 0$, we have \[1 - F_{\epsilon}(x) = \exp(2\epsilon) (1 - F_\epsilon(x + 2))\]
\end{lemma}

\begin{lemma}
    \label{lemma:f_equality}
    For $x \geq 0$, we have
    \[f_\epsilon(x) = \exp(2\epsilon) f_\epsilon(x + 2) \]
\end{lemma}

\subsection{For the proof of Theorem 1}
\begin{lemma} 
    \label{lemma:sl_tight_minus_infty} 
    \[\int_{-\infty}^{-2} f_{\epsilon}(x) F_{\epsilon}(x + 2)^\ell (1 - F_{\epsilon}(x))^m \dif x = \exp(2 \epsilon \ell) \int_{-\infty}^{-2} f_{\epsilon}(x) F_{\epsilon}(x)^\ell (1 - F_{\epsilon}(x))^m \dif x\]
\end{lemma}

\begin{proof}
    From Lemma \ref{lemma:F_equality_neg}, we have that 

    \begin{align*}
        \int_{-\infty}^{-2} f_{\epsilon}(x) F_{\epsilon}(x + 2)^\ell (1 - F_{\epsilon}(x))^m \dif x &= \int_{-\infty}^{-2} f_{\epsilon}(x) (\exp(2\epsilon) F_{\epsilon}(x))^\ell (1 - F_{\epsilon}(x))^m \dif x \\
        &= \exp(2\epsilon \ell) \int_{-\infty}^{-2} f_{\epsilon}(x) F_{\epsilon}(x)^\ell (1 - F_{\epsilon}(x))^m \dif x
    \end{align*}
\end{proof}

\begin{lemma}
    \label{lemma:sl_tight_plus_infty}
    \[\int_{0}^{\infty} f_{\epsilon}(x) F_{\epsilon}(x + 2)^\ell (1 - F_{\epsilon}(x))^m \dif x = \exp(2 \epsilon m) \int_{2}^{\infty} f_{\epsilon}(x) F_{\epsilon}(x)^\ell (1 - F_{\epsilon}(x))^m \dif x\]
\end{lemma}

\begin{proof}
    From Lemma \ref{lemma:F_equality_pos} and \ref{lemma:f_equality}, we have that

    \begin{align*}
        \int_{0}^{\infty} f_{\epsilon}(x) F_{\epsilon}(x + 2)^\ell (1 - F_{\epsilon}(x))^m \dif x &= \int_{0}^{\infty} \exp(2\epsilon)f_{\epsilon}(x + 2) F_{\epsilon}(x + 2)^\ell (\exp(2\epsilon)(1 - F_{\epsilon}(x + 2)))^m \dif x \\
        &= \exp(2\epsilon m) \int_{0}^{\infty} f_{\epsilon}(x + 2) F_{\epsilon}(x + 2)^\ell (1 - F_{\epsilon}(x + 2))^m \dif x\\
        &= \exp(2\epsilon (m + 1)) \int_{2}^{\infty} f_{\epsilon}(x) F_{\epsilon}(x)^\ell (1 - F_{\epsilon}(x))^m \dif x
    \end{align*}
\end{proof}

\begin{lemma}
    There exists a function $g: \N \to \mathbb{R}$ such that
    \label{lemma:sl_tight_minus_2}
    \[\int_{-2}^{0} f_{\epsilon}(x) F_{\epsilon}(x + 2)^\ell (1 - F_{\epsilon}(x))^m \dif x = g(\ell) \exp(2 \epsilon (m + 1)) \int_{-2}^{2} f_{\epsilon}(x) F_{\epsilon}(x)^\ell (1 - F_{\epsilon}(x))^m \dif x\]
    with $g(\ell) \to 1$ as $\ell \to \infty$.
\end{lemma}

\begin{proof}
    I'm not sure yet how to prove this, although I strongly suspect that the $(m + 1)$ term comes from the fact that $f_{\epsilon}(x)$ is the derivative of $-(1 - F_{\epsilon}(x))$, and it is taken to the $m$th power. Its integral should behave like a polynomial of degree $m + 1$ evaluated at $2$, which corresponds to $\exp(2\epsilon(m + 1))$. 
\end{proof}


\end{document}