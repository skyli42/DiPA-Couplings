

\section{Appendix A: Single Variable Programs}

\subsection{Couplings for Transitions and Straight Line Programs}

\begin{lemma}[Detailed version of lemma \ref{simplifiedIndTransitionCoupling}]\label{indTransitionCoupling}
    Let $X\brangle{1}\sim \Lap(\mu\brangle{1}, \frac{1}{d_x\varepsilon}), X\brangle{2}\sim\Lap(\mu\brangle{2}, \frac{1}{d_x\varepsilon})$ be random variables representing possible initial values of $\texttt{x}$ and let $t = (c, \sigma, \tau)$ be a transition from some valid transition alphabet $\Sigma_T$.
    Let $P(t) = (d_t, d_t')$.

    Let $\texttt{in}\brangle{1}\sim \texttt{in}\brangle{2}$ be an arbitrary valid adjacent input pair and let $o\brangle{1}$, $o\brangle{2}$ be random variables representing possible outputs of $t$ given inputs $\texttt{in}\brangle{1}$ and $\texttt{in}\brangle{2}$, respectively. 

    Then $\forall \varepsilon>0$ and for all $\gamma_x, \gamma_t, \gamma_t'\in [-1, 1]$ that satisfy the constraints \[
        \begin{cases}
          \gamma_t\leq\gamma_x & c = \lguard[\texttt{x}]\\
          \gamma_t\geq\gamma_x & c = \gguard[\texttt{x}]\\
          \gamma_t=0 & \sigma = \texttt{insample}\\
          \gamma_t'=0 & \sigma = \texttt{insample}'
        \end{cases},
      \]
      the lifting $o\brangle{1}\{(a, b): a=\sigma\implies b=\sigma\}^{\#d\varepsilon}o\brangle{2}$ is valid for $d = (|\mu\brangle{1}-\mu\brangle{2}+\gamma_x|)d_x+(|-\texttt{in}\brangle{1}+\texttt{in}\brangle{2}-\gamma_t|)d_t+(|-\texttt{in}\brangle{1}+\texttt{in}\brangle{2}-\gamma_t'|)d_t'$.
\end{lemma}

\begin{proof}
Fix $\varepsilon>0$.

We can first create the lifting $X\brangle{1}+\gamma_x (=)^{\#(|\mu\brangle{1}-\mu\brangle{2}+\gamma_x|)d_x\varepsilon}X\brangle{2}$. 

Additionally, create the lifting $z\brangle{1} (=)^{\#(|-\texttt{in}\brangle{1}+\texttt{in}\brangle{2}-\gamma_t|)d_t\varepsilon}z\brangle{2} - \texttt{in}\brangle{1}+\texttt{in}\brangle{2}-\gamma_t$, which is equivalent to creating the lifting $\texttt{insample}\brangle{1} +\gamma_t{(=)}^{\#(|-\texttt{in}\brangle{1}+\texttt{in}\brangle{2}-\gamma_t|)d_t\varepsilon}\texttt{insample}\brangle{2}$.

Finally, create the lifting $z'\brangle{1} (=)^{\#(|-\texttt{in}\brangle{1}+\texttt{in}\brangle{2}-\gamma_t'|)d_t'\varepsilon}z'\brangle{2} - \texttt{in}\brangle{1}+\texttt{in}\brangle{2}-\gamma_t'$. As before, this is equivalent to creating the lifting $\texttt{insample}'\brangle{1} +\gamma_t'{(=)}^{\#(|-\texttt{in}\brangle{1}+\texttt{in}\brangle{2}-\gamma_t'|)d_t'\varepsilon}\texttt{insample}'\brangle{2}$.

Thus, we emerge with three key statements to leverage:\begin{itemize}
    \item $X\brangle{1} + \gamma_x = X\brangle{2}$
    \item $z\brangle{1} = z\brangle{2} - \texttt{in}\brangle{1}+\texttt{in}\brangle{2}-\gamma_t$
    \item $z'\brangle{1} = z'\brangle{2} - \texttt{in}\brangle{1}+\texttt{in}\brangle{2}-\gamma_t'$
\end{itemize}

So if $c=\lguard[\texttt{x}]$ and $\gamma_t\leq \gamma_x$, then \begin{align*}
    \texttt{insample}\brangle{1}<X\brangle{1}&\implies \texttt{in}\brangle{1}+z\brangle{1}<X\brangle{1}\\
    &\implies \texttt{in}\brangle{1}+z\brangle{2}-\texttt{in}\brangle{1}+\texttt{in}\brangle{2}-\gamma_t<X\brangle{2}-\gamma_x\\
    &\implies \texttt{insample}\brangle{2}<X\brangle{2}
\end{align*}

Similarly, if $c=\gguard[\texttt{x}]$ and $\gamma_t\geq \gamma_x$, then \begin{align*}
    \texttt{insample}\brangle{1}\geq X\brangle{1}&\implies \texttt{in}\brangle{1}+z\brangle{1}\geq X\brangle{1}\\
    &\implies \texttt{in}\brangle{1}+z\brangle{2}-\texttt{in}\brangle{1}+\texttt{in}\brangle{2}-\gamma_t\geq X\brangle{2}-\gamma_x\\
    &\implies \texttt{insample}\brangle{2}\geq X\brangle{2}
\end{align*}

With these liftings, we have ensured that if the first run satisfies the guard of $t$, then the second run does as well. 

As noted, if $\sigma \in \Gamma$ and the first run taking transition $t$ implies that the second run does as well, then $o\brangle{1} = \sigma \implies o\brangle{2}=\sigma$ trivially.

Now, if $\sigma=\texttt{insample}$ and $\gamma_t=0$, then clearly we have that $\texttt{insample}\brangle{1}=\texttt{insample}\brangle{2}$, so for all $a\in \RR$, $o\brangle{1} = a\implies o\brangle{2} = a$.

Similarly, if $\sigma=\texttt{insample}'$ and $\gamma_t'=0$, we have that for all $a\in \RR$, $o\brangle{1} = a\implies o\brangle{2} = a$.

Thus, given the constraints \[
  \begin{cases}
    \gamma_t\leq\gamma_x & c = \lguard[\texttt{x}]\\
    \gamma_t\geq\gamma_x & c = \gguard[\texttt{x}]\\
    \gamma_t=0 & \sigma = \texttt{insample}\\
    \gamma_t'=0 & \sigma = \texttt{insample}'
  \end{cases},
\]
we have shown that the lifting $o\brangle{1}\{(a, b): a=\sigma\implies b=\sigma\}^{\#d\varepsilon}o\brangle{2}$ is valid, where the cost $d = (|\mu\brangle{1}-\mu\brangle{2}+\gamma_x|)d_x+(|-\texttt{in}\brangle{1}+\texttt{in}\brangle{2}-\gamma_t|)d_t+(|-\texttt{in}\brangle{1}+\texttt{in}\brangle{2}-\gamma_t'|)d_t'$. 

\end{proof}


\begin{lemma}[Detailed version of lemma \ref{simplifiedMultTransitionsCouplingProof}]\label{multTransitionsCouplingProof}
    Let $\rho = t_0\ldots t_{n-1}$ be a initialized SLP of length $n$. 
    Let $\texttt{in}\brangle{1}\sim \texttt{in}\brangle{2}$ be arbitrary adjacent input sequences of length $n$. Additionally, fix some potential output $\sigma$ of $\rho$ of length $n$ and let $\sigma\brangle{1}$, $\sigma\brangle{2}$ be random variables representing possible outputs of $\rho$ given inputs $\texttt{in}\brangle{1}$ and $\texttt{in}\brangle{2}$, respectively. Additionally, for all $t_i$, let $P(t_i) = (d_i, d_i')$.

    Then $\forall \varepsilon>0$ and for all $\{\gamma_i, \gamma_i'\}_{i=0}^{n-1}$ that, for all $i$, satisfy the constraints \[
        \begin{cases}
          \gamma_i\leq\gamma_{at(i)} & c_i = \lguard[\texttt{x}]\\
          \gamma_i\geq\gamma_{at(i)} & c_i = \gguard[\texttt{x}]\\
          \gamma_i=0 & \sigma_i = \texttt{insample}\\
          \gamma_i'=0 & \sigma_i = \texttt{insample}'
        \end{cases},
      \]
      the lifting $\sigma\brangle{1}\{(a, b): a=\sigma\implies b=\sigma\}^{\#d\varepsilon}\sigma\brangle{2}$ is valid for $d = \sum_{i=0}^{n-1}(|-\texttt{in}_i\brangle{1}+\texttt{in}_i\brangle{2}-\gamma_i|)d_i+(|-\texttt{in}_i\brangle{1}+\texttt{in}_i\brangle{2}-\gamma_i'|)d_i'$, and therefore $t$ is $d\varepsilon$-differentially private. 
\end{lemma}
\begin{proof}
    From the proof of lemma \ref{indTransitionCoupling}, we know that we can create the couplings $\texttt{insample}_i\brangle{1} +\gamma_i{(=)}^{\#(|-\texttt{in}_i\brangle{1}+\texttt{in}_i\brangle{2}-\gamma_i|)d_i\varepsilon}\texttt{insample}_i\brangle{2}$ and $\texttt{insample}_i'\brangle{1} +\gamma_i'{(=)}^{\#(|-\texttt{in}_i\brangle{1}+\texttt{in}_i\brangle{2}-\gamma_i'|)d_i'\varepsilon}\texttt{insample}_i'\brangle{2}$ for all $q_i$ in $\rho$. 

    Additionally, for some fixed $q_i$ in $\rho$, if we have the coupling $\texttt{x}_i\brangle{1}+\gamma_x (=)^{\#(|\hat{\mu_i}\brangle{1}-\hat{\mu_i}\brangle{2}+\gamma_x|)\hat{d_i}\varepsilon}x_i\brangle{2}$, where $\texttt{x}_i\brangle{1}\sim \Lap(\hat{\mu_i}\brangle{1}, \frac{1}{\hat{d_i}\varepsilon})$ and $\texttt{x}_i\brangle{2}\sim \Lap(\hat{\mu_i}\brangle{2}, \frac{1}{\hat{d_i}\varepsilon})$, then subject to the constraints \[
        \begin{cases}
          \gamma_i\leq\gamma_x & c_i = \lguard[\texttt{x}]\\
          \gamma_i\geq\gamma_x & c_i = \gguard[\texttt{x}]\\
          \gamma_i=0 & \sigma_i = \texttt{insample}_i\\
          \gamma_i'=0 & \sigma_i = \texttt{insample}_i'
        \end{cases},
      \]
    the coupling $\sigma_i\brangle{1}\{(a, b): a=\sigma_i\implies b=\sigma_i\}^{\#d\varepsilon}\sigma_i\brangle{2}$ is valid for some $d$. 

    Indeed, note that for all $i$, $\texttt{x}_i = \texttt{insample}_{at(i)}$ by the semantics of an SLP. Thus, we have that $\texttt{x}_i\brangle{1}+\gamma_x (=)^{\#(|-\texttt{in}_{at(i)}\brangle{1}+\texttt{in}_{at(i)}\brangle{2}+\gamma_{at(i)}|)d_{at(i)}\varepsilon}x_i\brangle{2}$, and we must satisfy the constraints \[
        \begin{cases}
          \gamma_i\leq\gamma_{at(i)} & c_i = \lguard[\texttt{x}]\\
          \gamma_i\geq\gamma_{at(i)} & c_i = \gguard[\texttt{x}]\\
          \gamma_i=0 & \sigma_i = \texttt{insample}_i\\
          \gamma_i'=0 & \sigma_i = \texttt{insample}_i'
        \end{cases}
      \]
      for all $i$.

    Thus, we can put all of these couplings together to show that the coupling $\sigma_i\brangle{1}\{(a, b): a=\sigma_i\implies b=\sigma_i\}^{\#d\varepsilon}\sigma_i\brangle{2}$ is valid for some $d>0$.

    In particular, note that we have created at most one pair of couplings (for $\texttt{insample}$ and $\texttt{insample}'$) for each $q_i$. Thus, the total coupling cost associated with each $q_i$ is at most $(|-\texttt{in}_i\brangle{1}+\texttt{in}_i\brangle{2}-\gamma_i|)d_i+(|-\texttt{in}_i\brangle{1}+\texttt{in}_i\brangle{2}-\gamma_i'|)d_i'$, 
    which gives us an overall coupling cost of $d = \sum_{i=0}^{n-1}(|-\texttt{in}_i\brangle{1}+\texttt{in}_i\brangle{2}-\gamma_i|)d_i+(|-\texttt{in}_i\brangle{1}+\texttt{in}_i\brangle{2}-\gamma_i'|)d_i'$.
\end{proof}

\subsection{DiPA and Programs}

We copy the definitions of leaking cycles, leaking pairs, disclosing cycles, and privacy violating paths from \cite{chadhaLinearTimeDecidability2021} for clarity. 

\begin{defn}[Leaking Cycles~\cite{chadhaLinearTimeDecidability2021}]
    A path $\rho = q_0\to\ldots \to q_n$ in a DiPA $A$ is a leaking path if there exist indices $i, j$ where $0\leq i < j < n$ such that the $i$'th transition $q_i\to q_{i+1}$ in $\rho$ is an assignment transition and the guard of the transition $q_j \to q_{j+1}$ is not $\texttt{true}$. If $\rho$ is also a cycle, then we call it a leaking cycle.
\end{defn}

\begin{defn}[\cite{chadhaLinearTimeDecidability2021}]
    A cycle $\rho$ in a DiPA $A$ is an \lcycle~if for some transition $q_i\to q_{i+1}$ in $\rho$, $\guard(q_i\to q_{i+1}) = \lguard[\texttt{x}]$. Similarly, $\rho$ is a \gcycle~if for some transition $q_i\to q_{i+1}$ in $\rho$, $\guard(q_i\to q_{i+1}) = \gguard[\texttt{x}]$.
    
    Additionally, a path $\rho$ of a DiPA $A$ is an \texttt{AL}-path (respectively, \texttt{AG}-path) if all assignment transitions in $\rho$ have guard $\lguard[\texttt{x}]$ (respectively, $\gguard[\texttt{x}]$)
\end{defn}

\begin{defn}[Leaking Pairs \cite{chadhaLinearTimeDecidability2021}]
    A pair of cycles $(C, C')$ is called a leaking pair if one of the following two conditions is satisfied.
    \begin{enumerate}
        \item $C$ is an \lcycle, $C'$ is a \gcycle~and there is an \texttt{AG}-path from a location in $C$ to a location in $C'$.
        \item $C$ is a \gcycle, $C'$ is an \lcycle~and there is an \texttt{AL}-path from a location in $C$ to a location in $C'$.
    \end{enumerate}
\end{defn}

\begin{defn}[Disclosing Cycles \cite{chadhaLinearTimeDecidability2021}]
    A cycle $C = q_0\to \ldots \to q_n \to q_0$ of a DiPA $A$ is a disclosing cycle if there is an $i$, $0 \leq i < |C|$ such that $q_i\in Q_{in}$ and the transition $q_i\to q_{i+1}$ that outputs either \texttt{insample} or \texttt{insample}'.
\end{defn}

\begin{defn}[Privacy Violating Paths \cite{chadhaLinearTimeDecidability2021}]
    We say that a path $\rho = q_0\to\ldots \to q_n $ of a DiPA $A$ is a privacy violating path if one of the following conditions hold:
    \begin{itemize}
        \item  $tail(\rho)$ is an \texttt{AG}-path (resp., \texttt{AL}-path) such that $last(\rho)$ is in a \gcycle~(resp., \lcycle) and the 0th transition $q_0\to q_1$ is an assignment transition that outputs \texttt{insample}.
        \item $\rho$ is an \texttt{AG}-path (resp., \texttt{AL}-path) such that $q_n$ is in a \gcycle~(resp., \lcycle) and the first transition $q_0\to q_1$ has guard $\lguard[\texttt{x}]$ (resp., $\gguard[\texttt{x}]$) and outputs \texttt{insample}
        \item $\rho$ is an \texttt{AG}-path (resp., \texttt{AL}-path) such that $q_0$ is in an \lcycle~(resp., \gcycle) and the last transition $q_{n-1}\to q_n$ has guard $\gguard[\texttt{x}]$ (resp., $\lguard[\texttt{x}]$) and outputs \texttt{insample}
    \end{itemize}
\end{defn}


\begin{lemma}\label{unsatisfiableImpliesNotWellformedLemma}
    If a periodic program $L$ generated by $G_L$ satisfies output distinction and there does not exist a coupling strategy $C$ that satisfies the privacy constraint system for $L$, then $G_L$ must contain either a leaking cycle, a leaking pair, a disclosing cycle, or a privacy violating path. 
\end{lemma}

\begin{proof}[Proof of lemma \ref{unsatisfiableImpliesNotWellformedLemma}]
    Let $L$ be a periodic program in $P$ that does not have a coupling strategy that satisfies the privacy constraint system.

    Consider a ``maximally'' satisfied coupling strategy $C=(\mathbf{\gamma}, \mathbf{\gamma}')$ for $L$; i.e. there is no other coupling strategy $C'$ for $L$ such that $C'$ satisfies more constraints than $C$. By lemma \ref{integersAreEnoughForATLemma}, we are allowed to only consider coupling strategies $C=(\gamma, \gamma')$ such that, for all $i\in AT(A)$, $\gamma_i \in \{-1, 0, 1\}$. 

    Fix some SLP $\rho$ in $A$ such that at least one constraint is not satisfied by $C$ as applied to $\rho$.

    By assumption, at least one constraint is unsatisfied by $C$. We will show that in every case, $A$ must contain at least one of a leaking cycle, leaking pair, disclosing cycle, or privacy violating path. By theorem \ref{DiPACounterexamplesThm}, this is sufficient to show that $A$ is not $d\varepsilon$-differentially private for any $d>0$.

    \textbf{Case 1: (1) is unsatisfied for $\gamma_i$}
    
    In this case, $c_i = \lguard[\texttt{x}]$ and $\gamma_i > \gamma_{at(i)}$. Note that $\gamma_{at(i)} \neq 1$. 

    We can assume that for all assignment transitions $t_{at(k)}$ in $\rho$ that $t_{at(k)}$ is not in a cycle, since otherwise there would be a leaking cycle in $A$. 

    \textbf{Case 1.1: $t_i$ is in a cycle}

    In this case, we can suppose that $t_i$ is not an assignment transition and $t_i$ does not output $\texttt{insample}$ or $\texttt{insample}'$, since otherwise either a leaking cycle or a disclosing cycle would clearly exist in $A$. We can thus additionally assume that constraint (5) is satisfied for $\gamma_i$. 
    
    Noe that the cycle containing $t_i$ is also an $\texttt{L}$-cycle by definition.

    Then attempting to resolve (1) for $\gamma_i$ by setting $\gamma_{at(i)} = 1$ must violate another constraint. In particular, either constraint (1) or (3) for $\gamma_{at(i)}$ or constraint (2) for some $\gamma_j$ such that $at(j) = at(i)$ must be newly violated. Note that constraint (5) for $\gamma_{at(i)}$ cannot be violated since we assumed that $t_{at(i)}$ is not in a cycle. 

    \textbf{Case 1.1.1: setting $\gamma_{at(i)} = 1$ violates constraint (1) for $\gamma_{at(i)}$}

    Let $t_{at(k)}$ be the earliest assignment transition before $t_{at(i)}$ such that, for all $at(k)\leq at(l)< at(i)$, $\gamma_{at(l)} <1$ and $c_{at(l)} = \lguard[\texttt{x}]$. Then there must be \textit{some} $\gamma_{at(l)}$ such that setting $\gamma_{at(l)} = 1$ would violate constraint (2) for some $\gamma_{l'}$ such that $at(l') = at(l)$. 

    Observe that $c_{l'} = \gguard[\texttt{x}]$ and there is an $\texttt{AL}$-path from $t_{l'}$ to $t_i$. 

    Then setting $\gamma_{l'}= 1$ must violate either constraint (3) or constraint (5) for $\gamma_{l'}$. If constraint (3) is violated, then $\gamma_{l'}$ is a transition with guard $\gguard[\texttt{x}]$ that outputs $\texttt{insample}$, so there is a privacy violating path from $t_{l'}$ to $t_i$. Otherwise if constraint (5) is violated, then $\gamma_{l'}$ is in a \gcycle, so there is a leaking pair composed of the cycles containing $t_{l'}$ and $t_i$, repectively. 

    \textbf{Case 1.1.2: Setting $\gamma_{at(i)}=1$ would violate (3) for $\gamma_{at(i)}$}

    Then $\gamma_{at(i)}$ is an assignment transition that outputs $\texttt{insample}$. Further, the path from $t_{at(i)}$ to $t_i$ is an $\texttt{AL}$-path, since there are no transitions on it. Thus, there is a privacy violating path from $t_i{at(i)}$ to $t_i$

    \textbf{Case 1.1.3: Setting $\gamma_{at(i)}=1$ would violate (2) for some $\gamma_j$ such that $at(j)= at(i)$}

    Note that, if $i<j$, the path from $t_i$ to $t_j$ (or vice versa, if $j<i$) is both an $\texttt{AL}$ and $\texttt{AG}$-path.

    Setting $\gamma_{j}= 1$ must violate either constraint (3) or constraint (5) for $\gamma_{j}$. 
    
    If constraint (3) is violated, then $\gamma_{j}$ is a transition with guard $\gguard[\texttt{x}]$ that outputs $\texttt{insample}$. Thus if $i<j$, there is a privacy violating path from $t_i$ to $t_j$ and if $j<i$, there is a privacy violating path from $t_j$ to $t_i$. 
    
    Otherwise if constraint (5) is violated, then $\gamma_{j}$ is in a \gcycle, so there is a leaking pair composed of the cycle containing $t_j$ and the cycle containing $t_i$ if $j<i$ or vice versa if $j>i$. 

    \textbf{Case 1.2: $t_i$ is not in a cycle}

    Note that $t_i$ must either be an assignment transition or output $\texttt{insample}$ or both, since otherwise, setting $\gamma_i = \gamma_{at(i)}$ would resolve constraint (1) for $\gamma_i$ without violating any other constraint. 

    \textbf{Case 1.2.1: $t_i$ outputs $\texttt{insample}$ and $t_i$ is an assignment transition}

    In this case, attempting to resolve constraint (1) without violating constraint (3) for $\gamma_i$ by setting $\gamma_i = \gamma_{at(i)} = 0$ must violate some other constraint. In particular, setting $\gamma_{at(i)} = 0$ can newly violate constraint (1) for $\gamma_{at(i)}$ or constraint (2) for some $\gamma_j$ such that $at(j) = at(i)$; note that setting $\gamma_{at(i)}=0$ cannot \textit{newly} violate constraint (1) for some $\gamma_j$ such that $at(j) = at(i)$. 
    Alternatively, setting $\gamma_i = 0$ could potentially newly violate either constraint (1) or constraint (2) for some $\gamma_j$ such that $at(j) = i$. 

    \textbf{Case 1.2.2.1: Setting $\gamma_{at(i)} =0$ violates constraint (1) for $\gamma_{at(i)}$}

    Let $t_{at(k)}$ be the earliest assignment transition before $t_{at(i)}$ such that, for all $at(k)\leq at(l)< at(i)$, $\gamma_{at(l)} = -1$ and $c_{at(l)} = \lguard[\texttt{x}]$. Then there must be \textit{some} $\gamma_{at(l)}$ such that setting $\gamma_{at(l)} = 0$ would violate constraint (2) for some $\gamma_{l'}$ such that $at(l') = at(l)$.  
    Additionally, note that setting $\gamma_{l'} = \gamma_{at(l)} = 0$ can only violate constraint (5) for $\gamma_{l'}$, since $\gamma_{l'}$ cannot be an assignment transition. 
    
    Thus, $t_{l'}$ is in a cycle, so the cycle containing $t_{l'}$ is a \gcycle. Note that the path from $t_{l'}$ to $t_i$ is an $\texttt{AL}$-path. Therefore, there is a privacy violating path from $t_{l'}$ to $t_i$.

    \textbf{Case 1.2.2.2: Setting $\gamma_{at(i)} =0$ violates constraint (2) for some $\gamma_j$ such that $at(j) = at(i)$}

    Note that $j\neq i$, meaning that $t_j$ is not an assignment transition. Then setting $\gamma_j = \gamma_{at(i)} = 0$ must violate constraint (5) for $\gamma_j$; this means that $t_j$ is in a \gcycle. 

    If $i<j$, then the path from $t_i$ to $t_j$ is an $\texttt{AL}$-path, so it is also a privacy violating path.

    Otherwise if $j<i$, then the path from $t_j$ to $t_i$ is an $\texttt{AG}$ path, so it is also a privacy violating path.

    \textbf{Case 1.2.2.3: Setting $\gamma_i =0$ violates constraint (1) for some $\gamma_j$ such that $at(j) = i$}

    If $\gamma_j$ is not an assignment transition, then setting $\gamma_j = \gamma_i = 0$ must violate constraint (5) for $\gamma_j$, so $t_j$ is in an \lcycle. Then there is a privacy violating path from $t_i$ to $t_j$, since the path from $t_{i+1}$ to $t_j$ is an $\texttt{AL}$-path by virtue of not containing any assignment transitions. 

    Otherwise if $t_j$ is an assignment transition, then $\gamma_j$ must originally be set to 1. Let $t_{at(k)}$ be the latest assignment after $t_{i}$ such that, for all $i\leq at(l)< at(k)$, $\gamma_{at(l)} = 1$ and $c_{at(l)} = \lguard[\texttt{x}]$. Then there must be \textit{some} $\gamma_{at(l)}$ such that setting $\gamma_{at(l)} = 0$ would violate constraint (1) for some $\gamma_{l'}$ such that $at(l') = at(l)$.  
    Additionally, note that setting $\gamma_{l'} = \gamma_{at(l)} = 0$ can only violate constraint (5) for $\gamma_{l'}$, since $\gamma_{l'}$ cannot be an assignment transition. 

    Then $\gamma_{l'}$ must be in an \lcycle. Since the path from $t_i$ to $t_{l'}$ is an $\texttt{AL}$-path, there is a privacy violating path from $t_i$ to $t_{l'}$.

    \textbf{Case 1.2.2.4: Setting $\gamma_i =0$ violates constraint (2) for some $\gamma_j$ such that $at(j) = i$}

    This case is exactly symmetric to case 1.2.2.3.

    \textbf{Case 1.2.2: $t_i$ outputs $\texttt{insample}$ and $t_i$ is not an assignment transition}

    We can assume that $\gamma_{at(i)} = -1$ originally, since otherwise, setting $\gamma_i = 0$ would resolve constraint (1) without violating any additional ones.

    Thus attempting to resolve constraint (1) while preserving constraint (3) for $\gamma_i$ by setting $\gamma_{at(i)}=\gamma_i =0$ must violate constraint (1) for $\gamma_{at(i)}$. 
   
    Let $t_{at(k)}$ be the earliest assignment transition before $t_{at(i)}$ such that, for all $at(k)\leq at(l)< at(i)$, $\gamma_{at(l)} = -1$ and $c_{at(l)} = \lguard[\texttt{x}]$. Then there must be some $\gamma_{at(l)}$ such that setting $\gamma_{at(l)} = 0$ would violate constraint (2) for some $\gamma_{l'}$ such that $at(l') = at(l)$.  
    Additionally, note that setting $\gamma_{l'} = \gamma_{at(l)} = 0$ can only violate constraint (5) for $\gamma_{l'}$, since $\gamma_{l'}$ cannot be an assignment transition. 
    
    Thus, $t_{l'}$ is in a cycle, so the cycle containing $t_{l'}$ is a \gcycle. Note that the path from $t_{l'}$ to $t_i$ is an $\texttt{AL}$-path. Therefore, there is a privacy violating path from $t_{l'}$ to $t_i$.

    \textbf{Case 1.2.3: $t_i$ does not output $\texttt{insample}$ and $t_i$ is an assignment transition}

    In this case, attempting to resolve (1) by setting $\gamma_{at(i)} = 1$ must violate either constraint (1) or (3) for $\gamma_{at(i)}$, or constraint (2) for some $\gamma_j$ such that $at(j) = at(i)$. 

    Additionally, note that $\gamma_{at(i)} \in \{0, -1\}$. 

    \textbf{Case 1.2.3.1: $\gamma_{at(i)} =0$}

    Since originally, $\gamma_i > \gamma_{at(i)} \implies \gamma_i = 1$, we know that setting $\gamma_i = \gamma_{at(i)} = 0$ must violate constraint (1) for some $\gamma_j$ such that $at(j) = i$. If $t_{j}$ is not an assignment transition, then setting $\gamma_{j} = 0$ can only violate constraint (5) for $\gamma_{j}$, meaning that $t_j$ is in an \lcycle.

    Otherwise, if $t_j$ is an assignment transition, let $t_{at(k)}$ be the latest assignment transition after $t_{i}$ such that for all $j\leq at(l)< at(k)$, $\gamma_{at(l)} =1$ and $c_{at(k)} = \lguard[\texttt{x}]$. Then there must exist some $at(l), j\leq at(l)< at(k)$ such that setting $\gamma_{at(l)}=0$ would violate constraint (1) for some non-assignment $\gamma_{l'}$ where $at(l') = at(l)$. 

    Further, setting $\gamma_{l'} = 0$ must then violate constraint (5) for $\gamma_{l'}$, so $t_{l'}$ is in an \lcycle. 

    Therefore, there exists a $\texttt{AL}$-path from $t_i$ to some transition $t$ in an \lcycle. 

    \textbf{Case 1.2.3.1.1: Setting $\gamma_{at(i)} = 1$ would violate constraint (1) for $\gamma_{at(i)}$}

    Let $t_{at(j)}$ be the earliest assignment transition before $t_{at(i)}$ such that for all $at(j)\leq at(k)< at(i)$, $\gamma_{at(k)} =0$ and $c_{at(k)} = \lguard[\texttt{x}]$. Then there must exist some $at(k), at(j)\leq at(k)< at(i)$ such that setting $\gamma_{at(k)}=1$ would violate constraint (2) for some non-assignment $\gamma_l$ where $at(l) = at(k)$, so $c_l = \gguard[\texttt{x}]$

    Note that there is an $\texttt{AL}$-path from $t_l$ to $t_i$, and therefore an $\texttt{AL}$-path from $t_l$ to some transition $t_{o}$ in an \lcycle. 

    Further, setting $\gamma_l = \gamma_{at(k)} = 1$ must then violate either constraint (3) or (5) for $\gamma_l$. 

    If constraint (3) is violated, then $t_l$ outputs $\texttt{insample}$, so there is a privacy violating from $t_l$ to $t_o$.

    If constraint (5) is violated, then $t_l$ is in a \gcycle, so there is a leaking pair consisting of the cycle containing $t_l$ and the cycle containing $t_o$. 

    \textbf{Case 1.2.3.1.2: Setting $\gamma_{at(i)}=1$ would violate constraint (3) for $\gamma_{at(i)}$}

    Note that there is an $\texttt{AL}$ path from $t_{at(i)}$ to some transition $t_j$ such that $t_j$ is in an \lcycle.

    Then $t_{at(i)}$ is an assignment transition that outputs $\texttt{insample}$, so there is a privacy violating path from $t_{at(i)}$ to $t_j$. 
    
    \textbf{Case 1.2.3.1.3: Setting $\gamma_{at(i)}=1$ would violate constraint (2) for some $\gamma_j$ such that $at(j) = at(i)$}

    As before, note that there is an $\texttt{AL}$ path from $t_{j}$ to some transition $t_k$ such that $t_k$ is in an \lcycle.

    Then trying to set $\gamma_j = \gamma_{at(i)} = 1$ must violate either constraint (3) or constraint (5) for $\gamma_j$. If constraint (3) is violated, then $t_j$ outputs $\texttt{insample}$, so there is a privacy violating from $t_j$ to $t_k$. If constraint (5) is violated, then $t_j$ is in a \gcycle, so there is a leaking pair consisting of the cycle containing $t_j$ and the cycle containing $t_k$.  

    \textbf{Case 1.2.3.2: $\gamma_{at(i)} = -1$}

    Note that $\gamma_i \in \{0, 1\}$.

    First, if $\gamma_i = 0$, then setting $\gamma_i = -1$ must newly violate constraint (1) for some $\gamma_j$ where $at(j) = i$. If $t_{j}$ is not an assignment transition, then setting $\gamma_{j} = -1$ can only newly violate constraint (3) for $\gamma_{j}$, meaning that $t_j$ outputs $\texttt{insample}$.

    Otherwise, if $t_j$ is an assignment transition, let $t_{at(k)}$ be the latest assignment transition after $t_{i}$ such that for all $j\leq at(l)< at(k)$, $\gamma_{at(l)} =0$ and $c_{at(k)} = \lguard[\texttt{x}]$. Then there must exist some $at(l), j\leq at(l)< at(k)$ such that setting $\gamma_{at(l)}=-1$ would newly violate constraint (1) for some non-assignment $\gamma_{l'}$ where $at(l') = at(l)$; as before, this means that $t_{l'}$ outputs $\texttt{insample}$.  

    Otherwise, if $\gamma_i = 1$, then setting $\gamma_i = -1$ must newly violate constraint (1) for some $\gamma_j$ where $at(j) = i$. If $t_{j}$ is not an assignment transition, then setting $\gamma_{j} = -1$ can only newly violate constraint (5) for $\gamma_{j}$, meaning that $t_j$ is in an \lcycle. 

    Otherwise, if $t_j$ is an assignment transition, let $t_{at(k)}$ be the latest assignment transition after $t_{i}$ such that for all $j\leq at(l)< at(k)$, $\gamma_{at(l)} =0$ and $c_{at(k)} = \lguard[\texttt{x}]$. Then there must exist some $at(l), j\leq at(l)< at(k)$ such that setting $\gamma_{at(l)}=-1$ would newly violate constraint (1) for some non-assignment $\gamma_{l'}$ where $at(l') = at(l)$; as before, this means that $t_{l'}$ is in an \lcycle.  

    Thus, if $\gamma_i =0$, then there is $\texttt{AL}$-path from $t_i$ to some other transition that has guard $\lguard[\texttt{x}]$ and outputs $\texttt{insample}$. Otherwise, if $\gamma_i = 1$, there is an $\texttt{AL}$-path from $t_i$ to some other transition that is in an \lcycle. 

    \textbf{Case 1.2.3.2.1: Setting $\gamma_{at(i)} = \gamma_i$ would violate constraint (1) for $\gamma_{at(i)}$}

    Let $t_{at(j)}$ be the earliest assignment transition before $t_{at(i)}$ such that for all $at(j)\leq at(k)< at(i)$, $\gamma_{at(k)} =-1$ and $c_{at(k)} = \lguard[\texttt{x}]$. Then there must exist some $at(k), at(j)\leq at(k)< at(i)$ such that setting $\gamma_{at(k)}=\gamma_i$ would newly violate constraint (2) for some non-assignment $\gamma_l$ where $at(l) = at(k)$.

    First note that $c_l = \gguard[\texttt{x}]$ and there is an $\texttt{AL}$-path from $t_l$ to $t_i$. 
    
    If $\gamma_i = 0$, then setting $\gamma_l = \gamma_{at(k)} = \gamma_i = 0$ can only newly violate constraint (5) for $\gamma_l$. Thus, $\gamma_l$ is in a \gcycle. Since $\gamma_i = 0$, there exists some $t_{l'}$ such that there is an $\texttt{AL}$-path from $t_i$ to $t_{l'}$ and $t_{l'}$ has guard $\lguard[\texttt{x}]$ and outputs $\texttt{insample}$. Thus, there is an $\texttt{AL}$ path from $t_l$ to $t_{l'}$, and so there is a privacy violating path from $t_l$ to $t_{l'}$. 
    
    If $\gamma_i = 1$, then setting $\gamma_l = \gamma_{at(k)} = \gamma_i = 1$ can newly violate constraints (3) or (5) for $\gamma_l$. Further, since $\gamma_i =1$, there exists some $t_{l'}$ such that there is an $\texttt{AL}$-path from $t_i$ to $t_{l'}$ and $t_{l'}$ is in an \lcycle. Thus, there is an $\texttt{AL}$ path from $t_l$ to $t_{l'}$.
    
    If constraint (3) is newly violated, then $t_l$ is a transition with guard $\gguard[\texttt{x}]$ that outputs $\texttt{insample}$. Thus, there is a privacy violating path from $t_l$ to $t_{l'}$. 

    If constraint (5) is newly violated, then $t_l$ is in a \gcycle. Thus, there is a leaking pair composed of the cycles containing $t_l$ and $t_{l'}$, respectively.

    \textbf{Case 1.2.3.2.2: Setting $\gamma_{at(i)}=\gamma_i$ would violate constraint (3) for $\gamma_{at(i)}$}

    First, $t_{at(i)}$ is an assignment transition that outputs $\texttt{insample}$. Since $\gamma_i = 1$, there exists some $t_j$ such that there is an $\texttt{AL}$-path from $t_{at(i)}$ to $t_j$ and $t_j$ is in an \lcycle. Then there is a privacy violating path from $t_{at(i)}$ to $t_j$.

    \textbf{Case 1.2.3.2.3: Setting $\gamma_{at(i)}=\gamma_i$ would violate constraint (2) for some $\gamma_j$ such that $at(j) = at(i)$}

    Observe that $t_j$ is not an assignment transition and has guard $\gguard[\texttt{x}]$. Additionally, there is an $\texttt{AL}$-path from $t_j$ to $t_i$ since there are no assignments between $t_j$ and $t_i$. 

    If $\gamma_i =0$, then setting $\gamma_j = \gamma_{at(i)} = 0$ can only newly violate constraint (5) for $\gamma_j$. Thus, $\gamma_j$ is in a \gcycle. Since $\gamma_i = 0$, there exists some $t_{k}$ such that there is an $\texttt{AL}$-path from $t_i$ to $t_{k}$. Thus, there is an $\texttt{AL}$ path from $t_j$ to $t_{k}$, and so there is a leaking pair composed of the cycles containing $t_j$ and $t_{k}$, respectively. 

    If $\gamma_i = 1$, then setting $\gamma_j = \gamma_{at(i)}=1$ can newly violate constraints (3) or (5) for $\gamma_l$. Further, since $\gamma_i =1$, there exists some $t_{k}$ such that there is an $\texttt{AL}$-path from $t_i$ to $t_{k}$ and $t_{k}$ is in an \lcycle. Thus, there is an $\texttt{AL}$ path from $t_j$ to $t_{k}$.
    
    If constraint (3) is newly violated, then $t_j$ is a transition with guard $\gguard[\texttt{x}]$ that outputs $\texttt{insample}$. Thus, there is a privacy violating path from $t_j$ to $t_{k}$. 

    If constraint (5) is newly violated, then $t_j$ is in a \gcycle. Thus, there is a leaking pair composed of the cycles containing $t_j$ and $t_{k}$, respectively.

    \textbf{Case 2: (2) is unsatisfied for $\gamma_i$}

    This case is exactly symmetric to case (1).

    \textbf{Case 3: (3) is unsatisfied for $\gamma_i$}

    First note that if $t_i$ is in a cycle, then that cycle will be a disclosing cycle because $t_i$ outputs $\texttt{insample}$. Thus, we will assume that $t_i$ is not in a cycle.

    Because $C$ is maximal, setting $\gamma_i=0$ must violate at least one of constraints (1) or (2) for $\gamma_i$ or (1) for some $\gamma_l$ such that $at(l) = i$.

    \textbf{Case 3.1: Satisfying (3) for $\gamma_i$ would violate (1) for $\gamma_i$}

    This means that $\gamma_{at(i)}<0\implies \gamma_{at(i)} = -1$. Further, $c_i = \lguard[\texttt{x}]$. Then changing $\gamma_{at(i)}=0$ can newly violate constraints (1) or (5) for $\gamma_{at(i)}$ or constraint (2) for some $\gamma_j$ such that $at(j) = at(i)$.

    If constraint (5) is newly violated, then $t_{at(i)}$ is in a cycle. In particular, the cycle must be a leaking cycle; if $t_i$ and $t_{at(i)}$ are both contained in a cycle, then it must be leaking because $c_i = \lguard[\texttt{x}]$. Otherwise, there still must be some transition in the cycle containing $t_{at(i)}$ that has a non-$\texttt{true}$ guard since otherwise a path from $t_{at(i)}$ to $t_i$ could not exist. 

    By similar reasoning, we can assume that for every assignment transition $t_{at(j)}$before $t_{at(i)}$ on a initialized path to $t_i$, $t_{at(j)}$ is not in a cycle. 

    If constraint (1) is newly violated for $\gamma_{at(i)}$, then $c_{at(i)} = \lguard[\texttt{x}]$. Let $t_{at(j)}$ be the earliest assignment transition before $t_{at(i)}$ such that $\gamma_{at(l)} = -1$ and for all assignment transitions $t_{at(k)}$ between $t_{at(j)}$ and $t_{at(i)}$, $c_{at(k)} = \lguard[\texttt{x}]$ and $\gamma_{at(k)} = -1$. 
    
    Then there must exist some assignment transition $t_{at(k)}$, $at(j)\leq at(k)\leq at(i)$ between $t_{at(j)}$ and $t_{at(i)}$ such that setting $\gamma_{at(k)} = 0$ would newly violate constraint (2) for some $l$ where $at(l) = at(k)$. In particular, this must be because $t_l$ is in a cycle and setting $\gamma_l = 0$ would violate constraint (5). Thus, $t_l$ is in a $\texttt{G}$-cycle. Then there is an $\texttt{AL}$-path from $t_l$ to $t_i$, creating a privacy violating path from $t_l$ to $t_i$. 

    If changing $\gamma_{at(i)}$ from $-1$ to $0$ means that constraint (2) would be newly violated for some $\gamma_j$ such that $at(j) = at(i)$, note that $\gamma_j < 0$ and $c_j = \gguard[\texttt{x}]$. 
    
    So setting $\gamma_j = 0$ can violate either (2) for some $\gamma_l$ where $at(l) = j$ or (5) for $\gamma_j$. 

    If setting $\gamma_j = 0$ would violate constraint (2) for some $\gamma_l$ where $at(l) = j$, then let let $t_{at(m)}$ be the latest assignment transition after $t_{at(j)}$ such that $\gamma_{at(m)} = -1$ and for all assignment transitions $t_{at(k)}$ between $t_{at(j)}$ and $t_{at(m)}$, $c_{at(k)} = \gguard[\texttt{x}]$ and $\gamma_{at(k)} = -1$. 
    
    Then there must exist some assignment transition $t_{at(k)}$, $at(j)\leq at(k)\leq at(m)$ between $t_{at(j)}$ and $t_{at(m)}$ such that setting $\gamma_{at(k)} = 0$ would newly violate constraint (2) for some $l'$ where $at(l') = at(k)$. 
    In particular, this must be because $t_{l'}$ is in a cycle and setting $\gamma_l = 0$ would violate constraint (5). Thus, $t_l$ is in a $\texttt{G}$-cycle. Then there is an $\texttt{AG}$-path from $t_i$ to $t_l$, creating a privacy violating path from $t_i$ to $t_l$. 

    Otherwise, if setting $\gamma_j = 0$ would violate constraint (5) for $\gamma_j$, then $t_j$ is in a $\texttt{G}$-cycle. We can assume that $j\neq i$ because otherwise, the cycle containing $t_j$ would be a disclosing cycle. Additionally, note that there are no assignment transitions between $t_i$ and $t_j$ or vice versa, since $at(j) = at(i)$.
    Thus, if $j<i$, then there is an $\texttt{AL}$-path from $t_j$ to $t_i$, which forms a privacy violating path. Symmetrically, if $i<j$, then there is an $\texttt{AG}-$path from $t_i$ to $t_j$, which again forms a privacy violating path. 

    \textbf{Case 3.2: Satisfying (3) for $\gamma_i$ would violate (2) for $\gamma_i$}

    This case is exactly symmetric to case (3a).

    \textbf{Case 3.3: Satisfying (3) for $\gamma_i$ would violate (1) for some $\gamma_l$ where $at(l) = i$}

    Note that $t_i$ must be an assignment transition. Further, we know that $\gamma_l>0$ and $c_l = \lguard[\texttt{x}]$. 
    
    Because $C$ is maximal, setting $\gamma_l=0$ would now violate either constraint (1) for some $\gamma_{l'}$ where $at(l') = l$ or constraint (5) for $\gamma_l$. Note that because $\gamma_l>0$, constraint (2) cannot be newly violated for some $\gamma_{l'}$ where $at(l') = l$.

    If constraint (5) would be newly violated for $\gamma_l$, then $\gamma_l$ is in an $\texttt{L}$-cycle. Additionally, note that the path from $t_{i+1}$ to $t_l$ is an $\texttt{AL}$-path, so there is a privacy violating path from $t_i$ to $t_l$. 

    Otherwise, if setting $\gamma_l = 0$ would violate constraint (1) for some $\gamma_{l'}$ where $at(l')=l$, let $t_{at(j)}$ be the latest assignment transition such that $c_{at(j)} = \lguard[\texttt{x}]$ and $\gamma_{at(j)}<1$ and, for all assignment transitions $t_{at(k)}$ between $t_l$ and $t_{at(j)}$, $c_{at(k)} = \lguard[\texttt{x}]$ and $\gamma_{at(k)}<1$. 

    If $at(j) = l$, then $l'$ is not an assignment transition. Then, setting $\gamma_{l'} = 0$ could only violate constraint (5). In this case, as before, there is a privacy violating path from $t_i$ to $t_l$. 

    Otherwise, since $C$ is maximal, we cannot set $\gamma_{at(k)}=0$ for any $l<at(k)\leq at(j)$ without violating another constraint. In particular, there must be some $at(k)$ such that setting $\gamma_{at(k)} = 0$ would violate constraint (1) for some $\gamma_{k'}$ such that $at(k') = at(k)$. Note that there must be an \texttt{AL}-path from $t_i$ to $t_{k'}$. Then, as before, there must be a privacy violating path from $t_i$ to $t_{k'}$. 


    \textbf{Case 3.4: Satisfying (3) for $\gamma_i$ would violate (2) for some $\gamma_l$ where $at(l) = i$} 

    This case is exactly symmetric to case (3c).

    \textbf{Case 4: (4) is unsatisfied for $\gamma_i'$}
    
    Because $C$ is maximal, setting $\gamma_i'=0$ must violate some other constraint. In particular, this must mean that constraint (6) is now violated. However, this would imply that $t_i$ is in a cycle, and so the cycle containing $t_i$ would be a disclosing cycle.

    \textbf{Case 5: (5) is unsatisfied for $t_i$:} Because $C$ is maximal, we know that if $\gamma_i = -\texttt{in}_i\brangle{1}+\texttt{in}_i\brangle{2}$ then another constraint must be violated. In particular, at least one of constraints (1), (2), or (3) must be violated for $\gamma_i$. 
    
    \textbf{Case 5.1: Satisfying (5) for $t_i$ would violate (1)}

    If (1) is now violated, then either $t_i$ is an assignment transition or $c_i = \lguard[\texttt{x}]$ and $\gamma_{at(i)}<1$. If $t_i$ is an assignment transition, then the cycle containing $t_i$ has a transition with a non-$\texttt{true}$ guard ($t_i$) and an assignment transition, so it must be a leaking cycle. 

    Otherwise, if $t_i$ is not an assignment transition, $c_i = \lguard[\texttt{x}]$, and constraint (1) is violated for $\gamma_i$, we must have that $\gamma_{at(i)}<1$ due to other constraints.
    
    Consider all assignment transitions in $\rho$ before $t_i$. Note that if any such assignment transition is in a cycle, then that cycle must be a leaking cycle since either the assignment transition is in the same cycle as $t_i$ or there must be some non-$\texttt{true}$ transition in the cycle because otherwise $t_i$ is unreachable.

    So assume that all assignment transitions in $\rho$ before $t_i$ are not in a cycle. Then if $c_{at(i)} \neq \lguard[\texttt{x}]$, because $C$ is maximal, this must mean that $t_{at(i)}$ outputs $\texttt{insample}$. Note that the path from $t_{at(i)+1}$ to $t_i$ is an $\texttt{AL}$-path (since there are no assignment transitions on it) and $t_i$ is in an $\texttt{L}$-cycle since $t_i$ is in a cycle and $c_i = \lguard[\texttt{x}]$. 
    Then the path from $t_{at(i)}$ (an assignment transition that outputs $\texttt{insample}$) to $t_i$ is a privacy violating path. 

    If $c_{at(i)} = \lguard[\texttt{x}]$, then let $c_{at(j)}$ be the earliest assignment transition such that $c_{at(j)} = \lguard[\texttt{x}]$ and $\gamma_{at(j)} < 1$ and, for all assignment transitions $t_{at(k)}$ between $t_{at(j)}$ and $t_i$, $c_{at(k)} = \lguard[\texttt{x}]$ and $\gamma_{at(k)} < 1$. Note that such an $t_{at(j)}$ must exist. 

    If $t_{at(j)} = t_{at(i)}$, then setting $\gamma_{at(i)} =1$ must violate either constraint (2) for some other $\gamma_l$ such that $at(l)=at(i)$, or constraint (3) for $\gamma_{at(i)}$. Without loss of generality, we will assume that $l\neq i$. If constraint (3) would be violated, then as before, there exists a privacy violating path from $t_{at(j)}$ to $t_i$. 
    If constraint (2) would be violated for some $\gamma_l$ such that $at(l)=at(i)$, then either $t_l$ must output $\texttt{insample}$ or $t_l$ must be in a cycle. 
    
    Suppose that $i<l$; then that the path from $t_i$ to $t_l$ is both an $\texttt{AG}$-path and an $\texttt{AL}$-path (since there are no assignment transitions on it). Thus, if $t_l$ outputs $\texttt{insample}$, there exists a privacy violating path from $t_i$ to $t_l$ and if $t_l$ is in a cycle, then the cycle containing $t_i$ and the cycle containing $t_l$ together make up a leaking pair, since the cycle containing $t_l$ is a $\texttt{G}$-cycle by definition. 
    Symmetrically, if $l>i$, then either the path from $t_l$ to $t_i$ is a privacy violating path or the cycle containing $t_l$ and the cycle containing $t_i$ make up a leaking pair.
    
    Otherwise, note that the path from $t_{at(j)}$ to $t_i$ is an $\texttt{AL}-$path. Since $C$ is maximal, we cannot set $\gamma_{at(k)}=1$ for $\gamma_{at(j)}$ or for any of the other assignment transitions $t_{at(k)}$ between $t_{at(j)}$ and $t_i$ without violating another constraint. 
    In particular, there must be some $t_{at(k)}$ where $at(j)\leq at(k)<i$ such that setting $\gamma_{at(k)} = 1$ would mean that either constraint (2) for some $\gamma_l$ such that $at(l) = at(k)$ or constraint (3) would be violated for $\gamma_{at(k)}$. 
    If constraint (3) would be violated for $\gamma_{at(k)}$ then $t_{at(k)}$ outputs $\texttt{insample}$, so as before, there is a privacy violating path from $t_{at(k)}$ to $t_i$. Otherwise if constraint (2) would be violated for some $\gamma_l$ such that $at(l) = at(k)$, then as before, $\gamma_l$ must either output $\texttt{insample}$ or $t_l$ is in a cycle. 
    Just like before, this means that there must be either a privacy violating path from $t_l$ to $t_i$ or the cycle containing $t_l$ and the cycle containing $t_i$ together make up a leaking pair. 

    \textbf{Case 5.2: Satisfying (5) for $t_i$ would violate (2)}

    This case is exactly symmetric to case (5a).

    \textbf{Case 5.3: Satisfying (5) for $t_i$ would violate (3)}

    If (3) would be violated, then $t_i$ must output $\texttt{insample}$ and be a non-public transition. Then the cycle containing $t_i$ must be a disclosing cycle. 
    
    \textbf{Case 6: (6) is unsatisfied for $t_i$:} Because $C$ is maximal, we know that if $\gamma_i' = -\texttt{in}_i\brangle{1}+\texttt{in}_i\brangle{2}$ then another constraint must be violated for $\gamma_i'$. In particular, constraint (4) must be violated, since no other constraint involves $\gamma_i'$. 
    Then $t_i$ is a transition in a cycle that outputs $\texttt{insample}'$, so $A$ has a disclosing cycle.
\end{proof}


\subsection{Minimizing Coupling Cost}

\begin{prop}\label{costDependspathProp}
    There exist SLPs $\rho'=\pi\cdot \pi$, $\rho^*=\pi\cdot\pi^*$ that share a prefix SLP $\pi$ such that for all optimal coupling strategies $C'$ for $\rho'$ and $C^*$ for $\rho^*$, $C'$ and $C^*$ differ on the shift assigned to some transition in $\pi$. 

    In other words, the optimal strategy $C$ must assign different coupling strategies to occurances of the same transition in different SLPs. 
\end{prop}

\begin{proof}
    First, we define the following transitions that we use to construct our counterexample:
    \begin{align*}
        t_{init} &= (\texttt{true}, \bot, \texttt{true})\\
        t_{geq1} &= (\gguard[\texttt{x}], \top, \texttt{false})\\
        t_{leq1} &= (\lguard[\texttt{x}], \bot, \texttt{false})\\
        t_{geq2} &= (\gguard[\texttt{x}], \top, \texttt{false})\\
        t_{leq2} &= (\lguard[\texttt{x}], \bot, \texttt{false})
    \end{align*}

    For all transitions $t$, let $P(t) = (1, 1)$.

    Let $\rho_1 = t_{init}t_{geq1}t_{geq2}^n$ and $\rho_2 = t_{init}t_{leq1}t_{leq2}^n$, where $n\in \NN$. Observe that $\rho_1$ and $\rho_2$ share the prefix $t_{init}$. 
    
    We make the following observations: 
    \begin{itemize}
        \item The cost of any coupling strategy for $\rho_1$ or $\rho_2$ must be at least 2:
        
        Let $C_{\rho_1} = (\gamma, \gamma')$ be a coupling strategy for $\rho_1$. We can bound its cost as follows: 
        \begin{align*}
            cost(C_{\rho_1}) &= \max_{\texttt{in}\brangle{1}\sim\texttt{in}\brangle{2}}\sum_{i=0}^{n+2}(|-\texttt{in}_i\brangle{1}+\texttt{in}_i\brangle{2}-\gamma_i(\texttt{in}_i\brangle{1}, \texttt{in}_i\brangle{2}))\\&\qquad+(|-\texttt{in}_i\brangle{1}+\texttt{in}_i\brangle{2}-\gamma_i'(\texttt{in}_i\brangle{1}, \texttt{in}_i\brangle{2})|)\\
            &\geq \max_{\texttt{in}\brangle{1}\sim\texttt{in}\brangle{2}} \sum_{i=0}^{n+2}(|-\texttt{in}_i\brangle{1}+\texttt{in}_i\brangle{2}-\gamma_i(\texttt{in}_i\brangle{1}, \texttt{in}_i\brangle{2})|)\\
            &= \max_{\Delta \in [-1, 1]^{n+2}} \sum_{i=0}^{n+2}(|\Delta_i-\gamma_i(0, \Delta_i)|)\\
            &\geq |1 - \gamma_0(0, 1)| + \sum_{i=1}^{n+2}|-1-\gamma_i(0, -1)|\\
            &= 1 - \gamma_0(0, 1) + \sum_{i=1}^{n+2} (1+\gamma_i(0, -1))\\
            &= 1 - \gamma_0(0, 1) + (n + 2) + \sum_{i=1}^{n+2}\gamma_i(0, -1)\\
            &\geq 1 - \gamma_0(0, 1) + (n + 2) + \sum_{i=1}^{n+2}\gamma_0(0, 1) \qquad \text{(privacy constraint)}\\
            &= (n + 3) + (n + 1) \gamma_0(0, 1)\\
            &\geq 2
        \end{align*}

    and by a similar argument, $cost(C_{\rho_2})\geq 2$ for any coupling strategy $C_{\rho_2}$ for $\rho_2$. 

    \item There exist coupling strategies $C_{\rho_1}^*$ and $C_{\rho_2}^*$ for $\rho_1$ and $\rho_2$, respectively, such that $cost(C_{\rho_1}^*) = cost(C_{\rho_2}^*) = 2$. 
    
    We will first describe $C_{\rho_1}^* = (\gamma, \gamma')$. Since no transition outputs \texttt{insample}, we can set $\gamma_i'(\texttt{in}\brangle{1}, \texttt{in}\brangle{2}) = \texttt{in}\brangle{2} - \texttt{in}\brangle{1}$ for all $i$ with no privacy cost. Define 
    \begin{align*}
        \gamma_0(\texttt{in}\brangle{1}, \texttt{in}\brangle{2}) &= -1 \\
        \gamma_i(\texttt{in}\brangle{1}, \texttt{in}\brangle{2}) &= \texttt{in}_i\brangle{2} - \texttt{in}_i\brangle{1} \qquad \text{for all $i>0$}
    \end{align*}
    We see that $C^*_{\rho_1}$ is valid, since $\gamma_i\geq \gamma_{0}$ for all $i>0$. Further, we see that 
    \begin{align*}
        cost(C^*_{\rho_1}) &= \max_{\texttt{in}\brangle{1}\sim\texttt{in}\brangle{2}}\sum_{i=0}^{n+2}(|-\texttt{in}_i\brangle{1}+\texttt{in}_i\brangle{2}-\gamma_i(\texttt{in}_i\brangle{1}, \texttt{in}_i\brangle{2}))\\&\qquad+(|-\texttt{in}_i\brangle{1}+\texttt{in}_i\brangle{2}-\gamma_i'(\texttt{in}_i\brangle{1}, \texttt{in}_i\brangle{2})|)\\
        &= \max_{\texttt{in}\brangle{1}\sim\texttt{in}\brangle{2}} |-\texttt{in}_0\brangle{1}+\texttt{in}_0\brangle{2}-\gamma_0(\texttt{in}_0\brangle{1}, \texttt{in}_0\brangle{2})| \\
        &= \max_{\texttt{in}\brangle{1}\sim\texttt{in}\brangle{2}} |-\texttt{in}_0\brangle{1}+\texttt{in}_0\brangle{2}+1|\\
        &\leq 2 
    \end{align*}
    showing that $cost(C^*_{\rho_1}) = 2$. Similarly, there is a coupling strategy $C^*_{\rho_2}$ for which $cost(C^*_{\rho_2}) = 2$.
    
    \item Any pair of coupling strategies $C_{\rho_1} = (\gamma^{(1)}, \gamma^{\prime(1)})$ and $C_{\rho_2}=(\gamma^{(2)}, \gamma^{\prime(2)})$ for $\rho_1$, $\rho_2$, respectively, that assign the same shifts to $t_{init}$ in both $\rho_1$ and $\rho_2$ must be such that $\max\{cost(C_{\rho_1}), cost(C_{\rho_2})\}>2$.
    
    For the sake of contradiction, suppose that $\max\{cost(C_{\rho_1}), cost(C_{\rho_2})\}=2$. Note that it is thus impossible for $\gamma_0^{(1)}(0, 1) \neq -1$, since then $cost(C_{\rho_1}) > 2$ in the same manner as above.  

    Thus, $\gamma_0^{(1)}(0, 1) = -1$ in $C_{\rho_1}$, which by hypothesis, means that $\gamma_0^{(2)}(0, 1) = -1$ as well. We have the privacy constraint $\gamma_i^{(2)} \leq \gamma_0^{(2)}$ for $i>0$, which also means that $\gamma_i = -1$ identically for all $i > 0$. However, this means that 
    \begin{align*}
        cost(C_{\rho_2}) &= \max_{\texttt{in}\brangle{1}\sim\texttt{in}\brangle{2}}\sum_{i=0}^{n+2}(|-\texttt{in}_i\brangle{1}+\texttt{in}_i\brangle{2}-\gamma_i^{(2)}(\texttt{in}_i\brangle{1}, \texttt{in}_i\brangle{2}))\\&\qquad+(|-\texttt{in}_i\brangle{1}+\texttt{in}_i\brangle{2}-\gamma_i^{\prime(2)}(\texttt{in}_i\brangle{1}, \texttt{in}_i\brangle{2})|)\\
        &\geq \max_{\texttt{in}\brangle{1}\sim\texttt{in}\brangle{2}} \sum_{i=0}^{n+2}(|-\texttt{in}_i\brangle{1}+\texttt{in}_i\brangle{2}-\gamma_i^{(2)}(\texttt{in}_i\brangle{1}, \texttt{in}_i\brangle{2})|)\\
        &\geq \max_{\Delta \in [-1, 1]^{n+2}} |\Delta_0 - \gamma_i^{(2)}(0, \Delta_0)| + \sum_{i=1}^{n+2}(|\Delta_i-\gamma_i^{(2)}(0, \Delta_i)|)\\
        &\geq |1 - \gamma_i^{(2)}(0, 1)| + \sum_{i=1}^{n+2}(|1+1|)\\
        &= 2 \cdot (n + 2)
    \end{align*}
    which is a contradiction. 

    \end{itemize}
    This completes the proof.     
\end{proof}

\begin{prop}(Precise statement of proposition \ref{quadraticPenaltyProp})
    There exist a family of sets of SLPs $\{B_n\}_{n\in \NN}$ for which the cost of any unified coupling strategy $C$ for SLPs in $B_n$ is in $\Omega(n^2)$, but for which there exist SLP-specific coupling strategies for $B_n$ suc   h that their total cost is in $O(n)$.
\end{prop}

\begin{proof}
    \sky{this construction is very confusing - asking Vishnu about it but remove if he doesn't get back to me}
    We first define the following transitions for all $1\leq i\leq n$: \begin{itemize}
        \item $t_{true}^{(i)} = (\texttt{true}, \bot, \texttt{true})$
        \item $t_{geq}^{(i)} = (\gguard[\texttt{x}],\bot, \texttt{true})$
        \item $t_{leq}^{(i)} = (\lguard[\texttt{x}], \top, \texttt{false})$
        \item $t_{loop}^{(i)} = (\lguard[\texttt{x}], \top, \texttt{false})$
    \end{itemize}

    For all transitions $t$, let $P(t) = (1, 1)$. 


    Consider an SLP $\rho = t_{true}^{(1)} t_{geq}^{(1)} \left(t_{leq}^{(1)}\right)^n \dots t_{true}^{(n)} t_{geq}^{(n)} \left(t_{leq}^{(n)}\right)^n$

    For each $i$, construct also the periodic program $L_i = L\left(t_{true}^{(n + i)} \left(t_{loop}^{(i)}\right)^* t_{geq}^{(i)}\right)$ such that each transition $t_{geq}^{(i)}$ is preceded by an arbitrary number of transitions with guard $\lguard$.
    
    Let $C$ be an SLP-independent coupling strategy for $B_n$ -- this menas that $C$ must assign the same shift to each transition in $B_n$. 

    From the privacy constraints on the periodic programs $L_i$, we see that $\gamma_{t_{true}^{n + i}}(1, 0) = 1$ from the same method as in \ref{costDependspathProp}, which then implies from the constraint $\gamma_{t_{true}^{n + i}} \leq \gamma_{t_{geq}^{(i)}}$ that $\gamma_{t_{geq}^{(i)}}(1, 0) = 1$. 

    Since the preceding assignment transition for $t_{leq}^{(i)}$ is given by $t_{geq}^{(i)}$, for which we have the privacy constraint $\gamma_{t_{geq}^{(i)}} \leq \gamma_{t_{leq}^{(i)}}$, we see that $\gamma_{t_{leq}^{(i)}}(1, 0) = 1$ as well. 

    Computing the cost of the coupling strategy assigned to $\rho$, which has $n \cdot (n + 2)$ transitions, we get 

    \begin{align*}
        cost(C_\rho) &\geq \max_{\Delta = \texttt{in}\brangle{2} - \texttt{in}\brangle{1}} \sum_{i=1}^{n} \big(|\Delta_{t_{true}^{(i)}} - \gamma_{t_{true}^{(i)}}(-\Delta_{t_{true}^{(i)}}, 0)| + |\Delta_{t_{geq}^{(i)}} - \gamma_{t_{geq}^{(i)}}(-\Delta_{t_{geq}^{(i)}}, 0)| \\ & \qquad \qquad \qquad \qquad + n \cdot |\Delta_{t_{leq}^{(i)}} - \gamma_{t_{leq}^{(i)}}(-\Delta_{t_{leq}^{(i)}}, 0)|\big)\\
        &\geq \sum_{i = 1}^n \left(|-1 - \gamma_{t_{true}^{(i)}}(1, 0)| + |-1 - \gamma_{t_{geq}^{(i)}}(1, 0)| + n \cdot |-1 - \gamma_{t_{leq}^{(i)}}(1, 0)|\right)\\
        &= \sum_{i = 1}^n \left(|-1 - \gamma_{t_{true}^{(i)}}(1, 0)| + |-1 - 1| + n \cdot |-1 - 1|\right)\\
        &\geq \sum_{i = 1}^n \left(2 n + 1\right)\\
        &= n \cdot (2 n + 1)
    \end{align*} 

    whereas there exists another coupling strategy $C_\rho^* = (\gamma^*, \gamma^{'*})$ for $\rho$ that would assign 
    \begin{align*}
        \gamma_{t_{true}^{(i)}}^*(\texttt{in}\brangle{1}, \texttt{in}\brangle{2}) &= -\texttt{in}\brangle{1} + \texttt{in}\brangle{2}\\
        \gamma_{t_{geq}^{(i)}}^*(\texttt{in}\brangle{1}, \texttt{in}\brangle{2}) &= 1\\
        \gamma_{t_{leq}^{(i)}}^*(\texttt{in}\brangle{1}, \texttt{in}\brangle{2}) &= -\texttt{in}\brangle{1} + \texttt{in}\brangle{2}
    \end{align*}
    which satisfies the privacy constraints \[\gamma_{t_{true}^{(i)}}^* \leq \gamma_{t_{geq}^{(i)}}^* \geq \gamma_{t_{leq}^{(i)}}^*\] for all $i$, which has cost $2n$.
    
    For the periodic programs $L_i$, the coupling strategy $C_{L_i}^* = (\gamma^*, \gamma^{'*})$ assigns

    \begin{align*}
        \gamma_{t_{true}^{(n + i)}}^*(\texttt{in}\brangle{1}, \texttt{in}\brangle{2}) &= 1\\
        \gamma_{t_{loop}^{(i)}}^*(\texttt{in}\brangle{1}, \texttt{in}\brangle{2}) &= -\texttt{in}\brangle{1} + \texttt{in}\brangle{2}\\
        \gamma_{t_{geq}^{(i)}}^*(\texttt{in}\brangle{1}, \texttt{in}\brangle{2}) &= 1
    \end{align*}

    which satisfies the privacy constraints \[\gamma_{t_{geq}^{(i)}}^* \geq \gamma_{t_{true}^{(n + i)}}^* \leq \gamma_{t_{loop}^{(i)}}^*\] for all $i$, and has cost $4$.
    
    Putting these strategies together, we have a coupling startegy $C^*$ for $B_n$ with cost $2n$, as opposed to at least $n (2n + 1)$ for any SLP-independent coupling strategy $C$.
\end{proof}



\begin{proof}[\proofname~of lemma \ref{finiteCostConstraintLemma}]

    ($\impliedby$)

    Let $T$ be the set of transitions $t_i$ in $L$ such that $t_i$ is \textbf{not} found under a star in $R_L$. 

    Fix a initialized SLP $\rho\in L$ and let $C_\rho$ be the coupling strategy for $\rho$ induced by $C$. 

    Let $D_\rho$ be the set of transitions $t_i\in \rho$ such that $t_i$ is under a star in $R_L$, i.e., $t_i\notin T$.  

    If the given constraint holds, then we know that $\max_{\texttt{in}\brangle{1}\sim\texttt{in}\brangle{2}}\sum_{i: t_i\in D_\rho}(|-\texttt{in}_i\brangle{1}+\texttt{in}_i\brangle{2}-\gamma_i|)d_i+(|-\texttt{in}_i\brangle{1}+\texttt{in}_i\brangle{2}-\gamma_i'|)d_i' = 0$

    So \begin{align*}
        cost(C_\rho) = \max_{\texttt{in}\brangle{1}\sim\texttt{in}\brangle{2}}&\sum_{i: t_i\in D_\rho}(|-\texttt{in}_i\brangle{1}+\texttt{in}_i\brangle{2}-\gamma_i|)d_i+(|-\texttt{in}_i\brangle{1}+\texttt{in}_i\brangle{2}-\gamma_i'|)d_i'\\
        &+\sum_{i: t_i\notin D_\rho}(|-\texttt{in}_i\brangle{1}+\texttt{in}_i\brangle{2}-\gamma_i|)d_i+(|-\texttt{in}_i\brangle{1}+\texttt{in}_i\brangle{2}-\gamma_i'|)d_i'\\
        = \max_{\texttt{in}\brangle{1}\sim\texttt{in}\brangle{2}}&\sum_{i: t_i\in T}(|-\texttt{in}_i\brangle{1}+\texttt{in}_i\brangle{2}-\gamma_i|)d_i+(|-\texttt{in}_i\brangle{1}+\texttt{in}_i\brangle{2}-\gamma_i'|)d_i'\\
        \leq \sum_{i:t_i\in T}(2d_i& + 2d_i')\\
        \leq |T|\max_{i:t_i\in T}&(2d_i + 2d_i')
    \end{align*}

    Thus, $cost(C)\leq |T|\max_{i:t_i\in T}(2d_i + 2d_i') <\infty$.

    ($\implies$)

    Let $t_i$ be a transition in $L$ under a star in $R_L$ such that $\gamma_i\neq -\texttt{in}\brangle{1}_i+\texttt{in}\brangle{2}_i$ or $\gamma_i'\neq  -\texttt{in}\brangle{1}_i+\texttt{in}\brangle{2}_i$.     Thus, $\exists \texttt{in}\brangle{1}\sim \texttt{in}\brangle{2}$ such that $(|-\texttt{in}_i\brangle{1}+\texttt{in}_i\brangle{2}-\gamma_i|)d_i+(|-\texttt{in}_i\brangle{1}+\texttt{in}_i\brangle{2}-\gamma_i'|)d_i'>0$.
    Fix such a $\texttt{in}\brangle{1}\sim \texttt{in}\brangle{2}$. 

    Then there exists some initialized SLP $\rho$ in $L$ of the form $a(bt_ic)^*d$ for some $a, b, c, d\in \Sigma_T^*$.
    
    Let $\rho_k=a(bt_ic)^kd$ be the corresponding initialized SLP in $L$ with $(bt_ic)$ iterated $k$ times. This is equivalent to iterating the cycle containing $t_i$ $k$ times. Then for all $k\in \NN$, \begin{align*}
        cost(\rho_k) \geq k((|-\texttt{in}_i\brangle{1}+\texttt{in}_i\brangle{2}-\gamma_i|)d_i+(|-\texttt{in}_i\brangle{1}+\texttt{in}_i\brangle{2}-\gamma_i'|)d_i'),
    \end{align*}
    so for all $M\in \RR$, $\exists \rho_k$ such that $cost(\rho_k) > M$.
\end{proof}


\begin{lemma}\label{cycleGammaConstraints}
    If a coupling strategy $C=(\mathbf{\gamma}, \mathbf{\gamma}')$ for a periodic program $L$ is valid and has finite cost, then the following must hold for all $i$:
    \begin{enumerate}
        \item If $t_i$ is in a cycle and $c_i = \lguard[\texttt{x}]$, then $\gamma_i = -\texttt{in}_i\brangle{1}+\texttt{in}_i\brangle{2}$ and $\gamma_{at(i)} = 1$.
        \item If $t_i$ is in a cycle and $c_i = \gguard[\texttt{x}]$, then $\gamma_i = -\texttt{in}_i\brangle{1}+\texttt{in}_i\brangle{2}$ and $\gamma_{at(i)} = -1$.
    \end{enumerate}
\end{lemma}
\begin{proof}
    We will show (1). (2) follows symmetrically.

    Consider some $t_i$ in a cycle where $c_i = \lguard[\texttt{x}]$. Because $C$ is has finite cost, we know from lemma \ref{finiteCostConstraintLemma} that $\gamma_i = -\texttt{in}_i\brangle{1}+\texttt{in}_i\brangle{2}$ for all  $\texttt{in}_i\brangle{1}\sim\texttt{in}_i\brangle{2}$. In particular, when $-\texttt{in}_i\brangle{1}+\texttt{in}_i\brangle{2}=1$, then $\gamma_i=1$. 
    
    Further, because $\gamma_{at(i)}$ must be greater than $\gamma_i$ for all $\texttt{in}_i\brangle{1}\sim\texttt{in}_i\brangle{2}$ for $C$ to be valid, we must have that $\gamma_{at(i)}=1$.
\end{proof}

\begin{lemma}\label{integersAreEnoughForATLemma}
    If a valid finite cost coupling strategy $C = (\gamma, \gamma')$ exists for a periodic program $L_\rho$, then there exists a valid finite cost coupling strategy $C^*= (\gamma^*, \gamma^{*\prime})$ such that for all $i\in AT(L_\rho)$, $\gamma_i^*\in \{-1, 0, 1\}$. 
\end{lemma}
\begin{proof}
    Since $L_\rho$ is differentially private, the cost $opt(L_\rho)$ of its optimal coupling strategy is finite. By Proposition \ref{prop:approx_opt_are_close}, we see also that $approx(L)$ is finite. Thus, there exists $\beta \in [-1, 1]^{n}$ which satisfies the approximate privacy constraints on $L_\rho$.

    Define 
    \[\gamma_i = \lfloor \beta_i \rfloor \]

    and notice that $\gamma_i, \gamma^{\prime}$ also satisfy the approximate privacy constraints on $L_\rho$:
    \begin{align*}
        \beta_{at(i)} \leq \beta_i &\implies \lfloor \beta_{at(i)} \rfloor \leq \lfloor \beta_i \rfloor \implies \gamma_{at(i)} \leq \gamma_i\\
        \beta_{at(i)} \geq \beta_i &\implies \lfloor \beta_{at(i)} \rfloor \geq \lfloor \beta_i \rfloor \implies \gamma_{at(i)} \geq \gamma_i\\
        \beta_{i} = 0 &\implies \lfloor \beta_{i} \rfloor = 0 \implies \gamma_{i} = 0\\
        \beta_{i} = 1 &\implies \lfloor \beta_{i} \rfloor = 1 \implies \gamma_{i} = 1\\
        \beta_{i} = -1 &\implies \lfloor \beta_{i} \rfloor = -1 \implies \gamma_{i} = -1
    \end{align*}
    Since $L_\rho$ is private, none of the transitions $t_i$ for $i \in AT(L_\rho)$ are in cycles by lemma \ref{unsatisfiableImpliesNotWellformedLemma}.
    
    By Proposition \ref{prop:approx_exists}, we see that there is a valid coupling strategy $C^* = (\gamma^*, \gamma^{*\prime})$ for $\gamma^*_i = \gamma_i \in \{-1, 0, 1\}$ for all $i \in AT(L_\rho)$.
\end{proof}




\begin{proof}[Proof of prop \ref{syntacticEquivalenceProp}]
    Let $G_P = (V, E)$ be a proper control flow graph that generates an output distinct program $P$. By definition, there exists some unique starting location $\ell_{init} \in V$.

    Then let $A_P = (V, \RR, C, \Gamma, \ell_{init}, X, P_A, \delta)$ where
    \begin{itemize}
        \item $\Sigma, C, X$ are as in the definition of a DiPA
        \item $P_A, \delta$ are (partial) functions defined below.
    \end{itemize}
    
    For all states $\ell\in V$, let $T_\ell$ be the set of transitions that label edges leaving $\ell$. Because of the shared noise condition, $\forall t, t'\in T_{\ell}, P(t) = P(t')$. Let $P(t) = (d_\ell, d_\ell')$ for some $t\in T_\ell$. Then $P_A(\ell) = (0, d_\ell, 0, d_{\ell}')$.

    For all $e = (\ell, \ell')\in E$, let $T(e) = (c, \sigma, \tau)$. We define $\delta(\ell, c) = (\ell', \sigma, \tau)$.

    Observe that because $G_P$ is proper, $\delta$ is a valid transition function for $A_P$. 

    Now let $A = (Q, \Sigma, C, \Gamma, q_{init}, X, P, \delta)$ be a DiPA and let $G_A = (V, E)$ be the underlying graph of $A$; observe that $q_{init}\in V$ is a unique initial state for $G_A$. Additionally, for all $e = (q, q')\in E$, let $\delta(q, c) = (q', \sigma, \tau)$. Then $T(e) = (c, \sigma, \tau)$ is the transition labeling $e$. 

    Because $\delta$ is a valid partial transition function, $G_A$ must be a proper control flow graph that generates an output distinct program. 
\end{proof}




\begin{prop}[Proposition \ref{ClassCouplingStrategiesAreEnoughProp}]
    If there exists a valid coupling strategy $C_\rho$ with cost $cost(C_\rho)$ for every SLP $\rho$ of periodic program $L$ and $\sup_{\rho\in L}cost(C_\rho)< \infty$, then there exists a valid class coupling strategy $C'$ for $L$ such that $cost(C') \leq \sup_{\rho\in L}cost(C_\rho)$. 
\end{prop}

\begin{proof}[Proof of proposition \ref{ClassCouplingStrategiesAreEnoughProp}]


    Because $\sup_{\rho\in L}cost(C_\rho)< \infty$, we can assume that there are no leaking cycles, disclosing cycles, leaking pairs, or privacy violating paths in $L$ by lemma \ref{unsatisfiableImpliesNotWellformedLemma}.


    For a given SLP $\rho$ and a coupling strategy $C_\rho$, recall that we effectively assign each transition $t_i$ in $\rho$ the cost $\max_{\Delta \in \{-1, 0, 1\}}|\Delta - \gamma_i(\Delta)| + |\Delta' - \gamma'_i(\Delta)|$. For convenience, we will shorthand this quantity as $\delta(\rho, t_i) + \delta'(\rho, t_i)$.


    For all $n\in \NN$, let $\rho_n$ be the SLP in $L$ with every cycle in $\rho_n$ repeated $n$ times. 

   Let $cycle(L)$ be the set of all transitions in $L$ that are contained within a cycle in $L$. Observe that for all $t\in cycle(L)$, \[
        \lim_{n\to\infty}\inf_{t_i\in\rho_n: t_i=t} \delta(\rho_n, t_i) = 0
    \]
    Informally, for every cycle transition $t$ in $L$, if the cycle it is contained in is iterated enough times, there must be some iteration $t_i$ of $t$ that is assigned costs approaching 0. 

    This can be shown by considering a transition $t$ in a cycle in $L$ whose minimum coupling cost is non-zero (i.e. $\inf_{\rho\in L; t_i=t} \delta(t_i) > 0$). Then for any finite $d>0$, there exists an SLP $\rho_n$ where $n>\lceil\frac{d}{\inf\delta(t_i)}\rceil+1$. Then $cost(C_{\rho^*})>d$, which implies that $\sup_{\rho\in L}cost(C_\rho) = \infty$, so the observation must hold. 

    Let $t_i$ be a transition in a cycle in $L$ and let $\mathcal{C}_i$ be the cycle containing $t_i$. 

    Then in particular, if $\mathcal{C}_i$ contains a transition with guard $\lguard[\texttt{x}]$, then for all $\psi>0$, there exists $n\in \NN$ such that for $\rho_n\in L$, $\gamma_{at(i)}> 1-\psi$ and if $\mathcal{C}_i$ contains a transition with guard $\lguard[\texttt{x}]$, then for all $\psi>0$, $\gamma_{at(i)}> -1+\psi$. 
    Informally, assignment transitions before an \lcycle\ have shifts that approach 1 and assignment transitions before a \gcycle\ have shifts that approach -1 in $\rho_n$ as $n\to\infty$. 

    Because we know that all coupling strategies $C_{\rho}$ are valid, this may also imply that other assignment transitions also have shifts that approach 1 or -1. 

    Further, if the shifts for an assignment transition $t_i$ approach 1, then the shifts for a transition $t_j$ such that $at(j) = i$ and $c_j = \gguard[\texttt{x}]$ must also approach 1; symmetrically, if the shifts for an assignment transition $t_i$ approach -1, then the shifts for a transition $t_j$ such that $at(j) = i$ and $c_j = \lguard[\texttt{x}]$ must also approach -1. 

    Let $T_1$ and $T_{-1}$ be the sets of assignment transitions in $L$ that approach 1 and -1, respectively. 

    Note that every other transition in $L$ is a non-cycle transition. Consider such a transition $t$ in $L$. Then for every SLP $\rho\in L$ and its corresponding coupling strategy $C_\rho$, there is exactly one shift assignment for $t$ because $t$ is not in a cycle. 

    Let the coupling strategy $C' = (\gamma, \gamma')$ be partially defined as follows: \begin{align*}
        \gamma(t_i)(\texttt{in}\brangle{1}, \texttt{in}\brangle{2}) &= \begin{cases}
            1 & t_i \in T_1\\
            -1 & t_i \in T_{-1}\\
            \texttt{in}\brangle{1}-\texttt{in}\brangle{2} & c_i = \lguard[\texttt{x}]\land t_{at(i)}\in T_1\\
            \texttt{in}\brangle{1}-\texttt{in}\brangle{2} & c_i = \gguard[\texttt{x}]\land t_{at(i)}\in T_{-1}\\
            1 & c_i = \gguard[\texttt{x}]\land t_{at(i)}\in T_1\\
            -1 & c_i = \lguard[\texttt{x}]\land t_{at(i)}\in T_{-1}
        \end{cases}\\
        \gamma'(t_i)(\texttt{in}\brangle{1}, \texttt{in}\brangle{2}) &=\begin{cases}
            0 & t_i\text{ outputs }\texttt{insample}'\\
            \texttt{in}\brangle{1}-\texttt{in}\brangle{2} & \text{otherwise}
        \end{cases}
    \end{align*}

    Let $T_{un}$ be the set of transitions in $L$ that are not assigned by $\gamma$ so far. Note that all transitions in $T_{un}$ are not in cycles. 

    Let $C^* = (\gamma^*, \gamma^{*\prime})$ be the minimal-cost valid class coupling strategy such that for all $t\notin T_{un}$, $\gamma^*(t) = \gamma(t)$. 

    In other words, $cost(C^*) = \inf_{\text{all such valid coupling strategies } C} cost(C)$. Note that $C^*$ is valid. 

    We additionally claim that $cost(C^*)\leq \sup_{\rho\in L}cost(C_\rho)$. The cost of $C^*$ can be separated into costs attributed to $\gamma^{*\prime}$, costs attributed to all transitions not in $T_{un}$ by $\gamma^*$, and costs attributed to all transitions in $T_{un}$ by $\gamma^*$. 
    
    First, note that the coupling cost attributed to $\gamma^{*\prime}$ in $C^*$ must be at most the maximimum coupling cost attributed to $\gamma'$ over all SLP-specific coupling strategies. From before, we additionally know that the cost attributed to all transitions $t\notin T_{un}$ by $\gamma^*$ is at most the supremum of the costs attributed to $t$ over all SLPs in $L$, since we take the limit of all such shifts for $\rho_n$ as $n\to\infty$.

    Finally, since all SLP-specific coupling strategies are valid, taking the remaining transition shifts to minimize the overall cost while retaining a valid coupling strategy sufficies. 
\end{proof}


\begin{proof}[Proof of \ref{prop:compute_opt_cost}]
    Assume that $L$ is differentially private. Then there exists a valid coupling strategy for $L$ with finite cost, showing that the constraints above are feasible, and a finite solution to the optimization problem exists.  

    We will specify a valid coupling strategy $C^* = (\gamma^*, {\gamma'}^*)$ for $L$ with the cost stated above, and then show it is optimal. Define $\Gamma: [-1, 1]^n \to [-1, 1]^n \times [-1, 1]^n$ as follows, where $\Gamma(\Delta)$ is a pair $(\gamma, \gamma')$: 
    \begin{align*}
        \Gamma(\Delta) = &\argmin_{\gamma, \gamma' \in [-1, 1]^n} \sum_{i = 1}^n \left(|\Delta_i - \gamma_i| d_i + |\Delta_i - \gamma_i'|d_i' \right)\\ 
        \text{subject to }
        &\ \gamma_{at(i)} \leq \gamma_i \text{ if } c_i = \gguard, \\
        &\ \gamma_{at(i)} \geq \gamma_i \text{ if } c_i = \lguard, \\
        &\ \gamma_i = 0 \text{ if } \sigma_i = \texttt{insample}, \\
        &\ \gamma_i' = 0 \text{ if } \sigma_i = \texttt{insample}'\\
        &\ \gamma_i = \gamma_i'= \Delta_i \text{ if } t_i \text{ is in a cycle}
    \end{align*}
    Then define \[(\gamma^*(\texttt{in}\brangle{1}, \texttt{in}\brangle{2}), {\gamma'}^*(\texttt{in}\brangle{1}, \texttt{in}\brangle{2})) = \Gamma(\texttt{in}\brangle{1} - \texttt{in}\brangle{2})\]

    Notice the following: 

    \begin{enumerate}
        \item $C^*$ is a valid coupling strategy for $L$, since the privacy constraints on $\gamma^*$ and ${\gamma'}^*$ are satisfied by construction.
        \item $C^*$ has the cost given by the solution to the optimization problem, since 
        \begin{align*}
            cost(C^*) &= \max_{\texttt{in}\brangle{1}\sim \texttt{in}\brangle{2}} \sum_{i = 1}^n  |\texttt{in}_i\brangle{1} - \texttt{in}_i\brangle{2} - \gamma_i^*(\texttt{in}\brangle{1}, \texttt{in}\brangle{2})| d_i \\ 
            \phantom{cost(C^*)} &\phantom{=\max_{\texttt{in}\brangle{1}\sim \texttt{in}\brangle{2}}\qquad } + |\texttt{in}_i\brangle{1} - \texttt{in}_i\brangle{2} - {\gamma'}_i^*(\texttt{in}\brangle{1}, \texttt{in}\brangle{2})|d_i' \\
            &= \max_{\texttt{in}\brangle{1}\sim \texttt{in}\brangle{2}} \sum_{i = 1}^n  |\texttt{in}_i\brangle{1} - \texttt{in}_i\brangle{2} - \Gamma_1(\texttt{in}\brangle{1} - \texttt{in}\brangle{2})_i| d_i \\ 
            \phantom{cost(C^*)} &\phantom{=\max_{\texttt{in}\brangle{1}\sim \texttt{in}\brangle{2}}\qquad } + |\texttt{in}_i\brangle{1} - \texttt{in}_i\brangle{2} - \Gamma_2(\texttt{in}\brangle{1} - \texttt{in}\brangle{2})_i|d_i' \\
            &= \max_{\Delta \in [-1, 1]^n} \sum_{i = 1}^n  |\Delta_i - \Gamma_1(\Delta)_i| d_i + |\Delta_i - \Gamma_2(\Delta)_i|d_i' \\
            &= \max_{\Delta \in [-1, 1]^n} \min_{\gamma, \gamma' \in [-1, 1]^n} \sum_{i = 1}^n  |\Delta_i - \gamma_i| d_i + |\Delta_i - \gamma_i'|d_i'
        \end{align*}
        \item $C^*$ is optimal, since for any valid coupling strategy $C = (\delta, \delta')$ for $L$, we have
        \begin{align*}
            cost(C) &= \max_{\texttt{in}\brangle{1}\sim \texttt{in}\brangle{2}} \sum_{i = 1}^n  |\texttt{in}_i\brangle{1} - \texttt{in}_i\brangle{2} - \delta_i(\texttt{in}\brangle{1}, \texttt{in}\brangle{2})| d_i \\
            \phantom{cost(C)} &\phantom{=\max_{\texttt{in}\brangle{1}\sim \texttt{in}\brangle{2}}\qquad } + |\texttt{in}_i\brangle{1} - \texttt{in}_i\brangle{2} - \delta_i'(\texttt{in}\brangle{1}, \texttt{in}\brangle{2})|d_i' \\
            &\geq \max_{\Delta \in [-1, 1]^n} \sum_{i = 1}^n  |\Delta_i - \delta_i(0, \Delta_i)| d_i + |\Delta_i - \delta_i'(0, \Delta_i)|d_i' \\
            &\geq \max_{\Delta \in [-1, 1]^n} \min_{\gamma, \gamma' \in [-1, 1]^n} \sum_{i = 1}^n \left(|\Delta_i - \gamma_i| d_i + |\Delta_i - \gamma_i'|d_i' \right)\\
            &= cost(C^*)
        \end{align*}
    \end{enumerate}
    which shows that the optimization problem computes the optimal cost of a coupling strategy for $L$ that satisfies the privacy constraints.
\end{proof}


\begin{proof}[Proof of \ref{prop:approx_exists}]
    Let 
    \[\gamma = \argmin_{\gamma \in [-1, 1]^n} \sum_{i \in I} \left(1 + |\gamma_i| \right) d_i\]
    subject to the constraints above. Define $C_L = (\gamma^*, {\gamma'}^*)$ where
    \begin{align*}
        \gamma_i^*(\texttt{in}\brangle{1}, \texttt{in}\brangle{2}) &= \begin{cases}
            \texttt{in}\brangle{1}_i - \texttt{in}\brangle{2}_i &\text{ if } t_i \text{ is in a cycle} \\
            \gamma_i &\text{ otherwise}
        \end{cases} \\[1em]
        {\gamma'}_i^*(\texttt{in}\brangle{1}, \texttt{in}\brangle{2}) &= \begin{cases}
            0 &\text{ if } t_i \text{ outputs \texttt{insample}} \\
            \texttt{in}\brangle{1}_i - \texttt{in}\brangle{2}_i &\text{ otherwise}
        \end{cases}
    \end{align*}
    Notice the following: 

    \begin{itemize}
        \item $C_L$ satisfies the privacy constraints, and so is valid.
        
        If $t_i$ is in a cycle with $c_i = \lguard$, then the constraints on $\gamma$ require that $\gamma_i = 1$, and so $1 = \gamma_i \leq \gamma_{at(i)} = 1$. As a result, we will satisfy the privacy constraint $\gamma_{at(i)}^* \geq \gamma_i^*$: 
        \[\gamma_i^* = \texttt{in}\brangle{1}_i - \texttt{in}\brangle{2}_i \leq 1 = \gamma_{at(i)}^*\]
        A similar argument holds for if $t_i$ is in a cycle with $c_i = \gguard$.

        All other privacy constraints are satisfied by construction.

        \item $C_L$ has the cost given by the solution to the optimization problem, since
        
        \begin{align*}
            cost(C_L) &= \max_{\texttt{in}\brangle{1}\sim \texttt{in}\brangle{2}} \sum_{i = 1}^n  |\texttt{in}_i\brangle{1} - \texttt{in}_i\brangle{2} - \gamma_i^*(\texttt{in}\brangle{1}, \texttt{in}\brangle{2})| d_i \\ 
            \phantom{cost(C_L)} &\phantom{=\max_{\texttt{in}\brangle{1}\sim \texttt{in}\brangle{2}}\qquad } + |\texttt{in}_i\brangle{1} - \texttt{in}_i\brangle{2} - {\gamma'}_i^*(\texttt{in}\brangle{1}, \texttt{in}\brangle{2})|d_i' \\
            &= \max_{\texttt{in}\brangle{1}\sim \texttt{in}\brangle{2}} \left(\sum_{i \in I} |\texttt{in}_i\brangle{1} - \texttt{in}_i\brangle{2} - \gamma_i^*(\texttt{in}\brangle{1}, \texttt{in}\brangle{2})| d_i\right) \\
            \phantom{cost(C_L)} &\phantom{=\max_{\texttt{in}\brangle{1}\sim \texttt{in}\brangle{2}}\qquad } + \left(\sum_{t_i \text{outputs \texttt{insample}}}|\texttt{in}_i\brangle{1} - \texttt{in}_i\brangle{2} - {\gamma'}_i^*(\texttt{in}\brangle{1}, \texttt{in}\brangle{2})|d_i'\right)\\
            &= \max_{\texttt{in}\brangle{1}\sim \texttt{in}\brangle{2}} \left(\sum_{i \in I} |\texttt{in}_i\brangle{1} - \texttt{in}_i\brangle{2} - \gamma_i| d_i\right) \\
            \phantom{cost(C_L)} &\phantom{=\max_{\texttt{in}\brangle{1}\sim \texttt{in}\brangle{2}}\qquad } + \left(\sum_{t_i \text{outputs \texttt{insample}}}|\texttt{in}_i\brangle{1} - \texttt{in}_i\brangle{2}|d_i'\right)\\
            &= \sum_{i \in I} (1 + |\gamma_i|) d_i + \sum_{t_i \text{outputs \texttt{insample}}} d_i' \\
            &= approx(L)
        \end{align*}
    \end{itemize}
\end{proof}

\begin{proof}[Proof of \ref{approximateSolutionPolyTimeProp}]
    To compute the solution to the minimization problem, we can set up the following linear program: 
    \begin{align*}
        \min_{\gamma, A_i \in [-1, 1]^n} &\sum_{i = 1}^n \left(1 + A_i \right) d_i \\ 
            \text{subject to } 
            &\ \gamma_{at(i)} \leq \gamma_i \text{ if } c_i = \lguard, \\
            &\ \gamma_{at(i)} \geq \gamma_i \text{ if } c_i = \gguard, \\
            &\ \gamma_i = 0 \text{ if } \sigma_i = \texttt{insample}, \\
            &\ \gamma_i = 1 \text{ if } t_i \text{ is in a cycle and has } c_i = \lguard,\\ 
            &\ \gamma_i = -1 \text{ if } t_i \text{ is in a cycle and has } c_i = \gguard,\\
            &\ \gamma_i \leq A_i, -\gamma_i \leq A_i \text{ for all } i \in \{1, \dots, n\} 
    \end{align*}
    This program can be solved using the ellipsoid method in polynomial time.
\end{proof}


\begin{proof}[Proof of \ref{prop:approx_opt_are_close}]
    We have $opt(L) \leq approx(L)$ by Proposition \ref{prop:compute_opt_cost}. Let $I$ be the set of transitions in $L$ that do $\textit{not}$ appear in a cycle. Then we have
    \begin{align*}
        opt(L) &= \max_{\Delta \in [-1, 1]^n} \min_{\gamma, \gamma' \in [-1, 1]^n} \sum_{i = 1}^n \left(|\Delta_i - \gamma_i| d_i + |\Delta_i - \gamma_i'|d_i' \right)\\
        &= \max_{\Delta \in [-1, 1]^n} \min_{\gamma, \gamma' \in [-1, 1]^n} \sum_{i \in I} \left(|\Delta_i - \gamma_i| d_i + |\Delta_i - \gamma_i'|d_i' \right)\\
        &= \max_{\Delta \in [-1, 1]^n} \min_{\gamma, \gamma' \in [-1, 1]^n} \left(\sum_{i \in I} \left(|\Delta_i - \gamma_i| d_i \right) + \sum_{t_i \text{outputs \texttt{insample}}} |\Delta_i| d_i' \right)\\
        &\geq \max_{\Delta \in [-1, 1]^n} \min_{\gamma, \gamma' \in [-1, 1]^n} \left(\sum_{i \in I} \left(|\gamma_i| - |\Delta_i| \right) d_i  + \sum_{t_i \text{outputs \texttt{insample}}} |\Delta_i| d_i' \right)\\
        &= \max_{\Delta \in [-1, 1]^n} \left(- \sum_{i \in I} |\Delta_i| d_i + \sum_{t_i \text{outputs \texttt{insample}}} |\Delta_i| d_i' + \min_{\gamma, \gamma' \in [-1, 1]^n} \sum_{i \in I}|\gamma_i| d_i \right)\\
        &\geq \min_{\gamma, \gamma' \in [-1, 1]^n} \sum_{i \in I}|\gamma_i| d_i \\
        &= approx(L) - \sum_{t_i \text{outputs \texttt{insample}}} d_i' - \sum_{i \in I} d_i'
    \end{align*}
    showing the second inequality.
\end{proof}


\begin{proof}[Proof of prop \ref{privacyConstraintGraphProp}]
    $(\implies)$ Let $L$ be differentially private. Then $approx(L) < \infty$, and so there exists $\gamma \in [-1, 1]^n$ such that the approximate privacy constraints are satisfied by $\gamma$. Aiming for a contradiction, assume that there exists a path ${\bf 1} \to v_{i_1} \to \dots \to v_{i_k} \to {\bf -1}$ in the privacy constraint graph of $L$. This corresponds to the sequence of privacy constraint inequalities
    \[1 \leq \gamma_{i_1} \leq \dots \leq \gamma_{i_k} \leq -1\]
    which is a contradiction, showing that no such $\gamma$ could exist. Therefore, there is no path from $\bf 1$ to $\bf -1$ in the privacy constraint graph of $L$.

    $(\impliedby)$ Let there exist no path from $\bf 1$ to $\bf -1$ in the privacy constraint graph of $L$. Define 
    \begin{align*}
        \gamma_i = \begin{cases}
            1 &\text{ if there exists a path from } {\bf 1} \text{ to } v_i \text{ in } G_L \\
            -1 &\text{ otherwise}
        \end{cases}
    \end{align*}
    We claim that the approximate privacy constraints are satisfied by $\gamma$.
    
    \begin{itemize}
        \item Consider the approximate privacy constraint $\gamma_i \leq \gamma_j$. This corresponds to the edge $(v_i, v_j)$ in $G_L$. If $\gamma_i = 1$, there is a path from $\bf 1$ to $v_i$, and so there is a path from $\bf 1$ to $v_j$, and so $\gamma_j = 1$, satisfying the constraint. If $\gamma_i = -1$, then any assignment of $\gamma_j$ satisfies the constraint. 
        \item Consider the constraint $\gamma_i = 1$. This corresponds to the edge $({\bf 1}, v_i)$ in $G_L$. Since there is a path from $\bf 1$ to $v_i$, we have $\gamma_i = 1$, satisfying the constraint.
        \item Consider the constraint $\gamma_i = -1$. This corresponds to the edge $(v_i, {\bf -1})$ in $G_L$. Since there is no path from $\bf 1$ to $\bf -1$ in $G_L$, there must be no path from $\bf 1$ to $v_i$. Thus, $\gamma_i = -1$, satisfying the constraint.
    \end{itemize}
    
    Thus, the approximate privacy constraints are satisfied by $\gamma$, which means that $approx(L) < \infty$ and $L$ is differentially private.
\end{proof}

\textbf{The following claims help prove prop \ref{programPrivacyConstraintGraphPathReq} and theorem \ref{LinearTimeDecidingPrograms}}

\begin{prop}
    \label{prop:paths_in_privacy_graph}
    Let $v_{i_0}, \dots, v_{i_k}$ be vertices in $G_P$ corresponding to transitions $t_{i_0}, \dots, t_{i_k}$ in $P$. If $v_{i_0} \to \dots \to v_{i_k}$ is a path in $G_P$, then there exists an SLP $\rho \in P$ such that 
    \begin{align*}
        t_{i_0} \cdots t_{i_k} \text{ is a subsequence of } \rho \text{ with } guard(t_{i_j}) = \gguard \text { for all } j \in \{1, \dots, k\}
    \end{align*}
    or
    \begin{align*}
        t_{i_k} \cdots t_{i_0} \text{ is a subsequence of } \rho \text{ with } guard(t_{i_j}) = \lguard \text { for all } j \in \{k - 1, \dots, 0\}
    \end{align*}
\end{prop}



\begin{proof}[Proof of \ref{prop:paths_in_privacy_graph}]
    We will use induction on the length of the path $v_{i_0} \to \dots \to v_{i_k}$ in $G_P$. 

    \begin{itemize}
        \item Base Case ($k = 1$)
        
        If $k = 1$, then the path $v_{i_0} \to v_{i_1}$ comprises of a single edge $(v_{i_0}, v_{i_1})$ in $G_P$. So, there is a periodic program $L$ for which $(v_{i_0}, v_{i_1}) \in G_L$, for which there is the privacy constraint $\gamma_{i_0} \leq \gamma_{i_1}$. 
        
        We either have that $i_0 = at(i_1)$ and $c_{i_1} = \gguard$, or $i_1 = at(i_0)$ and $c_{i_0} = \lguard$. In the first case, we have that $t_{i_0} t_{i_1}$ is a subsequence of some SLP $\rho$ in $L$ with $guard(t_{i_1}) = \gguard$. In the second case, we have that $t_{i_1} t_{i_0}$ is a subsequence of some SLP $\rho$ in $L$ with $guard(t_{i_0}) = \lguard$.

        \item Inductive Step ($k > 1$)
        
        By the inductive hypothesis, we have one of the following cases: 

        \begin{enumerate}
            \item There exists an SLP $\rho_1 \in P$ such that $t_{i_0} \cdots t_{i_{k - 1}}$ is a subsequence of $\rho_1$ with $guard(t_{i_j}) = \gguard$ for all $j \in \{1, \dots, k - 1\}$.
            
            Since we have the edge $(v_{i_{k - 1}}, v_{i_k})$ in $G_P$, there exists a periodic program $L$ for which $(v_{i_{k - 1}}, v_{i_k}) \in G_L$, for which there is the privacy constraint $\gamma_{i_{k - 1}} \leq \gamma_{i_k}$.

            We either have that $i_{k - 1} = at(i_k)$ and $c_{i_k} = \gguard$ ($t_{i_{k - 1}}$ precedes $t_{i_k}$ in $L$), or $i_k = at(i_{k - 1})$ and $c_{i_{k - 1}} = \lguard$ ($t_{i_k}$ precedes $t_{i_{k - 1}}$ in $L$). Notice, however, that we cannot have that $t_{i_k}$ precedes $t_{i_{k - 1}}$, since we have assumed $c_{i_{k - 1}} = \gguard$. 

            So, there exists an SLP $\rho_2 \in L$ such that $t_{i_{k - 1}} t_{i_k}$ is a subsequence of $\rho_2$ with $guard(t_{i_k}) = \gguard$. 

            Let $j_1$ be the index at which $t_{i_{k - 1}}$ appears in $\rho_1$, and $j_2$ be the index at which it appears in $\rho_2$. Then, the SLP $\rho_1[:j_1] \rho_2[j_2:]$ is an SLP in $P$ such that $t_{i_0} \cdots t_{i_k}$ is a subsequence of $\rho_1[:j_1] \rho_2[j_2:]$ with $guard(t_{i_j}) = \gguard$ for all $j \in \{1, \dots, k\}$.

            \item There exists an SLP $\rho_1 \in P$ such that $t_{i_{k - 1}} \cdots t_{i_0}$ is a subsequence of $\rho_1$ with $guard(t_{i_j}) = \lguard$ for all $j \in \{k - 2, \dots, 0\}$.
            
            Similar to the argument above, we either have that $t_{i_{k - 1}}$ precedes $t_{i_k}$, or $t_{i_k}$ precedes $t_{i_{k - 1}}$ in some periodic program. We cannot have that $t_{i_{k - 1}}$ precedes $t_{i_k}$, and so $c_{i_{k - 1}}$ is forced to be $\lguard$. We can then construct an SLP $\rho \in P$ such that $t_{i_k} \cdots t_{i_0}$ is a subsequence of $\rho$ with $guard(t_{i_j}) = \lguard$ for all $j \in \{k - 1, \dots, 0\}$. 
        \end{enumerate}
    \end{itemize}

    This completes the proof. 
\end{proof}


\begin{cor}
    The path $v_{i_0} \to \dots \to v_{i_k}$ is in $G_P$ if and only if there is a periodic program $L$ in $P$ such that $v_{i_0} \to \dots \to v_{i_k}$ is a path in $G_L$.
\end{cor}

\begin{proof}[Proof of \ref{programPrivacyConstraintGraphPathReq}]
    There is a path from $\bf 1$ to $\bf -1$ in the privacy constraint graph of $P$ if and only if there is a path from $\bf 1$ to $\bf -1$ in the privacy constraint graph of some periodic program $L$ in $P$ by Proposition \ref{prop:paths_in_privacy_graph}. This is true if and only if there exists some $L$ which is not differentially private, which is true if and only if $P$ is not differentially private.
\end{proof}

\section{Multivariable Programs}


\begin{lemma}[Precise Version of lemma \ref{simplifiedMvParallelCouplingsLemma}]\label{mvParallelCouplingsLemma}
    Let $\vec{X}\brangle{1} = (X_1\brangle{1}, \ldots X_k\brangle{1})$ where $X_i\brangle{1}\sim \Lap(\mu_{x_i}\brangle{1}, \frac{1}{d_{x_i}\varepsilon})$ are independent random variables and $\vec{X}\brangle{2} = (X_1\brangle{2}, \ldots X_k\brangle{2})$ where $X_i\brangle{2}\sim \Lap(\mu_{x_i}\brangle{2}, \frac{1}{d_{x_i}\varepsilon})$ are independent random variables be representations of two possible initial values of $\vec{X}$, respectively.

    Let $t = (c, \sigma, \tau)$ be a $k$v-transition such that $P(t) = (d_t, d_t')$.

    Let $\texttt{in}\brangle{1}\sim\texttt{in}\brangle{2}$ be an arbitrary adjacent input pair and let $o\brangle{1}$, $o\brangle{2}$ be random variables representing possible outputs of $t$ given inputs $\texttt{in}\brangle{1}$ and $\texttt{in}\brangle{2}$, respectively. 

    Then $\forall \varepsilon>0$ and for all $\gamma_{x_1}, \ldots, \gamma_{x_k}, \gamma_t^{(x_1)}, \ldots, \gamma_t^{(x_k)}, \gamma_t'$ that satisfy the constraints \[
        \begin{cases}
            \gamma_t^{(x_i)}\leq\gamma_{x_i} & c = \lguard[\texttt{x}_i]\\
            \gamma_t^{(x_i)}\geq\gamma_{x_i} & c = \gguard[\texttt{x}_i]\\
            \gamma_t^{(x_i)}=0 & \sigma = \texttt{insample}^{(\texttt{x}_i)}\\
            \gamma_t'=0 & \sigma = \texttt{insample}'
      \end{cases}
      \] for all $1\leq i\leq k$, the lifting $o\brangle{1}\{(a, b): a=\sigma\implies b=\sigma\}^{\#d\varepsilon}o\brangle{2}$ is valid for 
      $d = \sum_{i=1}^k\left(|\mu_{x_i}\brangle{1}-\mu_{x_i}\brangle{2}+\gamma_{x_i}|d_{x_i}+|\texttt{in}\brangle{1}-\texttt{in}\brangle{2}+\gamma_t^{(x_i)}|d_t\right)+|\texttt{in}\brangle{1}-\texttt{in}\brangle{2}+\gamma_t'|d_t'$.
\end{lemma}
\begin{proof}
    From lemma \ref{indTransitionCoupling}, we know that if these constraints are satisfied, we can create liftings such that for all $i$, $c^{(\texttt{x}_i)}$ is satisfied in run $\brangle{1}\implies c^{(\texttt{x}_i)}$ is satisfied in run $\brangle{2}$.
    
    Because $c$ is made of conjunctions and disjunctions of all $c^{(\texttt{x}_i)}$, this means that if $c$ is satisfied in run $\brangle{1}$, then $c$ must also be satisfied in $\brangle{2}$. 

    Finally, as before, if $\sigma \in \{\texttt{insample}^{(x_1)}\ldots \texttt{insample}^{(x_k)},\texttt{insample}'\}$, then for all $\sigma$, $o\brangle{1}=\sigma \implies o\brangle{2} = \sigma$. Further, note that we only need to couple $\texttt{insample}'\brangle{1}$ and $\texttt{insample}'\brangle{2}$ once for all variables. By simply adding up the costs from lemma \ref{indTransitionCoupling}, we complete the proof.
\end{proof}


\begin{proof}[Proof of lemma \ref{mvCrossCoupling}]
    Let $P(t) = (d_t, d_t')$. For convenience, we rewrite, for all $i$, $X_i = \mu_i + \zeta_i$, where $\zeta_i\sim \Lap(0, \frac{1}{d_x\varepsilon})$. For all $i$, we also write $\texttt{insample}^{(x_i)} = \texttt{in} + z_i$, where $z \sim \Lap(0, \frac{1}{d_t\varepsilon})$ and $\texttt{insample}' = \texttt{in} + z'$, where $z' \sim \Lap(0, \frac{1}{d_t'\varepsilon})$.

    Based on lemma \ref{mvParallelCouplingsLemma}, we will assume that we have constructed the following liftings for all $i$ and are aiming to ``extend'' them: \begin{itemize}
        \item $X_i\brangle{1} +\gamma_{x_i}(=)^{\#(|\mu_i\brangle{1}-\mu_i\brangle{2}+\gamma_{x_i}|d_x\varepsilon)}X_i\brangle{2}$
        \item $\texttt{insample}^{(\texttt{x}_i)} + \gamma_t^{(x_i)}(=)^{\#(|\texttt{in}\brangle{1}-\texttt{in}\brangle{2}+\gamma_t^{(x_i)}|d_t\varepsilon)}\texttt{insample}^{(\texttt{x}_i)}\brangle{2}$
        \item $\texttt{insample}' + \gamma_t'(=)^{\#(|\texttt{in}\brangle{1}-\texttt{in}\brangle{2}+\gamma_t'|d_t'\varepsilon)}\texttt{insample}'\brangle{2}$
    \end{itemize}

    \textbf{Case 1: } For all $i\neq j$, construct the lifting $z_i\brangle{1}(=)^{\#0}z_j\brangle{1}$, which is equivalent to the lifting $\texttt{insample}^{(x_i)}\brangle{1}(=)^{\#}\texttt{insample}^{(x_j)}\brangle{1}$. In particular, this guarantees that all $\texttt{insample}^{(x_i)}\brangle{1}$ are equal to each other. 
    Additionally, for all $i\neq j$, construct the lifting $\zeta_i\brangle{1}(=)^{\#0}\zeta_j\brangle{1}$, which is equivalent to the lifting $X_i\brangle{1} - X_j\brangle{1} (=)^{\#0}\mu_i\brangle{1}-\mu_j\brangle{1}$.

    First, suppose that $X_i\brangle{1} = \mu_i\brangle{1}$ for all $i$. Then if case one is true, the proposition that ``$c$ is satisfied in run $\brangle{1}$'' must always be false; thus, the implication ``$c$ is satisfied in run $\brangle{1}\implies c$ is satisfied in run $\brangle{2}$'' must always be true.

    Now observe that for any transition guard $c$, if the boolean expression produced from $c$ by setting all $\texttt{insample}^{(\texttt{x}_i)}$ equal to each other and setting $\texttt{x}_i = \mu_i\brangle{1}$ for all $i$ is a contradiction, then for any constant $a$, the boolean expression produced from $c$ by setting all $\texttt{insample}^{(\texttt{x}_i)}$ equal to each other and setting $\texttt{x}_i = \mu_i\brangle{1}+a$ for all $i$ must also be a contradiction. 

    Thus, the liftings we have constructed are sufficient to prove the proposition ``$c$ is satisfied in run $\brangle{1}\implies c$ is satisfied in run $\brangle{2}$'', and so, if the two output conditions are also satisfied, $\PP[\vec{X}\brangle{1}, t, \texttt{in}\brangle{1}, \sigma]\leq e^{d\varepsilon}\PP[\vec{X}\brangle{2}, t, \texttt{in}\brangle{2}, \sigma]$ for 
    $d = |\mu_i\brangle{1}-\mu_i\brangle{2}+\gamma_{x_i}|d_x\varepsilon + |\texttt{in}\brangle{1}-\texttt{in}\brangle{2}+\gamma_t^{(x_i)}|d_t\varepsilon+|\texttt{in}\brangle{1}-\texttt{in}\brangle{2}+\gamma_t'|d_t'\varepsilon$, i.e. for no ``additional'' cost.     
    
    \textbf{Case 2: } For all $i\neq j$, construct the lifting $z_i\brangle{2}(=)^{\#0}z_j\brangle{2}$, which is equivalent to the lifting $\texttt{insample}^{(x_i)}\brangle{2}(=)^{\#}\texttt{insample}^{(x_j)}\brangle{2}$. In particular, this guarantees that all $\texttt{insample}^{(x_i)}\brangle{2}$ are equal to each other. 
    Additionally, for all $i\neq j$, construct the lifting $\zeta_i\brangle{2}(=)^{\#0}\zeta_j\brangle{2}$, which is equivalent to the lifting $X_i\brangle{2} - X_j\brangle{2} (=)^{\#0}\mu_i\brangle{2}-\mu_j\brangle{2}$.

    As before, if case two is satisfied, then these liftings suffice to show that the proposition  ``$c$ is satisfied in run $\brangle{2}$'' must always be true; thus, the implication ``$c$ is satisfied in run $\brangle{1}\implies c$ is satisfied in run $\brangle{2}$'' must also always be true, and so $\PP[\vec{X}\brangle{1}, t, \texttt{in}\brangle{1}, \sigma]\leq e^{d\varepsilon}\PP[\vec{X}\brangle{2}, t, \texttt{in}\brangle{2}, \sigma]$ for 
    $d = |\mu_i\brangle{1}-\mu_i\brangle{2}+\gamma_{x_i}|d_x\varepsilon + |\texttt{in}\brangle{1}-\texttt{in}\brangle{2}+\gamma_t^{(x_i)}|d_t\varepsilon+|\texttt{in}\brangle{1}-\texttt{in}\brangle{2}+\gamma_t'|d_t'\varepsilon$.

\end{proof}

