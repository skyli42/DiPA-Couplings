
\subsection{Program Paths}

We now demonstrate how to concatenate transitions and their corresponding coupling strategies together into program \textit{paths}.

\begin{defn}[Paths]
    A path is a string composed of transitions. For a path $\rho = t_0\cdot t_1\cdot\ldots\cdot t_{n-1}$, if $t_0$ is of the form $t_0 = (\texttt{true}, \sigma, \texttt{true})$ for some $\sigma$, then we call $\rho$ a \textbf{complete} path.
\end{defn}

The length of a path $\rho$ is simply the number of transitions that are concatenated together to form $\rho$. 

We define some useful notation for dealing with paths and sequences more generally. 

Given a path (or sequence) $\rho = t_0\cdot t_1\cdot \ldots\cdot t_{n-1}$, the \textbf{tail} of $\rho$ is notated by $tail(\rho) = t_1\cdot \ldots\cdot t_{n-1}$. 
We may additionally use the notation $\rho_{i:j}$ to represent the subpath (or subsequence) $q_i\to q_{i+1}\to \ldots \to q_j$ of $\rho$. Using this notation, $tail(\rho) = \rho_{1:} = \rho_{1:n}$.

Whereas an individual transition reads in one real-valued input and outputted one output value, a path reads in a \textbf{sequence} of inputs and outputs a sequence of outputs, one for each transition in the path.

As before, we need to restrict the space of possible inputs to a path based on which locations in the path actually read in user input.
\begin{defn}
    For a path $\rho$ of length $n$, an input sequence $\texttt{in}\in \RR^n$ is valid if, for all $q_i$ in $\rho$ such that $q_i \in Q_{non}$, $\texttt{in}_i = 0$.  
\end{defn} 

We will assume that all input sequences are valid from now on. 


\sky{move this prop to much later}
Interestingly, the constraints on valid transition alphabets, specifically the constraints of determinism and output distinction, mean that outputs uniquely correspond to paths; in other words, given a valid transition alphabet, knowing an output sequence uniquely determines which path must have produced the output. 

\begin{prop}
    Let $\Sigma_T$ be a valid transition alphabet and let $\Gamma$ be the finite output alphabet associated with $\Sigma_T$. Let $O\subset (\Gamma\cup\{\texttt{insample}, \texttt{insample}'\})^*$ be the set of all possible outputs of complete paths over $\Sigma_T$. There exists an injection $f: \Sigma_T\to t_{init}\Sigma_T^*$ from the set of all possible outputs to complete paths over $\Sigma_T$. 
\end{prop}

\subsubsection{Path Semantics}

As with transitions, we can think of paths as very limited programs consisting of a series of transitions concatenated together with a persistent threshold variable $\texttt{x}$. Naturally, paths will now consider as input a \textbf{sequence} of real numbers, and similarly output a \textbf{sequence} of real numbers or symbols - each transition reads in an input and outputs some value.

The semantics of a path can be defined as a function mapping a subdistribution of program states and a real-valued input sequence to a subdistribution of final program states. 

More specifically, the semantics of a path $\rho = t_0t_1\cdots t_{n-1}$ are a function $\Phi_\rho: dist_\downarrow(S)\times \RR^n \to dist_\downarrow(S)$. $\Phi_\rho$ can be defined by composing transition semantics in the expected manner:

\[\Phi_\rho(s, \texttt{in}) = \begin{cases}
    \Phi_{t_0}(s, \texttt{in}_0)& |\rho| = 1\\
    \Phi_{t_{n-1}}(\Phi_{\rho_{0:n-1}}(s, \texttt{in}_{0:n-1}), \texttt{in}_{n-1})& |\rho| >1
\end{cases}\]
\sky{Is there a nicer way to write this that I'm forgetting}

Like with transitions, we denote the probability that a path $\rho = q_0\to\ldots\to q_n$ ``succeeds'' given an initial $\texttt{x}\in dist(\RR)$, input sequence $\texttt{in} \in \RR^n$, and possible output sequence $o \in (\Gamma\cup\Sigma)^n$as $\PP[\texttt{x}, \rho, \texttt{in}, o] = \int_{-\infty}^\infty\PP_O[\texttt{x} =x]\PP_O[(q_{n}, x, o)]dx$, where $O$ is the subdistribution produced by $\Phi_\rho((q, \texttt{x}, \lambda), \texttt{in})$.
\sky{Same issue with $\PP_O[\texttt{x}=x]$}

For a complete path $\rho$, note that the initial value of $\texttt{x}$ is irrelevant, so we will shorthand $\PP[\texttt{x}_0, \rho, \texttt{in}, \sigma]$ to $\PP[\rho, \texttt{in}, \sigma]$.


\subsubsection{Privacy}

By leveraging the construction of couplings for individual transitions, we can construct a set of approximate liftings for entire paths.

Because paths read in a \textit{sequence} of real-valued inputs, we need to slightly modify our definition of adjacency.

\begin{defn}[Adjacency for a sequence of inputs]
    Two input sequences $\{\texttt{in}_i\brangle{1}\}_{i=1}^n, \{\texttt{in}_i\brangle{2}\}_{i=1}^n$ of length $n$ are $\Delta$-adjacent (notated $\texttt{in}\brangle{1} \sim_{\Delta}\texttt{in}\brangle{2}$) if, for all $i\in [1\ldots n]$, $|\texttt{in}_i\brangle{1}-\texttt{in}_i\brangle{2}|\leq \Delta$. 

    As before, if $\Delta$ is not specified, we assume that $\Delta = 1$. 
\end{defn}

Thus, we have the following definition of privacy for complete paths:

\begin{defn}[$d\varepsilon$-differential privacy for a path]
    A complete path $\rho$ of length $n$ is $d\varepsilon$-differentially private for some $d>0$ if $\forall \varepsilon>0$, for all valid adjacent input sequences $\texttt{in}\brangle{1}\sim \texttt{in}\brangle{2}$ of length $n$ and all possible output sequences $\sigma$ of length $n$, $\PP[\rho, \texttt{in}\brangle{1}, \sigma]\leq e^{d\varepsilon}\PP[\rho, \texttt{in}\brangle{2}, \sigma]$.
\end{defn}

\sky{move these later}
We observe that because of the path-output correspondence, we can equivalently look at each path of a set in isolation:
\begin{prop}
    Let $S$ be a set of complete paths; $S$ is $d\varepsilon$-differentially private for some $d>0$ if and only if, for all paths $\rho\in S$, $\rho$ is $d\varepsilon$-differentially private.
\end{prop}

Note that, following \cite{chadhaLinearTimeDecidability2021}, we slightly redefine $\varepsilon$-differential privacy as $d\varepsilon$-differential privacy, treating $\varepsilon$ as a universal scaling parameter that can be fine-tuned by users for their own purposes. 
We argue that this definition is functionally equivalent, since if we are targeting $\varepsilon^*$-differential privacy overall, we can always take $\varepsilon = \frac{\varepsilon^*}{d}$.

\subsubsection{Concatenating couplings}

Just as individual transitions can be concatenated to form program paths, we can compose together couplings associated with each transition to produce a coupling proof of privacy for an entire path. 

If $o\brangle{1}, o\brangle{2}$ are random variables representing the output of $\rho$ given input sequences $\texttt{in}\brangle{1}$ and $\texttt{in}\brangle{2}$, respectively, 
then in order to show that a program path $\rho$ is differentially private we want to create the coupling $o\brangle{1}\{(a, b): a=\sigma\implies b=\sigma\}^{\#d\varepsilon}o\brangle{2}$ for some $d>0$ for all adjacent inputs $\texttt{in}\brangle{1}\sim\texttt{in}\brangle{2}$ and all possible outputs $\sigma$.

As it turns out, directly composing together the couplings from lemma \ref{indTransitionCoupling} are essentially sufficient; the constraints imposed upon shifts for a coupling for transition $t_i$ depend solely on the shift at the most recent \textbf{assignment transition} in $\rho$ (i.e. the most recent transition $t_j$ such that $\tau_j = \texttt{true}$). 
The coupling shifts for \textit{non-assignment transitions} can thus never impact each other. 

\begin{defn}[Assignment transitions]
    Let $A_\rho = \{t_i=(q_i, q_{i+1}, c_i, \sigma_i, \tau_i): \tau_i = \texttt{true}\}$ be the set of \textbf{assignment transitions} in a path $\rho$. Additionally, for every transition $t_i$ in $\rho$, let $t_{at(i)}$ be the most recent assignment transition in $\rho$; i.e., $at(i) = \max\{j<i: t_j\in A_\rho\}$. If such a $j$ does not exist, we set $at(i)=-1$. 
\end{defn}

In particular, note that for transition $t_i$, $\gamma_x = \gamma_{at(i)}$, where $\gamma_{-1}$ is the shift applied to the initial $\texttt{x}$-values $\texttt{x}_0\brangle{1}$ and $\texttt{x}_0\brangle{2}$ (for complete paths, note that $\gamma_{-1}$ is irrelevant).

Thus, for an individual transition $t_i$ of $\rho$, we have a family of valid coupling strategies $C_i(\gamma_{at(i)}, \gamma_i, \gamma_i')$. 

We can merge these coupling strategies together to create a proof of privacy for the entire path: 

\begin{lemma}\label{multTransitionsCouplingProof}
    Let $\rho = q_0\to \ldots \to q_n$ be a complete path of length $n$. 
    Let $\texttt{in}\brangle{1}\sim \texttt{in}\brangle{2}$ be arbitrary adjacent input sequences of length $n$. Additionally, fix some potential output $\sigma$ of $\rho$ of length $n$ and let $\sigma\brangle{1}$, $\sigma\brangle{2}$ be random variables representing possible outputs of $\rho$ given inputs $\texttt{in}\brangle{1}$ and $\texttt{in}\brangle{2}$, respectively. Additionally, for all $q_i$, let $P(q_i) = (d_i, d_i')$.

    Then $\forall \varepsilon>0$ and for all $\{\gamma_i, \gamma_i'\}_{i=0}^{n-1}$ that, for all $i$, satisfy the constraints \[
        \begin{cases}
          \gamma_i\leq\gamma_{at(i)} & c_i = \lguard[\texttt{x}]\\
          \gamma_i\geq\gamma_{at(i)} & c_i = \gguard[\texttt{x}]\\
          \gamma_i=0 & \sigma_i = \texttt{insample}\\
          \gamma_i'=0 & \sigma_i = \texttt{insample}'
        \end{cases},
      \]
      the lifting $\sigma\brangle{1}\{(a, b): a=\sigma\implies b=\sigma\}^{\#d\varepsilon}\sigma\brangle{2}$ is valid for $d = \sum_{i=0}^{n-1}(|-\texttt{in}_i\brangle{1}+\texttt{in}_i\brangle{2}-\gamma_i|)d_i+(|-\texttt{in}_i\brangle{1}+\texttt{in}_i\brangle{2}-\gamma_i'|)d_i'$, and therefore $t$ is $d\varepsilon$-differentially private. 
\end{lemma}
\begin{proof}
    From the proof of lemma \ref{indTransitionCoupling}, we know that we can create the couplings $\texttt{insample}_i\brangle{1} +\gamma_i{(=)}^{\#(|-\texttt{in}_i\brangle{1}+\texttt{in}_i\brangle{2}-\gamma_i|)d_i\varepsilon}\texttt{insample}_i\brangle{2}$ and $\texttt{insample}_i'\brangle{1} +\gamma_i'{(=)}^{\#(|-\texttt{in}_i\brangle{1}+\texttt{in}_i\brangle{2}-\gamma_i'|)d_i'\varepsilon}\texttt{insample}_i'\brangle{2}$ for all $q_i$ in $\rho$. 

    Additionally, for some fixed $q_i$ in $\rho$, if we have the coupling $\texttt{x}_i\brangle{1}+\gamma_x (=)^{\#(|\hat{\mu_i}\brangle{1}-\hat{\mu_i}\brangle{2}+\gamma_x|)\hat{d_i}\varepsilon}x_i\brangle{2}$, where $\texttt{x}_i\brangle{1}\sim \Lap(\hat{\mu_i}\brangle{1}, \frac{1}{\hat{d_i}\varepsilon})$ and $\texttt{x}_i\brangle{2}\sim \Lap(\hat{\mu_i}\brangle{2}, \frac{1}{\hat{d_i}\varepsilon})$, then subject to the constraints \[
        \begin{cases}
          \gamma_i\leq\gamma_x & c_i = \lguard[\texttt{x}]\\
          \gamma_i\geq\gamma_x & c_i = \gguard[\texttt{x}]\\
          \gamma_i=0 & \sigma_i = \texttt{insample}_i\\
          \gamma_i'=0 & \sigma_i = \texttt{insample}_i'
        \end{cases},
      \]
    the coupling $\sigma_i\brangle{1}\{(a, b): a=\sigma_i\implies b=\sigma_i\}^{\#d\varepsilon}\sigma_i\brangle{2}$ is valid for some $d$. 

    Indeed, note that for all $i$, $\texttt{x}_i = \texttt{insample}_{at(i)}$ by definition. Thus, we have that $\texttt{x}_i\brangle{1}+\gamma_x (=)^{\#(|-\texttt{in}_{at(i)}\brangle{1}+\texttt{in}_{at(i)}\brangle{2}+\gamma_{at(i)}|)d_{at(i)}\varepsilon}x_i\brangle{2}$, and we must satisfy the constraints \[
        \begin{cases}
          \gamma_i\leq\gamma_{at(i)} & c_i = \lguard[\texttt{x}]\\
          \gamma_i\geq\gamma_{at(i)} & c_i = \gguard[\texttt{x}]\\
          \gamma_i=0 & \sigma_i = \texttt{insample}_i\\
          \gamma_i'=0 & \sigma_i = \texttt{insample}_i'
        \end{cases}
      \]
      for all $i$.

    Thus, we can put all of these couplings together to show that the coupling $\sigma_i\brangle{1}\{(a, b): a=\sigma_i\implies b=\sigma_i\}^{\#d\varepsilon}\sigma_i\brangle{2}$ is valid for some $d>0$.

    In particular, note that we have created at most one pair of couplings (for $\texttt{insample}$ and $\texttt{insample}$) for each $q_i$. Thus, the total coupling cost associated with each $q_i$ is at most $(|-\texttt{in}_i\brangle{1}+\texttt{in}_i\brangle{2}-\gamma_i|)d_i+(|-\texttt{in}_i\brangle{1}+\texttt{in}_i\brangle{2}-\gamma_i'|)d_i'$, 
    which gives us an overall coupling cost of $d = \sum_{i=0}^{n-1}(|-\texttt{in}_i\brangle{1}+\texttt{in}_i\brangle{2}-\gamma_i|)d_i+(|-\texttt{in}_i\brangle{1}+\texttt{in}_i\brangle{2}-\gamma_i'|)d_i'$.
\end{proof}

As with individual transitions, lemma \ref{multTransitionsCouplingProof} implicitly defines an entire family of possible coupling proofs that demonstrate the privacy of a path.

\begin{defn}
    For a complete path $\rho$ of length $n$, \textbf{coupling strategy} is a tuple of two functions $\bm{\gamma}(\texttt{in}\brangle{1}, \texttt{in}\brangle{2}):\RR^n\times \RR^n\to [-1, 1]^n$ and $\bm{\gamma}'(\texttt{in}\brangle{1}, \texttt{in}\brangle{2}):\RR^n\times \RR^n\to [-1, 1]^n$ that produce shifts for each transition of $\rho$ for every possible pair of adjacent input sequences $\texttt{in}\brangle{1}\sim\texttt{in}\brangle{2}$. 
    If $\texttt{in}\brangle{1}$ and $\texttt{in}\brangle{2}$ are clear from context, we will often shorthand notating a coupling strategy as $\bm{\gamma}$ and $\bm{\gamma}'$. 
\end{defn}


\begin{defn}
    For a complete path $\rho$ of length $n$, a coupling strategy $C_\rho = (\bm{\gamma}, \bm{\gamma}')$ is \textbf{valid} if $\forall \texttt{in}\brangle{1}\sim\texttt{in}\brangle{2}$, $\bm{\gamma}(\texttt{in}\brangle{1}, \texttt{in}\brangle{2})$ and $\bm{\gamma}'(\texttt{in}\brangle{1}, \texttt{in}\brangle{2})$ satisfy the constraints \[
        \begin{cases}
          \gamma_i\leq\gamma_{at(i)} & c_i = \lguard[\texttt{x}]\\
          \gamma_i\geq\gamma_{at(i)} & c_i = \gguard[\texttt{x}]\\
          \gamma_i=0 & \sigma_i = \texttt{insample}\\
          \gamma_i'=0 & \sigma_i = \texttt{insample}'
        \end{cases}.
      \]
\end{defn}

Thus, if we have a \textbf{valid} coupling strategy $C$ for a path $\rho$, then immediately by lemma \ref{multTransitionsCouplingProof}, we have a proof that $\rho$ is $d\varepsilon$-differentially private. 

\sky{how necessary are these following defs/props}

\begin{defn}
    For a complete path $\rho$ of length $n$, the \textbf{cost} of a coupling strategy $C_\rho=(\bm{\gamma}, \bm{\gamma}')$ is \[cost(C_\rho) = \max_{\texttt{in}\brangle{1}\sim\texttt{in}\brangle{2}}\sum_{i=0}^{n-1}(|-\texttt{in}_i\brangle{1}+\texttt{in}_i\brangle{2}-\gamma_i|)d_i+(|-\texttt{in}_i\brangle{1}+\texttt{in}_i\brangle{2}-\gamma_i'|)d_i'.\]

    Additionally, let $G$ be the set of all valid coupling strategies $C_\rho=(\bm{\gamma}, \bm{\gamma}')$ for $\rho$. Then the \textbf{coupling cost} of $\rho$ is 
    \[cost(\rho) = \min_{(\bm{\gamma}, \bm{\gamma}')\in G}cost((\bm{\gamma}, \bm{\gamma}')).\]
\end{defn}

Naturally, the existence of a valid coupling strategy bounds the privacy cost of any path. 

\begin{cor}
    If $C_\rho=(\bm{\gamma}, \bm{\gamma}')$ is valid, then $\rho$ is $cost(C_\rho)\varepsilon$-differentially private.
\end{cor}

