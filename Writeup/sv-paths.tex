
\subsection{Straightline Programs}

We now demonstrate how to concatenate transitions and their corresponding coupling strategies together into program \textit{paths}.

\begin{defn}[Paths]
    A path is a sequence of transitions. For a path $\rho = t_0\cdot t_1\cdot\ldots\cdot t_{n-1}$, if $t_0$ is of the form $t_0 = (\texttt{true}, \sigma, \texttt{true})$ for some $\sigma$, then we call $\rho$ an \textbf{initialized} path.
\end{defn}

We use standard properties of sequences for paths, such as the notion of the length of a path and the tail of a path, in the expected manner. For a path $\rho$, we use the notation $\rho_{i:j}$ to represent the subpath  $t_i t_{i+1} \ldots t_{j-1}$ of $\rho$.

We directly lift the concept of inputs and outputs from individual transitions to paths; thus, each path reads in a \textbf{sequence} of inputs and outputs a sequence of outputs, one for each transition in the path. 

\subsubsection{Path Semantics}

The semantics of a path are defined as a function mapping a subdistribution of program states and a real-valued input sequence to a subdistribution of final program states. 

More specifically, the semantics of a path $\rho = t_0t_1\cdots t_{n-1}$ are a function $\Phi_\rho: dist_\downarrow(S)\times \RR^n \to dist_\downarrow(S)$. $\Phi_\rho$ can be defined by composing transition semantics in the expected manner:

\[\Phi_\rho(s, \texttt{in}) = \begin{cases}
    \Phi_{t_0}(s, \texttt{in}_0)& |\rho| = 1\\
    \Phi_{t_{n-1}}(\Phi_{\rho_{0:n-1}}(s, \texttt{in}_{0:n-1}), \texttt{in}_{n-1})& |\rho| >1
\end{cases}\]


Like with transitions, we denote the probability that a path $\rho = t_0 t_1\ldots t_{n-1}$ outputs a specific value given an initial $\texttt{x}\in\RR$, input sequence $\texttt{in} \in \RR^n$, and possible measurable output sequence $o \subseteq (\Gamma\cup\RR)^n$ as $\PP[\texttt{x}, \rho, \texttt{in}, o]$. As before, $\PP[\texttt{x}, t, \texttt{in}, o]$ is the marginal of $\Phi_\rho((\texttt{x}, \lambda), \texttt{in})$ on $(\cdot, o)$.

For a initialized path $\rho$, note that the initial value of $\texttt{x}$ is irrelevant, so we will shorthand $\PP[\texttt{x}_0, \rho, \texttt{in}, \sigma]$ to $\PP[\rho, \texttt{in}, \sigma]$.


\subsubsection{Privacy}

By leveraging the construction of couplings for individual transitions, we can construct a set of approximate liftings for paths.

Because paths read in a \textit{sequence} of real-valued inputs, we need to slightly modify our definition of valid adjacent inputs.

\begin{defn}[Validity for a sequence of inputs]
    Two input sequences $\{\texttt{in}_i\brangle{1}\}_{i=1}^n\sim\{\texttt{in}_i\brangle{2}\}_{i=1}^n$ are \textbf{valid} and \textbf{adjacent} input sequences for a path $\rho = t_0\ldots t_{n-1}$ if, for all $1\leq i\leq n$, 
\end{defn}

Thus, we have the following definition of privacy for initialized paths:

\begin{defn}[$d\varepsilon$-differential privacy for a path]
    A initialized path $\rho$ of length $n$ is $d\varepsilon$-differentially private for some $d>0$ if $\forall \varepsilon>0$, for all valid adjacent input sequences $\texttt{in}\brangle{1}\sim \texttt{in}\brangle{2}$ of length $n$ and all possible output sequences $\sigma$ of length $n$, $\PP[\rho, \texttt{in}\brangle{1}, \sigma]\leq e^{d\varepsilon}\PP[\rho, \texttt{in}\brangle{2}, \sigma]$.
\end{defn}

Note that, following \cite{chadhaLinearTimeDecidability2021}, we slightly redefine $\varepsilon$-differential privacy as $d\varepsilon$-differential privacy, treating $\varepsilon$ as a universal scaling parameter that can be fine-tuned by users for their own purposes. 
We argue that this definition is functionally equivalent, since if we are targeting $\varepsilon^*$-differential privacy overall, we can always take $\varepsilon = \frac{\varepsilon^*}{d}$.

\subsubsection{Sequential composition of couplings}

We can compose together couplings associated with each transition to produce a coupling proof of privacy for an entire path. 

If $o\brangle{1}, o\brangle{2}$ are random variables representing the output of a program path $\rho$ given input sequences $\texttt{in}\brangle{1}$ and $\texttt{in}\brangle{2}$, respectively, 
then if we can create the coupling $o\brangle{1}\{(a, b): a=\sigma\implies b=\sigma\}^{\#d\varepsilon}o\brangle{2}$ for some $d>0$ for all adjacent inputs $\texttt{in}\brangle{1}\sim\texttt{in}\brangle{2}$ and all possible outputs $\sigma$, we show that $\rho$ is differentially private.

As it turns out, directly composing together the couplings from lemma \ref{simplifiedIndTransitionCoupling} are essentially sufficient; the constraints imposed upon shifts for a coupling for transition $t_i$ depend solely on the shift at the most recent \textbf{assignment transition} in $\rho$ (i.e. the most recent transition $t_j$ such that $\tau_j = \texttt{true}$). 
The coupling shifts for \textit{non-assignment transitions} can thus never impact each other. 

We will let $A_\rho$ be the set of \textbf{assignment transitions} in a path $\rho$. Additionally, for every transition $t_i$ in $\rho$, let $t_{at(i)}$ be the most recent assignment transition in $\rho$ before $t_i$; i.e., $at(i) = \max\{j<i: t_j\in A_\rho\}$. If such a $j$ does not exist, we set $at(i)=-1$. 

In particular, note that for transition $t_i$, $\gamma_x = \gamma_{at(i)}$, where $\gamma_{-1}$ is the shift applied to the initial $\texttt{x}$-values $\texttt{x}_0\brangle{1}$ and $\texttt{x}_0\brangle{2}$ (for initialized paths, note that $\gamma_{-1}$ is irrelevant).

Thus, for an individual transition $t_i$ of $\rho$, we have a family of valid coupling strategies $C_i(\gamma_{at(i)}, \gamma_i, \gamma_i')$. 

We can merge these coupling strategies together to create a proof of privacy for the entire path: 

\begin{defn}[Coupling Strategies for Paths]
  For a initialized path $\rho$ of length $n$, a \textbf{valid coupling strategy} is a tuple of two functions $\bm{\gamma}(\texttt{in}\brangle{1}, \texttt{in}\brangle{2}):\RR^n\times \RR^n\to [-1, 1]^n$ and $\bm{\gamma}'(\texttt{in}\brangle{1}, \texttt{in}\brangle{2}):\RR^n\times \RR^n\to [-1, 1]^n$ 
  that produce shifts for each transition of $\rho$ for every possible pair of adjacent input sequences $\texttt{in}\brangle{1}\sim\texttt{in}\brangle{2}$ such that $\bm{\gamma}, \bm{\gamma'}$ satisfy the constraints \[
    \begin{cases}
      \gamma_i\leq\gamma_{at(i)} & c_i = \lguard[\texttt{x}]\\
      \gamma_i\geq\gamma_{at(i)} & c_i = \gguard[\texttt{x}]\\
      \gamma_i=0 & \sigma_i = \texttt{insample}\\
      \gamma_i'=0 & \sigma_i = \texttt{insample}'
    \end{cases}.
  \]
  If $\texttt{in}\brangle{1}$ and $\texttt{in}\brangle{2}$ are clear from context, we will often shorthand notating a coupling strategy as $\bm{\gamma}$ and $\bm{\gamma}'$. 
\end{defn}


\begin{lemma}\label{simplifiedMultTransitionsCouplingProof}
  For any initialized path $\rho = t_0\ldots t_{n-1}$, any possible output event $\sigma$ of $\rho$, and any two valid adjacent input sequences $\texttt{in}\brangle{1}\sim \texttt{in}\brangle{2}$, if we are given $2n$ real-valued ``shifts'' $\{\gamma_i, \gamma_i'\}_{i=0}^{n-1}$ that, for all $i$, satisfy the constraints \[
        \begin{cases}
          \gamma_i\leq\gamma_{at(i)} & c_i = \lguard[\texttt{x}]\\
          \gamma_i\geq\gamma_{at(i)} & c_i = \gguard[\texttt{x}]\\
          \gamma_i=0 & \sigma_i = \texttt{insample}\\
          \gamma_i'=0 & \sigma_i = \texttt{insample}'
        \end{cases},
      \]
      then we can construct an approximate lifting that proves that $\PP[\rho, \texttt{in}\brangle{1}, \sigma] \leq e^{d\varepsilon}\PP[\rho, \texttt{in}\brangle{2}, \sigma]$ for some bounded $d>0$. 
\end{lemma}


Thus, if we have a \textbf{valid} coupling strategy $C$ for a path $\rho$, then immediately by lemma \ref{multTransitionsCouplingProof}, we have a proof that $\rho$ is $d\varepsilon$-differentially private for some $d>0$; we call this $d$ the \textit{cost} of the coupling strategy for $\rho$. 

\begin{defn}
    For a initialized path $\rho$ of length $n$, the \textbf{cost} of a coupling strategy $C_\rho=(\bm{\gamma}, \bm{\gamma}')$ is \[cost(C_\rho) = \max_{\texttt{in}\brangle{1}\sim\texttt{in}\brangle{2}}\sum_{i=0}^{n-1}(|-\texttt{in}_i\brangle{1}+\texttt{in}_i\brangle{2}-\gamma_i|)d_i+(|-\texttt{in}_i\brangle{1}+\texttt{in}_i\brangle{2}-\gamma_i'|)d_i'.\]

    Additionally, let $G$ be the set of all valid coupling strategies $C_\rho=(\bm{\gamma}, \bm{\gamma}')$ for $\rho$. Then the \textbf{coupling cost} of $\rho$ is 
    \[cost(\rho) = \min_{(\bm{\gamma}, \bm{\gamma}')\in G}cost((\bm{\gamma}, \bm{\gamma}')).\]
\end{defn}

\begin{cor}
    If $C_\rho=(\bm{\gamma}, \bm{\gamma}')$ is a valid coupling strategy for a path $\rho$, then $\rho$ is $cost(C_\rho)\varepsilon$-differentially private.
\end{cor}

In particular, we observe that for any individual initialized path $\rho$, there always exists \textit{some} valid coupling strategy for $\rho$; so every path $\rho$ is always differentially private. 

Our program model fundamentally considers programs to be collections of individual paths; to this end, we extend the definition of privacy to sets of paths in the expected manner:

\begin{defn}
  Let $S$ be a set of initialized paths and let $O$ be a set of all possible outputs of paths in $S$. 
  Then $S$ is $d\varepsilon$-differentially private for some $d>0$ if, for all paths $\rho\in S$ and outputs $\sigma\in O$, $\forall \varepsilon>0$, for all valid adjacent input sequences $\texttt{in}\brangle{1}\sim \texttt{in}\brangle{2}$, $\PP[\rho, \texttt{in}\brangle{1}, \sigma]\leq e^{d\varepsilon}\PP[\rho, \texttt{in}\brangle{2}, \sigma]$.
\end{defn}



