
\subsection{Branching} 
{\color{red} considering cutting this section - the major function it serves is to emphasize that, excepting loops, paths should all have their own coupling strategies, which could just be explained in the program section}


\begin{defn}[Branching program]
    A branching program $B$ is a finite set of complete paths over a valid transition alphabet $\Sigma_T$.
\end{defn}

Equivalently, a branching program is a language over $\Sigma_T$ that can be represented by a regular expression only using concatenation and finite union; every word in the language must also be of the form $t_{init}\Sigma_T^*$ and satisfy the path condition. 

Branching programs exemplify our conception of programs as simply collections of possible program executions; indeed, we will demonstrate that for the purposes of privacy, we can do no better than treating them as collections of paths.

\subsubsection{Privacy}

We first extend the definition of coupling strategies to sets of paths in the natural manner.

\begin{defn}[Coupling strategies]
    A coupling strategy $C$ for a branching program $B$ is a collection of (path) coupling strategies where each complete path $\rho\in B$ is assigned a coupling strategy $C_\rho$. 
\end{defn}

\begin{defn}
    A coupling strategy $C$ for a branching program $B$ is valid if, for every constituent path coupling strategy $C_\rho$, $C_\rho$ is valid. 
\end{defn}

\begin{defn}
    The cost of a coupling strategy $C$ for a branched program $B$ is $\max_{\rho\in B}cost(\rho)$.
\end{defn}

Notably, for a general branching program, there is no ``smart'' way to combine coupling strategies together; that is, in order to obtain the optimal coupling cost, we must find different coupling strategies for each path in a branching program. We provide a simple counterexample.

\sky{note: rephrase this proposition slightly}
\begin{prop}\label{costDependspathProp}
    Optimal cost is dependent on path. There exists a valid transition alphabet $\Sigma_T$, a location space $Q$, and a branching program $B$ for which the optimal cost of a coupling strategy $C$ for $B$ is dependent on the path $\rho$. 
    
    In other words, the optimal strategy $C$ must assign different coupling strategies to occurances of the same transition in different paths. 
\end{prop}

\begin{proof}
    Let $Q = \{q_0, q_1, q_2, q_3\}$ consist only of input locations, each of which have both noise parameters equal to $1$. Let $\Sigma_T = \{t_{init}, t_{geq1}, t_{leq1}, t_{leq2}, t_{geq2}\}$ where 
    \begin{align*}
        t_{init} &= (q_0, q_1, \texttt{true}, \bot, 1)\\
        t_{geq1} &= (q_1, q_2, \gguard[\texttt{x}], \top, 0)\\
        t_{leq1} &= (q_1, q_3, \lguard[\texttt{x}], \bot, 0)\\
        t_{geq2} &= (q_2, q_2, \gguard[\texttt{x}], \top, 0)\\
        t_{leq2} &= (q_3, q_3, \lguard[\texttt{x}], \bot, 0)
    \end{align*}
    and let $B = \{t_{init}t_{geq1}t_{geq2}^n, t_{init}t_{leq1}t_{leq2}^n\}$ be the branching program consisting of two paths, each of which have $n$ repetitions of the cycle transitions $t_{geq2}$ and $t_{leq2}$, respectively.

    Let $\rho_1 = t_{init}t_{geq1}t_{geq2}^n$ and $\rho_2 = t_{init}t_{leq1}t_{leq2}^n$ be the two paths in $B$. Notice the following: 
    
    \begin{itemize}
        \item The cost of any coupling strategy for $B$ must be at least $2$.
        
        Let $C_{\rho_1}$ be a coupling strategy for $\rho_1$. We can bound its cost as follows: 
        \begin{align*}
            cost(C_{\rho_1}) &= \max_{\texttt{in}\brangle{1}\sim\texttt{in}\brangle{2}}\sum_{i=0}^{n+2}(|-\texttt{in}_i\brangle{1}+\texttt{in}_i\brangle{2}-\gamma_i(\texttt{in}_i\brangle{1}, \texttt{in}_i\brangle{2}))\\&\qquad+(|-\texttt{in}_i\brangle{1}+\texttt{in}_i\brangle{2}-\gamma_i'(\texttt{in}_i\brangle{1}, \texttt{in}_i\brangle{2})|)\\
            &\geq \max_{\texttt{in}\brangle{1}\sim\texttt{in}\brangle{2}} \sum_{i=0}^{n+2}(|-\texttt{in}_i\brangle{1}+\texttt{in}_i\brangle{2}-\gamma_i(\texttt{in}_i\brangle{1}, \texttt{in}_i\brangle{2})|)\\
            &= \max_{\Delta \in [-1, 1]^{n+2}} \sum_{i=0}^{n+2}(|\Delta_i-\gamma_i(0, \Delta_i)|)\\
            &\geq |1 - \gamma_0(0, 1)| + \sum_{i=1}^{n+2}|-1-\gamma_i(0, -1)|\\
            &= 1 - \gamma_0(0, 1) + \sum_{i=1}^{n+2} (1+\gamma_i(0, -1))\\
            &= 1 - \gamma_0(0, 1) + (n + 2) + \sum_{i=1}^{n+2}\gamma_i(0, -1)\\
            &\geq 1 - \gamma_0(0, 1) + (n + 2) + \sum_{i=1}^{n+2}\gamma_0(0, 1) \qquad \text{(privacy constraint)}\\
            &= (n + 3) + (n + 1) \gamma_0(0, 1)\\
            &\geq 2
        \end{align*}

    and by a similar argument, $cost(C_{\rho_2})\geq 2$ for any coupling strategy $C_{\rho_2}$.

    \item There exists a coupling strategy $C^*$ for $B$ such that $cost(C^*) = 2$.
    
    We will first describe $C_{\rho_1}^* = (\gamma, \gamma')$. Since no transition outputs \texttt{insample}, we can set $\gamma_i'(\texttt{in}\brangle{1}, \texttt{in}\brangle{2}) = \texttt{in}\brangle{2} - \texttt{in}\brangle{1}$ for all $i$ with no privacy cost. Define 
    \begin{align*}
        \gamma_0(\texttt{in}\brangle{1}, \texttt{in}\brangle{2}) &= -1 \\
        \gamma_i(\texttt{in}\brangle{1}, \texttt{in}\brangle{2}) &= \texttt{in}_i\brangle{2} - \texttt{in}_i\brangle{1} \qquad \text{for all $i>0$}
    \end{align*}
    We see that $C^*_{\rho_1}$ is valid, since $\gamma_i\geq \gamma_{0}$ for all $i>0$. Further, we see that 
    \begin{align*}
        cost(C^*_{\rho_1}) &= \max_{\texttt{in}\brangle{1}\sim\texttt{in}\brangle{2}}\sum_{i=0}^{n+2}(|-\texttt{in}_i\brangle{1}+\texttt{in}_i\brangle{2}-\gamma_i(\texttt{in}_i\brangle{1}, \texttt{in}_i\brangle{2}))\\&\qquad+(|-\texttt{in}_i\brangle{1}+\texttt{in}_i\brangle{2}-\gamma_i'(\texttt{in}_i\brangle{1}, \texttt{in}_i\brangle{2})|)\\
        &= \max_{\texttt{in}\brangle{1}\sim\texttt{in}\brangle{2}} |-\texttt{in}_0\brangle{1}+\texttt{in}_0\brangle{2}-\gamma_0(\texttt{in}_0\brangle{1}, \texttt{in}_0\brangle{2})| \\
        &= \max_{\texttt{in}\brangle{1}\sim\texttt{in}\brangle{2}} |-\texttt{in}_0\brangle{1}+\texttt{in}_0\brangle{2}+1|\\
        &\leq 2 
    \end{align*}
    showing that $cost(C^*_{\rho_1}) = 2$. Similarly, there is a coupling strategy $C^*_{\rho_2}$ for which $cost(C^*_{\rho_2}) = 2$. This shows that there is a coupling strategy $C^*$, consisting of $C^*_{\rho_1}$ and $C^*_{\rho_2}$, for which $cost(C^*) = 2$.
    
    \item Any coupling strategy $C$ that assigns the same coupling strategy to $t_{init}$ in both $\rho_1$ and $\rho_2$ must have cost $>2$.
    
    Let $C$ be as described, and assume that $C$ has optimal cost, ie. $cost(C) = 2$. If $\gamma_0(0, 1) \neq -1$ in $C_{\rho_1}$, then $cost(C_{\rho_1}) > 2$ in the same method as the above, a contradiction.  

    Thus, $\gamma_0(0, 1) = -1$ in $C_{\rho_1}$, which by hypothesis, means that $\gamma_0(0, 1) = -1$ in $C_{\rho_2}$. We have the privacy constraint $\gamma_i \leq \gamma_0$ on $\rho_2$, which also means that $\gamma_i = -1$ identically for all $i > 0$. However, this means that 
    \begin{align*}
        cost(C_{\rho_2}) &= \max_{\texttt{in}\brangle{1}\sim\texttt{in}\brangle{2}}\sum_{i=0}^{n+2}(|-\texttt{in}_i\brangle{1}+\texttt{in}_i\brangle{2}-\gamma_i(\texttt{in}_i\brangle{1}, \texttt{in}_i\brangle{2}))\\&\qquad+(|-\texttt{in}_i\brangle{1}+\texttt{in}_i\brangle{2}-\gamma_i'(\texttt{in}_i\brangle{1}, \texttt{in}_i\brangle{2})|)\\
        &\geq \max_{\texttt{in}\brangle{1}\sim\texttt{in}\brangle{2}} \sum_{i=0}^{n+2}(|-\texttt{in}_i\brangle{1}+\texttt{in}_i\brangle{2}-\gamma_i(\texttt{in}_i\brangle{1}, \texttt{in}_i\brangle{2})|)\\
        &\geq \max_{\Delta \in [-1, 1]^{n+2}} |\Delta_0 - \gamma_i(0, \Delta_0)| + \sum_{i=1}^{n+2}(|\Delta_i-\gamma_i(0, \Delta_i)|)\\
        &\geq |1 - \gamma_i(0, 1)| + \sum_{i=1}^{n+2}(|1+1|)\\
        &= 2 \cdot (n + 2)
    \end{align*}
    showing that $C$ is not optimal, a contradiction. Therefore $cost(C) > 2$.

    \end{itemize}
    
    The above observations show that the optimal coupling strategy for $B$ must necessarily assign different coupling strategies to $t_{init}$ in $\rho_1$ and $\rho_2$.
\end{proof}

Indeed, there exist a family of counterexamples such that the relative ``cost'' of choosing a unified coupling strategy for a branching program increases quadratically in the size of the branching program.


\begin{prop}
    There exist a family of branching programs $\{B_n\}_{n\in \NN}$ for which the cost of any path-independent coupling strategy $C$ for $B_n$ is in $\Omega(n^2)$, but for which there exists a path-dependent coupling strategy $C'$ for $B_n$ with cost in $O(n)$.
\end{prop}

\begin{proof}
    Construct $B_n$ as follows. Consider a path $\rho = t_{true}^{(1)} t_{geq}^{(1)} \left(t_{leq}^{(1)}\right)^n \dots t_{true}^{(n)} t_{geq}^{(n)} \left(t_{leq}^{(n)}\right)^n$ where:

    \begin{itemize}
        \item $t_{true}^{(i)}$ are assignment transitions with guard \texttt{true}.
        \item $t_{geq}^{(i)}$ are assignment transitions with guard $\gguard$.
        \item $t_{leq}^{(i)}$ are non-assignment transitions with guard $\lguard$.
        \item The noise parameters for all transitions are $1$.
        \item None of the transitions output \texttt{insample}.
    \end{itemize}

    For each $i$, construct also the looping branch $L_i = L\left(t_{true}^{(n + i)} \left(t_{loop}^{(i)}\right)^* t_{geq}^{(i)}\right)$ such that each transition $t_{geq}^{(i)}$ is preceded by an arbitrary number of transitions with guard $\lguard$.
    
    Let $C$ be a path-independent coupling strategy for $B_n$ -- this menas that $C$ must assign the same shift to each transition in $B_n$. 

    From the privacy constraints on the looping branches $L_i$, we see that $\gamma_{t_{true}^{n + i}}(1, 0) = 1$ from the same method as in \ref{costDependspathProp}, which then implies from the constraint $\gamma_{t_{true}^{n + i}} \leq \gamma_{t_{geq}^{(i)}}$ that $\gamma_{t_{geq}^{(i)}}(1, 0) = 1$. 

    Since the preceding assignment transition for $t_{leq}^{(i)}$ is given by $t_{geq}^{(i)}$, for which we have the privacy constraint $\gamma_{t_{geq}^{(i)}} \leq \gamma_{t_{leq}^{(i)}}$, we see that $\gamma_{t_{leq}^{(i)}}(1, 0) = 1$ as well. 

    Computing the cost of the coupling strategy assigned to $\rho$, which has $n \cdot (n + 2)$ transitions, we get 

    \begin{align*}
        cost(C_\rho) &\geq \max_{\Delta = \texttt{in}\brangle{2} - \texttt{in}\brangle{1}} \sum_{i=1}^{n} \big(|\Delta_{t_{true}^{(i)}} - \gamma_{t_{true}^{(i)}}(-\Delta_{t_{true}^{(i)}}, 0)| + |\Delta_{t_{geq}^{(i)}} - \gamma_{t_{geq}^{(i)}}(-\Delta_{t_{geq}^{(i)}}, 0)| \\ & \qquad \qquad \qquad \qquad + n \cdot |\Delta_{t_{leq}^{(i)}} - \gamma_{t_{leq}^{(i)}}(-\Delta_{t_{leq}^{(i)}}, 0)|\big)\\
        &\geq \sum_{i = 1}^n \left(|-1 - \gamma_{t_{true}^{(i)}}(1, 0)| + |-1 - \gamma_{t_{geq}^{(i)}}(1, 0)| + n \cdot |-1 - \gamma_{t_{leq}^{(i)}}(1, 0)|\right)\\
        &= \sum_{i = 1}^n \left(|-1 - \gamma_{t_{true}^{(i)}}(1, 0)| + |-1 - 1| + n \cdot |-1 - 1|\right)\\
        &\geq \sum_{i = 1}^n \left(2 n + 1\right)\\
        &= n \cdot (2 n + 1)
    \end{align*} 

    whereas there exists another coupling strategy $C_\rho^* = (\gamma^*, \gamma^{'*})$ for $\rho$ that would assign 
    \begin{align*}
        \gamma_{t_{true}^{(i)}}^*(\texttt{in}\brangle{1}, \texttt{in}\brangle{2}) &= -\texttt{in}\brangle{1} + \texttt{in}\brangle{2}\\
        \gamma_{t_{geq}^{(i)}}^*(\texttt{in}\brangle{1}, \texttt{in}\brangle{2}) &= 1\\
        \gamma_{t_{leq}^{(i)}}^*(\texttt{in}\brangle{1}, \texttt{in}\brangle{2}) &= -\texttt{in}\brangle{1} + \texttt{in}\brangle{2}
    \end{align*}
    which satisfies the privacy constraints \[\gamma_{t_{true}^{(i)}}^* \leq \gamma_{t_{geq}^{(i)}}^* \geq \gamma_{t_{leq}^{(i)}}^*\] for all $i$, which has cost $2n$.
    
    For the looping branches $L_i$, the coupling strategy $C_{L_i}^* = (\gamma^*, \gamma^{'*})$ assigns

    \begin{align*}
        \gamma_{t_{true}^{(n + i)}}^*(\texttt{in}\brangle{1}, \texttt{in}\brangle{2}) &= 1\\
        \gamma_{t_{loop}^{(i)}}^*(\texttt{in}\brangle{1}, \texttt{in}\brangle{2}) &= -\texttt{in}\brangle{1} + \texttt{in}\brangle{2}\\
        \gamma_{t_{geq}^{(i)}}^*(\texttt{in}\brangle{1}, \texttt{in}\brangle{2}) &= 1
    \end{align*}

    which satisfies the privacy constraints \[\gamma_{t_{geq}^{(i)}}^* \geq \gamma_{t_{true}^{(n + i)}}^* \leq \gamma_{t_{loop}^{(i)}}^*\] for all $i$, and has cost $4$.
    
    Putting these strategies together, we have a coupling startegy $C^*$ for $B_n$ with cost $2n$, as opposed to at least $n (2n + 1)$ for any path-independent coupling strategy $C$.
\end{proof}