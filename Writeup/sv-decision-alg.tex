
\subsection{A Linear-Time Algorithm for Deciding Privacy}\label{decisionSection}

We also provide an efficient (i.e. linear time) algorithm for solving the decision problem of privacy.

Because the privacy constraint system for a looping branch completely characterizes its privacy, our goal is to efficiently solve the system for each looping branch.

We begin with an algorithm for a single looping branch that encodes each constraint of the privacy constraint system in a graph which preserves some of the structure of the looping branch. 

\begin{defn}[Privacy constraint graph]
    Let $L$ be a looping branch generated by $G$ that contains $n$ distinct transitions $t_i$. The \textbf{privacy constraint graph} $\mathcal{P}_L = (V, E)$ of $L$ is a directed graph where: 
    \begin{itemize}
        \item $V = \{{\bf 1, -1}\} \cup \{v_i: t_i\in L\}$. Each $v_i \in V$ represents the shift $\gamma_i$ on $t_i$.
        \item For all $(t_i, t_j)$ such that the constraint $\gamma_i \leq \gamma_j$ is in the approximate privacy constraint system of $L$, $(v_i, v_j) \in E$.
        \item There are nodes ${\bf 1, -1}\ \in V$ such that: 
        \begin{itemize}
            \item $({\bf -1}, v)\in E$ for all $v \in V$.
            \item $(v, {\bf 1})\in E$ for all $v \in V$.
            \item For every transition $t_i = (c_i, \sigma_i, \tau_i)$ such that $\Psi^{-1}(t_i)$ is in a cycle in $G$ and $c_i = \lguard[\texttt{x}]$, $({\bf 1}, v_i) \in E$.
            \item For every transition $t_i = (c_i, \sigma_i, \tau_i)$ such that $\Psi^{-1}(t_i)$ is in a cycle in $G$ and $c_i = \gguard[\texttt{x}]$, $(v_i, {\bf -1}) \in E$. 
        \end{itemize}
    \end{itemize}
\end{defn}

As suggested by the definition, the existence of an edge $(v_i, v_j)\in E$ corresponds to a constraint forcing $\gamma_i \leq \gamma_j$. Additionally, we introduce constraints that ``force'' certain shifts to be either $1$ or $-1$. 
A path from $\mathbf{1}$ to $\mathbf{-1}$ thus implies that the constraint system must be unsatisfiable; we show that this is in fact a complete decision procedure.

\begin{prop}\label{privacyConstraintGraphProp}
    A looping branch $L$ is differentially private if and only if there does not exist a path from $\bf 1$ to $\bf -1$ in the privacy constraint graph of $L$.
\end{prop}

Fortunately, we can directly combine the privacy constraint graphs for individual looping branches together to produce a privacy constraint graph for a program, which provides the linear time algorithm that we desired.

\begin{prop}\label{programPrivacyConstraintGraphPathReq}
    A program $P$ is differentially private if and only if there does not exist a path from $\bf 1$ to $\bf -1$ in the privacy constraint graph $\mathcal{P}_P$ of $P$, where $\mathcal{P}_P$ is the union of the privacy constraint graphs $\mathcal{P}_L$ of each looping branch $L\subseteq P$.
\end{prop}

\begin{thm}\label{LinearTimeDecidingPrograms}
    Given a program $P$ generated by a graph $G$, we can decide if $P$ is differentially private in linear time in the size of $G$.
\end{thm}

\begin{proof}
    Constructing the privacy constraint graph $\mathcal{P}_P$ takes linear time in the size of $P$, on which we can perform a breadth first search to check whether there is a path from $\bf 1$ to $\bf -1$. This can be done in linear time in the size of $\mathcal{P}_P$, which is linear in the size of $G$.
\end{proof}

We observe that we match the runtime obtained by Chadha, Sistla, and Viswanathan~\cite{chadhaLinearTimeDecidability2021} for deciding the privacy of programs under this model. 


