\documentclass[12pt]{article}


\usepackage[shortlabels]{enumitem} 
\usepackage{amsmath,amsfonts,amssymb,amsthm,bm,mathrsfs}
\usepackage{fancyhdr}
\usepackage[margin=1in]{geometry}
\usepackage{parskip}
\usepackage{tikz}
\usepackage{algorithm}
\usepackage{algpseudocode}
\usepackage{stmaryrd}
% \usepackage{mdframed}
\usepackage{hyperref}
\usepackage{xcolor, soul}
\sethlcolor{cyan}


\newcommand{\NN}{\mathbb{N}}
\newcommand{\ZZ}{\mathbb{Z}}
\newcommand{\QQ}{\mathbb{Q}}
\newcommand{\RR}{\mathbb{R}}
\newcommand{\CC}{\mathbb{C}}
\newcommand{\PP}{\mathbb{P}}
\newcommand{\EE}{\mathbb{E}}
\newcommand{\notimplies}{\;\not\!\!\!\implies}
\newcommand{\gguard}[1][x]{\texttt{insample}\geq#1}
\newcommand{\lguard}[1][x]{\texttt{insample} < #1}
\newcommand{\mvgguard}[1][\texttt{x}_1]{\texttt{insample}^{(#1)}\geq#1}
\newcommand{\mvlguard}[1][\texttt{x}_1]{\texttt{insample}^{(#1)}<#1}
\newcommand{\gaguard}{n<N \text{ AND } \texttt{insample} \geq\ \texttt{x}}
\newcommand{\laguard}{n<N\text{ AND }\texttt{insample} < \texttt{x}}
\newcommand{\itgguard}{\texttt{input}\neq\tau\text{ AND } \texttt{insample}\geq\texttt{x}}
\newcommand{\itlguard}{\texttt{input}\neq\tau\text{ AND }\texttt{insample} < \texttt{x}}
\newcommand{\range}{\texttt{range}}
\newcommand{\brangle}[1]{\langle#1 \rangle}
\newcommand{\guard}{\texttt{guard}}
\newcommand{\trans}{\texttt{trans}}
\newcommand{\Lap}{\texttt{Lap}}
\newcommand{\gcycle}{\texttt{G}-cycle}
\newcommand{\lcycle}{\texttt{L}-cycle}
\newcommand{\sgn}{\texttt{sgn}}
\newcommand{\andtext}{\text{ AND }}
\newcommand{\ortext}{\text{ OR }}
\newcommand{\supp}{\texttt{supp}}

\newcommand{\im}{\texttt{im}}

\newcommand{\todo}[2]{\textcolor{#1}{\textbf{#2}}}
\newcommand{\sasho}[1]{\todo{blue}{Sasho: <<#1>>}}
\newcommand{\sky}[1]{\todo{green}{Sky: <<#1>>}}
\newcommand{\azadeh}[1]{\todo{red}{Azadeh: <<#1>>}}
\newcommand{\vishnu}[1]{\todo{magenta}{Vishnu: <<#1>>}}

\DeclareMathOperator*{\argmax}{arg\,max}
\DeclareMathOperator*{\argmin}{arg\,min}

\providecommand{\floor}[1]{\lfloor #1 \rfloor}
\newtheorem{thm}{Theorem}[section]
\newtheorem{lemma}[thm]{Lemma}
\newtheorem{prop}[thm]{Proposition}
\newtheorem{cor}[thm]{Corollary}
\newtheorem{obs}[thm]{Observation}
\theoremstyle{definition}
\newtheorem{defn}[thm]{Definition}
\newtheorem{const}[thm]{Construction}
\newtheorem{examp}[thm]{Example}
\newtheorem{conj}[thm]{Conjecture}
\newtheorem{rmk}[thm]{Remark}
\newtheorem{clm}[thm]{Claim}

\newcommand{\isto}{\stackrel{\sim}{\smash{\longrightarrow}\rule{0pt}{0.4ex}}} 
\graphicspath{ {./} }
\bibliographystyle{plain} 


\begin{document}


\section{Introduction}

Differential privacy (DP) is a mathematical framework for privacy that provides rigorous guarantees on the amount of data leakage that can occur when releasing statistical information about a dataset. Since its introduction in 2006~\cite{dworkCalibratingNoiseSensitivity2006a}, DP has become the gold standard for private statistical analysis. 
Both private companies such as Google and Apple and government bodies such as the United States Census Bureau have announced their adoption of DP algorithms for their data collection and release procedures \cite{HowWeRe,PrivacyFeaturesa,LearningPrivacyScalea,abowdCensusBureauAdopts2018a}. 

In brief, DP ensures that it is unlikely that an adversary can distinguish whether or not one person's data was used in a private computation. To do this, DP algorithms rely on randomization, especially through the addition of statistical noise. DP also allows for the \textit{quantification} of privacy; the amount of information revealed about any individual can be summarized in a ``privacy parameter'', usually denoted $\varepsilon$. 

One especially useful feature of DP is that DP algorithms \textit{compose} together, with only a linear degradation in privacy cost. In particular, this means that many DP algorithms can be constructed by composing well-known private ``primitive'' mechanisms together; these algorithms thus also lend themselves to straightforward proofs of correctness. 

However, it can be notoriously tricky to analyze algorithms that venture outside of standard compositional techniques. For example, previous implementations of differential privacy by Apple have been criticized for actually leaking much more information than claimed (as measured by $\varepsilon$)~\cite{tangPrivacyLossApple2017,gadottiPoolInferenceAttacks2022}. 
Famously, many different iterations of the Sparse Vector Technique (SVT) algorithm have been produced and supposedly proven correct, some of which were later shown to actually fail at protecting privacy at all~\cite{lyuUnderstandingSparseVector2016a}. 

The difficulty of \textit{ensuring} that DP algorithms are truly private has led to work developing tools to formally verify that DP algorithms meet their claimed privacy bounds. However, it is known that complete verification of differentially private algorithms is undecidable, even for a relatively limited class of programs~\cite{bartheDecidingDifferentialPrivacy2020}.

One approach to deciding the privacy of a program has thus been to limit the model of programs even further, such as ensuring that every program has can take input and output only from a finite domain, or limiting programs to branching on real-valued comparisons \cite{bartheDecidingDifferentialPrivacy2020,chadhaLinearTimeDecidability2021,chadhaDecidingDifferentialPrivacy2023}.

An orthogonal approach has been to develop heuristic or incomplete approaches to automatically generating proofs of privacy for DP algorithms; one especially popular tool for this approach is a construct known as an 
\textbf{approximate lifting}~\cite{bartheProvingDifferentialPrivacy2016,bartheDifferentialPrivacyComposition2013,hsuProbabilisticCouplingsProbabilistic2017,albarghouthiConstraintBasedSynthesisCoupling2018,albarghouthiSynthesizingCouplingProofs2017}. 
Approximate liftings are a generalization of probabilistic couplings, themselves a well-known technique in probability theory for analyzing relationships between random variables.
Approximate liftings allow for a more structured proof approach to many algorithms that themselves are not conducive to a standard compositional analysis, such as SVT.

We demonstrate that approximate liftings can also be applied to the problem of deciding privacy. Specifically, we construct a limited program model inspired by algorithms like SVT that allows for comparisons between a real-valued input query and a threshold value (for example, one could ask the query ``How many towns have a population over 10,000?'').
The program model is constructed as a regular language of program \textbf{transitions}; we demonstrate how to construct couplings for individual characters of an alphabet (program transitions), for words in the alphabet (``straight line programs''), and for full programs (regular languages) that can prove if a program is differentially private. 

Our family of coupling proofs can be summarized as a system of linear constraints; solving the system immediately constructs a proof of privacy for a program. Most notably, if this system is unsolvable, or, equivalently, if no such coupling proof exists, we show that there exist no possible proofs of privacy, i.e. the program is not differentially private. Thus, coupling proofs allow us to provide a \textit{decision procedure} for the privacy of programs in our model. 

In particular, the satisfiability of the linear constraint system can be decided efficiently: in linear time in the size of the program. Additionally, the structure of our linear constraint system allows for the \textit{optimization} of coupling based proofs; by solving the system, we attempt to provide tighter bounds on the privacy parameter $\varepsilon$ that a given proof shows a program satisfies. 
We conjecture that the upper bound given by the linear system is \textit{tight}; there is no possible better bound on privacy cost. 

Possibly surprisingly, we also demonstrate that a subclass of programs in our model is, in fact, exactly equivalent to a previously analyzed, automata-theoretic model known as DiPA \cite{chadhaLinearTimeDecidability2021}, which also provides a linear-time decision algorithm for privacy. 
Although our two algorithms are seemingly disparate, we show that there are, in fact, direct relationships between the structure of the coupling constraint system and the decision algorithm of \cite{chadhaLinearTimeDecidability2021}; we argue that reframing the arguments of \cite{chadhaLinearTimeDecidability2021} through the lens of coupling based proofs provides insight and a more intuitive approach. 

Further, we argue that coupling proofs are useful in part because of their generalizability; to this end, we first extend our program model to accomodate an arbitrary number of threshold variables to compare inputs to and show that the same techniques for constructing coupling proofs for single-variable programs also extend almost immediately to this expanded program model. 
Indeed, we also show that, for two variable programs, coupling proofs \textit{remain} complete for deciding privacy. We conjecture that this result can be extended to more than two variables, demonstrating that, in all cases, coupling proofs completely characterize programs under our extended model as well. 


In short, our contributions are:
\begin{itemize}
    \item We develop a simple program model centred around comparing an input to a threshold value for which we show that there is a simple and, most notably, complete class of privacy proofs (``coupling proofs'') built from approximate liftings.
    \item We provide a linear-time algorithm for deciding whether or not a program in our model is differentially private or not, and, for programs that are differentially private, provide methods for computing the minimal privacy cost of a coupling proof both directly and in approximation. 
    \item We show an equivalence between our program model and the automata-theoretic model DiPA \cite{chadhaLinearTimeDecidability2021} and discuss connections between the two proof approaches. 
    \item We demonstrate how this model, and along with it coupling proofs, can be extended to multiple threshold variable programs to prove their privacy and show that, in the case of two variable programs, completely characterize their privacy. 
\end{itemize}

\section{Preliminaries}

\subsection{Differential Privacy}

We begin by introduce the formal definition of differential privacy and key results about DP algorithms. 

Intuitively, for any output $\sigma$ of a private algorithm $A$, the probability of obtaining $\sigma$ for a dataset with some individual Alex should close (measured by a multiplicative factor) to the probability of obtaining $\sigma$ for a ``similar'' dataset (in particular, with Alex's data removed or changed).

In general, we work with \textbf{datasets} $\mathcal{X}\in X^n$ of size $n$ where $X$ is the set of all possible individual data points.

We first define what it means for datasets to be ``similar'' to each other. 

\begin{defn}
    Two datasets $\mathcal{X}=(x_1, \ldots, x_n), \mathcal{X}'=(x'_1, \ldots, x'_n)\in X^n$ are \textbf{adjacent} (denoted by $\mathcal{X}\sim\mathcal{X}'$) if $|\{i: x_i\neq x'_i\}|\leq 1$\footnote{A common variant is to define adjacency by the removal or addition of an entry, rather than by the modification of an entry}.
\end{defn}

This motivates the formal definition of differential privacy:

\begin{defn}[Pure Differential Privacy]
    For some $\varepsilon>0$, a randomized algorithm $A$ is $\varepsilon$-differentially private if, for all pairs of adjacent datasets $X\sim X'$ and all events $E \subseteq \im(A)$, \[\PP[A(X) \in E]\leq e^\varepsilon \PP[A(X')\in E]\]
    If there exists some $\varepsilon>0$ such that $A$ is $\varepsilon$-differentially private, we call $A$ ``differentially private'' or simply ``private'' without reference to a specific $\varepsilon$.
\end{defn}

The privacy parameter $\varepsilon$ is traditionally thought of as analogous to the ``privacy cost'' of a program; the larger $\varepsilon$ is, the more privacy is ``lost''. In our analysis of differentially private algorithms, we thus aim to minimize the ``cost'' of an algorithm, i.e. find a tight upper bound, if one exists, on $\varepsilon$ for an algorithm. 

We also introduce max divergence as a method of measuring ``true'' privacy cost. 

\begin{defn}[Max Divergence]
    For any two probability distributions $P, Q$ over a shared event space $E$, the max-divergence of $P$ and $Q$ is 
    $D_{\infty}(P||Q) = \max_{e\in E}\ln\left(\frac{P(e)}{Q(e)}\right)$.
\end{defn}

It immediately follows from the definition that an algorithm $A$ is $\varepsilon$-DP if and only if for all adjacent inputs $X\sim X'$, $D_{\infty}(A(X)||A(X'))\leq \varepsilon$. In particular, $D_{\infty}(A(X)||A(X'))\leq \varepsilon$ represents the tightest possible upper bound on $\varepsilon$ for $A$. 

An extremely useful property of differential privacy is that differentially private programs can be \textbf{sequentially composed} with a linear degradation in privacy:

\begin{thm}[Standard Composition \cite{dworkAlgorithmicFoundationsDifferential2014b}]
    If $A$ is $\varepsilon_1$-differentially private and, for all $\sigma$, $B(\sigma, \cdot)$ is $\varepsilon_2$-differentially private, then $B(A(X), X)$ is $\varepsilon_1+\varepsilon_2$-differentially private. 
\end{thm}

Composition therefore allows us to view privacy parameters $\varepsilon$ as a ``budget'' for privacy-leaking operations in a program. 

\subsubsection{Sensitivity and the Laplace Mechanism}

Because we are typically interested in analyzing \textit{functions} of a dataset (for example, the \textbf{average} age of a group), it is often useful to examine differential privacy through a similar model --- instead of comparing two adjacent datasets $X\sim X'$, we compare \textbf{queries} $f(X)$ and $f(X')$. In this world, we care about the \textit{sensitivity} of functions: how much a function \textit{changes} when considering adjacent inputs. 
The ($\ell_1$-)sensitivity of a function $f: X\to \RR$, denoted $\Delta f$, is defined as $\Delta f = \max_{X\sim X'}||f(X)-f(X')||_1$.

For any given function with a known sensitivity, we can construct a differentially private version of the function with the \textbf{Laplace Mechanism}. 

Recall that the Laplace distribution $\Lap(\mu, b)$ with mean $\mu$ and spread parameter $b$ is the probability distribution with probability density function $f(x) = \frac{1}{2b}\exp(-\frac{|x-\mu|}{b})$. If $\mu =0$, we may abbreviate $\Lap(0, b)$ as $\Lap(b)$. 

The Laplace Mechanism adds noise sampled from the Laplace distribution to a query result. In particular, the noise is dependent on the sensitivity of the input function; as expected, the higher the sensitivity of a function is, the more noise the Laplace mechanism will add to it.   

\begin{thm}[Theorem 3.6~\cite{dworkAlgorithmicFoundationsDifferential2014b}]
    For a function $f$ with sensitivity $\Delta$, $A(X) = f(X) + \Lap(\frac{\Delta}{\varepsilon})$ is $\varepsilon$-differentially private. 
\end{thm}

\subsection{Couplings and Liftings}

We now introduce probabilistic couplings and approximate liftings, which are probabilistic tools that allow for the structured creation of proofs of differential privacy.

Probabilistic couplings are a common tool in analyses of probabilistic processes that allow two otherwise independent processes to be correlated together and analyzed as a joint distribution; in particular, this is useful when attempting to prove properties about the \textbf{relationship} between two probabilistic processes. 

\begin{defn}[Couplings]
    A coupling between two distributions $A$ and $B$ is a joint distribution $C$ such that $\pi_1(C)=A$ and $\pi_2(C)=B$, where $\pi_1(C)$ and $\pi_2(C)$ are the first and second marginals of $C$, respectively. 
\end{defn}

A previous line of work extends couplings to reason about privacy in particular. The core construct introduced is the \textbf{approximate lifting}~\cite{bartheProvingDifferentialPrivacy2016,bartheDifferentialPrivacyComposition2013,hsuProbabilisticCouplingsProbabilistic2017,albarghouthiConstraintBasedSynthesisCoupling2018,albarghouthiSynthesizingCouplingProofs2017}:

\begin{defn}[$\varepsilon$-Lifting]
    Let $A_1, A_2$ be two sample spaces. We say a distribution $\mu_1$ on $A_1$ and $\mu_2$ on $A_2$ are related by the $\mathbf{\varepsilon}$\textbf{-lifting} of the relation $\Psi\subseteq A_1\times A_2$ (written $\mu_1(\Psi)^{\#\varepsilon}\mu_2$) if there exist two \textbf{witness distributions} $\mu_L, \mu_R$ on $A_1\times A_2$ such that\begin{enumerate}
        \item $\pi_1(\mu_L) = \mu_1$ and $\pi_2(\mu_R) = \mu_2$
        \item $\supp(\mu_L), \supp(\mu_R)\subseteq \Psi$
        \item $\sup_{E\subseteq A_1\times A_2}(\PP_{x\gets \mu_L}[x\in E]- e^\varepsilon \PP_{x\gets \mu_R}[x\in E])\leq 0$
    \end{enumerate}
\end{defn}

In some sense, approximate liftings can be considered ``half-couplings'', where ``half'' (the first marginal) of $\mu_L$ is coupled with ``half'' (the second marginal) of $\mu_R$. Approximate liftings also incur a ``privacy cost'' $\varepsilon$. 

As expected, there is a close connection between approximate liftings and differential privacy:

\begin{thm}[\cite{bartheProvingDifferentialPrivacy2016}]
    An algorithm $A(X)$ is $\varepsilon$-differentially private if and only if, for all adjacent inputs $X\sim X'$, $A(X)(=)^{\#\varepsilon}A(X')$.
\end{thm}

However, relaxing the lifted relation from equality to implication still allows us to prove that an algorithm is private; this proves useful in particular because constructing an approximate lifting of the equality relation can be intractable in practice. 

\begin{thm}[\cite{bartheProvingDifferentialPrivacy2016}]\label{implicationcouplingthm}
    If for all adjacent input sequences $X\sim X'$ and outputs $\sigma$ of $A$, $A(X)\{(a, b): a=\sigma\implies b=\sigma\}^{\#\varepsilon}A(X')$, then $A(X)$ is $\varepsilon-$differentially private.
\end{thm}

The existence of an implication coupling of this form is itself a direct proof of privacy---our goal will be to ``automatically'' construct such proofs. 

Many standard differential privacy results can be restated in terms of coupling-based proofs; we primarily leverage the facts that approximate liftings can be composed and that we can couple together Laplace random variables at the expected privacy cost. 

\begin{thm}[Composition of Liftings \cite{bartheProvingDifferentialPrivacy2016}]\label{liftingcomposition}
    Let $A_1, B_2, A_2, B_2$ be distributions over $S_1, T_1, S_2, T_2$, respectively and let $R_1\subseteq S_1\times T_1$, $R_2\subseteq S_2\times T_2$ be relations. If $A_1 R_1^{\#\varepsilon_1}B_1$ is a valid lifting and we can construct $A_2R_2^{\#\varepsilon_2}B_2$ under the assumption that the predicate $A_1 R_1 B_1$ is true, then $A_2 R_2^{\#\varepsilon_1+\varepsilon_2}B_2$.
\end{thm}

Composition for liftings operates differently than in the standard analysis of differentially private composition. Each lifting should be thought of as generating a logical assertion: in particular, the assertion is that the relation being lifted holds (for example, one could assert that two distributions $A_1$ and $B_1$ are equal by constructing the lifting $A_1 (=)^{\#\varepsilon}B_1$). 
Composition proceeds if, \textit{assuming that the first relation holds}, a second, likely more complex, lifting can be constructed. If this is true, then theorem \ref{liftingcomposition} asserts that the second lifting can be shown to hold \textit{unconditionally}, simply with a additive penalty to the privacy cost (specifically, the privacy cost increases by the privacy cost of the first lifting). 


The Laplace mechanism can also be restated in terms of approximate liftings: 
\begin{prop}[Laplace Mechanism for Liftings \cite{bartheProvingDifferentialPrivacy2016}]
    If $X_1\sim\Lap(\mu_1, \frac{1}{\varepsilon})$ and $X_2\sim\Lap(\mu_2, \frac{1}{\varepsilon})$, then $X_1(=)^{\#\varepsilon|\mu_1-\mu_2|}X_2$.
\end{prop}


Theorems \ref{implicationcouplingthm} and \ref{liftingcomposition} suggest the form of coupling proofs for privacy: given two ``runs'' of an algorithm on adjacent inputs, construct many smaller liftings between program variables in each run and compose these liftings together to show that a final implicatory lifting between the outputs of the two runs exists. 

\subsection{Proving SVT with couplings}

For illustrative purposes, we provide a lifting-based proof of privacy for a notoriously tricky algorithm, the Sparse Vector Technique (SVT), which is particularly for requiring an analysis that goes beyond standard composition. 
At a high level, SVT takes in a possibly infinite stream of input queries and a threshold value and outputs whether or not the input queries are above or below the threshold; see algorithm 1 for a full definition. 

However, unusually for differentially private algorithms, SVT can output a potentially unbounded number of ``below threshold'' queries before the first $c$ ``above threshold''s (or vice-versa), where $c$ is some constant set by the user; when $c=1$, SVT is also referred to as ``Above (or Below) Threshold''.
 Potential applications include, for example, checking that a series of inputs is within an expected range or, as the name suggests, privately determining which elements of a sparse vector are non-zero. 

Because SVT allows for a potentially unbounded number of ``below threshold'' query outputs, its analysis requires a non-standard approach; a naive composition approach that assigns a fixed cost to outputting the result of each query will immediately result in unbounded overall privacy cost. 
Indeed, the analysis of SVT is notoriously difficult, with multiple published attempts at privacy proofs that were later shown to be incorrect\footnote{A textbook analysis of SVT, along with a discussion of bugged versions and incorrect privacy proofs, can be found at \cite{lyuUnderstandingSparseVector2016a}}. 

However, re-analyzing SVT using approximate liftings is relatively simple.

\begin{algorithm}
    \hspace*{\algorithmicindent}\textbf{Input}: $\mathcal{X}\in X^n$, $T\in \RR$, $Q=q_1, \ldots \in {(X^n\to \RR)}^*$ with sensitivity $\Delta$, $c\in \NN$.
    \begin{algorithmic}[1]
        \caption{Sparse Vector Technique}\label{couplingAlg}
        \State $\varepsilon_1, \varepsilon_2 \gets \frac{\varepsilon}{2},
        \rho \gets \Lap(\frac{\Delta}{\varepsilon_1})$, $count \gets 0$
		\For{$q_i \in Q$} 
			\State $z\gets \Lap(\frac{2c\Delta}{\varepsilon_2})$
            \If{$q_i(\mathcal{X}) + z \geq T + \rho$}
                \State\textbf{output} $\top$
                \State$count\gets count+1$
                \If{$count \geq c$}
                    \State$\textbf{break}$
                \EndIf
            \Else
                \State\textbf{output} $\bot$
            \EndIf
		\EndFor
    \end{algorithmic}
\end{algorithm}


\begin{thm}
    Sparse Vector Technique is $\varepsilon$-differentially private. 
\end{thm}

\begin{proof}[Proof (adapted from an informal proof of \cite{bartheProvingDifferentialPrivacy2016})]
    Consider two runs of SVT with adjacent inputs $\mathcal{X}\sim\mathcal{X}'$, respectively. We are aiming to show that $SVT(\mathcal{X}, T, Q, c)\{(a, b): a=\sigma \implies b=\sigma\}^{\#\varepsilon}SVT(\mathcal{X}', T, Q, c)$ is a valid lifting. 

    Fix some output $\sigma \in \{\bot, \top\}^n$. Let $A = \{i:\sigma_i = \top\}$ be the indices of queries that are measured to be above the threshold. Note that $|A| = c$. 
    
    For every program variable $x$, let $x\brangle{1}$ and $x\brangle{2}$ represent the value of $x$ in $SVT(\mathcal{X}, T, Q, c)$ and $SVT(\mathcal{X}', T, Q, c)$, respectively, so, for example, $q_i(\mathcal{X})\brangle{1} = q_i(\mathcal{X})$ and $q_i(\mathcal{X})\brangle{2} = q_i(\mathcal{X}')$. 

    Let $\tilde{T}=T + \rho$. Then $\tilde{T} \sim \Lap(T, \frac{\Delta}{\varepsilon_1})$, so the lifting $\tilde{T}\brangle{1} +\Delta (=)^{\#\varepsilon_1}\tilde{T}\brangle{2}$ exists. 

    Let $S_i = q_i(\mathcal{X}) + z_i$, so $S_i \sim\Lap(q_i(\mathcal{X}), \frac{2c\Delta}{\varepsilon_2})$.

    For all $i$ such that $0\leq i < n$, $i\notin A$, we construct the lifting $z_i\brangle{1} (=)^{\#0}z_i\brangle{2}$. 

    Then note that because $\tilde{T}\brangle{1}+\Delta = \tilde{T}\brangle{2}$ and $z_i\brangle{1} = z_i \brangle{2}$, we know that $S_i\brangle{1} < \tilde{T}\brangle{1} \implies S_i\brangle{2} < \tilde{T}\brangle{2}$. This means that for all such $i$, if the condition on line 4 is not satisfied in the first run, then it also is not satisfied in the second run.

    For all $i\in A$, create the lifting $z_i\brangle{1}(=)^{\#\frac{\varepsilon_2}{c}}z_i\brangle{2} - q_i(\mathcal{X})+q_i(\mathcal{X}')-\Delta$, or equivalently, \\$S_i\brangle{1} +\Delta (=)^{\#\frac{\varepsilon_2}{c}} S_i\brangle{2}$. Note that the lifting has cost $\frac{\varepsilon_2}{c}$ since $|q_i(\mathcal{X})-q_i(\mathcal{X}')|\leq \Delta$. Like before, this means that if the condition on line 4 is satisfied in the first run, it must also be satisfied in the second run. 

    Then again because $\tilde{T}\brangle{1} +\Delta = \tilde{T}\brangle{2}$, $S_i\brangle{1} \geq \tilde{T}\brangle{1} \implies S_i\brangle{2} \geq \tilde{T}\brangle{2}$

    Thus, for all $i$, $SVT(\mathcal{X}, T, Q, c)_i = \sigma_i \implies SVT(\mathcal{X}', T, Q, c)_i = \sigma_i$, so $SVT(\mathcal{X}, T, Q, c)\{(a, b): a=\sigma \implies b=\sigma\}^{\#\varepsilon_1+\varepsilon_2}SVT(\mathcal{X}', T, Q, c)$.

    By Theorem \ref{implicationcouplingthm}, SVT is $\varepsilon$-differentially private. 
\end{proof}


\section{Deciding Privacy Using Couplings}

We now begin the process of building up a program model through the lens of regular languages. 
We will model programs as simply regular languages comprised of possible program paths (or executions) where each path corresponds to a word in the language. This approach simultaneously allows us to build approximate liftings for each path that prove the overall privacy of a program.

We begin with single transitions between two program locations, which correspond directly to characters in our alphabet. 

\sky{todo: add a note about treating $\varepsilon$ as a parameter to the program}


\subsection{Individual Transitions}

Transitions act as guarded statements whose guard is dependent on a persistent real-valued variable $\texttt{x}$ and a real-valued input to which random noise is added; we can thus formally define individual transitions as follows:

\begin{defn}[Transitions]\label{svTransDef}
    A transition is a tuple $t = (c, \sigma, \tau)$, where \begin{itemize}
        \item $c\in \{\texttt{true}, \lguard[\texttt{x}], \gguard[\texttt{x}]\}$ is a \textbf{guard} for the transition. We use $c(in, x)$ to denote the boolean function corresponding to substituting $\texttt{insample} = in$ and $\texttt{x} =x $ into $c$.
        \item $\sigma \in \Gamma\cup\{\texttt{insample}, \texttt{insample}'\}$ is the \textbf{output} of $t$, for some finite alphabet of symbols $\Gamma$.
        \item $\tau\in\{\texttt{true}, \texttt{false}\}$ is an \textbf{assignment control}: a boolean value indicating whether or not the stored value of $\texttt{x}$ will be updated when $t$ is taken.
    \end{itemize}
\end{defn}

We additionally associate two positive real-valued \textbf{noise parameters} $P(t) = (d, d')$ with each transition. 

Furthermore, we introduce a special type of transition called a \textbf{public} transition, which we use to represent reading publicly-available data. Every \textbf{public} transition must be of the form $t_{pub}= (\texttt{true}, \sigma, \tau)$ for some $\sigma \in \Gamma\cup\{\texttt{insample}, \texttt{insample}'\}$ and $\tau \in \{\texttt{true}, \texttt{false}\}$. We will call all other transitions \textbf{private}. 

\subsubsection{Transition Semantics}

We can think of each transition as defining an atomic program step: a transition reads a real valued input $\texttt{in}$, compares a noisy version of it (\texttt{insample}) to a threshold $\texttt{x}$, and, depending on the result of the comparison, outputs a value $\sigma$ and possibly updates the value of $\texttt{x}$.

The semantics of a transition are defined by a function that maps an initial program state and a real-valued input to a subsequent program state. 

A program state is a tuple consisting of a distribution of possible values for the program value $\texttt{x}$, and a distribution of possible values for the current output $\sigma$. 

Let $S = \RR\times (\Gamma \cup \RR)^*$ be the set of all program states. 
As expected, every possible input is simply an element of $\RR$.

Then the semantics of a transition $t$ can be defined as a function $\Phi_t:dist_\downarrow(S)\times \RR \to dist_\downarrow (S)$ that maps a subdistribution of initial program states and an input to a subdistribution of subsequent program states. 

More precisely, given $t = (c, \sigma, \tau)$ where $t$ has noise parameters $P(t) = (d_t, d_t')$, let $\texttt{insample}\sim \Lap(\texttt{in}, \frac{1}{d_t\varepsilon})$ and $\texttt{insample}'\sim \Lap(\texttt{in}, \frac{1}{d_t'\varepsilon})$ be independent Laplace random variables. 

Given an initial program state $(x, \sigma_0)\in S$, we define the distribution of states that $t$ maps $(x, \sigma_0)$ to as follows: if $\sigma \in \Gamma$,
\[\PP_{(x, \sigma_0)}[(x', \sigma_0\cdot \sigma)] = \begin{cases}
  \mathbb{I}[x = x']\PP[c(\texttt{insample}, x)] & \tau = \texttt{false}\\
  \PP[\texttt{insample} = x' \land c(\texttt{insample}, x)] & \tau = \texttt{true}
\end{cases}\]

Otherwise, if $\sigma=\texttt{insample}$, for all $y\in \RR$,
\[\PP_{(x, \sigma_0)}[(x', \sigma_0\cdot y)] = \begin{cases}
  \mathbb{I}[x = x']\PP[\texttt{insample} = y\land c(\texttt{insample}, x)] & \tau = \texttt{false}\\
  \mathbb{I}[x'=y]\PP[\texttt{insample} = x' \land c(\texttt{insample}, x)] & \tau = \texttt{true}
\end{cases}\]

Finally, if $\sigma=\texttt{insample}'$, for all $y\in \RR$
\[\PP_{(x, \sigma_0)}[(x', \sigma_0\cdot y)] = \begin{cases}
  \mathbb{I}[x = x']\PP[c(\texttt{insample}, x)]\PP[\texttt{insample}' = y] & \tau = \texttt{false}\\
  \PP[\texttt{insample} = x' \land c(\texttt{insample}, x)]\PP[\texttt{insample}' = y] & \tau = \texttt{true}
\end{cases}\]

Here, $\mathbb{I}[p]$ is used to denote the Iverson bracket, which evaluates to 1 if the predicate $p$ is true and 0 otherwise.

Then for any initial subdistribution of states $\theta$, $\Phi_t(\theta, \texttt{in})$ can be defined by weighting the sum of all subdistributions $\PP_s$ for $s\in S$, where the weight of each $\PP_s$ is given by the probability of $s$ given by $\theta$. 

More formally, for any initial output sequence $\sigma_0\in (\Gamma\cup\RR)^*$, let $S^{(\sigma_0)}$ be the set of program states $(x, \sigma)$ such that $\sigma = \sigma_0$. Then for any subdistribution of program states $\theta$, $\Phi_t(\theta, \texttt{in})$ is a subdistribution $O$ such that, for all $x'\in \RR$ and $\sigma \in \Gamma\cup\RR$,
\[
  \PP_O[(x', \sigma_0\cdot \sigma)] = \int_{S^{(\sigma_0)}}\PP_\theta[s]\PP_s[(x', \sigma_0\cdot \sigma)]ds
\]

Every other possible state is assigned probability 0 by $O$; with probability $1-\int_S \PP_O[s]ds$, we consider the transition to have halted.

We primarily are concerned with the probability that a transition outputs a value (i.e. its guard is satisfied) and, in particular, outputs a certain value $o$, where $o\subseteq \Gamma\cup\RR$ is a measurable event.

We denote this probability as $\PP[\texttt{x}, t, \texttt{in}, o]$, where $\texttt{x} \in dist(\RR)$ is the initial distribution of $\texttt{x}$, $t$ is a transition, $\texttt{in}\in \RR$ is a real-valued input, and $o\in \Gamma\cup \mathcal{P}(\RR)$ is a possible output of $t$. 
Specifically, let $O = \Phi_t((\texttt{x}, \lambda), \texttt{in})$, where $\texttt{x}\in \RR$ and $\lambda$ represents the empty string. Then $\PP[\texttt{x}, t, \texttt{in}, o]$ is the marginal of $\Phi_t((\texttt{x}, \lambda), \texttt{in})$ on $(\cdot, o)$.

\subsubsection{Couplings}

We now construct couplings for transitions with the aim of using them as building blocks for proofs of privacy.

First, we need to adapt standard privacy definitions to our specific setting; for example, recall that $\texttt{in}$, in reality, represents a \textbf{function} of some underlying dataset. This means that `closeness' in this context is defined as follows:

\begin{defn}[Adjacency and Validity]
    Two real-valued inputs $\texttt{in}\sim_{\Delta} \texttt{in}'$ are $\Delta$-adjacent if $|\texttt{in}-\texttt{in}'|\leq \Delta$. For a private transition $t$, $\texttt{in}\sim_{\Delta}\texttt{in}'$ is a \textbf{valid} pair of adjacent inputs if $\Delta = 1$ and for a public transition, $\texttt{in}\sim_{\Delta}\texttt{in}'$ is valid if $\Delta = 0$.
\end{defn}

Note that, in order to properly model public transitions (i.e. transitions whose input is public information), we require that all adjacent runs of the program provide the exact same input to a public transition. 

We show that, for any transition, we can construct approximate liftings such that each transition is proven ``private''. 

In particular, we construct liftings that are \textbf{parameterized} by three real values, $\gamma_x$, $\gamma_q$, and $\gamma_q'$. Specifically, we view choices of $\gamma_x, \gamma_q$, and $\gamma_q'$ as a \textbf{strategy} for proving that a transition is differentially private.

\begin{defn}[Valid Coupling Strategies]
    A \textbf{valid coupling strategy} for a transition $t = (c, \sigma, \tau)$ is a function $C_t:\RR^2\to[-1, 1]^3$ such that for any two valid adjacent inputs $\texttt{in}\brangle{1}\sim\texttt{in}\brangle{2}$, $C_t$ maps $(\texttt{in}\brangle{1},\texttt{in}\brangle{2})$ to a tuple $(\gamma_x, \gamma_t, \gamma_t')$ such that, for any valid adjacent pair of inputs, the constraints \[
        \begin{cases}
          \gamma_t\leq\gamma_x & c = \lguard[\texttt{x}]\\
          \gamma_t\geq\gamma_x & c = \gguard[\texttt{x}]\\
          \gamma_t=0 & \sigma = \texttt{insample}\\
          \gamma_t'=0 & \sigma = \texttt{insample}'
        \end{cases},
      \]
      are all satisfied. 
\end{defn}

For any \textbf{valid} coupling strategy, we can bound the difference of any transition outputting a specific value for any two valid adjacent inputs. 

To construct approximate liftings of a transition $t = (c, \sigma, \tau)$, we will analyze the behaviour of two different \textbf{runs} of $t$, one with input $\texttt{in}\brangle{1}$ and one with input $\texttt{in}\brangle{2}$. 

Our approach to couplings will be that for every Laplace-distributed variable, we will couple the value of the variable in one run with its value in the other \textbf{shifted} by some amount. 

We differentiate between the values of variables in the first and second run by using angle brackets $\brangle{k}$, so, for example, we will take $X\brangle{1}$ to be the value of $\texttt{x}$ at location $q$ in the run of $t$ with input $\texttt{in}\brangle{1}$ and $X\brangle{2}$ to be the value of $\texttt{x}$ in the run of $t$ with input $\texttt{in}\brangle{2}$. 

We thus want to create the lifting $o\brangle{1}\{(a, b): a=\sigma\implies b=\sigma\}o\brangle{2}$, where $o\brangle{1}$ and $o\brangle{2}$ are random variables representing the possible outputs of $t\brangle{1}$ and $t\brangle{2}$, respectively.

We must guarantee two things: that if the first transition's guard is satisfied, then the second transition's guard is also satisfied and that both runs output the same value $\sigma$ when the guard is satisfied. Note that if $c = \texttt{true}$, the first condition is trivially satisfied and when $\sigma\in \Gamma$, the second condition is trivially satisfied. 

This gives us our major coupling lemma, which allows us to define a coupling proof for every valid coupling strategy.

\begin{lemma}\label{simplifiedIndTransitionCoupling}
  For any transition $t$, any possible output event $\sigma$ of $t$ and any two valid adjacent inputs $\texttt{in}\brangle{1}\sim \texttt{in}\brangle{2}$, if we are given a valid coupling strategy, i.e. three real valued ``shifts'' $\gamma_x, \gamma_t, \gamma_t'\in [-1, 1]$ such that \[
    \begin{cases}
      \gamma_t\leq\gamma_x & c = \lguard[\texttt{x}]\\
      \gamma_t\geq\gamma_x & c = \gguard[\texttt{x}]\\
      \gamma_t=0 & \sigma = \texttt{insample}\\
      \gamma_t'=0 & \sigma = \texttt{insample}'
    \end{cases},
  \]
  then we can construct an approximate lifting that proves $\PP[X\brangle{1}, t, \texttt{in}\brangle{1}, \sigma]\leq e^{d\varepsilon}\PP[X\brangle{2}, t, \texttt{in}\brangle{2}, \sigma]$ for some bounded $d>0$ and initial threshold Laplace-distributed variables $X\brangle{1}$, $X\brangle{2}$.
\end{lemma}

A precise version of this lemma can be found in the appendix as lemma \ref{indTransitionCoupling}. 

Specifically, lemma \ref{indTransitionCoupling} gives us the bound $d \leq (|\mu\brangle{1}-\mu\brangle{2}+\gamma_x|)d_x+(|-\texttt{in}\brangle{1}+\texttt{in}\brangle{2}-\gamma_t|)d_t+(|-\texttt{in}\brangle{1}+\texttt{in}\brangle{2}-\gamma_t'|)d_t'$, where $\mu\brangle{1}$ and $\mu\brangle{2}$, respectively, are the means of $X\brangle{1}$ and $X\brangle{2}$.

We call $d$ the \textbf{cost} $cost(C)$ of any coupling strategy $C = (\gamma_x, \gamma_t, \gamma_t')$.



\subsection{Straightline Programs}

We now demonstrate how to concatenate transitions and their corresponding coupling strategies together into \textit{straight line programs} (SLPs).

\begin{defn}[Straight Line Programs]
    A straight line program (SLP) is a sequence of transitions. For an SLP $\rho = t_0\cdot t_1\cdot\ldots\cdot t_{n-1}$, if $t_0$ is of the form $t_0 = (\texttt{true}, \sigma, \texttt{true})$ for some $\sigma$, then we call $\rho$ an \textbf{initialized} SLP.
\end{defn}

We use standard properties of sequences for SLPs, such as the notion of the length of an SLP and the tail of an SLP, in the expected manner. For an SLP $\rho$, we use the notation $\rho_{i:j}$ to represent the subsequence  $t_i t_{i+1} \ldots t_{j-1}$ of $\rho$.

We directly lift the concept of inputs and outputs from individual transitions to SLPs; thus, each SLP reads in a \textbf{sequence} of inputs and outputs a sequence of outputs, one for each transition in the SLP. 

\subsubsection{Straight Line Program Semantics}

The semantics of an SLP are defined as a function mapping a subdistribution of program states and a real-valued input sequence to a subdistribution of final program states. 

More specifically, the semantics of an SLP $\rho = t_0t_1\cdots t_{n-1}$ are a function $\Phi_\rho: dist_\downarrow(S)\times \RR^n \to dist_\downarrow(S)$. $\Phi_\rho$ can be defined by composing transition semantics in the expected manner:

\[\Phi_\rho(s, \texttt{in}) = \begin{cases}
    \Phi_{t_0}(s, \texttt{in}_0)& |\rho| = 1\\
    \Phi_{t_{n-1}}(\Phi_{\rho_{0:n-1}}(s, \texttt{in}_{0:n-1}), \texttt{in}_{n-1})& |\rho| >1
\end{cases}\]


Like with transitions, we denote the probability that an SLP $\rho = t_0 t_1\ldots t_{n-1}$ outputs a specific value given an initial $\texttt{x}\in\RR$, input sequence $\texttt{in} \in \RR^n$, and possible measurable output sequence $o \subseteq (\Gamma\cup\RR)^n$ as $\PP[\texttt{x}, \rho, \texttt{in}, o]$. As before, $\PP[\texttt{x}, t, \texttt{in}, o]$ is the marginal of $\Phi_\rho((\texttt{x}, \lambda), \texttt{in})$ on $(\cdot, o)$.

For a initialized SLP $\rho$, note that the initial value of $\texttt{x}$ is irrelevant, so we will shorthand $\PP[\texttt{x}_0, \rho, \texttt{in}, \sigma]$ to $\PP[\rho, \texttt{in}, \sigma]$.


\subsubsection{Privacy}

By leveraging the construction of couplings for individual transitions, we can construct a set of approximate liftings for SLPs.

Because SLPs read in a \textit{sequence} of real-valued inputs, we need to slightly modify our definition of valid adjacent inputs.

\begin{defn}[Validity for a sequence of inputs]
    Two input sequences $\{\texttt{in}_i\brangle{1}\}_{i=1}^n\sim\{\texttt{in}_i\brangle{2}\}_{i=1}^n$ are \textbf{valid} and \textbf{adjacent} input sequences for an SLP $\rho = t_0\ldots t_{n-1}$ if, for all $1\leq i\leq n$, if $t_i$ is a private transition, then $\texttt{in}_i\brangle{1}\sim_1\texttt{in}_i\brangle{2}$ and if $t_i$ is a public transition, then $\texttt{in}_i\brangle{1}=\texttt{in}_i\brangle{2}$
\end{defn}

Thus, we have the following definition of privacy for initialized SLPs:

\begin{defn}[$d\varepsilon$-differential privacy for an SLP]
    An initialized SLP $\rho$ of length $n$ is $d\varepsilon$-differentially private for some $d>0$ if $\forall \varepsilon>0$, for all valid adjacent input sequences $\texttt{in}\brangle{1}\sim \texttt{in}\brangle{2}$ of length $n$ and all possible output sequences $\sigma$ of length $n$, $\PP[\rho, \texttt{in}\brangle{1}, \sigma]\leq e^{d\varepsilon}\PP[\rho, \texttt{in}\brangle{2}, \sigma]$.
\end{defn}

Note that, following \cite{chadhaLinearTimeDecidability2021}, we slightly redefine $\varepsilon$-differential privacy as $d\varepsilon$-differential privacy, treating $\varepsilon$ as a universal scaling parameter that can be fine-tuned by users for their own purposes. 
We argue that this definition is functionally equivalent, since if we are targeting $\varepsilon^*$-differential privacy overall, we can always take $\varepsilon = \frac{\varepsilon^*}{d}$.

\subsubsection{Sequential composition of couplings}

We can compose together a series of couplings associated with each transition to produce a coupling proof of privacy for an entire SLP. 

If $o\brangle{1}, o\brangle{2}$ are random variables representing the output of an SLP $\rho$ given input sequences $\texttt{in}\brangle{1}$ and $\texttt{in}\brangle{2}$, respectively, 
then if we can create the coupling $o\brangle{1}\{(a, b): a=\sigma\implies b=\sigma\}^{\#d\varepsilon}o\brangle{2}$ for some $d>0$ for all adjacent inputs $\texttt{in}\brangle{1}\sim\texttt{in}\brangle{2}$ and all possible outputs $\sigma$, we show that $\rho$ is differentially private.

As it turns out, directly composing together the couplings from lemma \ref{simplifiedIndTransitionCoupling} are sufficient; the constraints imposed upon shifts for a coupling for transition $t_i$ depend solely on the shift at the most recent \textbf{assignment transition} in $\rho$ (i.e. the most recent transition $t_j$ such that $\tau_j = \texttt{true}$). 
The coupling shifts for \textit{non-assignment transitions} can thus never impact each other. 

We will let $A_\rho$ be the set of \textbf{assignment transitions} in an SLP $\rho$. Additionally, for every transition $t_i$ in $\rho$, let $t_{at(i)}$ be the most recent assignment transition in $\rho$ before $t_i$; i.e., $at(i) = \max\{j<i: t_j\in A_\rho\}$. If such a $j$ does not exist, we set $at(i)=-1$. 

In particular, note that for transition $t_i$, $\gamma_x = \gamma_{at(i)}$, where $\gamma_{-1}$ is the shift applied to the initial $\texttt{x}$-values $\texttt{x}_0\brangle{1}$ and $\texttt{x}_0\brangle{2}$ (for initialized SLPs, note that $\gamma_{-1}$ is irrelevant).

Thus, for an individual transition $t_i$ of $\rho$, we have a family of valid coupling strategies $C_i(\gamma_{at(i)}, \gamma_i, \gamma_i')$. 

We can merge these coupling strategies together to create a proof of privacy for the entire SLP: 

\begin{defn}[Coupling Strategies for Straight Line Programs]
  For a initialized SLP $\rho$ of length $n$, a \textbf{valid coupling strategy} is a tuple of two functions $\bm{\gamma}(\texttt{in}\brangle{1}, \texttt{in}\brangle{2}):\RR^n\times \RR^n\to [-1, 1]^n$ and $\bm{\gamma}'(\texttt{in}\brangle{1}, \texttt{in}\brangle{2}):\RR^n\times \RR^n\to [-1, 1]^n$ 
  that produce shifts for each transition of $\rho$ for every possible pair of adjacent input sequences $\texttt{in}\brangle{1}\sim\texttt{in}\brangle{2}$ such that $\bm{\gamma}, \bm{\gamma'}$ satisfy the constraints \[
    \begin{cases}
      \gamma_i\leq\gamma_{at(i)} & c_i = \lguard[\texttt{x}]\\
      \gamma_i\geq\gamma_{at(i)} & c_i = \gguard[\texttt{x}]\\
      \gamma_i=0 & \sigma_i = \texttt{insample}\\
      \gamma_i'=0 & \sigma_i = \texttt{insample}'
    \end{cases}.
  \]
  If $\texttt{in}\brangle{1}$ and $\texttt{in}\brangle{2}$ are clear from context, we will often shorthand notating a coupling strategy as $\bm{\gamma}$ and $\bm{\gamma}'$. 
\end{defn}


\begin{lemma}\label{simplifiedMultTransitionsCouplingProof}
  For any initialized SLP $\rho = t_0\ldots t_{n-1}$, any possible output event $\sigma$ of $\rho$, and any two valid adjacent input sequences $\texttt{in}\brangle{1}\sim \texttt{in}\brangle{2}$, if we are given $2n$ real-valued ``shifts'' $\{\gamma_i, \gamma_i'\}_{i=0}^{n-1}$ that, for all $i$, satisfy the constraints \[
        \begin{cases}
          \gamma_i\leq\gamma_{at(i)} & c_i = \lguard[\texttt{x}]\\
          \gamma_i\geq\gamma_{at(i)} & c_i = \gguard[\texttt{x}]\\
          \gamma_i=0 & \sigma_i = \texttt{insample}\\
          \gamma_i'=0 & \sigma_i = \texttt{insample}'
        \end{cases},
      \]
      then we can construct an approximate lifting that proves that $\PP[\rho, \texttt{in}\brangle{1}, \sigma] \leq e^{d\varepsilon}\PP[\rho, \texttt{in}\brangle{2}, \sigma]$ for some bounded $d>0$. 
\end{lemma}


Thus, if we have a \textbf{valid} coupling strategy $C$ for an SLP $\rho$, then immediately by lemma \ref{multTransitionsCouplingProof}, we have a proof that $\rho$ is $d\varepsilon$-differentially private for some $d>0$; we call this $d$ the \textit{cost} of the coupling strategy for $\rho$. 

\begin{defn}
    For an initialized SLP $\rho$ of length $n$, the \textbf{cost} of a coupling strategy $C_\rho=(\bm{\gamma}, \bm{\gamma}')$ is \[cost(C_\rho) = \max_{\texttt{in}\brangle{1}\sim\texttt{in}\brangle{2}}\sum_{i=0}^{n-1}(|-\texttt{in}_i\brangle{1}+\texttt{in}_i\brangle{2}-\gamma_i|)d_i+(|-\texttt{in}_i\brangle{1}+\texttt{in}_i\brangle{2}-\gamma_i'|)d_i'.\]

    Additionally, let $G$ be the set of all valid coupling strategies $C_\rho=(\bm{\gamma}, \bm{\gamma}')$ for $\rho$. Then the \textbf{coupling cost} of $\rho$ is 
    \[cost(\rho) = \min_{(\bm{\gamma}, \bm{\gamma}')\in G}cost((\bm{\gamma}, \bm{\gamma}')).\]
\end{defn}

\begin{cor}\label{pathCostCor}
    If $C_\rho=(\bm{\gamma}, \bm{\gamma}')$ is a valid coupling strategy for an SLP $\rho$, then $\rho$ is $cost(C_\rho)\varepsilon$-differentially private.
\end{cor}

In particular, we observe that for any individual initialized SLP $\rho$, there always exists \textit{some} valid coupling strategy for $\rho$; so every SLP $\rho$ is always differentially private. 

Our program model fundamentally considers programs to be collections of individual SLPs; to this end, we extend the definition of privacy to sets of SLPs in the expected manner:

\begin{defn}\label{setOfPathsDPDefn}
  Let $S$ be a set of initialized SLPs and let $O$ be a set of all possible outputs of SLPs in $S$. 
  Then $S$ is $d\varepsilon$-differentially private for some $d>0$ if, for all SLPs $\rho\in S$ and outputs $\sigma\in O$, $\forall \varepsilon>0$, for all valid adjacent input sequences $\texttt{in}\brangle{1}\sim \texttt{in}\brangle{2}$, $\PP[\rho, \texttt{in}\brangle{1}, \sigma]\leq e^{d\varepsilon}\PP[\rho, \texttt{in}\brangle{2}, \sigma]$.
\end{defn}






\subsection{Syntactic structure}

We first introduce a structured program model in order to make the concept of loops more coherent. \sky{rewording needed}

We will model a structured program \sky{dif name needed} using control flow graphs; a program is represented by a directed graph $G = (V, E)$, where $V$ represents a set of program locations. 

In addition, we require that every edge in $G$ is labeled\footnote{These objects are also called control-flow automatons to distinguish them from traditional CFGs} with a transition $t = (c, \sigma, \tau)$. $G$ must contain a unique initial state $\ell_0 \in L$.

Additionally, to ensure that $G$ represents a coherent program, we require that $G = (V, E)$ satisfies the following conditions: 
\begin{itemize}
    \item \textbf{Determinism:} For all locations $\ell\in V$, if there exists an edge $(\ell, \ell')$ labeled by $t'=(c, \sigma', \tau')$, then no other edge $(\ell, \ell^*)$ labeled by a transition of the form $t^* = (c, \sigma^*, \tau^*)$ exists. 
    Additionally, if there exists an edge $(\ell, \ell')$ labeled by a transition of the form $(\texttt{true}, \sigma, \tau)$, then there does not exist another edge with source at $\ell$.
    \item \textbf{Shared Noise:} For all locations $\ell\in V$ and any two edges $(\ell, \ell')$ labeled by $t'=(c', \sigma', \tau')$ and $(\ell, \ell^*)$ labeled by $t^* = (c^*, \sigma^*, \tau^*)$, $P(t') = P(t^*)$. 
\end{itemize}

An \textbf{run} of $G$ thus takes the form of a string $\ell_0t_0\ell_1t_1\ldots t_{n-1}\ell_n$ for locations $\ell_i\in L$ and transitions $t_i$.

For any run of a structured program, we can transform it into a program path as previously defined by dropping all states from the run. For a run $r=\ell_0t_0\ell_1t_1\ldots t_{n-1}\ell_n$, we will use the function $\Psi(r) = t_0t_1\ldots t_{n-1}$ to denote this homomorphism. We observe that, because $G$ must be deterministic, $\Psi(r)$ is invertible. 

In particular, if $t_0t_1\ldots t_{n-1} = \Psi(r)$ for some run $r$ of $G = (V, E)$, then we abuse notation and define $\Psi^{-1}(t_i) = \ell_i\to\ell_{i+1}$ as a function that maps transitions to the edge that ``generated'' it. 


We may abuse notation and use a run $r$ as shorthand for the program path $\Psi(r)$ if context makes it clear which we refer to. 

We will say that $\{\Psi(r): r\text{ is a run of }G\}$ is the set of paths generated by a labeled graph $G$. 

\subsection{Loops}

In this section, we will introduce loops into our program model through the star operator.

\begin{defn}
    A looping branch $L$ is a set of complete paths generated by a structured program graph $G_L$ such that $L$ is the language described by a single union-free regular expression over a finite alphabet of transitions. 
\end{defn}

For a looping branch $L$, we will use $R_L$ to denote the minimal union-free regular expression that defines $L$. 

Intuitively, a looping branch is a single, straight-line path except that we allow for cycles to be inserted within the path. Looping branches are closely related to general ``star-dot'' or union-free regular languages and their associated 1-cycle-free-path-automata (see, for example, \cite{nagy2006union}). Indeed, the graph $G_L$, by definition, will also be the graph of a 1-cycle-free-path automaton. 

We will associate entire looping branches with a \textbf{single} coupling strategy.

\begin{defn}[Coupling strategy for a looping branch]
    Let $L$ be a looping branch generated by the graph $G_L = (V_L, E_L)$. Then a coupling strategy $C = (\gamma, \gamma')$ for a looping branch is a function $C:E_L\times\RR \times\RR\to [-1, 1]\times[-1, 1]$ that computes shifts for each transition-labeled edge in $L$ as a function of two adjacent inputs.
\end{defn}

We will sometimes call these coupling strategies ``class'' coupling strategies if necessary to distinguish them from coupling strategies for individual paths.

Observe that a class coupling strategy implicitly defines a coupling strategy for each path in $L$.

\begin{defn}[Induced Coupling Strategy]
    Given a coupling strategy $C = (\gamma, \gamma')$ for a looping branch $L$ and a specific path $\rho\in L$, the coupling strategy for $\rho=t_0\cdots t_{n-1}$ induced by $C$ is the pair of functions $\gamma_\rho, \gamma'_\rho$ such that 
    \begin{align*}
        \gamma_\rho(\texttt{in}\brangle{1}, \texttt{in}\brangle{2}) &= (\gamma(\Psi^{-1}(t_0))(\texttt{in}\brangle{1}, \texttt{in}\brangle{2}), \ldots,\gamma(\Psi^{-1}(t_{n-1}))(\texttt{in}\brangle{1}, \texttt{in}\brangle{2}) )\\
        \gamma'_\rho(\texttt{in}\brangle{1}, \texttt{in}\brangle{2}) &= (\gamma'(\Psi^{-1}(t_0))(\texttt{in}\brangle{1}, \texttt{in}\brangle{2}),\ldots,\gamma'(\Psi^{-1}(t_{n-1}))(\texttt{in}\brangle{1}, \texttt{in}\brangle{2}) )
    \end{align*}
\end{defn}

Perhaps surprisingly, we show that it is in fact optimal to only consider a single coupling strategy for an entire looping branch; not only is finding individual coupling strategies for every single path in a looping branch intractable, but it also does not lead to a better overall privacy cost. 

\begin{prop}\label{ClassCouplingStrategiesAreEnoughProp}
    If there exists a valid coupling strategy $C_\rho$ with cost $cost(C_\rho)$ for every path $\rho$ of looping branch $L$ and $\sup_{\rho\in L}cost(C_\rho)< \infty$, then there exists a valid class coupling strategy $C'$ for $L$ such that $cost(C') \leq \sup_{\rho\in L}cost(C_\rho)$. 
\end{prop}

Note that, because of the introduction of stars (i.e. cycles) to our model, it is possible for a looping branch to fail to be private for \textit{any} $d>0$; in other words, every coupling strategy for a looping branch has infinite cost. We can characterize whether or not a coupling strategy has infinite cost through another constraint:

\begin{lemma}\label{finiteCostConstraintLemma}
    For a looping branch $L$ generated by $G_L$, a valid coupling strategy $C = (\mathbf{\gamma}, \mathbf{\gamma}')$ has finite cost $cost(C)<\infty$ if and only if the following constraint applies for all $i$:
    \begin{itemize}
        \item If $t_i$ is contained within a star in $R_L$ (i.e. $\Psi^{-1}(t_i)$ is in a cycle in $G_L$), then $\gamma_i = -\texttt{in}\brangle{1}_i+\texttt{in}\brangle{2}_i$ and $\gamma_i' = -\texttt{in}\brangle{1}_i+\texttt{in}\brangle{2}_i$.
    \end{itemize}
\end{lemma}

Intuitively, a finite-cost coupling strategy must assign shifts such that every cycle transition has 0 privacy cost. 

In particular, we can combine this constraint that gives us \textit{finite cost} class coupling strategies with the four constraints that ensure that coupling strategies are valid.

\begin{defn}[Privacy Constraint System]\label{privacyConstraintSystem}
    Let $L$ be a looping branch generated by the graph $G_L$. If, for a candidate coupling strategy $C_L = (\gamma, \gamma')$ for $L$, the following constraints are satisfied for all $i$: \begin{enumerate}
        \item If $c_i = \lguard[\texttt{x}]$, then $\gamma_i\leq\gamma_{at(i)}$
        \item If $c_i = \gguard[\texttt{x}]$, then $\gamma_i\geq\gamma_{at(i)}$
        \item If $\sigma_i = \texttt{insample}$, then $\gamma_i=0$
        \item If $\sigma_i = \texttt{insample}'$, then $\gamma_i'=0$
        \item If $\Psi^{-1}(t_i)$ is in a cycle in $G_L$, then $\gamma_i = -\texttt{in}\brangle{1}_i+\texttt{in}\brangle{2}_i$
        \item If $\Psi^{-1}(t_i)$ is in a cycle in $G_L$, then $\gamma_i' = -\texttt{in}\brangle{1}_i+\texttt{in}\brangle{2}_i$
    \end{enumerate}
    then we say that $C$ satisfies the privacy constraint system for $L$. 
\end{defn}

As expected, if there exists a solution to the privacy constraint system for a looping branch $L$, then $L$ is private.

\begin{prop}\label{privacyFiniteCostProp}
    If there exists a coupling strategy $C$ for a looping branch $L$ that satisfies the privacy constraint system, then there exists a finite $d>0$ such that $L$ is $d\varepsilon$-differentially private. 
\end{prop}

Perhaps surprisingly, we will also later show that, under certain conditions, the privacy constraint system is complete; that is, if there does not exist a solution to the privacy constraint system, we can prove that there also does not exist any finite $d>0$ such that $L$ is $d\varepsilon$-differentially private. 


\subsection{Programs}

We now introduce our complete program model. 

\begin{defn}
    A program $P$ is a finite union of looping branches over a valid finite alphabet of transitions. 
\end{defn}

Equivalently, $P$ is a language described by a regular expression in union normal form such that each individual term describes a looping branch. 

As illustrated by branching programs, we cannot do any better with regards to coupling strategies than simply choosing one coupling strategy per looping branch. 

\begin{lemma}
    If, for every looping branch $L$ in $P$, there exists a valid coupling strategy $C_L$, then $P$ is $(\max_{L\subseteq P} cost(C_L))\varepsilon$-differentially private. 
\end{lemma}

\subsection{Deciding Privacy}

In this section, we discuss the boolean or decision problem of privacy; that is, deciding whether or not there exists \textit{any} finite $d>0$ such that a program is $d\varepsilon$-differentially private. 

Clearly, we can algorithmically show that at least some subset of differentially private programs are private through the use of couplings and coupling strategies. 

\begin{lemma}
    If, for every looping branch $L\subseteq P$ in a program $P$, there exists a coupling strategy $C_L$ that satisfies the privacy constraint system, then there exists some finite $d>0$ such that $P$ is $d\varepsilon$-differentially private.    
\end{lemma}
\begin{proof}
    Follows immediately from proposition \ref{privacyFiniteCostProp}.
\end{proof}

As previously mentioned, we show that coupling proofs are \textbf{complete} for programs of this form; every differentially private program can be proved to be private using couplings. 

\begin{lemma}\label{ProgramCounterexampleLemma}
    If, for some looping branch $L\subseteq P$ in a program $P$, there does not exist a coupling strategy $C_L$ that satisfies the privacy constraint system, then there does not exist any finite $d>0$ such that $P$ is $d\varepsilon$-differentially private.
\end{lemma}

To complete the proof, we introduce a previously analyzed program model known as DiPA, which we claim captures the same class of programs as our program model. 

\subsection{DiPA}

We now discuss a previously defined program model, DiPA, which turns out to be exactly equivalent to our own program model. The equivalence between DiPA and programs allows us to construct our completeness proof. 

\begin{defn}[\cite{chadhaLinearTimeDecidability2021}]
    A Differentially Private Automaton (DiPA) $A$ is an 8-tuple $(Q, \Sigma, C, \Gamma, q_{init}, X, P, \delta)$ where
    \begin{itemize}
        \item $Q$ is a finite set of locations partitioned into input locations $Q_{in}$ and non-input locations $Q_{non}$. 
        \item $\Sigma = \RR$ is the input alphabet
        \item $C = \{\texttt{true}, \lguard, \gguard\}$ is a set of guard conditions
        \item $\Gamma$ is a finite output alphabet
        \item $q_{init}\in Q$ is the initial location
        \item $X = \{\texttt{x}, \texttt{insample}, \texttt{insample}'\}$ is a set of variables
        \item $P: Q\to \QQ\times \QQ^{\geq 0}\times \QQ\times  \QQ^{\geq 0}$ is a parameter function that assigns sampling parameters for the Laplace distribution for each location
        \item $\delta:(Q\times C)\to (Q\times (\Gamma \cup \{\texttt{insample}, \texttt{insample}'\})\times \{\texttt{true}, \texttt{false}\})$ is a partial transition function. 
    \end{itemize}
    In addition, $\delta$ must satisfy some additional conditions:
    \begin{itemize}
        \item \textbf{Determinism:} For any location $q\in Q$, if $\delta(q,\texttt{true})$ is defined, then $\delta(q,\lguard)$ and $\delta(q,\gguard)$ are not defined. 

        \item \textbf{Output Distinction:} For any location $q\in Q$, if $\delta(q, \gguard) = (q_1, o_1, b_1)$ and $\delta(q, \lguard) = (q_2, o_2, b_2)$, then $o_1\neq o_2$ and at least one of $o_1\in \Gamma$ and $o_2\in \Gamma$ is true.

        \item \textbf{Initialization:} The initial location $q_0$ has only one outgoing transition of the form $\delta(q_0, \texttt{true}) = (q, o, \texttt{true})$.

        \item \textbf{Non-input transition:} From any $q\in Q_{non}$, if $\delta(q, c)$ is defined, then $c=\texttt{true}$.
    \end{itemize}
\end{defn}

A DiPA operates as follows: 
\begin{itemize}
    \item At each location, a real-valued input $\texttt{in}$ is read in and two variables $\texttt{insample}\sim \Lap(\texttt{in}, d\varepsilon)$ and $\texttt{insample}'\sim\Lap(\texttt{in}, d\varepsilon)$ are sampled.
    \item $\texttt{insample}$ is compared to the stored variable $\texttt{x}$, and depending on the guards of the transitions out of the current location, changes the current location and outputs a value. This value can either be $\texttt{insample}, \texttt{insample}'$, or a symbol from $\Gamma$.
    \item Finally, the  value of $\texttt{x}$ is optionally updated with the value of $\texttt{insample}$.
\end{itemize}

Just like with programs, a path $\rho$ in a DiPA $A$ reads a real-valued input sequence $\texttt{in}\in \RR^n$; the same definitions of adjacency and validity apply to DiPA input sequences.

We first establish notation for discussing the probabilities of different paths in a DiPA, which allows us to define $d\varepsilon$-differential privacy. 

\begin{defn} 
    Let $\rho$ be a path in a DiPA $A$, let $\texttt{in}$ be a valid input sequence and let $o$ be a possible output of $\rho$. In particular, if $\sigma_i\in \{\texttt{insample}, \texttt{insample}'\}$, then we require that $o_i$ is an \textit{interval} $(a, b)\subseteq \RR$, rather than simply a measurable set as before. 
    Then $\texttt{Pr}[x, \rho, \texttt{in}, o]$ is the probability of $\rho$ being taken with input sequence $\texttt{in}$ and outputting $o$. If the first location in $\rho$ is $q_{init}$, then $\texttt{Pr}[x, \rho, \texttt{in}, o]$ may be shortened to $\texttt{Pr}[\rho, \texttt{in}, o]$, since the initial value of $\texttt{x}$ is irrelevant.
\end{defn}

For a full definition of DiPA semantics, we refer back to the original work. 

\begin{defn}
    A DiPA $A$ is $d\varepsilon$-differentially private for some $d>0$ if for all paths $\rho$ in $A$, for all possible outputs $o$ of $\rho$ and valid adjacent input sequences $\texttt{in}\brangle{1}\sim \texttt{in}\brangle{2}$, \[
        \PP[\rho, \texttt{in}\brangle{1}, o]\leq e^{d\varepsilon} \PP[\rho, \texttt{in}\brangle{2}, o]
    \]
\end{defn}

Notably, we show an equivalence between programs and DiPAs.

\begin{prop}
    Every path $\rho$ through a DiPA $A$ is represented by a complete path $\hat{\rho}$ comprised of transitions from a valid transition alphabet $\Sigma_T$; further, the set of all possible paths through $A$ is a regular language over $\Sigma_T$.
\end{prop}
\begin{proof}
    Let $\rho = q_0\to q_1\to\ldots \to q_n$ be a path in a DiPA $A=(Q, \RR, C, \Gamma, q_0, X, P, \delta)$. 

    For all $i\in 0\ldots n-1$, there must be some $c_i$ such that $\delta(q_i, c_i) = (q_{i+1}, \sigma_i, \tau_i)$. Let $t_i = (q_i, q_{i+1}, c_i, \sigma_i, \tau_i)$ and let $\Sigma_\rho = \{t_i: i\in 0\ldots n-1\}$ be the set of all such transitions. Note that because $\delta$ satisfies the conditions of determinism, output distinction, initialization, and non-input transition, $\Sigma_\rho$ must as well. 
    Then let $\hat{\rho} = t_0\cdot t_1\cdot \ldots\cdot t_{i-1}$ be the representation of $\rho$ as a word over $\Sigma_\rho$. 

    Let $\Sigma_T = \bigcup_{\rho\in A}\Sigma_\rho$. Note that $\Sigma_T$ must have finite size because $A$ is a finite automaton and must still be a valid transition alphabet.

    Let $D = (Q, \Sigma_T, \delta_D, q_0, F=Q)$ be an NFA defined over the set of program locations $Q$ such that $\delta_D$ is defined as follows:
    Let $q\in Q$ be an arbitrary location. If $\delta(q, c) = (q', \sigma, \tau)$ is defined for some $c\in C$, let $\delta_D(q, (q, q', c, \sigma, \tau)) = q'$.

    Then clearly every path in $A$ is also a path in $D$ and vice versa; since every location in $D$ is an accepting location, $\mathcal{L}(D) = \{\hat{\rho}: \rho\in A\}$. Thus, the set of (representations of) all paths in $A$ must be a regular language. 
\end{proof}

\begin{prop}
    For every program $P$ over a valid transition alphabet $\Sigma_T$, there exists a corresponding DiPA $A$ such that there exists a path $\rho$ in $A$ if and only if its representation is in $P$. 
\end{prop}
\begin{proof}
    The DiPA can be directly constructed from $P$.
\end{proof}

As expected, the probability of a path ``succeeding'' in a program is the same as the probability of a path being traversed in a DiPA. 

\begin{prop}
    For all paths $\rho$ in a DiPA $A$ and for all input sequences $\texttt{in}$ and possible outputs $\sigma$ of $\rho$, $\PP[\rho, \texttt{in}, \sigma] = \PP[\hat{\rho}, \texttt{in}, \sigma]$.
\end{prop}


Interestingly, the privacy of any DiPA $A$ is completely characterized by four graph-theoretic structures---we refer to the original work for full definitions.

\begin{thm}[\cite{chadhaLinearTimeDecidability2021}]\label{DiPACounterexamplesThm}
    A DiPA $A$ does not have a leaking cycle, leaking pair, disclosing cycle, or privacy violating path if and only if there exists some $d>0$ such that for all $\varepsilon>0$, $A$ is $d\varepsilon$-differentially private. 
\end{thm}

This provides us with a method of demonstrating that a program is not $d\varepsilon$-differentially private for any $d>0$.

\begin{cor}
    If the corresponding DiPA $A$ to a program $P$ contains a leaking cycle, leaking pair, disclosing cycle, or privacy violating path, then does not exist a finite $d>0$ such that $P$ is $d\varepsilon$-differentially private. 
\end{cor}

Our key result relates the privacy constraint system and these graph-theoretic structures.

\begin{lemma}
    If, for a looping branch $L$, there does not exist a coupling strategy $C$ that satisfies the privacy constraint system for $L$, then the corresponding DiPA to $L$ must contain either a leaking cycle, a leaking pair, a disclosing cycle, or a privacy violating path. 
\end{lemma}


\subsection{A Linear-Time Algorithm for Deciding Privacy}\label{svDecisionAlgSection}

We also provide an efficient (i.e. linear time) algorithm for solving the decision problem of privacy.
Because the privacy constraint system for a periodic program completely characterizes its privacy, our goal is to efficiently decide if the system is satisfiable for individual periodic programs.

We begin with an algorithm for a single periodic program that encodes each constraint of the privacy constraint system in a graph which preserves some of the structure of the periodic program. 

\begin{defn}[Privacy constraint graph]
    Let $L$ be a periodic program generated by $G$ that contains $n$ distinct transitions $t_i$. The \textbf{privacy constraint graph} $\mathcal{P}_L = (V, E)$ of $L$ is a directed graph where: 
    \begin{itemize}
        \item $V = \{{\bf 1, -1}\} \cup \{v_i: t_i\in L\}$. Each $v_i \in V$ represents the shift $\gamma_i$ on $t_i$.
        \item For all $(t_i, t_j)$ such that the constraint $\gamma_i \leq \gamma_j$ is in the approximate privacy constraint system of $L$, $(v_i, v_j) \in E$.
        \item There are nodes ${\bf 1, -1}\ \in V$ such that: 
        \begin{itemize}
            \item $({\bf -1}, v)\in E$ for all $v \in V$.
            \item $(v, {\bf 1})\in E$ for all $v \in V$.
            \item For every transition $t_i = (c_i, \sigma_i, \tau_i)$ such that $\Psi^{-1}(t_i)$ is in a cycle in $G$ and $c_i = \lguard[\texttt{x}]$, $({\bf 1}, v_i) \in E$.
            \item For every transition $t_i = (c_i, \sigma_i, \tau_i)$ such that $\Psi^{-1}(t_i)$ is in a cycle in $G$ and $c_i = \gguard[\texttt{x}]$, $(v_i, {\bf -1}) \in E$. 
        \end{itemize}
    \end{itemize}
\end{defn}

As suggested by the definition, the existence of an edge $(v_i, v_j)\in E$ corresponds to a constraint forcing $\gamma_i \leq \gamma_j$. Additionally, it can be shown that certain shifts are ``forced'' to be either $1$ or $-1$; leading to edges to and from $\mathbf{1}$ and $\mathbf{-1}$. 
A path from $\mathbf{1}$ to $\mathbf{-1}$ thus implies that the constraint system must be unsatisfiable; we show that this is in fact a complete decision procedure.

\begin{prop}\label{privacyConstraintGraphProp}
    A periodic program $L$ is differentially private if and only if there does not exist a path from $\bf 1$ to $\bf -1$ in the privacy constraint graph of $L$.
\end{prop}

Fortunately, we can directly combine the privacy constraint graphs for individual periodic programs together to produce a privacy constraint graph for a program, which provides the linear time algorithm that we desired.

\begin{prop}\label{programPrivacyConstraintGraphPathReq}
    A program $P$ is differentially private if and only if there does not exist a path from $\bf 1$ to $\bf -1$ in the privacy constraint graph $\mathcal{P}_P$ of $P$, where $\mathcal{P}_P$ is the union of the privacy constraint graphs $\mathcal{P}_L$ of each periodic program $L\subseteq P$.
\end{prop}

\begin{thm}\label{LinearTimeDecidingPrograms}
    Given a program $P$ generated by a graph $G$, we can decide if $P$ is differentially private in linear time in the size of $G$.
\end{thm}

\begin{proof}
    Constructing the privacy constraint graph $\mathcal{P}_P$ takes linear time in the size of $P$, on which we can perform a breadth first search to check whether there is a path from $\bf 1$ to $\bf -1$. This can be done in linear time in the size of $\mathcal{P}_P$, which itself is linear in the size of $G$.
\end{proof}

We observe that we match the asymptotic runtime obtained by Chadha, Sistla, and Viswanathan~\cite{chadhaLinearTimeDecidability2021} for deciding the privacy of programs under this model. 




\section{Optimizing Privacy Cost with Couplings}


\subsection{Minimizing a privacy budget}

If we have a differentially private program, we'd also like to optimize its privacy cost. Because we must have separate coupling strategies for every looping branch of a program, we optimize the privacy cost of each looping branch independently.

\begin{prop}
    \label{prop:compute_opt_cost}
    Let $L$ be a looping branch in a program $P$ consisting of $n$ transitions. The cost $opt(L)$ of the optimal coupling strategy for $L$ can be computed by the following optimization problem: 
    \sky{should this be written without the $\Delta$? it needs to be explained at least}
    \begin{align*}
        opt(L) = \max_{\Delta \in [-1, 1]^n} &\min_{\gamma, \gamma' \in [-1, 1]^n} \sum_{i = 1}^n \left(|\Delta_i - \gamma_i| d_i + |\Delta_i - \gamma_i'|d_i' \right)\\ 
            \text{subject to }
            &\ \gamma_{at(i)} \leq \gamma_i \text{ if } c_i = \gguard, \\
            &\ \gamma_{at(i)} \geq \gamma_i \text{ if } c_i = \lguard, \\
            &\ \gamma_i = 0 \text{ if } \sigma_i = \texttt{insample}, \\
            &\ \gamma_i' = 0 \text{ if } \sigma_i = \texttt{insample}'\\
            &\ \gamma_i = \gamma_i'= \Delta_i \text{ if } t_i \text{ is in a cycle}
    \end{align*}
    If a looping branch $L$ is not differentially private, a solution to this optimization problem does not exist, and we write $opt(L) = \infty$.
\end{prop}

\begin{proof}
    Assume that $L$ is differentially private. Then there exists a valid coupling strategy for $L$ with finite cost, showing that the constraints above are feasible, and a finite solution to the optimization problem exists.  

    We will specify a valid coupling strategy $C^* = (\gamma^*, {\gamma'}^*)$ for $L$ with the cost stated above, and then show it is optimal. Define $\Gamma: [-1, 1]^n \to [-1, 1]^n \times [-1, 1]^n$ as follows, where $\Gamma(\Delta)$ is a pair $(\gamma, \gamma')$: 
    \begin{align*}
        \Gamma(\Delta) = &\argmin_{\gamma, \gamma' \in [-1, 1]^n} \sum_{i = 1}^n \left(|\Delta_i - \gamma_i| d_i + |\Delta_i - \gamma_i'|d_i' \right)\\ 
        \text{subject to }
        &\ \gamma_{at(i)} \leq \gamma_i \text{ if } c_i = \gguard, \\
        &\ \gamma_{at(i)} \geq \gamma_i \text{ if } c_i = \lguard, \\
        &\ \gamma_i = 0 \text{ if } \sigma_i = \texttt{insample}, \\
        &\ \gamma_i' = 0 \text{ if } \sigma_i = \texttt{insample}'\\
        &\ \gamma_i = \gamma_i'= \Delta_i \text{ if } t_i \text{ is in a cycle}
    \end{align*}
    Then define \[(\gamma^*(\texttt{in}\brangle{1}, \texttt{in}\brangle{2}), {\gamma'}^*(\texttt{in}\brangle{1}, \texttt{in}\brangle{2})) = \Gamma(\texttt{in}\brangle{1} - \texttt{in}\brangle{2})\]

    Notice the following: 

    \begin{enumerate}
        \item $C^*$ is a valid coupling strategy for $L$, since the privacy constraints on $\gamma^*$ and ${\gamma'}^*$ are satisfied by construction.
        \item $C^*$ has the cost given by the solution to the optimization problem, since 
        \begin{align*}
            cost(C^*) &= \max_{\texttt{in}\brangle{1}\sim \texttt{in}\brangle{2}} \sum_{i = 1}^n  |\texttt{in}_i\brangle{1} - \texttt{in}_i\brangle{2} - \gamma_i^*(\texttt{in}\brangle{1}, \texttt{in}\brangle{2})| d_i \\ 
            \phantom{cost(C^*)} &\phantom{=\max_{\texttt{in}\brangle{1}\sim \texttt{in}\brangle{2}}\qquad } + |\texttt{in}_i\brangle{1} - \texttt{in}_i\brangle{2} - {\gamma'}_i^*(\texttt{in}\brangle{1}, \texttt{in}\brangle{2})|d_i' \\
            &= \max_{\texttt{in}\brangle{1}\sim \texttt{in}\brangle{2}} \sum_{i = 1}^n  |\texttt{in}_i\brangle{1} - \texttt{in}_i\brangle{2} - \Gamma_1(\texttt{in}\brangle{1} - \texttt{in}\brangle{2})_i| d_i \\ 
            \phantom{cost(C^*)} &\phantom{=\max_{\texttt{in}\brangle{1}\sim \texttt{in}\brangle{2}}\qquad } + |\texttt{in}_i\brangle{1} - \texttt{in}_i\brangle{2} - \Gamma_2(\texttt{in}\brangle{1} - \texttt{in}\brangle{2})_i|d_i' \\
            &= \max_{\Delta \in [-1, 1]^n} \sum_{i = 1}^n  |\Delta_i - \Gamma_1(\Delta)_i| d_i + |\Delta_i - \Gamma_2(\Delta)_i|d_i' \\
            &= \max_{\Delta \in [-1, 1]^n} \min_{\gamma, \gamma' \in [-1, 1]^n} \sum_{i = 1}^n  |\Delta_i - \gamma_i| d_i + |\Delta_i - \gamma_i'|d_i'
        \end{align*}
        \item $C^*$ is optimal, since for any valid coupling strategy $C = (\delta, \delta')$ for $L$, we have
        \begin{align*}
            cost(C) &= \max_{\texttt{in}\brangle{1}\sim \texttt{in}\brangle{2}} \sum_{i = 1}^n  |\texttt{in}_i\brangle{1} - \texttt{in}_i\brangle{2} - \delta_i(\texttt{in}\brangle{1}, \texttt{in}\brangle{2})| d_i \\
            \phantom{cost(C)} &\phantom{=\max_{\texttt{in}\brangle{1}\sim \texttt{in}\brangle{2}}\qquad } + |\texttt{in}_i\brangle{1} - \texttt{in}_i\brangle{2} - \delta_i'(\texttt{in}\brangle{1}, \texttt{in}\brangle{2})|d_i' \\
            &\geq \max_{\Delta \in [-1, 1]^n} \sum_{i = 1}^n  |\Delta_i - \delta_i(0, \Delta_i)| d_i + |\Delta_i - \delta_i'(0, \Delta_i)|d_i' \\
            &\geq \max_{\Delta \in [-1, 1]^n} \min_{\gamma, \gamma' \in [-1, 1]^n} \sum_{i = 1}^n \left(|\Delta_i - \gamma_i| d_i + |\Delta_i - \gamma_i'|d_i' \right)\\
            &= cost(C^*)
        \end{align*}
    \end{enumerate}
    which shows that the optimization problem computes the optimal cost of a coupling strategy for $L$ that satisfies the privacy constraints.
\end{proof}

We observe that the inner problem is convex, and so the outer problem is that of convex maximization. 

In the absence of an immediate polynomial time algorithm for solving this optimization problem, we analyze an approximation of the problem. 

\begin{defn}
    Define the approximate privacy cost of a differentially private looping branch $L$ to be as follows. Let $I$ be the set of transitions in $L$ that do $\textit{not}$ appear in a cycle.  
    \begin{align*} 
        approx(L) = &\sum_{t_i \text{outputs \texttt{insample}}} d_i' + \min_{\gamma \in [-1, 1]^n} \sum_{i \in I} \left(1 + |\gamma_i| \right) d_i  \\
            \text{subject to } 
            &\ \gamma_{at(i)} \leq \gamma_i \text{ if } c_i = \gguard, \\
            &\ \gamma_{at(i)} \geq \gamma_i \text{ if } c_i = \lguard, \\
            &\ \gamma_i = 0 \text{ if } \sigma_i = \texttt{insample}, \\
            &\ \gamma_i = 1 \text{ if } t_i \text{ is in a cycle and has } c_i = \lguard,\\ 
            &\ \gamma_i = -1 \text{ if } t_i \text{ is in a cycle and has } c_i = \gguard
    \end{align*}
    which is the cost of a coupling strategy in which $\gamma$ and $\gamma'$ are constant with respect to $\texttt{in}\brangle{1}$ and $\texttt{in}\brangle{2}$ on non-cyclic transitions. 
\end{defn}

Solving this approximation still provides us with a valid coupling strategy.

\begin{prop}
    \label{prop:approx_exists}
    Given a differentially private looping branch $L$, there exists a valid coupling strategy $C_L$ for $L$ such that $cost(C_L) = approx(L)$.

    Moreover, if $\gamma \in [-1, 1]^n$ satisfies the approximate privacy constraints, then there is a valid coupling strategy $C_L = (\gamma^*, {\gamma'}^*)$ for $L$ such that $\gamma_i^*(\texttt{in}\brangle{1}, \texttt{in}\brangle{2}) = \gamma_i$ for all $t_i$ that do not appear in cycles. 
\end{prop}


\begin{proof}
    Let 
    \[\gamma = \argmin_{\gamma \in [-1, 1]^n} \sum_{i \in I} \left(1 + |\gamma_i| \right) d_i\]
    subject to the constraints above. Define $C_L = (\gamma^*, {\gamma'}^*)$ where
    \begin{align*}
        \gamma_i^*(\texttt{in}\brangle{1}, \texttt{in}\brangle{2}) &= \begin{cases}
            \texttt{in}\brangle{1}_i - \texttt{in}\brangle{2}_i &\text{ if } t_i \text{ is in a cycle} \\
            \gamma_i &\text{ otherwise}
        \end{cases} \\[1em]
        {\gamma'}_i^*(\texttt{in}\brangle{1}, \texttt{in}\brangle{2}) &= \begin{cases}
            0 &\text{ if } t_i \text{ outputs \texttt{insample}} \\
            \texttt{in}\brangle{1}_i - \texttt{in}\brangle{2}_i &\text{ otherwise}
        \end{cases}
    \end{align*}
    Notice the following: 

    \begin{itemize}
        \item $C_L$ satisfies the privacy constraints, and so is valid.
        
        If $t_i$ is in a cycle with $c_i = \lguard$, then the constraints on $\gamma$ require that $\gamma_i = 1$, and so $1 = \gamma_i \leq \gamma_{at(i)} = 1$. As a result, we will satisfy the privacy constraint $\gamma_{at(i)}^* \geq \gamma_i^*$: 
        \[\gamma_i^* = \texttt{in}\brangle{1}_i - \texttt{in}\brangle{2}_i \leq 1 = \gamma_{at(i)}^*\]
        A similar argument holds for if $t_i$ is in a cycle with $c_i = \gguard$.

        All other privacy constraints are satisfied by construction.

        \item $C_L$ has the cost given by the solution to the optimization problem, since
        
        \begin{align*}
            cost(C_L) &= \max_{\texttt{in}\brangle{1}\sim \texttt{in}\brangle{2}} \sum_{i = 1}^n  |\texttt{in}_i\brangle{1} - \texttt{in}_i\brangle{2} - \gamma_i^*(\texttt{in}\brangle{1}, \texttt{in}\brangle{2})| d_i \\ 
            \phantom{cost(C_L)} &\phantom{=\max_{\texttt{in}\brangle{1}\sim \texttt{in}\brangle{2}}\qquad } + |\texttt{in}_i\brangle{1} - \texttt{in}_i\brangle{2} - {\gamma'}_i^*(\texttt{in}\brangle{1}, \texttt{in}\brangle{2})|d_i' \\
            &= \max_{\texttt{in}\brangle{1}\sim \texttt{in}\brangle{2}} \left(\sum_{i \in I} |\texttt{in}_i\brangle{1} - \texttt{in}_i\brangle{2} - \gamma_i^*(\texttt{in}\brangle{1}, \texttt{in}\brangle{2})| d_i\right) \\
            \phantom{cost(C_L)} &\phantom{=\max_{\texttt{in}\brangle{1}\sim \texttt{in}\brangle{2}}\qquad } + \left(\sum_{t_i \text{outputs \texttt{insample}}}|\texttt{in}_i\brangle{1} - \texttt{in}_i\brangle{2} - {\gamma'}_i^*(\texttt{in}\brangle{1}, \texttt{in}\brangle{2})|d_i'\right)\\
            &= \max_{\texttt{in}\brangle{1}\sim \texttt{in}\brangle{2}} \left(\sum_{i \in I} |\texttt{in}_i\brangle{1} - \texttt{in}_i\brangle{2} - \gamma_i| d_i\right) \\
            \phantom{cost(C_L)} &\phantom{=\max_{\texttt{in}\brangle{1}\sim \texttt{in}\brangle{2}}\qquad } + \left(\sum_{t_i \text{outputs \texttt{insample}}}|\texttt{in}_i\brangle{1} - \texttt{in}_i\brangle{2}|d_i'\right)\\
            &= \sum_{i \in I} (1 + |\gamma_i|) d_i + \sum_{t_i \text{outputs \texttt{insample}}} d_i' \\
            &= approx(L)
        \end{align*}
    \end{itemize}
\end{proof}

Importantly, the approximate LP is solvable in polynomial time through an application of the ellipsoid method.


\begin{prop}
    The approximate privacy cost of $L$ can be computed in time polynomial in $n$, the number of transitions in $L$.
\end{prop}

\begin{proof}
    To compute the solution to the minimization problem, we can set up the following linear program: 
    \begin{align*}
        \min_{\gamma, A_i \in [-1, 1]^n} &\sum_{i = 1}^n \left(1 + A_i \right) d_i \\ 
            \text{subject to } 
            &\ \gamma_{at(i)} \leq \gamma_i \text{ if } c_i = \lguard, \\
            &\ \gamma_{at(i)} \geq \gamma_i \text{ if } c_i = \gguard, \\
            &\ \gamma_i = 0 \text{ if } \sigma_i = \texttt{insample}, \\
            &\ \gamma_i = 1 \text{ if } t_i \text{ is in a cycle and has } c_i = \lguard,\\ 
            &\ \gamma_i = -1 \text{ if } t_i \text{ is in a cycle and has } c_i = \gguard,\\
            &\ \gamma_i \leq A_i, -\gamma_i \leq A_i \text{ for all } i \in \{1, \dots, n\} 
    \end{align*}
    This program can be solved using the ellipsoid method in polynomial time.
\end{proof}

Further, we obtain an approximation factor bounded by a term linear in the number of non-cyclic transitions in $L$. 

\begin{prop}
    \label{prop:approx_opt_are_close}
    For a looping branch $L$ with $n$ distinct transitions, we have 
    \[opt(L) \leq approx(L) \leq opt(L) + \sum_{i \in I }^n d_i + \sum_{t_i \text{outputs \texttt{insample}}} d_i'\]
    where $I$ is the set of transitions in $L$ that do $\textit{not}$ appear in a cycle.
\end{prop}

\begin{proof}
    We have $opt(L) \leq approx(L)$ by Proposition \ref{prop:compute_opt_cost}. Let $I$ be the set of transitions in $L$ that do $\textit{not}$ appear in a cycle. Then we have
    \begin{align*}
        opt(L) &= \max_{\Delta \in [-1, 1]^n} \min_{\gamma, \gamma' \in [-1, 1]^n} \sum_{i = 1}^n \left(|\Delta_i - \gamma_i| d_i + |\Delta_i - \gamma_i'|d_i' \right)\\
        &= \max_{\Delta \in [-1, 1]^n} \min_{\gamma, \gamma' \in [-1, 1]^n} \sum_{i \in I} \left(|\Delta_i - \gamma_i| d_i + |\Delta_i - \gamma_i'|d_i' \right)\\
        &= \max_{\Delta \in [-1, 1]^n} \min_{\gamma, \gamma' \in [-1, 1]^n} \left(\sum_{i \in I} \left(|\Delta_i - \gamma_i| d_i \right) + \sum_{t_i \text{outputs \texttt{insample}}} |\Delta_i| d_i' \right)\\
        &\geq \max_{\Delta \in [-1, 1]^n} \min_{\gamma, \gamma' \in [-1, 1]^n} \left(\sum_{i \in I} \left(|\gamma_i| - |\Delta_i| \right) d_i  + \sum_{t_i \text{outputs \texttt{insample}}} |\Delta_i| d_i' \right)\\
        &= \max_{\Delta \in [-1, 1]^n} \left(- \sum_{i \in I} |\Delta_i| d_i + \sum_{t_i \text{outputs \texttt{insample}}} |\Delta_i| d_i' + \min_{\gamma, \gamma' \in [-1, 1]^n} \sum_{i \in I}|\gamma_i| d_i \right)\\
        &\geq \min_{\gamma, \gamma' \in [-1, 1]^n} \sum_{i \in I}|\gamma_i| d_i \\
        &= approx(L) - \sum_{t_i \text{outputs \texttt{insample}}} d_i' - \sum_{i \in I} d_i'
    \end{align*}
    showing the second inequality.
\end{proof}

We conjecture that the optimal coupling cost, as represented by the un-approximated LP, exactly matches the ``true'' privacy cost of a program. In this case, coupling proofs would truly be ``complete'' for this program model. 

\begin{conj}
    The optimal coupling cost is the ``true'' privacy cost of a looping branch. That is, 
    \begin{align*}
        opt(L) = \sup_{\rho \in L} \sup_{\texttt{in}\brangle{1} \sim \texttt{in}\brangle{2}} D_\infty(\PP[\rho, \texttt{in}\brangle{1}, o]\; ||\; \PP[\rho, \texttt{in}\brangle{2}, o])
    \end{align*}
    representing the worst-case privacy loss over all possible paths in $L$ and all possible adjacent inputs.
\end{conj}

\subsection{Deciding Privacy}

We now turn to algorithms for the decision problem of privacy.

We can check whether a looping branch is differentially private by checking whether the approximate privacy cost of the looping branch is finite. This is true if and only if the \textit{approximate privacy constraints} are feasible.

We will give an algorithm to check whether the approximate privacy constraints are feasible. Although this is equivalent to solving a 2SAT instance, we will encode these constraints in a graph which preserves some of the structure of our program. 

\begin{defn}[Privacy constraint graph]
    Let $L$ be a looping branch in a program $P$. The \textbf{privacy constraint graph} $G_L = (V, E)$ of $L$ is a directed graph where: 
    \begin{itemize}
        \item For every transition $t_i$ in $L$, there is a vertex $v_i \in V$ representing the shift $\gamma_i$ on $t_i$.
        \item For every pair of transitions $(t_i, t_j)$ which have the privacy constraint $\gamma_i \leq \gamma_j$, there is an edge $(v_i, v_j) \in E$.
        \item There are nodes ${\bf 1, -1}\ \in V$ such that: 
        \begin{itemize}
            \item The edges $({\bf -1}, v)$ exist for all $v \in V$.
            \item The edges $(v, {\bf 1})$ exist for all $v \in V$.
            \item For every transition $t_i$ with the privacy constraint $\gamma_i = 1$, there is an edge $({\bf 1}, v_i) \in E$.
            \item For every transition $t_i$ with the privacy constraint $\gamma_i = -1$, there is an edge $(v_i, {\bf -1}) \in E$. 
        \end{itemize}
    \end{itemize}

\end{defn}

\begin{prop}
    A looping branch $L$ is differentially private if and only if there does not exist a path from $\bf 1$ to $\bf -1$ in the privacy constraint graph of $L$.
\end{prop}

\begin{proof}
    $(\implies)$ Let $L$ be differentially private. Then $approx(L) < \infty$, and so there exists $\gamma \in [-1, 1]^n$ such that the approximate privacy constraints are satisfied by $\gamma$. Aiming for a contradiction, assume that there exists a path ${\bf 1} \to v_{i_1} \to \dots \to v_{i_k} \to {\bf -1}$ in the privacy constraint graph of $L$. This corresponds to the sequence of privacy constraint inequalities
    \[1 \leq \gamma_{i_1} \leq \dots \leq \gamma_{i_k} \leq -1\]
    which is a contradiction, showing that no such $\gamma$ could exist. Therefore, there is no path from $\bf 1$ to $\bf -1$ in the privacy constraint graph of $L$.

    $(\impliedby)$ Let there exist no path from $\bf 1$ to $\bf -1$ in the privacy constraint graph of $L$. Define 
    \begin{align*}
        \gamma_i = \begin{cases}
            1 &\text{ if there exists a path from } {\bf 1} \text{ to } v_i \text{ in } G_L \\
            -1 &\text{ otherwise}
        \end{cases}
    \end{align*}
    We claim that the approximate privacy constraints are satisfied by $\gamma$. \vishnu{Need to do a better job of stating what approximate privacy constraints are and why they only depend on $\gamma \in [-1, 1]^n$ and not some function of $\texttt{in}\brangle{1}$ and $\texttt{in}\brangle{2}$}.
    
    \begin{itemize}
        \item Consider the approximate privacy constraint $\gamma_i \leq \gamma_j$. This corresponds to the edge $(v_i, v_j)$ in $G_L$. If $\gamma_i = 1$, there is a path from $\bf 1$ to $v_i$, and so there is a path from $\bf 1$ to $v_j$, and so $\gamma_j = 1$, satisfying the constraint. If $\gamma_i = -1$, then any assignment of $\gamma_j$ satisfies the constraint. 
        \item Consider the constraint $\gamma_i = 1$. This corresponds to the edge $({\bf 1}, v_i)$ in $G_L$. Since there is a path from $\bf 1$ to $v_i$, we have $\gamma_i = 1$, satisfying the constraint.
        \item Consider the constraint $\gamma_i = -1$. This corresponds to the edge $(v_i, {\bf -1})$ in $G_L$. Since there is no path from $\bf 1$ to $\bf -1$ in $G_L$, there must be no path from $\bf 1$ to $v_i$. Thus, $\gamma_i = -1$, satisfying the constraint.
    \end{itemize}
    
    Thus, the approximate privacy constraints are satisfied by $\gamma$, which means that $approx(L) < \infty$ and $L$ is differentially private.
\end{proof}

We have specified a linear time algorithm only to check whether a given looping branch is private. However, we can check whether \textit{all} looping branches are private at once in linear time by combining the privacy constraint graphs for each looping branch into a single graph.

\begin{defn}
    The privacy constraint graph $G_P$ of a program $P$ is the union of the privacy constraint graphs of each looping branch in $P$.
\end{defn}

\begin{prop}
    \label{prop:paths_in_privacy_graph}
    Let $v_{i_0}, \dots, v_{i_k}$ be vertices in $G_P$ corresponding to transitions $t_{i_0}, \dots, t_{i_k}$ in $P$. If $v_{i_0} \to \dots \to v_{i_k}$ is a path in $G_P$, then there exists a path $\rho \in P$ such that 
    \begin{align*}
        t_{i_0} \cdots t_{i_k} \text{ is a subsequence of } \rho \text{ with } guard(t_{i_j}) = \gguard \text { for all } j \in \{1, \dots, k\}
    \end{align*}
    or
    \begin{align*}
        t_{i_k} \cdots t_{i_0} \text{ is a subsequence of } \rho \text{ with } guard(t_{i_j}) = \lguard \text { for all } j \in \{k - 1, \dots, 0\}
    \end{align*}
\end{prop}


\begin{proof}
    We will use induction on the length of the path $v_{i_0} \to \dots \to v_{i_k}$ in $G_P$. 

    \begin{itemize}
        \item Base Case ($k = 1$)
        
        If $k = 1$, then the path $v_{i_0} \to v_{i_1}$ comprises of a single edge $(v_{i_0}, v_{i_1})$ in $G_P$. So, there is a looping branch $L$ for which $(v_{i_0}, v_{i_1}) \in G_L$, for which there is the privacy constraint $\gamma_{i_0} \leq \gamma_{i_1}$. 
        
        We either have that $i_0 = at(i_1)$ and $c_{i_1} = \gguard$, or $i_1 = at(i_0)$ and $c_{i_0} = \lguard$. In the first case, we have that $t_{i_0} t_{i_1}$ is a subsequence of some path $\rho$ in $L$ with $guard(t_{i_1}) = \gguard$. In the second case, we have that $t_{i_1} t_{i_0}$ is a subsequence of some path $\rho$ in $L$ with $guard(t_{i_0}) = \lguard$.

        \item Inductive Step ($k > 1$)
        
        By the inductive hypothesis, we have one of the following cases: 

        \begin{enumerate}
            \item There exists a path $\rho_1 \in P$ such that $t_{i_0} \cdots t_{i_{k - 1}}$ is a subsequence of $\rho_1$ with $guard(t_{i_j}) = \gguard$ for all $j \in \{1, \dots, k - 1\}$.
            
            Since we have the edge $(v_{i_{k - 1}}, v_{i_k})$ in $G_P$, there exists a looping branch $L$ for which $(v_{i_{k - 1}}, v_{i_k}) \in G_L$, for which there is the privacy constraint $\gamma_{i_{k - 1}} \leq \gamma_{i_k}$.

            We either have that $i_{k - 1} = at(i_k)$ and $c_{i_k} = \gguard$ ($t_{i_{k - 1}}$ precedes $t_{i_k}$ in $L$), or $i_k = at(i_{k - 1})$ and $c_{i_{k - 1}} = \lguard$ ($t_{i_k}$ precedes $t_{i_{k - 1}}$ in $L$). Notice, however, that we cannot have that $t_{i_k}$ precedes $t_{i_{k - 1}}$, since we have assumed $c_{i_{k - 1}} = \gguard$. 

            So, there exists a path $\rho_2 \in L$ such that $t_{i_{k - 1}} t_{i_k}$ is a subsequence of $\rho_2$ with $guard(t_{i_k}) = \gguard$. 

            Let $j_1$ be the index at which $t_{i_{k - 1}}$ appears in $\rho_1$, and $j_2$ be the index at which it appears in $\rho_2$. Then, the path $\rho_1[:j_1] \rho_2[j_2:]$ is a path in $P$ such that $t_{i_0} \cdots t_{i_k}$ is a subsequence of $\rho_1[:j_1] \rho_2[j_2:]$ with $guard(t_{i_j}) = \gguard$ for all $j \in \{1, \dots, k\}$.

            \item There exists a path $\rho_1 \in P$ such that $t_{i_{k - 1}} \cdots t_{i_0}$ is a subsequence of $\rho_1$ with $guard(t_{i_j}) = \lguard$ for all $j \in \{k - 2, \dots, 0\}$.
            
            Similar to the argument above, we either have that $t_{i_{k - 1}}$ precedes $t_{i_k}$, or $t_{i_k}$ precedes $t_{i_{k - 1}}$ in some looping branch. We cannot have that $t_{i_{k - 1}}$ precedes $t_{i_k}$, and so $c_{i_{k - 1}}$ is forced to be $\lguard$. We can then construct a path $\rho \in P$ such that $t_{i_k} \cdots t_{i_0}$ is a subsequence of $\rho$ with $guard(t_{i_j}) = \lguard$ for all $j \in \{k - 1, \dots, 0\}$. 
        \end{enumerate}
    \end{itemize}

    This completes the proof. 
\end{proof}


\begin{cor}
    The path $v_{i_0} \to \dots \to v_{i_k}$ is in $G_P$ if and only if there is a looping branch $L$ in $P$ such that $v_{i_0} \to \dots \to v_{i_k}$ is a path in $G_L$.
\end{cor}

\begin{prop}
    A program $P$ is differentially private if and only if there does not exist a path from $\bf 1$ to $\bf -1$ in the privacy constraint graph of $P$.
\end{prop}

\begin{proof}
    There is a path from $\bf 1$ to $\bf -1$ in the privacy constraint graph of $P$ if and only if there is a path from $\bf 1$ to $\bf -1$ in the privacy constraint graph of some looping branch $L$ in $P$ by Proposition \ref{prop:paths_in_privacy_graph}. This is true if and only if there exists some $L$ which is not differentially private, which is true if and only if $P$ is not differentially private.
\end{proof}

\sky{clarify what ``size of'' $P$ is - ambiguous}

\begin{thm}
    Given a program $P$, we can check whether $P$ is differentially private in linear time in the size of $P$.
\end{thm}

\begin{proof}
    Constructing the privacy constraint graph $G_P$ takes linear time in the size of $P$, on which we can perform a breadth first search to check whether there is a path from $\bf 1$ to $\bf -1$. This can be done in linear time in the size of $G_P$, which is linear in the size of $P$.
\end{proof}

\sky{add a note about this matching DiPA bounds}





\section{Multiple Threshold Variable Programs}

We have shown the \textit{completeness} of coupling-based proofs of privacy for single-variable programs. However, it is also notable that coupling-based proofs allow for straightforward \textit{generalizations} to different program models that follow a similar paradigm. 

In particular, we show that coupling proofs can similarly be used to prove the privacy of an extended program model that allows for an \textit{arbitrary} number of threshold variables to compare inputs to. Indeed, we can again construct a series of ``coupling strategies'' parameterized by real-valued ``shifts''; we also show that, in the special case where there are two threshold variables, the existence of finite-cost coupling strategies again completely characterizes two-variable programs.
We finally conjecture that coupling proofs also characterize $k$-threshold variable programs in general. 

\subsection{Multivariable Transitions}

Our basic building block for programs with multiple threshold variables will be a single program \textit{transition}, as in the single variable case. 

\begin{defn}[$k$-variable guards]
    Let $\texttt{x}_1, \ldots \texttt{x}_k$ be real-valued program variables. Then a \textbf{$k$-variable guard} is a boolean statement $c = c^{(\texttt{x}_1)}\oplus_1 c^{(\texttt{x}_2)}\oplus_2\ldots\oplus_{k-1}c^{(\texttt{x}_k)}$ where for all $i$, \begin{itemize}
        \item $c^{(\texttt{x}_i)}\in \{\texttt{true}, \mvlguard[\texttt{x}_i], \mvgguard[\texttt{x}_i]\}$
        \item $\oplus_i \in \{\land, \lor\}$
    \end{itemize}
    Without loss of generality, we will assume that if the entire guard evaluates to a tautology, then $c^{(\texttt{x}_i)}=\texttt{true}$ for all $\texttt{x}_i$. Additionally, we will assume that all boolean operations in $c$ are left-associative. 
    
    
    % For example, we will disallow any guards of the form $c = \ldots \texttt{true}\lor \lguard[\texttt{x}_i]\ldots$, since the guard would simplify to $\texttt{true}$ (in which case, we suppose that for all $i$, $c^{(\texttt{x}_i)} = \texttt{true}$ and $\oplus_i = \land$). 

    Let $\mathcal{C}^{(k)}$ be the set of all possible guards with $k$ variables $\texttt{x}_1, \ldots \texttt{x}_k$.
\end{defn}

A multiple variable transition can thus be defined analogously single variable transitions; in particular, we will again take $\Gamma$ to be some finite alphabet of symbols that, along with real numbers, comprise the possible outputs of a transition. 

\begin{defn}[$k$-variable transitions]
    A $k$-variable transition ($k$v-transition) is a tuple $(c, \sigma, \tau)$ where \begin{itemize}
        \item $c\in\mathcal{C}^{(k)}$ is a transition guard.
        \item $\sigma\in\Gamma\cup\{\texttt{insample}^{(\texttt{x}_1)}, \ldots, \texttt{insample}^{(\texttt{x}_k)}, \texttt{insample}'\}$ is the output of the transition
        \item $\tau \in \{0\} \cup [k]$ indicates whether to assign $\texttt{insample}^{(\texttt{x}_\tau)}$ into no variable (when $\tau = 0$) or $\texttt{x}_\tau$. In particular, note that only a single variable can be assigned into at a time and that every variable $\texttt{x}_i$ can only take its ``corresponding'' input value $\texttt{insample}^{(\texttt{x}_i)}$. 
    \end{itemize}
\end{defn}

We again associate every transition $t$ with two real-valued noise parameters $P(t) = (d, d')$.

\subsubsection{$k$-Variable Transition Semantics}

The semantics of $k$-variable transitions are defined analogously to single variable transitions. Specifically, a program state is a tuple consisting of a value for every threshold variable $\texttt{x}_i$ and a distribution of possible values for the current output $\sigma$. Let $S =\RR^k\times (\Gamma\cup \RR)^*$ be the set all possible program states. As expected, every possible input is simply an element of $\RR$. 

Then the semantics of a $k$v-transition $t$ can be defined as a function $\Phi_t: dist_\downarrow(S)\times \RR\to dist_\downarrow(S)$ that maps an subdistribution of initial program states and an input to a subdistribution of subsequent program states.

We sketch how $k$-variable transition semantics are defined informally; the precise semantics, as defined by $\Phi_t$, are analogous to the single variable case.

Given some threshold values $\texttt{x}_i\in \RR^k$, a transition $t = (c, \sigma, \tau)$, and spread parameter values $P(t) = (d, d')$, $t$ reads a real number input $\texttt{in}$ and 
samples $k$ \textbf{independent} random variables $z^{(\texttt{x}_1)}\sim\Lap(0, \frac{1}{d\varepsilon}),\ldots, z^{(\texttt{x}_k)}\sim\Lap(0, \frac{1}{d\varepsilon})$ for comparing a noised version of the input to each threshold variable as well as one random variable $z' \sim\Lap(0, \frac{1}{d\varepsilon})$ to potentially be used for outputting a re-noised version of the input. 

Using these noise variables, $t$ then assigns $k$ variables $\texttt{insample}^{(\texttt{x}_i)} = \texttt{in} + z^{(x)}$ and an additional variable $\texttt{insample}' = \texttt{in} + z'$. 
If the guard $c$ is satisfied when comparing $\texttt{insample}^{(\texttt{x}_i)}$ to $\texttt{x}_i$ for all $i$, then the transition outputs $\sigma$ and, if $\tau\neq 0$, reassigns $\texttt{x}_{\tau} = \texttt{insample}^{(\texttt{x}_\tau)}$.

We again denote the probability that a transition $t=(c, \sigma, \tau)$ outputs a specific measurable output event $o$ as $\PP[\vec{\texttt{x}}, t, \texttt{in}, o]$, where $\vec{\texttt{x}}\in \RR^k$ is a vector of initial values of all threshold values $\texttt{x}_i$, $\texttt{in}\in \RR$ is a real-valued input, and $o\subseteq \Gamma\cup\RR$is a possible measurable output event of $t$.

Specifically, if $\vec{\texttt{x}}\in \RR^k$ and $o$ is a measurable output event of $t$, then $\PP[\vec{\texttt{x}}, t, \texttt{in}, o]$ is the marginal of $\Phi_t((\vec{\texttt{x}}, \lambda), \texttt{in})$ on $(\cdot, o)$.

\subsection{Multivariable Couplings}

In this section, we introduce two approaches for constructing couplings for $k$v-transitions; the first approach constructs single variable couplings independently or ``in parallel'' for each variable, while the second coupling approach couples different program variables together to create contradictions or tautologies. 

We show that if we can create couplings for each variable of a $k$v-transition in isolation, then we can immediately create a coupling for a $k$v-transition. 

We must first define what it means to isolate a variable in a transition: 

\begin{defn}[Isolating a variable]
    Let $t = (c, \sigma, \tau)$ be a $k$-variable transition. For all $i$, $t^{(\texttt{x}_i)}$ is the single variable transition $t^{(\texttt{x}_i)} = (c^{(\texttt{x}_i)}, \sigma^{(\texttt{x}_i)}, \tau^{(\texttt{x}_i)})$, where $c^{(\texttt{x}_i)}$ is the $\texttt{x}_i$-component of $c$ as defined above, $\sigma^{(\texttt{x}_i)} = \begin{cases}
        \sigma & \sigma \in \Gamma\\
        \texttt{insample} & \sigma = \texttt{insample}^{(\texttt{x}_i)}\\
        \texttt{insample}' & \sigma = \texttt{insample}'\\
        \bot & \text{otherwise}
    \end{cases}$ and $\tau^{(\texttt{x}_i)} = \begin{cases}
        \texttt{true} & \tau = i\\
        \texttt{false} & \tau \neq i
    \end{cases}$, where $\bot$ is a unique ``junk'' symbol. We call $t^{(\texttt{x}_i)}$ the $\texttt{x}_i$-\textbf{isolated} version of $t$. 
\end{defn}

By lemma \ref{simplifiedIndTransitionCoupling}, we know that, given a $k$v-transition $t$, we can create couplings for each $\texttt{x}_i$-isolated version of $t$. We demonstrate that this is sufficient to create couplings for $t$ as a whole. 

As in the single variable case, we create couplings from \textbf{coupling strategies}, i.e. a collection of shifts $C = (\gamma_{x_1}, \ldots, \gamma_{x_k}, \gamma_t^{(x_1)}, \ldots, \gamma_t^{(x_k)}, \gamma_t')$. Such a collection is valid if it satisfies the constraints below. 

\begin{lemma}\label{simplifiedMvParallelCouplingsLemma}
    For all $\varepsilon>0$, for any $k$v-transition $t$, measurable output event $\sigma$ of $t$, and adjacent inputs $\texttt{in}\brangle{1}\sim\texttt{in}\brangle{2}$, if we are given $2k+1$ real number ``shifts'' $\gamma_{x_1}, \ldots, \gamma_{x_k}, \gamma_t^{(x_1)}, \ldots, \gamma_t^{(x_k)}, \gamma_t'$ such that for all $1\leq i\leq k$, \[
        \begin{cases}
            \gamma_t^{(x_i)}\leq\gamma_{x_i} & c^{(\texttt{x}_i)} = \mvlguard[\texttt{x}_i]\\
            \gamma_t^{(x_i)}\geq\gamma_{x_i} & c^{(\texttt{x}_i)} = \mvgguard[\texttt{x}_i]\\
            \gamma_t^{(x_i)}=0 & \sigma = \texttt{insample}^{(\texttt{x}_i)}\\
            \gamma_t'=0 & \sigma = \texttt{insample}'
      \end{cases},
      \]
      then we can construct an approximate lifting that proves $\PP[\vec{X}\brangle{1}, t, \texttt{in}\brangle{1}, \sigma]\leq e^{d\varepsilon}\PP[\vec{X}\brangle{2}, t, \texttt{in}\brangle{2}, \sigma]$ for some bounded $d>0$ and initial threshold Laplace-distributed variables $\vec{X}\brangle{1}$, $\vec{X}\brangle{2}\in \RR^k$.
\end{lemma}

To be precise, we show that $d = \sum_{i=1}^k\left(|\mu_{x_i}\brangle{1}-\mu_{x_i}\brangle{2}+\gamma_{x_i}|d_{x_i}+|\texttt{in}\brangle{1}-\texttt{in}\brangle{2}+\gamma_t^{(x_i)}|d_t\right)+|\texttt{in}\brangle{1}-\texttt{in}\brangle{2}+\gamma_t'|d_t'$;
the \textit{cost} for a $k$v-coupling strategy $C$ is thus \[cost(C) = \sup_{\texttt{in}\brangle{1}\sim\texttt{in}\brangle{2}}\sum_{i=1}^k\left(|\mu_{x_i}\brangle{1}-\mu_{x_i}\brangle{2}+\gamma_{x_i}|d_{x_i}+|\texttt{in}\brangle{1}-\texttt{in}\brangle{2}+\gamma_t^{(x_i)}|d_t\right)+|\texttt{in}\brangle{1}-\texttt{in}\brangle{2}+\gamma_t'|d_t'\] (see lemma \ref{mvParallelCouplingsLemma} in the appendix for a full proof).

This provides an extremely straightforward method of combining coupling strategies for different variables together.

\begin{cor}
    For a $k$v-transition $t$, if, for all $1\leq i\leq k$, there exists a coupling strategy $C_i$ such that $C_i$ is a valid coupling strategy for the isolated transition $t^{(x_i)}$, then there exists a valid coupling strategy $C$ for $t$ such that $cost(C)\leq \sum_{i=1}^k cost(C_i)$.
\end{cor}

\begin{proof}
    Let $t = (c, \sigma, \tau)$ and for all $i$, let $C_i = (\gamma_{x_i}, \gamma_t^{(x_i)}, \gamma_t^{(x_i)\prime})$ be the single variable coupling strategy for the isolated transition $t^{(x_i)} = (c^{(x_i)}, \sigma^{(x_i)}, \tau^{(x_i)})$. 
    
    Take $C = (\gamma_{x_1}, \ldots, \gamma_{x_k}, \gamma_t^{(x_1)}, \ldots, \gamma_t^{(x_k)}, \gamma_t')$, where $\gamma_t' = 0$.

    Because $C_i$ is valid for all $i$, we know that if $c^{(x_i)} = \lguard$, then $\gamma_t^{(x_i)}\leq \gamma_{x_i}$ and if $c^{(x_i)} = \gguard$, then $\gamma_t^{(x_i)}\geq \gamma_{x_i}$. Similarly, if $\sigma = \texttt{insample}^{(x_i)}$, then $\sigma^{(x_i)} = \texttt{insample}$, so $\gamma_t^{(x_i)} = 0$. Thus, for all $i$, the first three conditions of validity of $C$ are satisfied. Finally, since $\gamma_t' = 0$, the fourth condition of validity is satisfied as well, so $C$ is valid.
    
    Observer that for all $\texttt{in}\brangle{1}\sim\texttt{in}\brangle{2}$, $\sum_{i=1}^k(|\texttt{in}\brangle{1}-\texttt{in}\brangle{2}|)d_t'\geq (|\texttt{in}\brangle{1}-\texttt{in}\brangle{2}|)d_t'$. 
    So \begin{alignat*}{2}
        \sum_{i=1}^k cost(C_i) &= \sum_{i=1}^k\sup_{\texttt{in}\brangle{1}\sim\texttt{in}\brangle{2}}(&(|\mu_{x_i}\brangle{1}-\mu_{x_i}\brangle{2}+\gamma_{x_i}|)d_{x_i}+(|\texttt{in}\brangle{1}-\texttt{in}\brangle{2}+\gamma_t^{(x_i)}|)d_t\\ & &+(|\texttt{in}\brangle{1}-\texttt{in}\brangle{2}+\gamma_t^{(x_i)\prime}|)d_t')\\
        &\geq \sup_{\texttt{in}\brangle{1}\sim\texttt{in}\brangle{2}}\sum_{i=1}^k(&(|\mu_{x_i}\brangle{1}-\mu_{x_i}\brangle{2}+\gamma_{x_i}|)d_{x_i}+(|\texttt{in}\brangle{1}-\texttt{in}\brangle{2}+\gamma_t^{(x_i)}|)d_t)\\
         & &+(|\texttt{in}\brangle{1}-\texttt{in}\brangle{2}|)d_t'\\
        &= cost(C) &
    \end{alignat*}
\end{proof}

\subsubsection{Cross-Couplings}

We introduce cross-couplings, which couple together \textit{different} program variables to produce valid approximate liftings. 

Intuitively, cross-couplings allow us to construct liftings for certain transitions ``for free'' in a manner compatible with existing liftings, dependent on the initial threshold distributions. 

In particular, cross-couplings can be applied to transitions whose guards correspond to checking if an input is within either the empty set or the entire real line, which are either always true or always false. By deriving either a tautology or a contradiction, the implication statement that must hold for a valid approximate lifting is shown to be trivially satisfied. 

The construction of cross-couplings for a transition is dependent both on any existing \textit{shifts} (from, for example, parallel single variable couplings) and the initial distribution of program variables --- in particular, the construction depends on the means of the initial distributions of each program variable. 

We say that a collection of means $\{\mu_i\brangle{1}, \mu_i\brangle{2}\}_{i=1}^k$ and shifts $\gamma_{x_1},\ldots,\gamma_{x_k}$ \textbf{allow for a cross coupling} for a transition $t$ if either of the first two conditions in the following lemma (lemma \ref{mvCrossCoupling}) are satisfied for $\mu$ and $\gamma$. 

\begin{lemma}\label{mvCrossCoupling}
    Let $\vec{X}\brangle{1} = (X_1\brangle{1}, \ldots X_k\brangle{1})$ where $X_i\brangle{1}\sim \Lap(\mu_i\brangle{1}, \frac{1}{d_x\varepsilon})$ are independent random variables and $\vec{X}\brangle{2} = (X_1\brangle{2}, \ldots X_k\brangle{2})$ where $X_i\brangle{2}\sim \Lap(\mu_{x_i}\brangle{2}, \frac{1}{d_x\varepsilon})$ 
    are independent random variables be such that, for all $i$, $\mu_i\brangle{1}\sim \mu_i\brangle{2}$ are adjacent input values.

    Then for any coupling strategy $\gamma_{x_1}, \ldots, \gamma_{x_k}, \gamma_t^{(x_1)}, \ldots, \gamma_t^{(x_k)}, \gamma_t'$ for a transition $t = (c, \sigma,\tau)$, if one of the following is true: \begin{enumerate}
        \item The boolean expression produced from $c$ by setting all $\texttt{insample}^{(\texttt{x}_i)}$ equal to each other and setting $\texttt{x}_i = \mu_i\brangle{1}$ for all $i$ is a contradiction.
        \item The boolean expression produced from $c$ by setting all $\texttt{insample}^{(\texttt{x}_i)}$ equal to each other and setting $\texttt{x}_i = \mu_i\brangle{2}$ for all $i$ is a tautology.
    \end{enumerate}
    and, additionally, the following two conditions hold: \begin{itemize}
        \item If $\sigma = \texttt{insample}^{(\texttt{x}_i)}$, then $\gamma_t^{(\texttt{x}_i)}=0$
            \item If $\sigma = \texttt{insample}'$, then $\gamma_t'=0$
    \end{itemize}
    then we can construct an approximate lifting that proves $\PP[\vec{X}\brangle{1}, t, \texttt{in}\brangle{1}, \sigma]\leq e^{d\varepsilon}\PP[\vec{X}\brangle{2}, t, \texttt{in}\brangle{2}, \sigma]$ for some $d>0$. 
\end{lemma}

Note that cross-couplings require every single component of the initial variable distribution to have the same spread parameter.

\subsection{Multivariable Paths}

By sequentially concatenating $k$v-transitions, we can again create $k$-variable path programs, which represent executions of a program in our program model. 

\begin{defn}[$k$-variable paths]
    A $k$-variable path ($k$v-path) is a finite string of $k$v-transitions. Analogously to single variable paths, we call a $k$v-path $\rho = t_0t_1\ldots t_{n-1}$, \textbf{initialized} if, for all $0\leq i < k$, $t_i = (\texttt{true}, \sigma_i, i+1)$ for some $\sigma_i$.
\end{defn}

The semantics of a $k$v-path are again defined by composing the semantics of each individual transition in the path; we denote the probability of a path $\rho$ outputting a specific measurable output sequence event $\sigma$ given initial threshold distributions $\vec{\texttt{x}}$ and input sequence $\texttt{in}$ as $\PP[\vec{\texttt{x}}, \rho, \texttt{in}, \sigma]$, which we shorthand to $\PP[\rho, \texttt{in}, \sigma]$ when $\rho$ is an initialized path.

Analogously to single variable paths, we will use the notation $t_{at_j(i)}$ to refer to the assignment transition for variable $\texttt{x}_j$ that immediately precedes transition $t_i$ within a path. 

As previously noted, cross-couplings require that the spread parameter of threshold variables are identical across variables; we thus say that a path $\rho = t_0\ldots t_{n-1}$ \textbf{allows for cross-couplings} if there exists some constant $d_{at}>0$ such that, for every assignment transition $t_i$ of $\rho$, $P(t_i) = (d_{at}, d'_i)$. 


Combining our ``parallel'' and ``cross'' coupling strategies allows us to create $k$v-path coupling strategies:

\begin{defn}[$k$-variable Coupling Strategies]
    A $k$v-coupling strategy for a $k$v-path $\rho$ of length $n$ is a collection of shifts $\{\gamma_i^{(\texttt{x}_1)},\ldots, \gamma_i^{(\texttt{x}_k)}, \gamma_i'\}_{i=0}^{n-1}$ such that every $\{\gamma_i^{(\texttt{x}_1)},\ldots, \gamma_i^{(\texttt{x}_k)}, \gamma_i'\}_{i=0}^{n-1}$ is a function of two adjacent input sequences $\texttt{in}\brangle{1}\sim \texttt{in}\brangle{2}$ with range $[-1, 1]$. 
    We call a coupling strategy \textbf{valid} if, for \textbf{all} input sequences $\texttt{in}\brangle{1}\sim\texttt{in}\brangle{2}$, it satisfies the constraints in lemma \ref{mvPathCouplingLemma}.
\end{defn}

As expected, if a coupling strategy satisfies certain conditions (see lemma \ref{mvPathCouplingLemma}), then we can use it to prove that a path is private. 

\begin{lemma}\label{mvPathCouplingLemma}
    Let $\rho = t_0\ldots t_{n-1}$ be a initialized $k$v-path of length $n$ where $t_i = (c_i, \sigma_i, \tau_i)$ and $P(t_i) = (d_i, d'_i)$ for all $i$. 
    Let $\texttt{in}\brangle{1}\sim \texttt{in}\brangle{2}$ be arbitrary adjacent input sequences of length $n$. Additionally, fix some potential measurable output sequence event $\sigma$ of $\rho$ of length $n$.

    Then $\forall \varepsilon>0$ and for all $\{\gamma_i^{(\texttt{x}_1)},\ldots, \gamma_i^{(\texttt{x}_k)}, \gamma_i'\}_{i=0}^{n-1}$ that, for all $0\leq i\leq n-1$ and $1\leq j\leq k$ satisfy the following constraints:\begin{enumerate}
        \item If $c_i$ is satisfied in run $\brangle{1}$, then $c_i$ is satisfied in run $\brangle{2}$; i.e. at least one of the following is true:\begin{enumerate}
            \item $\{\texttt{in}_{at_1(i)}\brangle{1}, \texttt{in}_{at_1(i)}\brangle{2}, \ldots, \texttt{in}_{at_k(i)}\brangle{1}, \texttt{in}_{at_k(i)}\brangle{2}\}$ and $\gamma_{at_1(i)}^{(\texttt{x}_1)}, \ldots, \gamma_{at_k(i)}^{(\texttt{x}_k)}$ allow for a cross coupling for $t_i$.
            \item For all $1\leq j \leq k$, if $c_i^{(\texttt{x}_j)} = \mvlguard[\texttt{x}_j]$, then $\gamma_i^{(\texttt{x}_j)}\leq \gamma^{(\texttt{x}_j)}_{at_j(i)}$ and if $c_i^{(\texttt{x}_j)} = \mvgguard[\texttt{x}_j]$, then $\gamma_i^{(\texttt{x}_j)}\geq \gamma^{(\texttt{x}_j)}_{at_j(i)}$.
        \end{enumerate}
        \item If $t_i$ outputs the specific value $o_i$ in run $\brangle{1}$, then $t_i$ also outputs $o_i$ in run $\brangle{2}$; i.e. both of the following must be true: \begin{enumerate}
            \item If $\sigma_i = \texttt{insample}^{(\texttt{x}_j)}$, then $\gamma_i^{(\texttt{x}_j)}=0$
            \item If $\sigma_i = \texttt{insample}'$, then $\gamma_i'=0$
        \end{enumerate}
    \end{enumerate}
     we can construct an approximate lifting that proves $\PP[\rho, \texttt{in}\brangle{1}, \sigma]\leq e^{d\varepsilon}\PP[\rho, \texttt{in}\brangle{2}, \sigma]$ for $d = \sum_{i=0}^{n-1}\left(|\texttt{in}\brangle{2}-\texttt{in}\brangle{1}-\gamma_i'|d_i'+\sum_{j=1}^k|\texttt{in}\brangle{2}-\texttt{in}\brangle{1}-\gamma_i^{(\texttt{x}_j)}|d_i\right)$.
\end{lemma}

\begin{proof}
    Follows from lemmas \ref{simplifiedMvParallelCouplingsLemma} and \ref{mvCrossCoupling} exactly as lemma \ref{multTransitionsCouplingProof} follows from lemma \ref{indTransitionCoupling}.
\end{proof}

We denote the cost of a coupling strategy $C=\{\gamma_i^{(\texttt{x}_1)},\ldots, \gamma_i^{(\texttt{x}_k)}, \gamma_i'\}_{i=0}^{n-1}$ for a $k$v-path $\rho$ as $cost(C) = \max_{\texttt{in}\brangle{1}\sim\texttt{in}\brangle{2}}\sum_{i=0}^{n-1}\left(|\texttt{in}\brangle{2}-\texttt{in}\brangle{1}-\gamma_i'|d_i'+\sum_{j=1}^k|\texttt{in}\brangle{2}-\texttt{in}\brangle{1}-\gamma_i^{(\texttt{x}_j)}|d_i\right)$.

\subsection{$k$v-Programs}

As in the single variable case, we model programs that can be represented by a finite control flow graph $G = (V, E)$, where $V$ is a finite set of program locations, and each edge $e\in E$ is labeled with a function $T(e)$ such that $T(e)$ is a $k$v-transition. 

For clarity, we call control flow graphs for $k$-variable programs $k$v-CFGs. 

\begin{defn}
    A $k$v-CFG $G = (V, E)$ is \textbf{proper} if it satisfies the following conditions: 
    \begin{itemize}
        \item \textbf{Initialization:} The first $k$ transitions $t_1\ldots t_k$ of any execution of $G$ must be such that $t_i$ is a transition with guard $\texttt{true}$ that assigns into $\texttt{x}_i$.
        
        More precisely, $V$ must contain unique initial locations $\ell_{init}^{(x_1)}, \ldots \ell_{init}^{(x_k)}\in V$ such that for all $1\leq i \leq k$, there exists exactly one edge $e_{init}^{(x_i)}\in E$ that has a source at $\ell_{init}^{(x_i)}$. Additionally, for all $1\leq i <k$, $e_{init}^{(x_i)} = (\ell_{init}^{(x_{i})}, \ell_{init}^{(x_{i+1})})$. Finally, for all $1\leq i \leq k$, $T(e_{init}^{(x_i)})$ must be of the form $(\texttt{true}, \sigma_i, i)$ for some $\sigma_i$.
        \item \textbf{Determinism:} For all locations $\ell\in V$, if there exist distinct edges $(\ell, \ell')$ labeled by $t'=(c', \sigma', \tau')$ and $(\ell, \ell^*)$ labeled by $t^* = (c^*, \sigma^*, \tau^*)$, then $c'$ and $c^*$ must be logically disjoint; i.e. the boolean expression $c' \land c^*$ must be a contradiction. 
        In particular, note that this means that if there exists an edge $(\ell, \ell')\in E$ labeled by a transition of the form $(\texttt{true}, \sigma, \tau)$, then there does not exist another edge in $E$ with source at $\ell$.
        \item \textbf{Shared Noise:} For all locations $\ell\in V$ and any two edges $(\ell, \ell')$ labeled by $t'=(c', \sigma', \tau')$ and $(\ell, \ell^*)$ labeled by $t^* = (c^*, \sigma^*, \tau^*)$, $P(t') = P(t^*)$. 
        % \item \textbf{Public Input:} For all locations $\ell\in V$, if there exists some edge $e = (\ell, \ell') \in E$ such that $e$ is labeled by a public transition, then every other edge from $\ell$ must also be labeled by a public transition. 
        \item \textbf{Cross Coupling Compatibility:} There exists some constant $d_{at}>0$ such that for every edge $e\in E$ labeled by the transition $t = (c, \sigma, \tau)$, if $\tau \neq 0$ (i.e. $t$ is an assignment transition), $P(t) = (d_{at}, d'_t)$ for some $d'_t >0$.
    \end{itemize}
\end{defn}

We will again use $\Psi(r)$ to denote the forgetful homomorphism from an execution $r$ of $G$ that drops all program locations from $r$ to produce a $k$v-path and say that $\{\Psi(r): r\text{ is a execution of }G\}$ is the set of $k$v-paths \textbf{generated} by a proper $k$v-CFG $G$. 

Naturally, $k$v-programs can be defined using a $k$v-CFG in analogy to the single variable case. 

\begin{defn}
    A $k$v-program $P$ is a language over a finite alphabet of $k$v-transitions generated by a proper $k$v-CFG $G$. 
\end{defn}

Observe that a $k$v-program is still a regular language, simply over a finite alphabet of $k$v-transitions rather than single variable transitions. Thus, we again apply the decomposition result of \cite{afoninMinimalUnionFreeDecompositions2009} and analyze every $k$v-program as a finite collection of \textbf{trajectories}.

\begin{defn}[$k$-variable Trajectories]
    A $k$v-trajectory $L$ is a union-free regular language generated by a proper $k$v-CFG $G$. 
\end{defn}

We again associate a single coupling strategy with every $k$v-trajectory:

\begin{defn}[Coupling strategy for a $k$v-trajectory]
    Let $L$ be a $k$v-trajectory generated by the graph $G_L = (V_L, E_L)$, where $m = |E_L|$. Then a coupling strategy $C = \{\gamma_i^{(\texttt{x}_1)},\ldots, \gamma_i^{(\texttt{x}_k)}, \gamma_i'\}_{i=0}^{m-1}$ for $L$ is a function $C:E_L\times\RR \times\RR\to [-1, 1]^{k+1}$ that computes $k+1$ shifts for each transition-labeled edge in $L$ as a function of two adjacent inputs.
\end{defn}

A coupling strategy for a $k$v-trajectory $L$ induces a coupling strategy for each $k$v-path in $L$ in the same manner as the single variable case (see definition \ref{svInducedCouplingStrategy}). 

As expected, the cost $cost(C)$ of a coupling strategy $C$ for a $k$v-trajectory $L$ is the supremum of the costs of the $k$v-path coupling strategies induced by $C$ over all paths in $L$.

We can also directly translate the constraints for a coupling strategy to have finite cost from single variable trajectories to define the multivariable privacy constraint system.

\begin{defn}\label{mvPrivacyConstraintSystem}
    Let $L$ be a $k$v-trajectory generated by $G_L = (V_L, E_L)$ and let $C = \{\gamma_i^{(\texttt{x}_1)},\ldots, \gamma_i^{(\texttt{x}_k)}, \gamma_i'\}_{i=0}^{m-1}$ be a coupling strategy for $L$. If, for every path $\rho$ in $L$, the coupling strategy for $\rho$ induced by $C$ satisfies the constraints from lemma \ref{mvPathCouplingLemma} as well as the following constraints for all input sequences $\texttt{in}\brangle{1}\sim\texttt{in}\brangle{2}$ and all $i$: \begin{enumerate}
        \setcounter{enumi}{2}
        \item If $\Psi^{-1}(t_i)$ is in a cycle in $G_L$, then for all $1\leq j\leq k$, $\gamma_i^{(\texttt{x}_j)} = -\texttt{in}\brangle{1}_i+\texttt{in}\brangle{2}_i$
        \item If $\Psi^{-1}(t_i)$ is in a cycle in $G_L$, then $\gamma_i' = -\texttt{in}\brangle{1}_i+\texttt{in}\brangle{2}_i$
    \end{enumerate}
    then we say that $C$ satisfies the privacy constraint system for $L$. 
\end{defn}

\begin{lemma}
    If a coupling strategy $C$ satisfies the privacy constraint system for a $k$v-trajectory $L$, then $cost(C)<\infty$ and $L$ is $cost(C)\varepsilon$-differentially private. 
\end{lemma}

\begin{proof}
    Because the privacy constraint system includes constraints for validity by definition, we know that $C$ is a valid coupling strategy. In particular, by lemma \ref{mvPathCouplingLemma}, this means that $C$ will produce a proof that $L$ is $cost(C)\varepsilon$-differentially private. 
    
    It remains to show that $C$ has finite cost. For every path $\rho\in L$, let $\rho^{(\texttt{x}_i)}$ be the single variable path created from $\rho$ by isolating every transition to the variable $\texttt{x}_i$. 

    Because of constraints (3) and (4), we know that, for all $\texttt{x}_i$, $\sup_{\rho\in L}\sum_{i=0}^{|\rho|-1}|\texttt{in}_i\brangle{2}-\texttt{in}_i\brangle{1}-\gamma_i^{(\texttt{x}_i)}|d_i + |\texttt{in}_i\brangle{2}-\texttt{in}_i\brangle{1}-\gamma_i'|d_i'$ is finite by applying lemma \ref{finiteCostConstraintLemma} to $\rho^{(\texttt{x}_i)}$ and a single variable coupling strategy constructed using $(\gamma^{(\texttt{x}_i)}, \gamma')$. 

    This immediately implies that $cost(C)$ is finite as well, since there are a finite number of program variables. 
\end{proof}

Thus, we can extend corollary \ref{svProgramPrivacyCorollary} to multiple variable programs. 

\begin{lemma}
    If, for every trajectory $L$ in $P$, there exists a valid coupling strategy $C_L$, then $P$ is $cost(P) = (\max_{L\subseteq P} cost(C_L))\varepsilon$-differentially private. In particular, if for every $L\subseteq P$ there exists a coupling strategy for $L$ that satisfies the privacy constraint system, then $P$ is $d\varepsilon$-differentially private for some finite $d>0$.
\end{lemma}

\subsection{Demonstrating Violations of Privacy for 2-variable programs}

We have shown that, using parallel and cross-couplings, couplings can be used to generate proofs of privacy for any $k$-variable program that satisfies the privacy constraint system. 

Unlike the single variable case, we do not show that the privacy constraint system is complete for an arbitrary number of variables. We discuss some of the difficulties in generalizing violations of privacy to $k$ variables in section \ref{generalizingToKVariables}.

However, we do have completeness in the special case when we have $k=2$ variables. We show that in this case (whose variables we now label $\texttt{x}$ and $\texttt{y}$) the privacy constraint system for a 2v-program is complete for output-distinct programs. 

As with the single variable case, if the privacy constraint system is unsatisfiable for a program $P$ generated by $G$, we demonstrate that there must exist specific graph structures that correspond to a violation of privacy.

In general, these graph structures directly correspond to the single variable case; indeed, with one exception, we can identify these structures by looking at the generating graph of a program \textit{isolated} to a single variable. 

We will use $G^{(\texttt{x}_i)}$ to denote the $k$v-CFG $G = (V, E)$ where each edge label $T(e)$ is replaced by the $\texttt{x}_i$-isolation of $T(e)$. 

Identifying leaking cycles, disclosing cycles, and privacy violating paths in any variable $\texttt{x}_i$ for the $\texttt{x}$-isolated graph $G$ is sufficient to produce a violation of privacy in the two variable case; we must account for a special kind of leaking pair that turns out to be provably private using cross-couplings. 

\begin{defn}
    Let $G = (V, E)$ be a 2v-transition labeled graph with variables $\texttt{x}$, $\texttt{y}$ and let $(C, C')$ be a leaking pair in either $G^{(\texttt{x})}$ or $G^{(\texttt{y})}$ such that there is a path from $C$ to $C'$ with no assignment transitions into the other variable. For every edge $e\in E$, let $t_e = (c_e, \sigma_e, \tau_e)$ be the 2v-transition that labels $e$. 
    If at least one of the following is true for $C, C'$: \begin{itemize}
        \item For all $e\in C\cup C'$, $c_e \in \{\texttt{true}, \mvlguard[\texttt{x}]\land\mvgguard[\texttt{y}], \mvgguard[\texttt{x}]\land\mvlguard[\texttt{y}]\}$
        \item For all $e\in C\cup C'$, $c_e \in \{\texttt{true}, \mvlguard[\texttt{x}]\lor\mvgguard[\texttt{y}], \mvgguard[\texttt{x}]\lor\mvlguard[\texttt{y}]\}$
\end{itemize}
then we say $(C, C')$ is a cancelling leaking pair. 
\end{defn}

Intuitively, the reason that cancelling leaking pairs are private is that for any threshold variable distributions with means at $\mu_x,\mu_y$, at least one of $\mu_x\geq\mu_y$ or $\mu_x\leq \mu_y$ must always be true no matter what input sequence we are given. 
Thus, for example, either all of the transitions with guard $\mvlguard[\texttt{x}]\land\mvgguard[\texttt{y}]$ or all of the transitions with guard $\mvgguard[\texttt{x}]\land\mvlguard[\texttt{y}]$ allow for a ``free'' cross coupling, while the other set of transitions can be coupled for finite cost in a standard manner.

Incorporating this exception, we show that an unsatisfiable coupling strategy still implies the presence of these ``bad'' graph structures. 

\begin{lemma}
    If no coupling strategy for a 2v-trajectory $L$ satisfies the privacy constraint system, then there exists a leaking cycle, non-cancelling leaking pair, disclosing cycle, or privacy violating path in either $\texttt{x}$ or $\texttt{y}$ in $L$.
\end{lemma}
\begin{proof}
    Suppose that we have some maximally satisfied coupling strategy $C$ for $L$. There must be some constraint that is violated by $C$. Note that if constraint (11) is violated, then there must be a leaking cycle in $L$ since we only allow cross-couplings between assignment transitions. 

    Thus, we can assume that constraint (11) is not violated. 

    By lemma \ref{ProgramCounterexampleThm}, there must then be either a leaking cycle, leaking pair, disclosing cycle, or privacy violating path with respect to a single variable. 

    We will show that, if there only exist leaking pairs in $L$, then at least one leaking pair must be a non-cancelling leaking pair. 

    For the sake of contradiction, suppose that every leaking pair $\kappa, \kappa'$ in $L$ is a cancelling leaking pair. 

    By definition, this means that for every two transitions $t_i, t_j$ in $\kappa, \kappa'$, $at_x(i) = at_x(j)$ and $at_y(i) = at_y(j)$.

    Without loss of generality, we will assume that every non-$\texttt{true}$ transition in $\kappa$ must either have guard $\mvlguard[\texttt{x}]\land\mvgguard[\texttt{y}]$ or $\mvlguard[\texttt{x}]\lor\mvgguard[\texttt{y}]$. Thus, every non-$\texttt{true}$ transition in $\kappa'$ must either have guard $\mvgguard[\texttt{x}]\land\mvlguard[\texttt{y}]$ or $\mvgguard[\texttt{x}]\lor\mvlguard[\texttt{y}]$, respectively. 

    Let $t_i$ be an arbitrary non-$\texttt{true}$ transition in $\kappa$ and $t_j$ be an arbitrary non-$\texttt{true}$ transition in $\kappa'$.

    Consider the case where $c_i = \mvlguard[\texttt{x}]\land\mvgguard[\texttt{y}]$ and $c_j = \mvgguard[\texttt{x}]\land\mvlguard[\texttt{y}]$; the other case is symmetric.

    Observe that at least one of $\texttt{in}_{at_x(i)}\brangle{1}\leq \texttt{in}_{at_y(i)}\brangle{1}$ or $\texttt{in}_{at_y(i)}\brangle{1}\geq \texttt{in}_{at_y(i)}\brangle{1}$ must be true. Suppose that $\texttt{in}_{at_x(i)}\brangle{1}\leq \texttt{in}_{at_y(i)}\brangle{1}$. The case where $\texttt{in}_{at_y(i)}\brangle{1}\geq \texttt{in}_{at_y(i)}\brangle{1}$ is symmetric. 

    Then we can set $\gamma_{at_x(i)} = -1$, $\gamma_{at_y(i)}=1$, and $\gamma_{(at_x(i), at_y(i))} =  -\max(0, \texttt{in}_{at_x(i)}\brangle{1}+ \gamma_{at_x(i)}^{(x)}-\texttt{in}_{at_y(i)}\brangle{1}-\gamma_{at_y(i)}^{(y)})$. 
    
    Because there do not exist disclosing cycles, leaking cycles, or privacy violating paths in either variable, this cannot violate any further constraints. Thus, $C$ is not maximal, which is our contradiction, so $L$ must contain either a leaking cycle, disclosing cycle, privacy violating path, or non-cancelling leaking pair in at least one variable.
\end{proof}

Finally, through a careful analysis, we extend the single variable DiPA counterexample results to two variables to show that the presence of these bad graph structures in a generating graph $G$ for a program $P$ implies that $P$ is not differentially private. 

As in the single variable case, this result holds for the class of programs whose paths can be uniquely identified with specific outputs, which we call output-distinct (see definition \ref{outputDistinctionDef}).


\begin{lemma}
    Consider a 2v-program $P$ generated by $G$. If $G$ satisfies output distinction and there exists a leaking cycle, disclosing cycle, or privacy violating path in a single variable or a non-cancelling leaking pair in either $\texttt{x}$ or $\texttt{y}$, then $P$ is not $d\varepsilon$-differentially private for any $d>0$. 
\end{lemma}

We conclude that the privacy constraint system completely characterizes privacy even for 2 variable programs.

\begin{thm}
    A 2v-program $P$ is $d\varepsilon$-differentially private for some $d>0$ if and only if there exists coupling strategy for every trajectory of $P$ that satisfies the privacy constraint system.
\end{thm}

\subsection{Beyond Two Variables}\label{generalizingToKVariables}

We have shown that for an arbitrary number of variables, we can produce a set of candidate coupling proofs that can potentially prove that a $k$v-program is private. As in the single variable case, these coupling proofs can be characterized by a system of linear constraints; if the system is satisfiable for a $k$v-program, then the program is differentially private. Additionally, we showed that the constraint system is complete for 2 variable programs specifically. 

We conjecture this constraint system is still in fact complete for $k>2$ variables; our proofs for counterexamples to privacy for $2$v-programs are currently specific to two variable programs, but it is possible that they can be extended through a more careful analysis. In particular, it may be possible to ``group'' pairs of single-variable guards together and analyze them as a single unit to reduce the general $k$-variable problem to a 2-variable problem. 

In a similar vein, we conjecture that it is possible to extend the definition of a multi-variable transition guard to encompass \textit{any} boolean function of inputs of the form $\{\texttt{true}, \mvlguard[\texttt{x}_i], \mvgguard[\texttt{x}_i]\}$; indeed, satisfying the same system of linear constraints would still produce correct proofs of differential privacy, but it is unclear how to demonstrate completeness for guards that can be arbitrary boolean functions. 


\section{Conclusion}
We have shown how to use coupling techniques to prove privacy for a class of SVT-like programs first defined in \cite{chadhaLinearTimeDecidability2021} and discovered that couplings additionally characterize this class. We additionally showed that this can be done tractably, and that couplings can help provide lower bounds on privacy costs of these algorithms. 

Future work most naturally would focus on extensions of the program model. For the model, potential areas include removing the requirement for output to be deterministic of a path through the automaton, which would allow for algorithms such as Report Noisy Max to be captured by the model. Similarly, the alphabet of the automaton could be expanded to incorporate more than comparisons between two real numbers. 
Such extensions would naturally also require extensions of the class of couplings we define here, which are limited to ``shifts''. 

Additionally, we believe that couplings should completely characterize GDiPAs as well as DiPAs; proving this requires showing that a lack of well-formedness in any single variable generates a counterexample to privacy. 
In this vein, we would like to explore using couplings to \textit{disprove} privacy; the fact that shift couplings completely characterize DiPAs hints at the possibility of ``anti-couplings'' to generate counterexamples.

\section{Related Work}

\textbf{Deciding Differential Privacy}

Due to the known undecidability result of \cite{bartheDecidingDifferentialPrivacy2020}, decision procedures for checking the privacy of putatively DP algorithms must inherently limit their program model. As mentioned, previous authors have developed a program model (DiPA) directly equivalent to our program model \cite{chadhaLinearTimeDecidability2021}; 
an extension to this model introduced multiple threshold variables, but only allowed for conjunctions between variable guards and required input variables to be correlated \cite{chadhaDecidingDifferentialPrivacy2023}. Interestingly, \cite{chadhaDecidingDifferentialPrivacy2023} still finds a decision procedure for privacy even with an arbitrary number of variables. 
Similarly, \cite{bartheDecidingDifferentialPrivacy2020} develop a decision procedure for programs with finite domains and ranges under a Markov chain based program model, exploiting a decidable fragment of the first-order theory of the reals with exponentiation. 

Other results have established that the problem of deciding differential privacy, even for more limited models, is in general hard. For example, \cite{chadhaDecidingDifferentialPrivacy2023} show that deciding privacy for their $k$-variable model is $PSPACE$-complete, \cite{gaboardiComplexityVerifyingLoopFree2020} find that the decision problem of privacy for loop-free programs is $coNP^{\#P}$-complete,
and \cite{bunComplexityVerifyingBoolean2022} show that the same problem for boolean programs is $PSPACE$-complete. 

There is also a long line of work oriented toward verifying DP using program logics (see, for example \cite{reedDistanceMakesTypes2010,wangCheckDPAutomatedIntegrated2020,wangProvingDifferentialPrivacy2019,zhangTestingDifferentialPrivacy2020,zhangLightDPAutomatingDifferential2017}). 
This approach allows for privacy proofs to be generated for a larger class of programs, but at the cost of completeness; while tools like \cite{wangCheckDPAutomatedIntegrated2020} can produce counterexamples to privacy, they do not guarantee that either a proof or counterexample can be produced in all cases.

\textbf{Probabilistic Couplings and Approximate Liftings}

Probabilistic couplings were first developed by Wolfgang Doeblin in the 1930's and have since become widely applied in statistics and probability theory more generally; for examples of their applications, see a standard reference text like \cite{lindvallLecturesCouplingMethod2002}. 

The use of probabilistic couplings has become more common in a computer science context in recent years; the specific line of work connecting approximate liftings and differential privacy begun with \cite{bartheProvingDifferentialPrivacy2016}; 
further work since then has expanded the scope of approximate liftings \cite{bartheRelationalStarLiftings2019,hsuProbabilisticCouplingsProbabilistic2017} and used liftings as the basis for constructive proofs of privacy of programs~\cite{albarghouthiSynthesizingCouplingProofs2017,albarghouthiConstraintBasedSynthesisCoupling2018}.


\textbf{Privacy Lower Bounds}

Some work has been done on automatically finding lower bounds on privacy cost through the generation of ``costly'' inputs, providing a mechanism for ``completing'' privacy proof generation techniques. 
\cite{bichselDPFinderFindingDifferential2018,bichselDPSniperBlackBoxDiscovery2021,dingDetectingViolationsDifferential2018,niuDPOptIdentifyHigh2022} all provide methods of automatically constructing counterexample pairs of inputs that lower bound the privacy parameter of an algorithm, either through static analysis of an algorithm or black-box access to the algorithm. 
Perhaps most notably, recent work has focused on ``auditing'' differentially-private machine learning algorithms \cite{luGeneralFrameworkAuditing2022,loknaGroupAttackAuditing2023,loknaGroupAttackAuditing2023,steinkePrivacyAuditingOne2023}; with black-box access to a machine learning model, an auditing protocol checks if the model satisfies its claimed privacy bounds by attempting to produce counterexamples that demonstrate a higher privacy lower bound than claimed. 

{\color{red} need to reformat some citations at some point}
\bibliography{./dipalibrary}



\section{Appendix A: Single Variable Programs}

\subsection{Couplings for Transitions and Straight Line Programs}

\begin{lemma}[Detailed version of lemma \ref{simplifiedIndTransitionCoupling}]\label{indTransitionCoupling}
    Let $X\brangle{1}\sim \Lap(\mu\brangle{1}, \frac{1}{d_x\varepsilon}), X\brangle{2}\sim\Lap(\mu\brangle{2}, \frac{1}{d_x\varepsilon})$ be random variables representing possible initial values of $\texttt{x}$ and let $t = (c, \sigma, \tau)$ be a transition from some valid transition alphabet $\Sigma_T$.
    Let $P(t) = (d_t, d_t')$.

    Let $\texttt{in}\brangle{1}\sim \texttt{in}\brangle{2}$ be an arbitrary valid adjacent input pair and let $o\brangle{1}$, $o\brangle{2}$ be random variables representing possible outputs of $t$ given inputs $\texttt{in}\brangle{1}$ and $\texttt{in}\brangle{2}$, respectively. 

    Then $\forall \varepsilon>0$ and for all $\gamma_x, \gamma_t, \gamma_t'\in [-1, 1]$ that satisfy the constraints \[
        \begin{cases}
          \gamma_t\leq\gamma_x & c = \lguard[\texttt{x}]\\
          \gamma_t\geq\gamma_x & c = \gguard[\texttt{x}]\\
          \gamma_t=0 & \sigma = \texttt{insample}\\
          \gamma_t'=0 & \sigma = \texttt{insample}'
        \end{cases},
      \]
      the lifting $o\brangle{1}\{(a, b): a=\sigma\implies b=\sigma\}^{\#d\varepsilon}o\brangle{2}$ is valid for $d = (|\mu\brangle{1}-\mu\brangle{2}+\gamma_x|)d_x+(|-\texttt{in}\brangle{1}+\texttt{in}\brangle{2}-\gamma_t|)d_t+(|-\texttt{in}\brangle{1}+\texttt{in}\brangle{2}-\gamma_t'|)d_t'$.
\end{lemma}

\begin{proof}
Fix $\varepsilon>0$.

We can first create the lifting $X\brangle{1}+\gamma_x (=)^{\#(|\mu\brangle{1}-\mu\brangle{2}+\gamma_x|)d_x\varepsilon}X\brangle{2}$. 

Additionally, create the lifting $z\brangle{1} (=)^{\#(|-\texttt{in}\brangle{1}+\texttt{in}\brangle{2}-\gamma_t|)d_t\varepsilon}z\brangle{2} - \texttt{in}\brangle{1}+\texttt{in}\brangle{2}-\gamma_t$, which is equivalent to creating the lifting $\texttt{insample}\brangle{1} +\gamma_t{(=)}^{\#(|-\texttt{in}\brangle{1}+\texttt{in}\brangle{2}-\gamma_t|)d_t\varepsilon}\texttt{insample}\brangle{2}$.

Finally, create the lifting $z'\brangle{1} (=)^{\#(|-\texttt{in}\brangle{1}+\texttt{in}\brangle{2}-\gamma_t'|)d_t'\varepsilon}z'\brangle{2} - \texttt{in}\brangle{1}+\texttt{in}\brangle{2}-\gamma_t'$. As before, this is equivalent to creating the lifting $\texttt{insample}'\brangle{1} +\gamma_t'{(=)}^{\#(|-\texttt{in}\brangle{1}+\texttt{in}\brangle{2}-\gamma_t'|)d_t'\varepsilon}\texttt{insample}'\brangle{2}$.

Thus, we emerge with three key statements to leverage:\begin{itemize}
    \item $X\brangle{1} + \gamma_x = X\brangle{2}$
    \item $z\brangle{1} = z\brangle{2} - \texttt{in}\brangle{1}+\texttt{in}\brangle{2}-\gamma_t$
    \item $z'\brangle{1} = z'\brangle{2} - \texttt{in}\brangle{1}+\texttt{in}\brangle{2}-\gamma_t'$
\end{itemize}

So if $c=\lguard[\texttt{x}]$ and $\gamma_t\leq \gamma_x$, then \begin{align*}
    \texttt{insample}\brangle{1}<X\brangle{1}&\implies \texttt{in}\brangle{1}+z\brangle{1}<X\brangle{1}\\
    &\implies \texttt{in}\brangle{1}+z\brangle{2}-\texttt{in}\brangle{1}+\texttt{in}\brangle{2}-\gamma_t<X\brangle{2}-\gamma_x\\
    &\implies \texttt{insample}\brangle{2}<X\brangle{2}
\end{align*}

Similarly, if $c=\gguard[\texttt{x}]$ and $\gamma_t\geq \gamma_x$, then \begin{align*}
    \texttt{insample}\brangle{1}\geq X\brangle{1}&\implies \texttt{in}\brangle{1}+z\brangle{1}\geq X\brangle{1}\\
    &\implies \texttt{in}\brangle{1}+z\brangle{2}-\texttt{in}\brangle{1}+\texttt{in}\brangle{2}-\gamma_t\geq X\brangle{2}-\gamma_x\\
    &\implies \texttt{insample}\brangle{2}\geq X\brangle{2}
\end{align*}

With these liftings, we have ensured that if the first run satisfies the guard of $t$, then the second run does as well. 

As noted, if $\sigma \in \Gamma$ and the first run taking transition $t$ implies that the second run does as well, then $o\brangle{1} = \sigma \implies o\brangle{2}=\sigma$ trivially.

Now, if $\sigma=\texttt{insample}$ and $\gamma_t=0$, then clearly we have that $\texttt{insample}\brangle{1}=\texttt{insample}\brangle{2}$, so for all $a\in \RR$, $o\brangle{1} = a\implies o\brangle{2} = a$.

Similarly, if $\sigma=\texttt{insample}'$ and $\gamma_t'=0$, we have that for all $a\in \RR$, $o\brangle{1} = a\implies o\brangle{2} = a$.

Thus, given the constraints \[
  \begin{cases}
    \gamma_t\leq\gamma_x & c = \lguard[\texttt{x}]\\
    \gamma_t\geq\gamma_x & c = \gguard[\texttt{x}]\\
    \gamma_t=0 & \sigma = \texttt{insample}\\
    \gamma_t'=0 & \sigma = \texttt{insample}'
  \end{cases},
\]
we have shown that the lifting $o\brangle{1}\{(a, b): a=\sigma\implies b=\sigma\}^{\#d\varepsilon}o\brangle{2}$ is valid, where the cost $d = (|\mu\brangle{1}-\mu\brangle{2}+\gamma_x|)d_x+(|-\texttt{in}\brangle{1}+\texttt{in}\brangle{2}-\gamma_t|)d_t+(|-\texttt{in}\brangle{1}+\texttt{in}\brangle{2}-\gamma_t'|)d_t'$. 

\end{proof}


\begin{lemma}[Detailed version of lemma \ref{simplifiedMultTransitionsCouplingProof}]\label{multTransitionsCouplingProof}
    Let $\rho = t_0\ldots t_{n-1}$ be a initialized SLP of length $n$. 
    Let $\texttt{in}\brangle{1}\sim \texttt{in}\brangle{2}$ be arbitrary adjacent input sequences of length $n$. Additionally, fix some potential output $\sigma$ of $\rho$ of length $n$ and let $\sigma\brangle{1}$, $\sigma\brangle{2}$ be random variables representing possible outputs of $\rho$ given inputs $\texttt{in}\brangle{1}$ and $\texttt{in}\brangle{2}$, respectively. Additionally, for all $t_i$, let $P(t_i) = (d_i, d_i')$.

    Then $\forall \varepsilon>0$ and for all $\{\gamma_i, \gamma_i'\}_{i=0}^{n-1}$ that, for all $i$, satisfy the constraints \[
        \begin{cases}
          \gamma_i\leq\gamma_{at(i)} & c_i = \lguard[\texttt{x}]\\
          \gamma_i\geq\gamma_{at(i)} & c_i = \gguard[\texttt{x}]\\
          \gamma_i=0 & \sigma_i = \texttt{insample}\\
          \gamma_i'=0 & \sigma_i = \texttt{insample}'
        \end{cases},
      \]
      the lifting $\sigma\brangle{1}\{(a, b): a=\sigma\implies b=\sigma\}^{\#d\varepsilon}\sigma\brangle{2}$ is valid for $d = \sum_{i=0}^{n-1}(|-\texttt{in}_i\brangle{1}+\texttt{in}_i\brangle{2}-\gamma_i|)d_i+(|-\texttt{in}_i\brangle{1}+\texttt{in}_i\brangle{2}-\gamma_i'|)d_i'$, and therefore $t$ is $d\varepsilon$-differentially private. 
\end{lemma}
\begin{proof}
    From the proof of lemma \ref{indTransitionCoupling}, we know that we can create the couplings $\texttt{insample}_i\brangle{1} +\gamma_i{(=)}^{\#(|-\texttt{in}_i\brangle{1}+\texttt{in}_i\brangle{2}-\gamma_i|)d_i\varepsilon}\texttt{insample}_i\brangle{2}$ and $\texttt{insample}_i'\brangle{1} +\gamma_i'{(=)}^{\#(|-\texttt{in}_i\brangle{1}+\texttt{in}_i\brangle{2}-\gamma_i'|)d_i'\varepsilon}\texttt{insample}_i'\brangle{2}$ for all $q_i$ in $\rho$. 

    Additionally, for some fixed $q_i$ in $\rho$, if we have the coupling $\texttt{x}_i\brangle{1}+\gamma_x (=)^{\#(|\hat{\mu_i}\brangle{1}-\hat{\mu_i}\brangle{2}+\gamma_x|)\hat{d_i}\varepsilon}x_i\brangle{2}$, where $\texttt{x}_i\brangle{1}\sim \Lap(\hat{\mu_i}\brangle{1}, \frac{1}{\hat{d_i}\varepsilon})$ and $\texttt{x}_i\brangle{2}\sim \Lap(\hat{\mu_i}\brangle{2}, \frac{1}{\hat{d_i}\varepsilon})$, then subject to the constraints \[
        \begin{cases}
          \gamma_i\leq\gamma_x & c_i = \lguard[\texttt{x}]\\
          \gamma_i\geq\gamma_x & c_i = \gguard[\texttt{x}]\\
          \gamma_i=0 & \sigma_i = \texttt{insample}_i\\
          \gamma_i'=0 & \sigma_i = \texttt{insample}_i'
        \end{cases},
      \]
    the coupling $\sigma_i\brangle{1}\{(a, b): a=\sigma_i\implies b=\sigma_i\}^{\#d\varepsilon}\sigma_i\brangle{2}$ is valid for some $d$. 

    Indeed, note that for all $i$, $\texttt{x}_i = \texttt{insample}_{at(i)}$ by the semantics of an SLP. Thus, we have that $\texttt{x}_i\brangle{1}+\gamma_x (=)^{\#(|-\texttt{in}_{at(i)}\brangle{1}+\texttt{in}_{at(i)}\brangle{2}+\gamma_{at(i)}|)d_{at(i)}\varepsilon}x_i\brangle{2}$, and we must satisfy the constraints \[
        \begin{cases}
          \gamma_i\leq\gamma_{at(i)} & c_i = \lguard[\texttt{x}]\\
          \gamma_i\geq\gamma_{at(i)} & c_i = \gguard[\texttt{x}]\\
          \gamma_i=0 & \sigma_i = \texttt{insample}_i\\
          \gamma_i'=0 & \sigma_i = \texttt{insample}_i'
        \end{cases}
      \]
      for all $i$.

    Thus, we can put all of these couplings together to show that the coupling $\sigma_i\brangle{1}\{(a, b): a=\sigma_i\implies b=\sigma_i\}^{\#d\varepsilon}\sigma_i\brangle{2}$ is valid for some $d>0$.

    In particular, note that we have created at most one pair of couplings (for $\texttt{insample}$ and $\texttt{insample}'$) for each $q_i$. Thus, the total coupling cost associated with each $q_i$ is at most $(|-\texttt{in}_i\brangle{1}+\texttt{in}_i\brangle{2}-\gamma_i|)d_i+(|-\texttt{in}_i\brangle{1}+\texttt{in}_i\brangle{2}-\gamma_i'|)d_i'$, 
    which gives us an overall coupling cost of $d = \sum_{i=0}^{n-1}(|-\texttt{in}_i\brangle{1}+\texttt{in}_i\brangle{2}-\gamma_i|)d_i+(|-\texttt{in}_i\brangle{1}+\texttt{in}_i\brangle{2}-\gamma_i'|)d_i'$.
\end{proof}

\subsection{DiPA and Programs}

We copy the definitions of leaking cycles, leaking pairs, disclosing cycles, and privacy violating paths from \cite{chadhaLinearTimeDecidability2021} for clarity. 

\begin{defn}[Leaking Cycles~\cite{chadhaLinearTimeDecidability2021}]
    A path $\rho = q_0\to\ldots \to q_n$ in a DiPA $A$ is a leaking path if there exist indices $i, j$ where $0\leq i < j < n$ such that the $i$'th transition $q_i\to q_{i+1}$ in $\rho$ is an assignment transition and the guard of the transition $q_j \to q_{j+1}$ is not $\texttt{true}$. If $\rho$ is also a cycle, then we call it a leaking cycle.
\end{defn}

\begin{defn}[\cite{chadhaLinearTimeDecidability2021}]
    A cycle $\rho$ in a DiPA $A$ is an \lcycle~if for some transition $q_i\to q_{i+1}$ in $\rho$, $\guard(q_i\to q_{i+1}) = \lguard[\texttt{x}]$. Similarly, $\rho$ is a \gcycle~if for some transition $q_i\to q_{i+1}$ in $\rho$, $\guard(q_i\to q_{i+1}) = \gguard[\texttt{x}]$.
    
    Additionally, a path $\rho$ of a DiPA $A$ is an \texttt{AL}-path (respectively, \texttt{AG}-path) if all assignment transitions in $\rho$ have guard $\lguard[\texttt{x}]$ (respectively, $\gguard[\texttt{x}]$)
\end{defn}

\begin{defn}[Leaking Pairs \cite{chadhaLinearTimeDecidability2021}]
    A pair of cycles $(C, C')$ is called a leaking pair if one of the following two conditions is satisfied.
    \begin{enumerate}
        \item $C$ is an \lcycle, $C'$ is a \gcycle~and there is an \texttt{AG}-path from a location in $C$ to a location in $C'$.
        \item $C$ is a \gcycle, $C'$ is an \lcycle~and there is an \texttt{AL}-path from a location in $C$ to a location in $C'$.
    \end{enumerate}
\end{defn}

\begin{defn}[Disclosing Cycles \cite{chadhaLinearTimeDecidability2021}]
    A cycle $C = q_0\to \ldots \to q_n \to q_0$ of a DiPA $A$ is a disclosing cycle if there is an $i$, $0 \leq i < |C|$ such that $q_i\in Q_{in}$ and the transition $q_i\to q_{i+1}$ that outputs either \texttt{insample} or \texttt{insample}'.
\end{defn}

\begin{defn}[Privacy Violating Paths \cite{chadhaLinearTimeDecidability2021}]
    We say that a path $\rho = q_0\to\ldots \to q_n $ of a DiPA $A$ is a privacy violating path if one of the following conditions hold:
    \begin{itemize}
        \item  $tail(\rho)$ is an \texttt{AG}-path (resp., \texttt{AL}-path) such that $last(\rho)$ is in a \gcycle~(resp., \lcycle) and the 0th transition $q_0\to q_1$ is an assignment transition that outputs \texttt{insample}.
        \item $\rho$ is an \texttt{AG}-path (resp., \texttt{AL}-path) such that $q_n$ is in a \gcycle~(resp., \lcycle) and the first transition $q_0\to q_1$ has guard $\lguard[\texttt{x}]$ (resp., $\gguard[\texttt{x}]$) and outputs \texttt{insample}
        \item $\rho$ is an \texttt{AG}-path (resp., \texttt{AL}-path) such that $q_0$ is in an \lcycle~(resp., \gcycle) and the last transition $q_{n-1}\to q_n$ has guard $\gguard[\texttt{x}]$ (resp., $\lguard[\texttt{x}]$) and outputs \texttt{insample}
    \end{itemize}
\end{defn}


\begin{lemma}\label{unsatisfiableImpliesNotWellformedLemma}
    If a periodic program $L$ generated by $G_L$ satisfies output distinction and there does not exist a coupling strategy $C$ that satisfies the privacy constraint system for $L$, then $G_L$ must contain either a leaking cycle, a leaking pair, a disclosing cycle, or a privacy violating path. 
\end{lemma}

\begin{proof}[Proof of lemma \ref{unsatisfiableImpliesNotWellformedLemma}]
    Let $L$ be a periodic program in $P$ that does not have a coupling strategy that satisfies the privacy constraint system.

    Consider a ``maximally'' satisfied coupling strategy $C=(\mathbf{\gamma}, \mathbf{\gamma}')$ for $L$; i.e. there is no other coupling strategy $C'$ for $L$ such that $C'$ satisfies more constraints than $C$. By lemma \ref{integersAreEnoughForATLemma}, we are allowed to only consider coupling strategies $C=(\gamma, \gamma')$ such that, for all $i\in AT(A)$, $\gamma_i \in \{-1, 0, 1\}$. 

    Fix some SLP $\rho$ in $A$ such that at least one constraint is not satisfied by $C$ as applied to $\rho$.

    By assumption, at least one constraint is unsatisfied by $C$. We will show that in every case, $A$ must contain at least one of a leaking cycle, leaking pair, disclosing cycle, or privacy violating path. By theorem \ref{DiPACounterexamplesThm}, this is sufficient to show that $A$ is not $d\varepsilon$-differentially private for any $d>0$.

    \textbf{Case 1: (1) is unsatisfied for $\gamma_i$}
    
    In this case, $c_i = \lguard[\texttt{x}]$ and $\gamma_i > \gamma_{at(i)}$. Note that $\gamma_{at(i)} \neq 1$. 

    We can assume that for all assignment transitions $t_{at(k)}$ in $\rho$ that $t_{at(k)}$ is not in a cycle, since otherwise there would be a leaking cycle in $A$. 

    \textbf{Case 1.1: $t_i$ is in a cycle}

    In this case, we can suppose that $t_i$ is not an assignment transition and $t_i$ does not output $\texttt{insample}$ or $\texttt{insample}'$, since otherwise either a leaking cycle or a disclosing cycle would clearly exist in $A$. We can thus additionally assume that constraint (5) is satisfied for $\gamma_i$. 
    
    Noe that the cycle containing $t_i$ is also an $\texttt{L}$-cycle by definition.

    Then attempting to resolve (1) for $\gamma_i$ by setting $\gamma_{at(i)} = 1$ must violate another constraint. In particular, either constraint (1) or (3) for $\gamma_{at(i)}$ or constraint (2) for some $\gamma_j$ such that $at(j) = at(i)$ must be newly violated. Note that constraint (5) for $\gamma_{at(i)}$ cannot be violated since we assumed that $t_{at(i)}$ is not in a cycle. 

    \textbf{Case 1.1.1: setting $\gamma_{at(i)} = 1$ violates constraint (1) for $\gamma_{at(i)}$}

    Let $t_{at(k)}$ be the earliest assignment transition before $t_{at(i)}$ such that, for all $at(k)\leq at(l)< at(i)$, $\gamma_{at(l)} <1$ and $c_{at(l)} = \lguard[\texttt{x}]$. Then there must be \textit{some} $\gamma_{at(l)}$ such that setting $\gamma_{at(l)} = 1$ would violate constraint (2) for some $\gamma_{l'}$ such that $at(l') = at(l)$. 

    Observe that $c_{l'} = \gguard[\texttt{x}]$ and there is an $\texttt{AL}$-path from $t_{l'}$ to $t_i$. 

    Then setting $\gamma_{l'}= 1$ must violate either constraint (3) or constraint (5) for $\gamma_{l'}$. If constraint (3) is violated, then $\gamma_{l'}$ is a transition with guard $\gguard[\texttt{x}]$ that outputs $\texttt{insample}$, so there is a privacy violating path from $t_{l'}$ to $t_i$. Otherwise if constraint (5) is violated, then $\gamma_{l'}$ is in a \gcycle, so there is a leaking pair composed of the cycles containing $t_{l'}$ and $t_i$, repectively. 

    \textbf{Case 1.1.2: Setting $\gamma_{at(i)}=1$ would violate (3) for $\gamma_{at(i)}$}

    Then $\gamma_{at(i)}$ is an assignment transition that outputs $\texttt{insample}$. Further, the path from $t_{at(i)}$ to $t_i$ is an $\texttt{AL}$-path, since there are no transitions on it. Thus, there is a privacy violating path from $t_i{at(i)}$ to $t_i$

    \textbf{Case 1.1.3: Setting $\gamma_{at(i)}=1$ would violate (2) for some $\gamma_j$ such that $at(j)= at(i)$}

    Note that, if $i<j$, the path from $t_i$ to $t_j$ (or vice versa, if $j<i$) is both an $\texttt{AL}$ and $\texttt{AG}$-path.

    Setting $\gamma_{j}= 1$ must violate either constraint (3) or constraint (5) for $\gamma_{j}$. 
    
    If constraint (3) is violated, then $\gamma_{j}$ is a transition with guard $\gguard[\texttt{x}]$ that outputs $\texttt{insample}$. Thus if $i<j$, there is a privacy violating path from $t_i$ to $t_j$ and if $j<i$, there is a privacy violating path from $t_j$ to $t_i$. 
    
    Otherwise if constraint (5) is violated, then $\gamma_{j}$ is in a \gcycle, so there is a leaking pair composed of the cycle containing $t_j$ and the cycle containing $t_i$ if $j<i$ or vice versa if $j>i$. 

    \textbf{Case 1.2: $t_i$ is not in a cycle}

    Note that $t_i$ must either be an assignment transition or output $\texttt{insample}$ or both, since otherwise, setting $\gamma_i = \gamma_{at(i)}$ would resolve constraint (1) for $\gamma_i$ without violating any other constraint. 

    \textbf{Case 1.2.1: $t_i$ outputs $\texttt{insample}$ and $t_i$ is an assignment transition}

    In this case, attempting to resolve constraint (1) without violating constraint (3) for $\gamma_i$ by setting $\gamma_i = \gamma_{at(i)} = 0$ must violate some other constraint. In particular, setting $\gamma_{at(i)} = 0$ can newly violate constraint (1) for $\gamma_{at(i)}$ or constraint (2) for some $\gamma_j$ such that $at(j) = at(i)$; note that setting $\gamma_{at(i)}=0$ cannot \textit{newly} violate constraint (1) for some $\gamma_j$ such that $at(j) = at(i)$. 
    Alternatively, setting $\gamma_i = 0$ could potentially newly violate either constraint (1) or constraint (2) for some $\gamma_j$ such that $at(j) = i$. 

    \textbf{Case 1.2.2.1: Setting $\gamma_{at(i)} =0$ violates constraint (1) for $\gamma_{at(i)}$}

    Let $t_{at(k)}$ be the earliest assignment transition before $t_{at(i)}$ such that, for all $at(k)\leq at(l)< at(i)$, $\gamma_{at(l)} = -1$ and $c_{at(l)} = \lguard[\texttt{x}]$. Then there must be \textit{some} $\gamma_{at(l)}$ such that setting $\gamma_{at(l)} = 0$ would violate constraint (2) for some $\gamma_{l'}$ such that $at(l') = at(l)$.  
    Additionally, note that setting $\gamma_{l'} = \gamma_{at(l)} = 0$ can only violate constraint (5) for $\gamma_{l'}$, since $\gamma_{l'}$ cannot be an assignment transition. 
    
    Thus, $t_{l'}$ is in a cycle, so the cycle containing $t_{l'}$ is a \gcycle. Note that the path from $t_{l'}$ to $t_i$ is an $\texttt{AL}$-path. Therefore, there is a privacy violating path from $t_{l'}$ to $t_i$.

    \textbf{Case 1.2.2.2: Setting $\gamma_{at(i)} =0$ violates constraint (2) for some $\gamma_j$ such that $at(j) = at(i)$}

    Note that $j\neq i$, meaning that $t_j$ is not an assignment transition. Then setting $\gamma_j = \gamma_{at(i)} = 0$ must violate constraint (5) for $\gamma_j$; this means that $t_j$ is in a \gcycle. 

    If $i<j$, then the path from $t_i$ to $t_j$ is an $\texttt{AL}$-path, so it is also a privacy violating path.

    Otherwise if $j<i$, then the path from $t_j$ to $t_i$ is an $\texttt{AG}$ path, so it is also a privacy violating path.

    \textbf{Case 1.2.2.3: Setting $\gamma_i =0$ violates constraint (1) for some $\gamma_j$ such that $at(j) = i$}

    If $\gamma_j$ is not an assignment transition, then setting $\gamma_j = \gamma_i = 0$ must violate constraint (5) for $\gamma_j$, so $t_j$ is in an \lcycle. Then there is a privacy violating path from $t_i$ to $t_j$, since the path from $t_{i+1}$ to $t_j$ is an $\texttt{AL}$-path by virtue of not containing any assignment transitions. 

    Otherwise if $t_j$ is an assignment transition, then $\gamma_j$ must originally be set to 1. Let $t_{at(k)}$ be the latest assignment after $t_{i}$ such that, for all $i\leq at(l)< at(k)$, $\gamma_{at(l)} = 1$ and $c_{at(l)} = \lguard[\texttt{x}]$. Then there must be \textit{some} $\gamma_{at(l)}$ such that setting $\gamma_{at(l)} = 0$ would violate constraint (1) for some $\gamma_{l'}$ such that $at(l') = at(l)$.  
    Additionally, note that setting $\gamma_{l'} = \gamma_{at(l)} = 0$ can only violate constraint (5) for $\gamma_{l'}$, since $\gamma_{l'}$ cannot be an assignment transition. 

    Then $\gamma_{l'}$ must be in an \lcycle. Since the path from $t_i$ to $t_{l'}$ is an $\texttt{AL}$-path, there is a privacy violating path from $t_i$ to $t_{l'}$.

    \textbf{Case 1.2.2.4: Setting $\gamma_i =0$ violates constraint (2) for some $\gamma_j$ such that $at(j) = i$}

    This case is exactly symmetric to case 1.2.2.3.

    \textbf{Case 1.2.2: $t_i$ outputs $\texttt{insample}$ and $t_i$ is not an assignment transition}

    We can assume that $\gamma_{at(i)} = -1$ originally, since otherwise, setting $\gamma_i = 0$ would resolve constraint (1) without violating any additional ones.

    Thus attempting to resolve constraint (1) while preserving constraint (3) for $\gamma_i$ by setting $\gamma_{at(i)}=\gamma_i =0$ must violate constraint (1) for $\gamma_{at(i)}$. 
   
    Let $t_{at(k)}$ be the earliest assignment transition before $t_{at(i)}$ such that, for all $at(k)\leq at(l)< at(i)$, $\gamma_{at(l)} = -1$ and $c_{at(l)} = \lguard[\texttt{x}]$. Then there must be some $\gamma_{at(l)}$ such that setting $\gamma_{at(l)} = 0$ would violate constraint (2) for some $\gamma_{l'}$ such that $at(l') = at(l)$.  
    Additionally, note that setting $\gamma_{l'} = \gamma_{at(l)} = 0$ can only violate constraint (5) for $\gamma_{l'}$, since $\gamma_{l'}$ cannot be an assignment transition. 
    
    Thus, $t_{l'}$ is in a cycle, so the cycle containing $t_{l'}$ is a \gcycle. Note that the path from $t_{l'}$ to $t_i$ is an $\texttt{AL}$-path. Therefore, there is a privacy violating path from $t_{l'}$ to $t_i$.

    \textbf{Case 1.2.3: $t_i$ does not output $\texttt{insample}$ and $t_i$ is an assignment transition}

    In this case, attempting to resolve (1) by setting $\gamma_{at(i)} = 1$ must violate either constraint (1) or (3) for $\gamma_{at(i)}$, or constraint (2) for some $\gamma_j$ such that $at(j) = at(i)$. 

    Additionally, note that $\gamma_{at(i)} \in \{0, -1\}$. 

    \textbf{Case 1.2.3.1: $\gamma_{at(i)} =0$}

    Since originally, $\gamma_i > \gamma_{at(i)} \implies \gamma_i = 1$, we know that setting $\gamma_i = \gamma_{at(i)} = 0$ must violate constraint (1) for some $\gamma_j$ such that $at(j) = i$. If $t_{j}$ is not an assignment transition, then setting $\gamma_{j} = 0$ can only violate constraint (5) for $\gamma_{j}$, meaning that $t_j$ is in an \lcycle.

    Otherwise, if $t_j$ is an assignment transition, let $t_{at(k)}$ be the latest assignment transition after $t_{i}$ such that for all $j\leq at(l)< at(k)$, $\gamma_{at(l)} =1$ and $c_{at(k)} = \lguard[\texttt{x}]$. Then there must exist some $at(l), j\leq at(l)< at(k)$ such that setting $\gamma_{at(l)}=0$ would violate constraint (1) for some non-assignment $\gamma_{l'}$ where $at(l') = at(l)$. 

    Further, setting $\gamma_{l'} = 0$ must then violate constraint (5) for $\gamma_{l'}$, so $t_{l'}$ is in an \lcycle. 

    Therefore, there exists a $\texttt{AL}$-path from $t_i$ to some transition $t$ in an \lcycle. 

    \textbf{Case 1.2.3.1.1: Setting $\gamma_{at(i)} = 1$ would violate constraint (1) for $\gamma_{at(i)}$}

    Let $t_{at(j)}$ be the earliest assignment transition before $t_{at(i)}$ such that for all $at(j)\leq at(k)< at(i)$, $\gamma_{at(k)} =0$ and $c_{at(k)} = \lguard[\texttt{x}]$. Then there must exist some $at(k), at(j)\leq at(k)< at(i)$ such that setting $\gamma_{at(k)}=1$ would violate constraint (2) for some non-assignment $\gamma_l$ where $at(l) = at(k)$, so $c_l = \gguard[\texttt{x}]$

    Note that there is an $\texttt{AL}$-path from $t_l$ to $t_i$, and therefore an $\texttt{AL}$-path from $t_l$ to some transition $t_{o}$ in an \lcycle. 

    Further, setting $\gamma_l = \gamma_{at(k)} = 1$ must then violate either constraint (3) or (5) for $\gamma_l$. 

    If constraint (3) is violated, then $t_l$ outputs $\texttt{insample}$, so there is a privacy violating from $t_l$ to $t_o$.

    If constraint (5) is violated, then $t_l$ is in a \gcycle, so there is a leaking pair consisting of the cycle containing $t_l$ and the cycle containing $t_o$. 

    \textbf{Case 1.2.3.1.2: Setting $\gamma_{at(i)}=1$ would violate constraint (3) for $\gamma_{at(i)}$}

    Note that there is an $\texttt{AL}$ path from $t_{at(i)}$ to some transition $t_j$ such that $t_j$ is in an \lcycle.

    Then $t_{at(i)}$ is an assignment transition that outputs $\texttt{insample}$, so there is a privacy violating path from $t_{at(i)}$ to $t_j$. 
    
    \textbf{Case 1.2.3.1.3: Setting $\gamma_{at(i)}=1$ would violate constraint (2) for some $\gamma_j$ such that $at(j) = at(i)$}

    As before, note that there is an $\texttt{AL}$ path from $t_{j}$ to some transition $t_k$ such that $t_k$ is in an \lcycle.

    Then trying to set $\gamma_j = \gamma_{at(i)} = 1$ must violate either constraint (3) or constraint (5) for $\gamma_j$. If constraint (3) is violated, then $t_j$ outputs $\texttt{insample}$, so there is a privacy violating from $t_j$ to $t_k$. If constraint (5) is violated, then $t_j$ is in a \gcycle, so there is a leaking pair consisting of the cycle containing $t_j$ and the cycle containing $t_k$.  

    \textbf{Case 1.2.3.2: $\gamma_{at(i)} = -1$}

    Note that $\gamma_i \in \{0, 1\}$.

    First, if $\gamma_i = 0$, then setting $\gamma_i = -1$ must newly violate constraint (1) for some $\gamma_j$ where $at(j) = i$. If $t_{j}$ is not an assignment transition, then setting $\gamma_{j} = -1$ can only newly violate constraint (3) for $\gamma_{j}$, meaning that $t_j$ outputs $\texttt{insample}$.

    Otherwise, if $t_j$ is an assignment transition, let $t_{at(k)}$ be the latest assignment transition after $t_{i}$ such that for all $j\leq at(l)< at(k)$, $\gamma_{at(l)} =0$ and $c_{at(k)} = \lguard[\texttt{x}]$. Then there must exist some $at(l), j\leq at(l)< at(k)$ such that setting $\gamma_{at(l)}=-1$ would newly violate constraint (1) for some non-assignment $\gamma_{l'}$ where $at(l') = at(l)$; as before, this means that $t_{l'}$ outputs $\texttt{insample}$.  

    Otherwise, if $\gamma_i = 1$, then setting $\gamma_i = -1$ must newly violate constraint (1) for some $\gamma_j$ where $at(j) = i$. If $t_{j}$ is not an assignment transition, then setting $\gamma_{j} = -1$ can only newly violate constraint (5) for $\gamma_{j}$, meaning that $t_j$ is in an \lcycle. 

    Otherwise, if $t_j$ is an assignment transition, let $t_{at(k)}$ be the latest assignment transition after $t_{i}$ such that for all $j\leq at(l)< at(k)$, $\gamma_{at(l)} =0$ and $c_{at(k)} = \lguard[\texttt{x}]$. Then there must exist some $at(l), j\leq at(l)< at(k)$ such that setting $\gamma_{at(l)}=-1$ would newly violate constraint (1) for some non-assignment $\gamma_{l'}$ where $at(l') = at(l)$; as before, this means that $t_{l'}$ is in an \lcycle.  

    Thus, if $\gamma_i =0$, then there is $\texttt{AL}$-path from $t_i$ to some other transition that has guard $\lguard[\texttt{x}]$ and outputs $\texttt{insample}$. Otherwise, if $\gamma_i = 1$, there is an $\texttt{AL}$-path from $t_i$ to some other transition that is in an \lcycle. 

    \textbf{Case 1.2.3.2.1: Setting $\gamma_{at(i)} = \gamma_i$ would violate constraint (1) for $\gamma_{at(i)}$}

    Let $t_{at(j)}$ be the earliest assignment transition before $t_{at(i)}$ such that for all $at(j)\leq at(k)< at(i)$, $\gamma_{at(k)} =-1$ and $c_{at(k)} = \lguard[\texttt{x}]$. Then there must exist some $at(k), at(j)\leq at(k)< at(i)$ such that setting $\gamma_{at(k)}=\gamma_i$ would newly violate constraint (2) for some non-assignment $\gamma_l$ where $at(l) = at(k)$.

    First note that $c_l = \gguard[\texttt{x}]$ and there is an $\texttt{AL}$-path from $t_l$ to $t_i$. 
    
    If $\gamma_i = 0$, then setting $\gamma_l = \gamma_{at(k)} = \gamma_i = 0$ can only newly violate constraint (5) for $\gamma_l$. Thus, $\gamma_l$ is in a \gcycle. Since $\gamma_i = 0$, there exists some $t_{l'}$ such that there is an $\texttt{AL}$-path from $t_i$ to $t_{l'}$ and $t_{l'}$ has guard $\lguard[\texttt{x}]$ and outputs $\texttt{insample}$. Thus, there is an $\texttt{AL}$ path from $t_l$ to $t_{l'}$, and so there is a privacy violating path from $t_l$ to $t_{l'}$. 
    
    If $\gamma_i = 1$, then setting $\gamma_l = \gamma_{at(k)} = \gamma_i = 1$ can newly violate constraints (3) or (5) for $\gamma_l$. Further, since $\gamma_i =1$, there exists some $t_{l'}$ such that there is an $\texttt{AL}$-path from $t_i$ to $t_{l'}$ and $t_{l'}$ is in an \lcycle. Thus, there is an $\texttt{AL}$ path from $t_l$ to $t_{l'}$.
    
    If constraint (3) is newly violated, then $t_l$ is a transition with guard $\gguard[\texttt{x}]$ that outputs $\texttt{insample}$. Thus, there is a privacy violating path from $t_l$ to $t_{l'}$. 

    If constraint (5) is newly violated, then $t_l$ is in a \gcycle. Thus, there is a leaking pair composed of the cycles containing $t_l$ and $t_{l'}$, respectively.

    \textbf{Case 1.2.3.2.2: Setting $\gamma_{at(i)}=\gamma_i$ would violate constraint (3) for $\gamma_{at(i)}$}

    First, $t_{at(i)}$ is an assignment transition that outputs $\texttt{insample}$. Since $\gamma_i = 1$, there exists some $t_j$ such that there is an $\texttt{AL}$-path from $t_{at(i)}$ to $t_j$ and $t_j$ is in an \lcycle. Then there is a privacy violating path from $t_{at(i)}$ to $t_j$.

    \textbf{Case 1.2.3.2.3: Setting $\gamma_{at(i)}=\gamma_i$ would violate constraint (2) for some $\gamma_j$ such that $at(j) = at(i)$}

    Observe that $t_j$ is not an assignment transition and has guard $\gguard[\texttt{x}]$. Additionally, there is an $\texttt{AL}$-path from $t_j$ to $t_i$ since there are no assignments between $t_j$ and $t_i$. 

    If $\gamma_i =0$, then setting $\gamma_j = \gamma_{at(i)} = 0$ can only newly violate constraint (5) for $\gamma_j$. Thus, $\gamma_j$ is in a \gcycle. Since $\gamma_i = 0$, there exists some $t_{k}$ such that there is an $\texttt{AL}$-path from $t_i$ to $t_{k}$. Thus, there is an $\texttt{AL}$ path from $t_j$ to $t_{k}$, and so there is a leaking pair composed of the cycles containing $t_j$ and $t_{k}$, respectively. 

    If $\gamma_i = 1$, then setting $\gamma_j = \gamma_{at(i)}=1$ can newly violate constraints (3) or (5) for $\gamma_l$. Further, since $\gamma_i =1$, there exists some $t_{k}$ such that there is an $\texttt{AL}$-path from $t_i$ to $t_{k}$ and $t_{k}$ is in an \lcycle. Thus, there is an $\texttt{AL}$ path from $t_j$ to $t_{k}$.
    
    If constraint (3) is newly violated, then $t_j$ is a transition with guard $\gguard[\texttt{x}]$ that outputs $\texttt{insample}$. Thus, there is a privacy violating path from $t_j$ to $t_{k}$. 

    If constraint (5) is newly violated, then $t_j$ is in a \gcycle. Thus, there is a leaking pair composed of the cycles containing $t_j$ and $t_{k}$, respectively.

    \textbf{Case 2: (2) is unsatisfied for $\gamma_i$}

    This case is exactly symmetric to case (1).

    \textbf{Case 3: (3) is unsatisfied for $\gamma_i$}

    First note that if $t_i$ is in a cycle, then that cycle will be a disclosing cycle because $t_i$ outputs $\texttt{insample}$. Thus, we will assume that $t_i$ is not in a cycle.

    Because $C$ is maximal, setting $\gamma_i=0$ must violate at least one of constraints (1) or (2) for $\gamma_i$ or (1) for some $\gamma_l$ such that $at(l) = i$.

    \textbf{Case 3.1: Satisfying (3) for $\gamma_i$ would violate (1) for $\gamma_i$}

    This means that $\gamma_{at(i)}<0\implies \gamma_{at(i)} = -1$. Further, $c_i = \lguard[\texttt{x}]$. Then changing $\gamma_{at(i)}=0$ can newly violate constraints (1) or (5) for $\gamma_{at(i)}$ or constraint (2) for some $\gamma_j$ such that $at(j) = at(i)$.

    If constraint (5) is newly violated, then $t_{at(i)}$ is in a cycle. In particular, the cycle must be a leaking cycle; if $t_i$ and $t_{at(i)}$ are both contained in a cycle, then it must be leaking because $c_i = \lguard[\texttt{x}]$. Otherwise, there still must be some transition in the cycle containing $t_{at(i)}$ that has a non-$\texttt{true}$ guard since otherwise a path from $t_{at(i)}$ to $t_i$ could not exist. 

    By similar reasoning, we can assume that for every assignment transition $t_{at(j)}$before $t_{at(i)}$ on a initialized path to $t_i$, $t_{at(j)}$ is not in a cycle. 

    If constraint (1) is newly violated for $\gamma_{at(i)}$, then $c_{at(i)} = \lguard[\texttt{x}]$. Let $t_{at(j)}$ be the earliest assignment transition before $t_{at(i)}$ such that $\gamma_{at(l)} = -1$ and for all assignment transitions $t_{at(k)}$ between $t_{at(j)}$ and $t_{at(i)}$, $c_{at(k)} = \lguard[\texttt{x}]$ and $\gamma_{at(k)} = -1$. 
    
    Then there must exist some assignment transition $t_{at(k)}$, $at(j)\leq at(k)\leq at(i)$ between $t_{at(j)}$ and $t_{at(i)}$ such that setting $\gamma_{at(k)} = 0$ would newly violate constraint (2) for some $l$ where $at(l) = at(k)$. In particular, this must be because $t_l$ is in a cycle and setting $\gamma_l = 0$ would violate constraint (5). Thus, $t_l$ is in a $\texttt{G}$-cycle. Then there is an $\texttt{AL}$-path from $t_l$ to $t_i$, creating a privacy violating path from $t_l$ to $t_i$. 

    If changing $\gamma_{at(i)}$ from $-1$ to $0$ means that constraint (2) would be newly violated for some $\gamma_j$ such that $at(j) = at(i)$, note that $\gamma_j < 0$ and $c_j = \gguard[\texttt{x}]$. 
    
    So setting $\gamma_j = 0$ can violate either (2) for some $\gamma_l$ where $at(l) = j$ or (5) for $\gamma_j$. 

    If setting $\gamma_j = 0$ would violate constraint (2) for some $\gamma_l$ where $at(l) = j$, then let let $t_{at(m)}$ be the latest assignment transition after $t_{at(j)}$ such that $\gamma_{at(m)} = -1$ and for all assignment transitions $t_{at(k)}$ between $t_{at(j)}$ and $t_{at(m)}$, $c_{at(k)} = \gguard[\texttt{x}]$ and $\gamma_{at(k)} = -1$. 
    
    Then there must exist some assignment transition $t_{at(k)}$, $at(j)\leq at(k)\leq at(m)$ between $t_{at(j)}$ and $t_{at(m)}$ such that setting $\gamma_{at(k)} = 0$ would newly violate constraint (2) for some $l'$ where $at(l') = at(k)$. 
    In particular, this must be because $t_{l'}$ is in a cycle and setting $\gamma_l = 0$ would violate constraint (5). Thus, $t_l$ is in a $\texttt{G}$-cycle. Then there is an $\texttt{AG}$-path from $t_i$ to $t_l$, creating a privacy violating path from $t_i$ to $t_l$. 

    Otherwise, if setting $\gamma_j = 0$ would violate constraint (5) for $\gamma_j$, then $t_j$ is in a $\texttt{G}$-cycle. We can assume that $j\neq i$ because otherwise, the cycle containing $t_j$ would be a disclosing cycle. Additionally, note that there are no assignment transitions between $t_i$ and $t_j$ or vice versa, since $at(j) = at(i)$.
    Thus, if $j<i$, then there is an $\texttt{AL}$-path from $t_j$ to $t_i$, which forms a privacy violating path. Symmetrically, if $i<j$, then there is an $\texttt{AG}-$path from $t_i$ to $t_j$, which again forms a privacy violating path. 

    \textbf{Case 3.2: Satisfying (3) for $\gamma_i$ would violate (2) for $\gamma_i$}

    This case is exactly symmetric to case (3a).

    \textbf{Case 3.3: Satisfying (3) for $\gamma_i$ would violate (1) for some $\gamma_l$ where $at(l) = i$}

    Note that $t_i$ must be an assignment transition. Further, we know that $\gamma_l>0$ and $c_l = \lguard[\texttt{x}]$. 
    
    Because $C$ is maximal, setting $\gamma_l=0$ would now violate either constraint (1) for some $\gamma_{l'}$ where $at(l') = l$ or constraint (5) for $\gamma_l$. Note that because $\gamma_l>0$, constraint (2) cannot be newly violated for some $\gamma_{l'}$ where $at(l') = l$.

    If constraint (5) would be newly violated for $\gamma_l$, then $\gamma_l$ is in an $\texttt{L}$-cycle. Additionally, note that the path from $t_{i+1}$ to $t_l$ is an $\texttt{AL}$-path, so there is a privacy violating path from $t_i$ to $t_l$. 

    Otherwise, if setting $\gamma_l = 0$ would violate constraint (1) for some $\gamma_{l'}$ where $at(l')=l$, let $t_{at(j)}$ be the latest assignment transition such that $c_{at(j)} = \lguard[\texttt{x}]$ and $\gamma_{at(j)}<1$ and, for all assignment transitions $t_{at(k)}$ between $t_l$ and $t_{at(j)}$, $c_{at(k)} = \lguard[\texttt{x}]$ and $\gamma_{at(k)}<1$. 

    If $at(j) = l$, then $l'$ is not an assignment transition. Then, setting $\gamma_{l'} = 0$ could only violate constraint (5). In this case, as before, there is a privacy violating path from $t_i$ to $t_l$. 

    Otherwise, since $C$ is maximal, we cannot set $\gamma_{at(k)}=0$ for any $l<at(k)\leq at(j)$ without violating another constraint. In particular, there must be some $at(k)$ such that setting $\gamma_{at(k)} = 0$ would violate constraint (1) for some $\gamma_{k'}$ such that $at(k') = at(k)$. Note that there must be an \texttt{AL}-path from $t_i$ to $t_{k'}$. Then, as before, there must be a privacy violating path from $t_i$ to $t_{k'}$. 


    \textbf{Case 3.4: Satisfying (3) for $\gamma_i$ would violate (2) for some $\gamma_l$ where $at(l) = i$} 

    This case is exactly symmetric to case (3c).

    \textbf{Case 4: (4) is unsatisfied for $\gamma_i'$}
    
    Because $C$ is maximal, setting $\gamma_i'=0$ must violate some other constraint. In particular, this must mean that constraint (6) is now violated. However, this would imply that $t_i$ is in a cycle, and so the cycle containing $t_i$ would be a disclosing cycle.

    \textbf{Case 5: (5) is unsatisfied for $t_i$:} Because $C$ is maximal, we know that if $\gamma_i = -\texttt{in}_i\brangle{1}+\texttt{in}_i\brangle{2}$ then another constraint must be violated. In particular, at least one of constraints (1), (2), or (3) must be violated for $\gamma_i$. 
    
    \textbf{Case 5.1: Satisfying (5) for $t_i$ would violate (1)}

    If (1) is now violated, then either $t_i$ is an assignment transition or $c_i = \lguard[\texttt{x}]$ and $\gamma_{at(i)}<1$. If $t_i$ is an assignment transition, then the cycle containing $t_i$ has a transition with a non-$\texttt{true}$ guard ($t_i$) and an assignment transition, so it must be a leaking cycle. 

    Otherwise, if $t_i$ is not an assignment transition, $c_i = \lguard[\texttt{x}]$, and constraint (1) is violated for $\gamma_i$, we must have that $\gamma_{at(i)}<1$ due to other constraints.
    
    Consider all assignment transitions in $\rho$ before $t_i$. Note that if any such assignment transition is in a cycle, then that cycle must be a leaking cycle since either the assignment transition is in the same cycle as $t_i$ or there must be some non-$\texttt{true}$ transition in the cycle because otherwise $t_i$ is unreachable.

    So assume that all assignment transitions in $\rho$ before $t_i$ are not in a cycle. Then if $c_{at(i)} \neq \lguard[\texttt{x}]$, because $C$ is maximal, this must mean that $t_{at(i)}$ outputs $\texttt{insample}$. Note that the path from $t_{at(i)+1}$ to $t_i$ is an $\texttt{AL}$-path (since there are no assignment transitions on it) and $t_i$ is in an $\texttt{L}$-cycle since $t_i$ is in a cycle and $c_i = \lguard[\texttt{x}]$. 
    Then the path from $t_{at(i)}$ (an assignment transition that outputs $\texttt{insample}$) to $t_i$ is a privacy violating path. 

    If $c_{at(i)} = \lguard[\texttt{x}]$, then let $c_{at(j)}$ be the earliest assignment transition such that $c_{at(j)} = \lguard[\texttt{x}]$ and $\gamma_{at(j)} < 1$ and, for all assignment transitions $t_{at(k)}$ between $t_{at(j)}$ and $t_i$, $c_{at(k)} = \lguard[\texttt{x}]$ and $\gamma_{at(k)} < 1$. Note that such an $t_{at(j)}$ must exist. 

    If $t_{at(j)} = t_{at(i)}$, then setting $\gamma_{at(i)} =1$ must violate either constraint (2) for some other $\gamma_l$ such that $at(l)=at(i)$, or constraint (3) for $\gamma_{at(i)}$. Without loss of generality, we will assume that $l\neq i$. If constraint (3) would be violated, then as before, there exists a privacy violating path from $t_{at(j)}$ to $t_i$. 
    If constraint (2) would be violated for some $\gamma_l$ such that $at(l)=at(i)$, then either $t_l$ must output $\texttt{insample}$ or $t_l$ must be in a cycle. 
    
    Suppose that $i<l$; then that the path from $t_i$ to $t_l$ is both an $\texttt{AG}$-path and an $\texttt{AL}$-path (since there are no assignment transitions on it). Thus, if $t_l$ outputs $\texttt{insample}$, there exists a privacy violating path from $t_i$ to $t_l$ and if $t_l$ is in a cycle, then the cycle containing $t_i$ and the cycle containing $t_l$ together make up a leaking pair, since the cycle containing $t_l$ is a $\texttt{G}$-cycle by definition. 
    Symmetrically, if $l>i$, then either the path from $t_l$ to $t_i$ is a privacy violating path or the cycle containing $t_l$ and the cycle containing $t_i$ make up a leaking pair.
    
    Otherwise, note that the path from $t_{at(j)}$ to $t_i$ is an $\texttt{AL}-$path. Since $C$ is maximal, we cannot set $\gamma_{at(k)}=1$ for $\gamma_{at(j)}$ or for any of the other assignment transitions $t_{at(k)}$ between $t_{at(j)}$ and $t_i$ without violating another constraint. 
    In particular, there must be some $t_{at(k)}$ where $at(j)\leq at(k)<i$ such that setting $\gamma_{at(k)} = 1$ would mean that either constraint (2) for some $\gamma_l$ such that $at(l) = at(k)$ or constraint (3) would be violated for $\gamma_{at(k)}$. 
    If constraint (3) would be violated for $\gamma_{at(k)}$ then $t_{at(k)}$ outputs $\texttt{insample}$, so as before, there is a privacy violating path from $t_{at(k)}$ to $t_i$. Otherwise if constraint (2) would be violated for some $\gamma_l$ such that $at(l) = at(k)$, then as before, $\gamma_l$ must either output $\texttt{insample}$ or $t_l$ is in a cycle. 
    Just like before, this means that there must be either a privacy violating path from $t_l$ to $t_i$ or the cycle containing $t_l$ and the cycle containing $t_i$ together make up a leaking pair. 

    \textbf{Case 5.2: Satisfying (5) for $t_i$ would violate (2)}

    This case is exactly symmetric to case (5a).

    \textbf{Case 5.3: Satisfying (5) for $t_i$ would violate (3)}

    If (3) would be violated, then $t_i$ must output $\texttt{insample}$ and be a non-public transition. Then the cycle containing $t_i$ must be a disclosing cycle. 
    
    \textbf{Case 6: (6) is unsatisfied for $t_i$:} Because $C$ is maximal, we know that if $\gamma_i' = -\texttt{in}_i\brangle{1}+\texttt{in}_i\brangle{2}$ then another constraint must be violated for $\gamma_i'$. In particular, constraint (4) must be violated, since no other constraint involves $\gamma_i'$. 
    Then $t_i$ is a transition in a cycle that outputs $\texttt{insample}'$, so $A$ has a disclosing cycle.
\end{proof}


\subsection{Minimizing Coupling Cost}

\begin{prop}\label{costDependspathProp}
    There exist SLPs $\rho'=\pi\cdot \pi$, $\rho^*=\pi\cdot\pi^*$ that share a prefix SLP $\pi$ such that for all optimal coupling strategies $C'$ for $\rho'$ and $C^*$ for $\rho^*$, $C'$ and $C^*$ differ on the shift assigned to some transition in $\pi$. 

    In other words, the optimal strategy $C$ must assign different coupling strategies to occurances of the same transition in different SLPs. 
\end{prop}

\begin{proof}
    First, we define the following transitions that we use to construct our counterexample:
    \begin{align*}
        t_{init} &= (\texttt{true}, \bot, \texttt{true})\\
        t_{geq1} &= (\gguard[\texttt{x}], \top, \texttt{false})\\
        t_{leq1} &= (\lguard[\texttt{x}], \bot, \texttt{false})\\
        t_{geq2} &= (\gguard[\texttt{x}], \top, \texttt{false})\\
        t_{leq2} &= (\lguard[\texttt{x}], \bot, \texttt{false})
    \end{align*}

    For all transitions $t$, let $P(t) = (1, 1)$.

    Let $\rho_1 = t_{init}t_{geq1}t_{geq2}^n$ and $\rho_2 = t_{init}t_{leq1}t_{leq2}^n$, where $n\in \NN$. Observe that $\rho_1$ and $\rho_2$ share the prefix $t_{init}$. 
    
    We make the following observations: 
    \begin{itemize}
        \item The cost of any coupling strategy for $\rho_1$ or $\rho_2$ must be at least 2:
        
        Let $C_{\rho_1} = (\gamma, \gamma')$ be a coupling strategy for $\rho_1$. We can bound its cost as follows: 
        \begin{align*}
            cost(C_{\rho_1}) &= \max_{\texttt{in}\brangle{1}\sim\texttt{in}\brangle{2}}\sum_{i=0}^{n+2}(|-\texttt{in}_i\brangle{1}+\texttt{in}_i\brangle{2}-\gamma_i(\texttt{in}_i\brangle{1}, \texttt{in}_i\brangle{2}))\\&\qquad+(|-\texttt{in}_i\brangle{1}+\texttt{in}_i\brangle{2}-\gamma_i'(\texttt{in}_i\brangle{1}, \texttt{in}_i\brangle{2})|)\\
            &\geq \max_{\texttt{in}\brangle{1}\sim\texttt{in}\brangle{2}} \sum_{i=0}^{n+2}(|-\texttt{in}_i\brangle{1}+\texttt{in}_i\brangle{2}-\gamma_i(\texttt{in}_i\brangle{1}, \texttt{in}_i\brangle{2})|)\\
            &= \max_{\Delta \in [-1, 1]^{n+2}} \sum_{i=0}^{n+2}(|\Delta_i-\gamma_i(0, \Delta_i)|)\\
            &\geq |1 - \gamma_0(0, 1)| + \sum_{i=1}^{n+2}|-1-\gamma_i(0, -1)|\\
            &= 1 - \gamma_0(0, 1) + \sum_{i=1}^{n+2} (1+\gamma_i(0, -1))\\
            &= 1 - \gamma_0(0, 1) + (n + 2) + \sum_{i=1}^{n+2}\gamma_i(0, -1)\\
            &\geq 1 - \gamma_0(0, 1) + (n + 2) + \sum_{i=1}^{n+2}\gamma_0(0, 1) \qquad \text{(privacy constraint)}\\
            &= (n + 3) + (n + 1) \gamma_0(0, 1)\\
            &\geq 2
        \end{align*}

    and by a similar argument, $cost(C_{\rho_2})\geq 2$ for any coupling strategy $C_{\rho_2}$ for $\rho_2$. 

    \item There exist coupling strategies $C_{\rho_1}^*$ and $C_{\rho_2}^*$ for $\rho_1$ and $\rho_2$, respectively, such that $cost(C_{\rho_1}^*) = cost(C_{\rho_2}^*) = 2$. 
    
    We will first describe $C_{\rho_1}^* = (\gamma, \gamma')$. Since no transition outputs \texttt{insample}, we can set $\gamma_i'(\texttt{in}\brangle{1}, \texttt{in}\brangle{2}) = \texttt{in}\brangle{2} - \texttt{in}\brangle{1}$ for all $i$ with no privacy cost. Define 
    \begin{align*}
        \gamma_0(\texttt{in}\brangle{1}, \texttt{in}\brangle{2}) &= -1 \\
        \gamma_i(\texttt{in}\brangle{1}, \texttt{in}\brangle{2}) &= \texttt{in}_i\brangle{2} - \texttt{in}_i\brangle{1} \qquad \text{for all $i>0$}
    \end{align*}
    We see that $C^*_{\rho_1}$ is valid, since $\gamma_i\geq \gamma_{0}$ for all $i>0$. Further, we see that 
    \begin{align*}
        cost(C^*_{\rho_1}) &= \max_{\texttt{in}\brangle{1}\sim\texttt{in}\brangle{2}}\sum_{i=0}^{n+2}(|-\texttt{in}_i\brangle{1}+\texttt{in}_i\brangle{2}-\gamma_i(\texttt{in}_i\brangle{1}, \texttt{in}_i\brangle{2}))\\&\qquad+(|-\texttt{in}_i\brangle{1}+\texttt{in}_i\brangle{2}-\gamma_i'(\texttt{in}_i\brangle{1}, \texttt{in}_i\brangle{2})|)\\
        &= \max_{\texttt{in}\brangle{1}\sim\texttt{in}\brangle{2}} |-\texttt{in}_0\brangle{1}+\texttt{in}_0\brangle{2}-\gamma_0(\texttt{in}_0\brangle{1}, \texttt{in}_0\brangle{2})| \\
        &= \max_{\texttt{in}\brangle{1}\sim\texttt{in}\brangle{2}} |-\texttt{in}_0\brangle{1}+\texttt{in}_0\brangle{2}+1|\\
        &\leq 2 
    \end{align*}
    showing that $cost(C^*_{\rho_1}) = 2$. Similarly, there is a coupling strategy $C^*_{\rho_2}$ for which $cost(C^*_{\rho_2}) = 2$.
    
    \item Any pair of coupling strategies $C_{\rho_1} = (\gamma^{(1)}, \gamma^{\prime(1)})$ and $C_{\rho_2}=(\gamma^{(2)}, \gamma^{\prime(2)})$ for $\rho_1$, $\rho_2$, respectively, that assign the same shifts to $t_{init}$ in both $\rho_1$ and $\rho_2$ must be such that $\max\{cost(C_{\rho_1}), cost(C_{\rho_2})\}>2$.
    
    For the sake of contradiction, suppose that $\max\{cost(C_{\rho_1}), cost(C_{\rho_2})\}=2$. Note that it is thus impossible for $\gamma_0^{(1)}(0, 1) \neq -1$, since then $cost(C_{\rho_1}) > 2$ in the same manner as above.  

    Thus, $\gamma_0^{(1)}(0, 1) = -1$ in $C_{\rho_1}$, which by hypothesis, means that $\gamma_0^{(2)}(0, 1) = -1$ as well. We have the privacy constraint $\gamma_i^{(2)} \leq \gamma_0^{(2)}$ for $i>0$, which also means that $\gamma_i = -1$ identically for all $i > 0$. However, this means that 
    \begin{align*}
        cost(C_{\rho_2}) &= \max_{\texttt{in}\brangle{1}\sim\texttt{in}\brangle{2}}\sum_{i=0}^{n+2}(|-\texttt{in}_i\brangle{1}+\texttt{in}_i\brangle{2}-\gamma_i^{(2)}(\texttt{in}_i\brangle{1}, \texttt{in}_i\brangle{2}))\\&\qquad+(|-\texttt{in}_i\brangle{1}+\texttt{in}_i\brangle{2}-\gamma_i^{\prime(2)}(\texttt{in}_i\brangle{1}, \texttt{in}_i\brangle{2})|)\\
        &\geq \max_{\texttt{in}\brangle{1}\sim\texttt{in}\brangle{2}} \sum_{i=0}^{n+2}(|-\texttt{in}_i\brangle{1}+\texttt{in}_i\brangle{2}-\gamma_i^{(2)}(\texttt{in}_i\brangle{1}, \texttt{in}_i\brangle{2})|)\\
        &\geq \max_{\Delta \in [-1, 1]^{n+2}} |\Delta_0 - \gamma_i^{(2)}(0, \Delta_0)| + \sum_{i=1}^{n+2}(|\Delta_i-\gamma_i^{(2)}(0, \Delta_i)|)\\
        &\geq |1 - \gamma_i^{(2)}(0, 1)| + \sum_{i=1}^{n+2}(|1+1|)\\
        &= 2 \cdot (n + 2)
    \end{align*}
    which is a contradiction. 

    \end{itemize}
    This completes the proof.     
\end{proof}

\begin{prop}(Precise statement of proposition \ref{quadraticPenaltyProp})
    There exist a family of sets of SLPs $\{B_n\}_{n\in \NN}$ for which the cost of any unified coupling strategy $C$ for SLPs in $B_n$ is in $\Omega(n^2)$, but for which there exist SLP-specific coupling strategies for $B_n$ such that their total cost is in $O(n)$.
\end{prop}

\begin{proof}
    We first define the following transitions for all $1\leq i\leq n$: \begin{itemize}
        \item $t_{true}^{(i)} = (\texttt{true}, \bot, \texttt{true})$
        \item $t_{geq}^{(i)} = (\gguard[\texttt{x}],\bot, \texttt{true})$
        \item $t_{leq}^{(i)} = (\lguard[\texttt{x}], \top, \texttt{false})$
        \item $t_{loop}^{(i)} = (\lguard[\texttt{x}], \top, \texttt{false})$
    \end{itemize}

    For all transitions $t$, let $P(t) = (1, 1)$. 


    Consider an SLP $\rho = t_{true}^{(1)} t_{geq}^{(1)} \left(t_{leq}^{(1)}\right)^n \dots t_{true}^{(n)} t_{geq}^{(n)} \left(t_{leq}^{(n)}\right)^n$

    For each $i$, construct also the periodic program $L_i = t_{true}^{(n + i)} \left(t_{loop}^{(i)}\right)^* t_{geq}^{(i)}$ such that each transition $t_{geq}^{(i)}$ is preceded by an arbitrary number of transitions with guard $\lguard$. In particular, we note that $t_{geq}^{(i)}$ is a transition shared by $L_i$ and $\rho$. 

    Let $B_n = \{\rho\}\cup \bigcup_{i=1}^n L_i$. It is worth noting that we do not require $B_n$ to be a set of SLPs generated by a proper CFG; it is possible to extend this construction to such a set relatively straightforwardly by creating a unique initial start transition and connecting it to the beginnings of each SLP in $B_n$.
    
    We will analyze a ``SLP-independent'' set of coupling strategies, meaning that every transition in $B_n$ must be assigned the same shifts by every coupling strategy in the set; this is specifically relevant for transitions that are shared between SLPs (here, $t_{geq}^{(i)}$). 
    For convenience, we treat this set of coupling strategies as a single coupling strategy $C$ that assigns shifts to each individual transition; $C$ induces coupling strategies for SLPs in the same manner as definition \ref{svInducedCouplingStrategy}. 

    We label shifts in $C$ for a transition $t$ as $\gamma_t, \gamma'_t$. Since none of the transitions in $B_n$ output $\texttt{insample}'$, we ignore $\gamma'_t$ for the rest of the proof. 

    There are several constraints on shifts in $C$. 

    First, consider the periodic program $L_i$. In order for $L_i$ to be private, we must have that $\gamma_{t_{true}^{n + i}}(1, 0) = 1$ from the same argument as in the proof of proposition \ref{costDependspathProp}, which then implies from the constraint $\gamma_{t_{true}^{n + i}} \leq \gamma_{t_{geq}^{(i)}}$ that $\gamma_{t_{geq}^{(i)}}(1, 0) = 1$. 

    Now, since the assignment transition immediately preceding $t_{leq}^{(i)}$ in $\rho$ is $t_{geq}^{(i)}$, for which we have the privacy constraint $\gamma_{t_{geq}^{(i)}} \leq \gamma_{t_{leq}^{(i)}}$, we see that $\gamma_{t_{leq}^{(i)}}(1, 0) = 1$ as well. 

    We can thus compute the cost of the coupling strategy $C_\rho$ that $C$ induces on $\rho$, which has $n \cdot (n + 2)$ transitions. 

    \begin{align*}
        cost(C_\rho) &\geq \max_{\Delta = \texttt{in}\brangle{2} - \texttt{in}\brangle{1}} \sum_{i=1}^{n} \big(|\Delta_{t_{true}^{(i)}} - \gamma_{t_{true}^{(i)}}(-\Delta_{t_{true}^{(i)}}, 0)| + |\Delta_{t_{geq}^{(i)}} - \gamma_{t_{geq}^{(i)}}(-\Delta_{t_{geq}^{(i)}}, 0)| \\ & \qquad \qquad \qquad \qquad + n \cdot |\Delta_{t_{leq}^{(i)}} - \gamma_{t_{leq}^{(i)}}(-\Delta_{t_{leq}^{(i)}}, 0)|\big)\\
        &\geq \sum_{i = 1}^n \left(|-1 - \gamma_{t_{true}^{(i)}}(1, 0)| + |-1 - \gamma_{t_{geq}^{(i)}}(1, 0)| + n \cdot |-1 - \gamma_{t_{leq}^{(i)}}(1, 0)|\right)\\
        &= \sum_{i = 1}^n \left(|-1 - \gamma_{t_{true}^{(i)}}(1, 0)| + |-1 - 1| + n \cdot |-1 - 1|\right)\\
        &\geq \sum_{i = 1}^n \left(2 n + 1\right)\\
        &= n \cdot (2 n + 1)
    \end{align*} 

    Note that this also provides a lower bound on the overall privacy cost of any SLP in $B_n$. 

    However, there exists an SLP-independent coupling strategy $C_\rho^* = (\gamma^*, \gamma^{'*})$ for $\rho$ that assigns
    \begin{align*}
        \gamma_{t_{true}^{(i)}}^*(\texttt{in}\brangle{1}, \texttt{in}\brangle{2}) &= -\texttt{in}\brangle{1} + \texttt{in}\brangle{2}\\
        \gamma_{t_{geq}^{(i)}}^*(\texttt{in}\brangle{1}, \texttt{in}\brangle{2}) &= 1\\
        \gamma_{t_{leq}^{(i)}}^*(\texttt{in}\brangle{1}, \texttt{in}\brangle{2}) &= -\texttt{in}\brangle{1} + \texttt{in}\brangle{2}
    \end{align*}
    Note that this satisfies the necessary privacy constraints $\gamma_{t_{true}^{(i)}}^* \leq \gamma_{t_{geq}^{(i)}}^* \geq \gamma_{t_{leq}^{(i)}}^*$ for all $i$, and that $C_\rho^*$ has cost $2n$.
    
    Similarly for any periodic program $L_i$, we can construct the coupling strategy $C_{L_i}^* = (\gamma^*, \gamma^{'*})$, which assigns

    \begin{align*}
        \gamma_{t_{true}^{(n + i)}}^*(\texttt{in}\brangle{1}, \texttt{in}\brangle{2}) &= 1\\
        \gamma_{t_{loop}^{(i)}}^*(\texttt{in}\brangle{1}, \texttt{in}\brangle{2}) &= -\texttt{in}\brangle{1} + \texttt{in}\brangle{2}\\
        \gamma_{t_{geq}^{(i)}}^*(\texttt{in}\brangle{1}, \texttt{in}\brangle{2}) &= 1
    \end{align*}

    Again, note that $C_{L_i}^*$ satisfies the privacy constraints $\gamma_{t_{geq}^{(i)}}^* \geq \gamma_{t_{true}^{(n + i)}}^* \leq \gamma_{t_{loop}^{(i)}}^*$ for all $i$, and has cost $4$.
    
    Thus, we can create a set of SLP-dependent coupling strategies for $B_n$ with maximal cost $2n$ for any SLP in $B_n$, which is a quadratic improvement over the SLP-independent case. 
\end{proof}



\begin{proof}[\proofname~of lemma \ref{finiteCostConstraintLemma}]

    ($\impliedby$)

    Let $T$ be the set of transitions $t_i$ in $L$ such that $t_i$ is \textbf{not} found under a star in $R_L$. 

    Fix a initialized SLP $\rho\in L$ and let $C_\rho$ be the coupling strategy for $\rho$ induced by $C$. 

    Let $D_\rho$ be the set of transitions $t_i\in \rho$ such that $t_i$ is under a star in $R_L$, i.e., $t_i\notin T$.  

    If the given constraint holds, then we know that $\max_{\texttt{in}\brangle{1}\sim\texttt{in}\brangle{2}}\sum_{i: t_i\in D_\rho}(|-\texttt{in}_i\brangle{1}+\texttt{in}_i\brangle{2}-\gamma_i|)d_i+(|-\texttt{in}_i\brangle{1}+\texttt{in}_i\brangle{2}-\gamma_i'|)d_i' = 0$

    So \begin{align*}
        cost(C_\rho) = \max_{\texttt{in}\brangle{1}\sim\texttt{in}\brangle{2}}&\sum_{i: t_i\in D_\rho}(|-\texttt{in}_i\brangle{1}+\texttt{in}_i\brangle{2}-\gamma_i|)d_i+(|-\texttt{in}_i\brangle{1}+\texttt{in}_i\brangle{2}-\gamma_i'|)d_i'\\
        &+\sum_{i: t_i\notin D_\rho}(|-\texttt{in}_i\brangle{1}+\texttt{in}_i\brangle{2}-\gamma_i|)d_i+(|-\texttt{in}_i\brangle{1}+\texttt{in}_i\brangle{2}-\gamma_i'|)d_i'\\
        = \max_{\texttt{in}\brangle{1}\sim\texttt{in}\brangle{2}}&\sum_{i: t_i\in T}(|-\texttt{in}_i\brangle{1}+\texttt{in}_i\brangle{2}-\gamma_i|)d_i+(|-\texttt{in}_i\brangle{1}+\texttt{in}_i\brangle{2}-\gamma_i'|)d_i'\\
        \leq \sum_{i:t_i\in T}(2d_i& + 2d_i')\\
        \leq |T|\max_{i:t_i\in T}&(2d_i + 2d_i')
    \end{align*}

    Thus, $cost(C)\leq |T|\max_{i:t_i\in T}(2d_i + 2d_i') <\infty$.

    ($\implies$)

    Let $t_i$ be a transition in $L$ under a star in $R_L$ such that $\gamma_i\neq -\texttt{in}\brangle{1}_i+\texttt{in}\brangle{2}_i$ or $\gamma_i'\neq  -\texttt{in}\brangle{1}_i+\texttt{in}\brangle{2}_i$.     Thus, $\exists \texttt{in}\brangle{1}\sim \texttt{in}\brangle{2}$ such that $(|-\texttt{in}_i\brangle{1}+\texttt{in}_i\brangle{2}-\gamma_i|)d_i+(|-\texttt{in}_i\brangle{1}+\texttt{in}_i\brangle{2}-\gamma_i'|)d_i'>0$.
    Fix such a $\texttt{in}\brangle{1}\sim \texttt{in}\brangle{2}$. 

    Then there exists some initialized SLP $\rho$ in $L$ of the form $a(bt_ic)^*d$ for some $a, b, c, d\in \Sigma_T^*$.
    
    Let $\rho_k=a(bt_ic)^kd$ be the corresponding initialized SLP in $L$ with $(bt_ic)$ iterated $k$ times. This is equivalent to iterating the cycle containing $t_i$ $k$ times. Then for all $k\in \NN$, \begin{align*}
        cost(\rho_k) \geq k((|-\texttt{in}_i\brangle{1}+\texttt{in}_i\brangle{2}-\gamma_i|)d_i+(|-\texttt{in}_i\brangle{1}+\texttt{in}_i\brangle{2}-\gamma_i'|)d_i'),
    \end{align*}
    so for all $M\in \RR$, $\exists \rho_k$ such that $cost(\rho_k) > M$.
\end{proof}


\begin{lemma}\label{cycleGammaConstraints}
    If a coupling strategy $C=(\mathbf{\gamma}, \mathbf{\gamma}')$ for a periodic program $L$ is valid and has finite cost, then the following must hold for all $i$:
    \begin{enumerate}
        \item If $t_i$ is in a cycle and $c_i = \lguard[\texttt{x}]$, then $\gamma_i = -\texttt{in}_i\brangle{1}+\texttt{in}_i\brangle{2}$ and $\gamma_{at(i)} = 1$.
        \item If $t_i$ is in a cycle and $c_i = \gguard[\texttt{x}]$, then $\gamma_i = -\texttt{in}_i\brangle{1}+\texttt{in}_i\brangle{2}$ and $\gamma_{at(i)} = -1$.
    \end{enumerate}
\end{lemma}
\begin{proof}
    We will show (1). (2) follows symmetrically.

    Consider some $t_i$ in a cycle where $c_i = \lguard[\texttt{x}]$. Because $C$ is has finite cost, we know from lemma \ref{finiteCostConstraintLemma} that $\gamma_i = -\texttt{in}_i\brangle{1}+\texttt{in}_i\brangle{2}$ for all  $\texttt{in}_i\brangle{1}\sim\texttt{in}_i\brangle{2}$. In particular, when $-\texttt{in}_i\brangle{1}+\texttt{in}_i\brangle{2}=1$, then $\gamma_i=1$. 
    
    Further, because $\gamma_{at(i)}$ must be greater than $\gamma_i$ for all $\texttt{in}_i\brangle{1}\sim\texttt{in}_i\brangle{2}$ for $C$ to be valid, we must have that $\gamma_{at(i)}=1$.
\end{proof}

\begin{lemma}\label{integersAreEnoughForATLemma}
    If a valid finite cost coupling strategy $C = (\gamma, \gamma')$ exists for a periodic program $L_\rho$, then there exists a valid finite cost coupling strategy $C^*= (\gamma^*, \gamma^{*\prime})$ such that for all $i\in AT(L_\rho)$, $\gamma_i^*\in \{-1, 0, 1\}$. 
\end{lemma}
\begin{proof}
    Since $L_\rho$ is differentially private, the cost $opt(L_\rho)$ of its optimal coupling strategy is finite. By Proposition \ref{prop:approx_opt_are_close}, we see also that $approx(L)$ is finite. Thus, there exists $\beta \in [-1, 1]^{n}$ which satisfies the approximate privacy constraints on $L_\rho$.

    Define 
    \[\gamma_i = \lfloor \beta_i \rfloor \]

    and notice that $\gamma_i, \gamma^{\prime}$ also satisfy the approximate privacy constraints on $L_\rho$:
    \begin{align*}
        \beta_{at(i)} \leq \beta_i &\implies \lfloor \beta_{at(i)} \rfloor \leq \lfloor \beta_i \rfloor \implies \gamma_{at(i)} \leq \gamma_i\\
        \beta_{at(i)} \geq \beta_i &\implies \lfloor \beta_{at(i)} \rfloor \geq \lfloor \beta_i \rfloor \implies \gamma_{at(i)} \geq \gamma_i\\
        \beta_{i} = 0 &\implies \lfloor \beta_{i} \rfloor = 0 \implies \gamma_{i} = 0\\
        \beta_{i} = 1 &\implies \lfloor \beta_{i} \rfloor = 1 \implies \gamma_{i} = 1\\
        \beta_{i} = -1 &\implies \lfloor \beta_{i} \rfloor = -1 \implies \gamma_{i} = -1
    \end{align*}
    Since $L_\rho$ is private, none of the transitions $t_i$ for $i \in AT(L_\rho)$ are in cycles by lemma \ref{unsatisfiableImpliesNotWellformedLemma}.
    
    By Proposition \ref{prop:approx_exists}, we see that there is a valid coupling strategy $C^* = (\gamma^*, \gamma^{*\prime})$ for $\gamma^*_i = \gamma_i \in \{-1, 0, 1\}$ for all $i \in AT(L_\rho)$.
\end{proof}




\begin{proof}[Proof of prop \ref{syntacticEquivalenceProp}]
    Let $G_P = (V, E)$ be a proper control flow graph that generates an output distinct program $P$. By definition, there exists some unique starting location $\ell_{init} \in V$.

    Then let $A_P = (V, \RR, C, \Gamma, \ell_{init}, X, P_A, \delta)$ where
    \begin{itemize}
        \item $\Sigma, C, X$ are as in the definition of a DiPA
        \item $P_A, \delta$ are (partial) functions defined below.
    \end{itemize}
    
    For all states $\ell\in V$, let $T_\ell$ be the set of transitions that label edges leaving $\ell$. Because of the shared noise condition, $\forall t, t'\in T_{\ell}, P(t) = P(t')$. Let $P(t) = (d_\ell, d_\ell')$ for some $t\in T_\ell$. Then $P_A(\ell) = (0, d_\ell, 0, d_{\ell}')$.

    For all $e = (\ell, \ell')\in E$, let $T(e) = (c, \sigma, \tau)$. We define $\delta(\ell, c) = (\ell', \sigma, \tau)$.

    Observe that because $G_P$ is proper, $\delta$ is a valid transition function for $A_P$. 

    Now let $A = (Q, \Sigma, C, \Gamma, q_{init}, X, P, \delta)$ be a DiPA and let $G_A = (V, E)$ be the underlying graph of $A$; observe that $q_{init}\in V$ is a unique initial state for $G_A$. Additionally, for all $e = (q, q')\in E$, let $\delta(q, c) = (q', \sigma, \tau)$. Then $T(e) = (c, \sigma, \tau)$ is the transition labeling $e$. 

    Because $\delta$ is a valid partial transition function, $G_A$ must be a proper control flow graph that generates an output distinct program. 
\end{proof}




\begin{prop}[Proposition \ref{ClassCouplingStrategiesAreEnoughProp}]
    If there exists a valid coupling strategy $C_\rho$ with cost $cost(C_\rho)$ for every SLP $\rho$ of periodic program $L$ and $\sup_{\rho\in L}cost(C_\rho)< \infty$, then there exists a valid class coupling strategy $C'$ for $L$ such that $cost(C') \leq \sup_{\rho\in L}cost(C_\rho)$. 
\end{prop}

\begin{proof}[Proof of proposition \ref{ClassCouplingStrategiesAreEnoughProp}]


    Because $\sup_{\rho\in L}cost(C_\rho)< \infty$, we can assume that there are no leaking cycles, disclosing cycles, leaking pairs, or privacy violating paths in $L$ by lemma \ref{unsatisfiableImpliesNotWellformedLemma}.


    For a given SLP $\rho$ and a coupling strategy $C_\rho$, recall that we effectively assign each transition $t_i$ in $\rho$ the cost $\max_{\Delta \in \{-1, 0, 1\}}|\Delta - \gamma_i(\Delta)| + |\Delta' - \gamma'_i(\Delta)|$. For convenience, we will shorthand this quantity as $\delta(\rho, t_i) + \delta'(\rho, t_i)$.


    For all $n\in \NN$, let $\rho_n$ be the SLP in $L$ with every cycle in $\rho_n$ repeated $n$ times. 

   Let $cycle(L)$ be the set of all transitions in $L$ that are contained within a cycle in $L$. Observe that for all $t\in cycle(L)$, \[
        \lim_{n\to\infty}\inf_{t_i\in\rho_n: t_i=t} \delta(\rho_n, t_i) = 0
    \]
    Informally, for every cycle transition $t$ in $L$, if the cycle it is contained in is iterated enough times, there must be some iteration $t_i$ of $t$ that is assigned costs approaching 0. 

    This can be shown by considering a transition $t$ in a cycle in $L$ whose minimum coupling cost is non-zero (i.e. $\inf_{\rho\in L; t_i=t} \delta(t_i) > 0$). Then for any finite $d>0$, there exists an SLP $\rho_n$ where $n>\lceil\frac{d}{\inf\delta(t_i)}\rceil+1$. Then $cost(C_{\rho^*})>d$, which implies that $\sup_{\rho\in L}cost(C_\rho) = \infty$, so the observation must hold. 

    Let $t_i$ be a transition in a cycle in $L$ and let $\mathcal{C}_i$ be the cycle containing $t_i$. 

    Then in particular, if $\mathcal{C}_i$ contains a transition with guard $\lguard[\texttt{x}]$, then for all $\psi>0$, there exists $n\in \NN$ such that for $\rho_n\in L$, $\gamma_{at(i)}> 1-\psi$ and if $\mathcal{C}_i$ contains a transition with guard $\lguard[\texttt{x}]$, then for all $\psi>0$, $\gamma_{at(i)}> -1+\psi$. 
    Informally, assignment transitions before an \lcycle\ have shifts that approach 1 and assignment transitions before a \gcycle\ have shifts that approach -1 in $\rho_n$ as $n\to\infty$. 

    Because we know that all coupling strategies $C_{\rho}$ are valid, this may also imply that other assignment transitions also have shifts that approach 1 or -1. 

    Further, if the shifts for an assignment transition $t_i$ approach 1, then the shifts for a transition $t_j$ such that $at(j) = i$ and $c_j = \gguard[\texttt{x}]$ must also approach 1; symmetrically, if the shifts for an assignment transition $t_i$ approach -1, then the shifts for a transition $t_j$ such that $at(j) = i$ and $c_j = \lguard[\texttt{x}]$ must also approach -1. 

    Let $T_1$ and $T_{-1}$ be the sets of assignment transitions in $L$ that approach 1 and -1, respectively. 

    Note that every other transition in $L$ is a non-cycle transition. Consider such a transition $t$ in $L$. Then for every SLP $\rho\in L$ and its corresponding coupling strategy $C_\rho$, there is exactly one shift assignment for $t$ because $t$ is not in a cycle. 

    Let the coupling strategy $C' = (\gamma, \gamma')$ be partially defined as follows: \begin{align*}
        \gamma(t_i)(\texttt{in}\brangle{1}, \texttt{in}\brangle{2}) &= \begin{cases}
            1 & t_i \in T_1\\
            -1 & t_i \in T_{-1}\\
            \texttt{in}\brangle{1}-\texttt{in}\brangle{2} & c_i = \lguard[\texttt{x}]\land t_{at(i)}\in T_1\\
            \texttt{in}\brangle{1}-\texttt{in}\brangle{2} & c_i = \gguard[\texttt{x}]\land t_{at(i)}\in T_{-1}\\
            1 & c_i = \gguard[\texttt{x}]\land t_{at(i)}\in T_1\\
            -1 & c_i = \lguard[\texttt{x}]\land t_{at(i)}\in T_{-1}
        \end{cases}\\
        \gamma'(t_i)(\texttt{in}\brangle{1}, \texttt{in}\brangle{2}) &=\begin{cases}
            0 & t_i\text{ outputs }\texttt{insample}'\\
            \texttt{in}\brangle{1}-\texttt{in}\brangle{2} & \text{otherwise}
        \end{cases}
    \end{align*}

    Let $T_{un}$ be the set of transitions in $L$ that are not assigned by $\gamma$ so far. Note that all transitions in $T_{un}$ are not in cycles. 

    Let $C^* = (\gamma^*, \gamma^{*\prime})$ be the minimal-cost valid class coupling strategy such that for all $t\notin T_{un}$, $\gamma^*(t) = \gamma(t)$. 

    In other words, $cost(C^*) = \inf_{\text{all such valid coupling strategies } C} cost(C)$. Note that $C^*$ is valid. 

    We additionally claim that $cost(C^*)\leq \sup_{\rho\in L}cost(C_\rho)$. The cost of $C^*$ can be separated into costs attributed to $\gamma^{*\prime}$, costs attributed to all transitions not in $T_{un}$ by $\gamma^*$, and costs attributed to all transitions in $T_{un}$ by $\gamma^*$. 
    
    First, note that the coupling cost attributed to $\gamma^{*\prime}$ in $C^*$ must be at most the maximimum coupling cost attributed to $\gamma'$ over all SLP-specific coupling strategies. From before, we additionally know that the cost attributed to all transitions $t\notin T_{un}$ by $\gamma^*$ is at most the supremum of the costs attributed to $t$ over all SLPs in $L$, since we take the limit of all such shifts for $\rho_n$ as $n\to\infty$.

    Finally, since all SLP-specific coupling strategies are valid, taking the remaining transition shifts to minimize the overall cost while retaining a valid coupling strategy sufficies. 
\end{proof}


\begin{proof}[Proof of \ref{prop:compute_opt_cost}]
    Assume that $L$ is differentially private. Then there exists a valid coupling strategy for $L$ with finite cost, showing that the constraints above are feasible, and a finite solution to the optimization problem exists.  

    We will specify a valid coupling strategy $C^* = (\gamma^*, {\gamma'}^*)$ for $L$ with the cost stated above, and then show it is optimal. Define $\Gamma: [-1, 1]^n \to [-1, 1]^n \times [-1, 1]^n$ as follows, where $\Gamma(\Delta)$ is a pair $(\gamma, \gamma')$: 
    \begin{align*}
        \Gamma(\Delta) = &\argmin_{\gamma, \gamma' \in [-1, 1]^n} \sum_{i = 1}^n \left(|\Delta_i - \gamma_i| d_i + |\Delta_i - \gamma_i'|d_i' \right)\\ 
        \text{subject to }
        &\ \gamma_{at(i)} \leq \gamma_i \text{ if } c_i = \gguard, \\
        &\ \gamma_{at(i)} \geq \gamma_i \text{ if } c_i = \lguard, \\
        &\ \gamma_i = 0 \text{ if } \sigma_i = \texttt{insample}, \\
        &\ \gamma_i' = 0 \text{ if } \sigma_i = \texttt{insample}'\\
        &\ \gamma_i = \gamma_i'= \Delta_i \text{ if } t_i \text{ is in a cycle}
    \end{align*}
    Then define \[(\gamma^*(\texttt{in}\brangle{1}, \texttt{in}\brangle{2}), {\gamma'}^*(\texttt{in}\brangle{1}, \texttt{in}\brangle{2})) = \Gamma(\texttt{in}\brangle{1} - \texttt{in}\brangle{2})\]

    Notice the following: 

    \begin{enumerate}
        \item $C^*$ is a valid coupling strategy for $L$, since the privacy constraints on $\gamma^*$ and ${\gamma'}^*$ are satisfied by construction.
        \item $C^*$ has the cost given by the solution to the optimization problem, since 
        \begin{align*}
            cost(C^*) &= \max_{\texttt{in}\brangle{1}\sim \texttt{in}\brangle{2}} \sum_{i = 1}^n  |\texttt{in}_i\brangle{1} - \texttt{in}_i\brangle{2} - \gamma_i^*(\texttt{in}\brangle{1}, \texttt{in}\brangle{2})| d_i \\ 
            \phantom{cost(C^*)} &\phantom{=\max_{\texttt{in}\brangle{1}\sim \texttt{in}\brangle{2}}\qquad } + |\texttt{in}_i\brangle{1} - \texttt{in}_i\brangle{2} - {\gamma'}_i^*(\texttt{in}\brangle{1}, \texttt{in}\brangle{2})|d_i' \\
            &= \max_{\texttt{in}\brangle{1}\sim \texttt{in}\brangle{2}} \sum_{i = 1}^n  |\texttt{in}_i\brangle{1} - \texttt{in}_i\brangle{2} - \Gamma_1(\texttt{in}\brangle{1} - \texttt{in}\brangle{2})_i| d_i \\ 
            \phantom{cost(C^*)} &\phantom{=\max_{\texttt{in}\brangle{1}\sim \texttt{in}\brangle{2}}\qquad } + |\texttt{in}_i\brangle{1} - \texttt{in}_i\brangle{2} - \Gamma_2(\texttt{in}\brangle{1} - \texttt{in}\brangle{2})_i|d_i' \\
            &= \max_{\Delta \in [-1, 1]^n} \sum_{i = 1}^n  |\Delta_i - \Gamma_1(\Delta)_i| d_i + |\Delta_i - \Gamma_2(\Delta)_i|d_i' \\
            &= \max_{\Delta \in [-1, 1]^n} \min_{\gamma, \gamma' \in [-1, 1]^n} \sum_{i = 1}^n  |\Delta_i - \gamma_i| d_i + |\Delta_i - \gamma_i'|d_i'
        \end{align*}
        \item $C^*$ is optimal, since for any valid coupling strategy $C = (\delta, \delta')$ for $L$, we have
        \begin{align*}
            cost(C) &= \max_{\texttt{in}\brangle{1}\sim \texttt{in}\brangle{2}} \sum_{i = 1}^n  |\texttt{in}_i\brangle{1} - \texttt{in}_i\brangle{2} - \delta_i(\texttt{in}\brangle{1}, \texttt{in}\brangle{2})| d_i \\
            \phantom{cost(C)} &\phantom{=\max_{\texttt{in}\brangle{1}\sim \texttt{in}\brangle{2}}\qquad } + |\texttt{in}_i\brangle{1} - \texttt{in}_i\brangle{2} - \delta_i'(\texttt{in}\brangle{1}, \texttt{in}\brangle{2})|d_i' \\
            &\geq \max_{\Delta \in [-1, 1]^n} \sum_{i = 1}^n  |\Delta_i - \delta_i(0, \Delta_i)| d_i + |\Delta_i - \delta_i'(0, \Delta_i)|d_i' \\
            &\geq \max_{\Delta \in [-1, 1]^n} \min_{\gamma, \gamma' \in [-1, 1]^n} \sum_{i = 1}^n \left(|\Delta_i - \gamma_i| d_i + |\Delta_i - \gamma_i'|d_i' \right)\\
            &= cost(C^*)
        \end{align*}
    \end{enumerate}
    which shows that the optimization problem computes the optimal cost of a coupling strategy for $L$ that satisfies the privacy constraints.
\end{proof}


\begin{proof}[Proof of \ref{prop:approx_exists}]
    Let 
    \[\gamma = \argmin_{\gamma \in [-1, 1]^n} \sum_{i \in I} \left(1 + |\gamma_i| \right) d_i\]
    subject to the constraints above. Define $C_L = (\gamma^*, {\gamma'}^*)$ where
    \begin{align*}
        \gamma_i^*(\texttt{in}\brangle{1}, \texttt{in}\brangle{2}) &= \begin{cases}
            \texttt{in}\brangle{1}_i - \texttt{in}\brangle{2}_i &\text{ if } t_i \text{ is in a cycle} \\
            \gamma_i &\text{ otherwise}
        \end{cases} \\[1em]
        {\gamma'}_i^*(\texttt{in}\brangle{1}, \texttt{in}\brangle{2}) &= \begin{cases}
            0 &\text{ if } t_i \text{ outputs \texttt{insample}} \\
            \texttt{in}\brangle{1}_i - \texttt{in}\brangle{2}_i &\text{ otherwise}
        \end{cases}
    \end{align*}
    Notice the following: 

    \begin{itemize}
        \item $C_L$ satisfies the privacy constraints, and so is valid.
        
        If $t_i$ is in a cycle with $c_i = \lguard$, then the constraints on $\gamma$ require that $\gamma_i = 1$, and so $1 = \gamma_i \leq \gamma_{at(i)} = 1$. As a result, we will satisfy the privacy constraint $\gamma_{at(i)}^* \geq \gamma_i^*$: 
        \[\gamma_i^* = \texttt{in}\brangle{1}_i - \texttt{in}\brangle{2}_i \leq 1 = \gamma_{at(i)}^*\]
        A similar argument holds for if $t_i$ is in a cycle with $c_i = \gguard$.

        All other privacy constraints are satisfied by construction.

        \item $C_L$ has the cost given by the solution to the optimization problem, since
        
        \begin{align*}
            cost(C_L) &= \max_{\texttt{in}\brangle{1}\sim \texttt{in}\brangle{2}} \sum_{i = 1}^n  |\texttt{in}_i\brangle{1} - \texttt{in}_i\brangle{2} - \gamma_i^*(\texttt{in}\brangle{1}, \texttt{in}\brangle{2})| d_i \\ 
            \phantom{cost(C_L)} &\phantom{=\max_{\texttt{in}\brangle{1}\sim \texttt{in}\brangle{2}}\qquad } + |\texttt{in}_i\brangle{1} - \texttt{in}_i\brangle{2} - {\gamma'}_i^*(\texttt{in}\brangle{1}, \texttt{in}\brangle{2})|d_i' \\
            &= \max_{\texttt{in}\brangle{1}\sim \texttt{in}\brangle{2}} \left(\sum_{i \in I} |\texttt{in}_i\brangle{1} - \texttt{in}_i\brangle{2} - \gamma_i^*(\texttt{in}\brangle{1}, \texttt{in}\brangle{2})| d_i\right) \\
            \phantom{cost(C_L)} &\phantom{=\max_{\texttt{in}\brangle{1}\sim \texttt{in}\brangle{2}}\qquad } + \left(\sum_{t_i \text{outputs \texttt{insample}}}|\texttt{in}_i\brangle{1} - \texttt{in}_i\brangle{2} - {\gamma'}_i^*(\texttt{in}\brangle{1}, \texttt{in}\brangle{2})|d_i'\right)\\
            &= \max_{\texttt{in}\brangle{1}\sim \texttt{in}\brangle{2}} \left(\sum_{i \in I} |\texttt{in}_i\brangle{1} - \texttt{in}_i\brangle{2} - \gamma_i| d_i\right) \\
            \phantom{cost(C_L)} &\phantom{=\max_{\texttt{in}\brangle{1}\sim \texttt{in}\brangle{2}}\qquad } + \left(\sum_{t_i \text{outputs \texttt{insample}}}|\texttt{in}_i\brangle{1} - \texttt{in}_i\brangle{2}|d_i'\right)\\
            &= \sum_{i \in I} (1 + |\gamma_i|) d_i + \sum_{t_i \text{outputs \texttt{insample}}} d_i' \\
            &= approx(L)
        \end{align*}
    \end{itemize}
\end{proof}

\begin{proof}[Proof of \ref{approximateSolutionPolyTimeProp}]
    To compute the solution to the minimization problem, we can set up the following linear program: 
    \begin{align*}
        \min_{\gamma, A_i \in [-1, 1]^n} &\sum_{i = 1}^n \left(1 + A_i \right) d_i \\ 
            \text{subject to } 
            &\ \gamma_{at(i)} \leq \gamma_i \text{ if } c_i = \lguard, \\
            &\ \gamma_{at(i)} \geq \gamma_i \text{ if } c_i = \gguard, \\
            &\ \gamma_i = 0 \text{ if } \sigma_i = \texttt{insample}, \\
            &\ \gamma_i = 1 \text{ if } t_i \text{ is in a cycle and has } c_i = \lguard,\\ 
            &\ \gamma_i = -1 \text{ if } t_i \text{ is in a cycle and has } c_i = \gguard,\\
            &\ \gamma_i \leq A_i, -\gamma_i \leq A_i \text{ for all } i \in \{1, \dots, n\} 
    \end{align*}
    This program can be solved using the ellipsoid method in polynomial time.
\end{proof}


\begin{proof}[Proof of \ref{prop:approx_opt_are_close}]
    We have $opt(L) \leq approx(L)$ by Proposition \ref{prop:compute_opt_cost}. Let $I$ be the set of transitions in $L$ that do $\textit{not}$ appear in a cycle. Then we have
    \begin{align*}
        opt(L) &= \max_{\Delta \in [-1, 1]^n} \min_{\gamma, \gamma' \in [-1, 1]^n} \sum_{i = 1}^n \left(|\Delta_i - \gamma_i| d_i + |\Delta_i - \gamma_i'|d_i' \right)\\
        &= \max_{\Delta \in [-1, 1]^n} \min_{\gamma, \gamma' \in [-1, 1]^n} \sum_{i \in I} \left(|\Delta_i - \gamma_i| d_i + |\Delta_i - \gamma_i'|d_i' \right)\\
        &= \max_{\Delta \in [-1, 1]^n} \min_{\gamma, \gamma' \in [-1, 1]^n} \left(\sum_{i \in I} \left(|\Delta_i - \gamma_i| d_i \right) + \sum_{t_i \text{outputs \texttt{insample}}} |\Delta_i| d_i' \right)\\
        &\geq \max_{\Delta \in [-1, 1]^n} \min_{\gamma, \gamma' \in [-1, 1]^n} \left(\sum_{i \in I} \left(|\gamma_i| - |\Delta_i| \right) d_i  + \sum_{t_i \text{outputs \texttt{insample}}} |\Delta_i| d_i' \right)\\
        &= \max_{\Delta \in [-1, 1]^n} \left(- \sum_{i \in I} |\Delta_i| d_i + \sum_{t_i \text{outputs \texttt{insample}}} |\Delta_i| d_i' + \min_{\gamma, \gamma' \in [-1, 1]^n} \sum_{i \in I}|\gamma_i| d_i \right)\\
        &\geq \min_{\gamma, \gamma' \in [-1, 1]^n} \sum_{i \in I}|\gamma_i| d_i \\
        &= approx(L) - \sum_{t_i \text{outputs \texttt{insample}}} d_i' - \sum_{i \in I} d_i'
    \end{align*}
    showing the second inequality.
\end{proof}


\begin{proof}[Proof of prop \ref{privacyConstraintGraphProp}]
    $(\implies)$ Let $L$ be differentially private. Then $approx(L) < \infty$, and so there exists $\gamma \in [-1, 1]^n$ such that the approximate privacy constraints are satisfied by $\gamma$. Aiming for a contradiction, assume that there exists a path ${\bf 1} \to v_{i_1} \to \dots \to v_{i_k} \to {\bf -1}$ in the privacy constraint graph of $L$. This corresponds to the sequence of privacy constraint inequalities
    \[1 \leq \gamma_{i_1} \leq \dots \leq \gamma_{i_k} \leq -1\]
    which is a contradiction, showing that no such $\gamma$ could exist. Therefore, there is no path from $\bf 1$ to $\bf -1$ in the privacy constraint graph of $L$.

    $(\impliedby)$ Let there exist no path from $\bf 1$ to $\bf -1$ in the privacy constraint graph of $L$. Define 
    \begin{align*}
        \gamma_i = \begin{cases}
            1 &\text{ if there exists a path from } {\bf 1} \text{ to } v_i \text{ in } G_L \\
            -1 &\text{ otherwise}
        \end{cases}
    \end{align*}
    We claim that the approximate privacy constraints are satisfied by $\gamma$.
    
    \begin{itemize}
        \item Consider the approximate privacy constraint $\gamma_i \leq \gamma_j$. This corresponds to the edge $(v_i, v_j)$ in $G_L$. If $\gamma_i = 1$, there is a path from $\bf 1$ to $v_i$, and so there is a path from $\bf 1$ to $v_j$, and so $\gamma_j = 1$, satisfying the constraint. If $\gamma_i = -1$, then any assignment of $\gamma_j$ satisfies the constraint. 
        \item Consider the constraint $\gamma_i = 1$. This corresponds to the edge $({\bf 1}, v_i)$ in $G_L$. Since there is a path from $\bf 1$ to $v_i$, we have $\gamma_i = 1$, satisfying the constraint.
        \item Consider the constraint $\gamma_i = -1$. This corresponds to the edge $(v_i, {\bf -1})$ in $G_L$. Since there is no path from $\bf 1$ to $\bf -1$ in $G_L$, there must be no path from $\bf 1$ to $v_i$. Thus, $\gamma_i = -1$, satisfying the constraint.
    \end{itemize}
    
    Thus, the approximate privacy constraints are satisfied by $\gamma$, which means that $approx(L) < \infty$ and $L$ is differentially private.
\end{proof}

\textbf{The following claims help prove prop \ref{programPrivacyConstraintGraphPathReq} and theorem \ref{LinearTimeDecidingPrograms}}

\begin{prop}
    \label{prop:paths_in_privacy_graph}
    Let $v_{i_0}, \dots, v_{i_k}$ be vertices in $G_P$ corresponding to transitions $t_{i_0}, \dots, t_{i_k}$ in $P$. If $v_{i_0} \to \dots \to v_{i_k}$ is a path in $G_P$, then there exists an SLP $\rho \in P$ such that 
    \begin{align*}
        t_{i_0} \cdots t_{i_k} \text{ is a subsequence of } \rho \text{ with } guard(t_{i_j}) = \gguard \text { for all } j \in \{1, \dots, k\}
    \end{align*}
    or
    \begin{align*}
        t_{i_k} \cdots t_{i_0} \text{ is a subsequence of } \rho \text{ with } guard(t_{i_j}) = \lguard \text { for all } j \in \{k - 1, \dots, 0\}
    \end{align*}
\end{prop}



\begin{proof}[Proof of \ref{prop:paths_in_privacy_graph}]
    We will use induction on the length of the path $v_{i_0} \to \dots \to v_{i_k}$ in $G_P$. 

    \begin{itemize}
        \item Base Case ($k = 1$)
        
        If $k = 1$, then the path $v_{i_0} \to v_{i_1}$ comprises of a single edge $(v_{i_0}, v_{i_1})$ in $G_P$. So, there is a periodic program $L$ for which $(v_{i_0}, v_{i_1}) \in G_L$, for which there is the privacy constraint $\gamma_{i_0} \leq \gamma_{i_1}$. 
        
        We either have that $i_0 = at(i_1)$ and $c_{i_1} = \gguard$, or $i_1 = at(i_0)$ and $c_{i_0} = \lguard$. In the first case, we have that $t_{i_0} t_{i_1}$ is a subsequence of some SLP $\rho$ in $L$ with $guard(t_{i_1}) = \gguard$. In the second case, we have that $t_{i_1} t_{i_0}$ is a subsequence of some SLP $\rho$ in $L$ with $guard(t_{i_0}) = \lguard$.

        \item Inductive Step ($k > 1$)
        
        By the inductive hypothesis, we have one of the following cases: 

        \begin{enumerate}
            \item There exists an SLP $\rho_1 \in P$ such that $t_{i_0} \cdots t_{i_{k - 1}}$ is a subsequence of $\rho_1$ with $guard(t_{i_j}) = \gguard$ for all $j \in \{1, \dots, k - 1\}$.
            
            Since we have the edge $(v_{i_{k - 1}}, v_{i_k})$ in $G_P$, there exists a periodic program $L$ for which $(v_{i_{k - 1}}, v_{i_k}) \in G_L$, for which there is the privacy constraint $\gamma_{i_{k - 1}} \leq \gamma_{i_k}$.

            We either have that $i_{k - 1} = at(i_k)$ and $c_{i_k} = \gguard$ ($t_{i_{k - 1}}$ precedes $t_{i_k}$ in $L$), or $i_k = at(i_{k - 1})$ and $c_{i_{k - 1}} = \lguard$ ($t_{i_k}$ precedes $t_{i_{k - 1}}$ in $L$). Notice, however, that we cannot have that $t_{i_k}$ precedes $t_{i_{k - 1}}$, since we have assumed $c_{i_{k - 1}} = \gguard$. 

            So, there exists an SLP $\rho_2 \in L$ such that $t_{i_{k - 1}} t_{i_k}$ is a subsequence of $\rho_2$ with $guard(t_{i_k}) = \gguard$. 

            Let $j_1$ be the index at which $t_{i_{k - 1}}$ appears in $\rho_1$, and $j_2$ be the index at which it appears in $\rho_2$. Then, the SLP $\rho_1[:j_1] \rho_2[j_2:]$ is an SLP in $P$ such that $t_{i_0} \cdots t_{i_k}$ is a subsequence of $\rho_1[:j_1] \rho_2[j_2:]$ with $guard(t_{i_j}) = \gguard$ for all $j \in \{1, \dots, k\}$.

            \item There exists an SLP $\rho_1 \in P$ such that $t_{i_{k - 1}} \cdots t_{i_0}$ is a subsequence of $\rho_1$ with $guard(t_{i_j}) = \lguard$ for all $j \in \{k - 2, \dots, 0\}$.
            
            Similar to the argument above, we either have that $t_{i_{k - 1}}$ precedes $t_{i_k}$, or $t_{i_k}$ precedes $t_{i_{k - 1}}$ in some periodic program. We cannot have that $t_{i_{k - 1}}$ precedes $t_{i_k}$, and so $c_{i_{k - 1}}$ is forced to be $\lguard$. We can then construct an SLP $\rho \in P$ such that $t_{i_k} \cdots t_{i_0}$ is a subsequence of $\rho$ with $guard(t_{i_j}) = \lguard$ for all $j \in \{k - 1, \dots, 0\}$. 
        \end{enumerate}
    \end{itemize}

    This completes the proof. 
\end{proof}


\begin{cor}
    The path $v_{i_0} \to \dots \to v_{i_k}$ is in $G_P$ if and only if there is a periodic program $L$ in $P$ such that $v_{i_0} \to \dots \to v_{i_k}$ is a path in $G_L$.
\end{cor}

\begin{proof}[Proof of \ref{programPrivacyConstraintGraphPathReq}]
    There is a path from $\bf 1$ to $\bf -1$ in the privacy constraint graph of $P$ if and only if there is a path from $\bf 1$ to $\bf -1$ in the privacy constraint graph of some periodic program $L$ in $P$ by Proposition \ref{prop:paths_in_privacy_graph}. This is true if and only if there exists some $L$ which is not differentially private, which is true if and only if $P$ is not differentially private.
\end{proof}

\section{Multivariable Programs}


\begin{lemma}[Precise Version of lemma \ref{simplifiedMvParallelCouplingsLemma}]\label{mvParallelCouplingsLemma}
    Let $\vec{X}\brangle{1} = (X_1\brangle{1}, \ldots X_k\brangle{1})$ where $X_i\brangle{1}\sim \Lap(\mu_{x_i}\brangle{1}, \frac{1}{d_{x_i}\varepsilon})$ are independent random variables and $\vec{X}\brangle{2} = (X_1\brangle{2}, \ldots X_k\brangle{2})$ where $X_i\brangle{2}\sim \Lap(\mu_{x_i}\brangle{2}, \frac{1}{d_{x_i}\varepsilon})$ are independent random variables be representations of two possible initial values of $\vec{X}$, respectively.

    Let $t = (c, \sigma, \tau)$ be a $k$v-transition such that $P(t) = (d_t, d_t')$.

    Let $\texttt{in}\brangle{1}\sim\texttt{in}\brangle{2}$ be an arbitrary adjacent input pair and let $o\brangle{1}$, $o\brangle{2}$ be random variables representing possible outputs of $t$ given inputs $\texttt{in}\brangle{1}$ and $\texttt{in}\brangle{2}$, respectively. 

    Then $\forall \varepsilon>0$ and for all $\gamma_{x_1}, \ldots, \gamma_{x_k}, \gamma_t^{(x_1)}, \ldots, \gamma_t^{(x_k)}, \gamma_t'$ that satisfy the constraints \[
        \begin{cases}
            \gamma_t^{(x_i)}\leq\gamma_{x_i} & c = \lguard[\texttt{x}_i]\\
            \gamma_t^{(x_i)}\geq\gamma_{x_i} & c = \gguard[\texttt{x}_i]\\
            \gamma_t^{(x_i)}=0 & \sigma = \texttt{insample}^{(\texttt{x}_i)}\\
            \gamma_t'=0 & \sigma = \texttt{insample}'
      \end{cases}
      \] for all $1\leq i\leq k$, the lifting $o\brangle{1}\{(a, b): a=\sigma\implies b=\sigma\}^{\#d\varepsilon}o\brangle{2}$ is valid for 
      $d = \sum_{i=1}^k\left(|\mu_{x_i}\brangle{1}-\mu_{x_i}\brangle{2}+\gamma_{x_i}|d_{x_i}+|\texttt{in}\brangle{1}-\texttt{in}\brangle{2}+\gamma_t^{(x_i)}|d_t\right)+|\texttt{in}\brangle{1}-\texttt{in}\brangle{2}+\gamma_t'|d_t'$.
\end{lemma}
\begin{proof}
    From lemma \ref{indTransitionCoupling}, we know that if these constraints are satisfied, we can create liftings such that for all $i$, $c^{(\texttt{x}_i)}$ is satisfied in run $\brangle{1}\implies c^{(\texttt{x}_i)}$ is satisfied in run $\brangle{2}$.
    
    Because $c$ is made of conjunctions and disjunctions of all $c^{(\texttt{x}_i)}$, this means that if $c$ is satisfied in run $\brangle{1}$, then $c$ must also be satisfied in $\brangle{2}$. 

    Finally, as before, if $\sigma \in \{\texttt{insample}^{(x_1)}\ldots \texttt{insample}^{(x_k)},\texttt{insample}'\}$, then for all $\sigma$, $o\brangle{1}=\sigma \implies o\brangle{2} = \sigma$. Further, note that we only need to couple $\texttt{insample}'\brangle{1}$ and $\texttt{insample}'\brangle{2}$ once for all variables. By simply adding up the costs from lemma \ref{indTransitionCoupling}, we complete the proof.
\end{proof}


\begin{proof}[Proof of lemma \ref{mvCrossCoupling}]
    Let $P(t) = (d_t, d_t')$. For convenience, we rewrite, for all $i$, $X_i = \mu_i + \zeta_i$, where $\zeta_i\sim \Lap(0, \frac{1}{d_x\varepsilon})$. For all $i$, we also write $\texttt{insample}^{(x_i)} = \texttt{in} + z_i$, where $z \sim \Lap(0, \frac{1}{d_t\varepsilon})$ and $\texttt{insample}' = \texttt{in} + z'$, where $z' \sim \Lap(0, \frac{1}{d_t'\varepsilon})$.

    Based on lemma \ref{mvParallelCouplingsLemma}, we will assume that we have constructed the following liftings for all $i$ and are aiming to ``extend'' them: \begin{itemize}
        \item $X_i\brangle{1} +\gamma_{x_i}(=)^{\#(|\mu_i\brangle{1}-\mu_i\brangle{2}+\gamma_{x_i}|d_x\varepsilon)}X_i\brangle{2}$
        \item $\texttt{insample}^{(\texttt{x}_i)} + \gamma_t^{(x_i)}(=)^{\#(|\texttt{in}\brangle{1}-\texttt{in}\brangle{2}+\gamma_t^{(x_i)}|d_t\varepsilon)}\texttt{insample}^{(\texttt{x}_i)}\brangle{2}$
        \item $\texttt{insample}' + \gamma_t'(=)^{\#(|\texttt{in}\brangle{1}-\texttt{in}\brangle{2}+\gamma_t'|d_t'\varepsilon)}\texttt{insample}'\brangle{2}$
    \end{itemize}

    \textbf{Case 1: } For all $i\neq j$, construct the lifting $z_i\brangle{1}(=)^{\#0}z_j\brangle{1}$, which is equivalent to the lifting $\texttt{insample}^{(x_i)}\brangle{1}(=)^{\#}\texttt{insample}^{(x_j)}\brangle{1}$. In particular, this guarantees that all $\texttt{insample}^{(x_i)}\brangle{1}$ are equal to each other. 
    Additionally, for all $i\neq j$, construct the lifting $\zeta_i\brangle{1}(=)^{\#0}\zeta_j\brangle{1}$, which is equivalent to the lifting $X_i\brangle{1} - X_j\brangle{1} (=)^{\#0}\mu_i\brangle{1}-\mu_j\brangle{1}$.

    First, suppose that $X_i\brangle{1} = \mu_i\brangle{1}$ for all $i$. Then if case one is true, the proposition that ``$c$ is satisfied in run $\brangle{1}$'' must always be false; thus, the implication ``$c$ is satisfied in run $\brangle{1}\implies c$ is satisfied in run $\brangle{2}$'' must always be true.

    Now observe that for any transition guard $c$, if the boolean expression produced from $c$ by setting all $\texttt{insample}^{(\texttt{x}_i)}$ equal to each other and setting $\texttt{x}_i = \mu_i\brangle{1}$ for all $i$ is a contradiction, then for any constant $a$, the boolean expression produced from $c$ by setting all $\texttt{insample}^{(\texttt{x}_i)}$ equal to each other and setting $\texttt{x}_i = \mu_i\brangle{1}+a$ for all $i$ must also be a contradiction. 

    Thus, the liftings we have constructed are sufficient to prove the proposition ``$c$ is satisfied in run $\brangle{1}\implies c$ is satisfied in run $\brangle{2}$'', and so, if the two output conditions are also satisfied, $\PP[\vec{X}\brangle{1}, t, \texttt{in}\brangle{1}, \sigma]\leq e^{d\varepsilon}\PP[\vec{X}\brangle{2}, t, \texttt{in}\brangle{2}, \sigma]$ for 
    $d = |\mu_i\brangle{1}-\mu_i\brangle{2}+\gamma_{x_i}|d_x\varepsilon + |\texttt{in}\brangle{1}-\texttt{in}\brangle{2}+\gamma_t^{(x_i)}|d_t\varepsilon+|\texttt{in}\brangle{1}-\texttt{in}\brangle{2}+\gamma_t'|d_t'\varepsilon$, i.e. for no ``additional'' cost.     
    
    \textbf{Case 2: } For all $i\neq j$, construct the lifting $z_i\brangle{2}(=)^{\#0}z_j\brangle{2}$, which is equivalent to the lifting $\texttt{insample}^{(x_i)}\brangle{2}(=)^{\#}\texttt{insample}^{(x_j)}\brangle{2}$. In particular, this guarantees that all $\texttt{insample}^{(x_i)}\brangle{2}$ are equal to each other. 
    Additionally, for all $i\neq j$, construct the lifting $\zeta_i\brangle{2}(=)^{\#0}\zeta_j\brangle{2}$, which is equivalent to the lifting $X_i\brangle{2} - X_j\brangle{2} (=)^{\#0}\mu_i\brangle{2}-\mu_j\brangle{2}$.z

    As before, if case two is satisfied, then these liftings suffice to show that the proposition  ``$c$ is satisfied in run $\brangle{2}$'' must always be true; thus, the implication ``$c$ is satisfied in run $\brangle{1}\implies c$ is satisfied in run $\brangle{2}$'' must also always be true, and so $\PP[\vec{X}\brangle{1}, t, \texttt{in}\brangle{1}, \sigma]\leq e^{d\varepsilon}\PP[\vec{X}\brangle{2}, t, \texttt{in}\brangle{2}, \sigma]$ for 
    $d = |\mu_i\brangle{1}-\mu_i\brangle{2}+\gamma_{x_i}|d_x\varepsilon + |\texttt{in}\brangle{1}-\texttt{in}\brangle{2}+\gamma_t^{(x_i)}|d_t\varepsilon+|\texttt{in}\brangle{1}-\texttt{in}\brangle{2}+\gamma_t'|d_t'\varepsilon$.

\end{proof}


\begin{proof}[Proof of \ref{2vUnsatisfiableImpliesNotWellformedLemma}]
    Suppose that we have some maximally satisfied coupling strategy $C$ for $L$. There must be some constraint that is violated by $C$.

    By lemma \ref{ProgramCounterexampleThm}, there must be either a leaking cycle, leaking pair, disclosing cycle, or privacy violating path with respect to a single variable. 

    We show that, if there only exist leaking pairs in $L$, then at least one leaking pair must be a non-cancelling leaking pair. In every other case, there exists a problematic graph structure and the proof is complete. 

    For the sake of contradiction, suppose that every leaking pair $\kappa, \kappa'$ in $L$ is a cancelling leaking pair. By definition, this means that for every two transitions $t_i, t_j$ in $\kappa, \kappa'$, $at_x(i) = at_x(j)$ and $at_y(i) = at_y(j)$. 

    Without loss of generality, we will assume that every non-$\texttt{true}$ transition in $\kappa$ must either have guard $\mvlguard[\texttt{x}]\land\mvgguard[\texttt{y}]$ or $\mvlguard[\texttt{x}]\lor\mvgguard[\texttt{y}]$. Thus, every non-$\texttt{true}$ transition in $\kappa'$ must either have guard $\mvgguard[\texttt{x}]\land\mvlguard[\texttt{y}]$ or $\mvgguard[\texttt{x}]\lor\mvlguard[\texttt{y}]$, respectively. 

    Let $t_i$ be an arbitrary non-$\texttt{true}$ transition in $\kappa$ and $t_j$ be an arbitrary non-$\texttt{true}$ transition in $\kappa'$.

    Consider the case where $c_i = \mvlguard[\texttt{x}]\land\mvgguard[\texttt{y}]$ and $c_j = \mvgguard[\texttt{x}]\land\mvlguard[\texttt{y}]$; the other case is symmetric.

    Observe that at least one of $\texttt{in}_{at_x(i)}\brangle{1}\leq \texttt{in}_{at_y(i)}\brangle{1}$ or $\texttt{in}_{at_y(i)}\brangle{1}\geq \texttt{in}_{at_y(i)}\brangle{1}$ must be true.
    
    Suppose that $\texttt{in}_{at_x(i)}\brangle{1}\leq \texttt{in}_{at_y(i)}\brangle{1}$. The case where $\texttt{in}_{at_y(i)}\brangle{1}\geq \texttt{in}_{at_y(i)}\brangle{1}$ is symmetric. 
    Then we can set $\gamma_{at_x(i)}^{(x)} = -1$ and $\gamma_{at_y(i)}^{(y)}=1$; since $\texttt{in}_{at_x(i)}\brangle{1}\leq \texttt{in}_{at_y(i)}\brangle{1}$, $c_i\brangle{1}$ is a contradiction, fulfilling condition (1) of lemma \ref{mvCrossCoupling}. 

    Further, we can assign $\gamma_j^{(x)} = \gamma_j^{(y)}=\texttt{in}\brangle{2}-\texttt{in}\brangle{1}$ freely; thus, we have shown that, with regards to leaking pairs, no constraints of the privacy constraint system can be violated.    
    
    In particular, because there do not exist disclosing cycles, leaking cycles, or privacy violating paths in either variable, this cannot violate any further constraints. Thus, $C$ is not maximal, which is our contradiction, so $L$ must contain either a leaking cycle, disclosing cycle, privacy violating path, or non-cancelling leaking pair in at least one variable.
\end{proof}



\end{document} 