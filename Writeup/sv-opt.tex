
\subsection{Minimizing a privacy budget}

Beyond whether or not valid coupling strategies exist for a program, we'd also like to optimize their privacy costs. Since we break down each program into a finite number of looping branches, each with a separate coupling strategy, we optimize the cost of a coupling strategy for every looping branch independently

We first define a linear program that computes the optimal cost of any coupling strategy for a looping branch. 

\begin{prop}
    \label{prop:compute_opt_cost}
    Let $L$ be a looping branch containing $n$ distinct transitions $t_i = (c_i, \sigma_i, \tau_i)$ such that each $t_i$ has spread parameters $P(t_i) = (d_i, d'_i)$. The cost $opt(L)$ of the optimal coupling strategy for $L$ can be computed by solving the following optimization problem: 
    \begin{align*}
        opt(L) = \max_{\texttt{in}\brangle{2}-\texttt{in}\brangle{1} \in [-1, 1]^n} &\min_{\gamma, \gamma' \in [-1, 1]^n} \sum_{i = 1}^n \left(|\texttt{in}_i\brangle{2}-\texttt{in}_i\brangle{1} - \gamma_i| d_i + |\texttt{in}_i\brangle{2}-\texttt{in}_i\brangle{1} - \gamma_i'|d_i' \right)\\ 
            \text{subject to }
            &\ \gamma_{at(i)} \leq \gamma_i \text{ if } c_i = \gguard, \\
            &\ \gamma_{at(i)} \geq \gamma_i \text{ if } c_i = \lguard, \\
            &\ \gamma_i = 0 \text{ if } \sigma_i = \texttt{insample}, \\
            &\ \gamma_i' = 0 \text{ if } \sigma_i = \texttt{insample}'\\
            &\ \gamma_i = \gamma_i'= \texttt{in}_i\brangle{2}-\texttt{in}_i\brangle{1} \text{ if } t_i \text{ is in a cycle}
    \end{align*}
    If a looping branch $L$ is not differentially private, a solution to this optimization problem does not exist, and we write $opt(L) = \infty$.
\end{prop}

We observe that the inner problem is convex, and so the outer problem is that of convex maximization. 

In the absence of an immediate polynomial time algorithm for solving this optimization problem, we consider an approximation of the problem for looping branches that are known to be private. 

\begin{defn}[Approximate Privacy Cost]
    Let $L$ be a differentially private looping branch generated by $G$ containing $n$ distinct transitions $t_i = (c_i, \sigma_i, \tau_i)$ such that each $t_i$ has spread parameters $P(t_i) = (d_i, d'_i)$.
    Let $I$ be the set of transitions in $L$ that do \textit{not} appear in a cycle in $G$. The \textbf{approximate privacy cost} $approx(L)$ of $L$ is defined as:
    \begin{align*} 
        approx(L) = &\sum_{t_i \text{outputs \texttt{insample}'}} d_i' + \min_{\gamma \in [-1, 1]^n} \sum_{t_i \in I} \left(1 + |\gamma_i| \right) d_i  \\
            \text{subject to } 
            &\ \gamma_{at(i)} \leq \gamma_i \text{ if } c_i = \gguard, \\
            &\ \gamma_{at(i)} \geq \gamma_i \text{ if } c_i = \lguard, \\
            &\ \gamma_i = 0 \text{ if } \sigma_i = \texttt{insample}, \\
            &\ \gamma_i = 1 \text{ if } t_i \text{ is in a cycle and } c_i = \lguard,\\ 
            &\ \gamma_i = -1 \text{ if } t_i \text{ is in a cycle and } c_i = \gguard
    \end{align*}
\end{defn}
Observe that this is the cost of a coupling strategy $(\gamma, \gamma')$ such that $\gamma$ and $\gamma'$ do not depend on $\texttt{in}\brangle{1}$ or $\texttt{in}\brangle{2}$ for non-cyclic transitions. 

Importantly, we show that the approximate LP is solvable in polynomial time through an application of the ellipsoid method.

\begin{prop}\label{approximateSolutionPolyTimeProp}
    The approximate privacy cost of a differentially private looping branch $L$ can be computed in time polynomial in $n$, the number of distinct transitions in $L$.
\end{prop}

As hoped, solving this approximation still provides us with a valid coupling strategy.

\begin{prop}
    \label{prop:approx_exists}
    Given a differentially private looping branch $L$, there exists a valid coupling strategy $C_L$ for $L$ such that $cost(C_L) = approx(L)$.

    Moreover, if $\gamma \in [-1, 1]^n$ satisfies the approximate privacy constraints, then there is a valid coupling strategy $C_L = (\gamma^*, {\gamma'}^*)$ for $L$ such that $\gamma_i^*(\texttt{in}\brangle{1}, \texttt{in}\brangle{2}) = \gamma_i$ for all $t_i$ that do not appear in cycles. 
\end{prop}

Further, we obtain an approximation factor bounded by a term linear in the number of non-cyclic transitions in $L$. 

\begin{prop}
    \label{prop:approx_opt_are_close}
    For a looping branch $L$ with $n$ distinct transitions, we have 
    \[opt(L) \leq approx(L) \leq opt(L) + \sum_{i \in I }^n d_i + \sum_{t_i \text{outputs \texttt{insample}}} d_i'\]
    where $I$ is the set of transitions in $L$ that do $\textit{not}$ appear in a cycle.
\end{prop}

We conjecture that the optimal coupling cost, as represented by the linear program from \ref{prop:compute_opt_cost}, exactly matches the ``true'' privacy cost of a program, as represented by the KL-divergence between the distributions of a looping branch's outputs on adjacent input sequences. In this case, coupling proofs would provide an exact bound on the privacy cost of any program that can be represented by our model. 

\begin{conj}
    For a looping branch $L$, the optimal coupling cost $opt(L)$ of $L$ matches the ``true'' privacy cost, i.e.:
    \begin{align*}
        opt(L) = \sup_{\rho \in L} \sup_{\texttt{in}\brangle{1} \sim \texttt{in}\brangle{2}} D_\infty(\PP[\rho, \texttt{in}\brangle{1}, \cdot]\; ||\; \PP[\rho, \texttt{in}\brangle{2}, \cdot])
    \end{align*}
    representing the worst-case privacy loss over all paths in $L$, all valid adjacent inputs, and all measurable output events.
\end{conj}

\subsection{Deciding Privacy}

We also provide an efficient (i.e. linear time) algorithm for solving the decision problem of privacy.

In particular, we check whether a looping branch is differentially private by checking whether the approximate privacy cost of the looping branch is finite. This is true if and only if the \textit{approximate privacy constraints} are feasible.

Note that it is possible to directly encode the approximate privacy constraints as an instance of 2SAT. For a more intuitive algorithm, we encode these constraints in a graph which preserves some of the structure of programs. 

\begin{defn}[Privacy constraint graph]
    Let $L$ be a looping branch that contains $n$ distinct transitions $t_i$. The \textbf{privacy constraint graph} $\mathcal{P}_L = (V, E)$ of $L$ is a directed graph where: 
    \begin{itemize}
        \item $V = \{{\bf 1, -1}\} \cup \{v_i: t_i\in L\}$. Each $v_i \in V$ represents the shift $\gamma_i$ on $t_i$.
        \item For all $(t_i, t_j)$ such that the constraint $\gamma_i \leq \gamma_j$ is in the approximate privacy constraint system of $L$, $(v_i, v_j) \in E$.
        \item There are nodes ${\bf 1, -1}\ \in V$ such that: 
        \begin{itemize}
            \item $({\bf -1}, v)\in E$ for all $v \in V$.
            \item $(v, {\bf 1})\in E$ for all $v \in V$.
            \item For every transition $t_i$ with the privacy constraint $\gamma_i = 1$ in the approximate privacy constraint system of $L$, $({\bf 1}, v_i) \in E$.
            \item For every transition $t_i$ with the privacy constraint $\gamma_i = -1$ in the approximate privacy constraint system of $L$, $(v_i, {\bf -1}) \in E$. 
        \end{itemize}
    \end{itemize}

\end{defn}

\begin{prop}\label{privacyConstraintGraphProp}
    A looping branch $L$ is differentially private if and only if there does not exist a path from $\bf 1$ to $\bf -1$ in the privacy constraint graph of $L$.
\end{prop}

We can directly combine the privacy constraint graphs for individual looping branches together to produce a privacy constraint graph for a program. 

\begin{prop}\label{programPrivacyConstraintGraphPathReq}
    A program $P$ is differentially private if and only if there does not exist a path from $\bf 1$ to $\bf -1$ in the privacy constraint graph $\mathcal{P}_P$ of $P$, where $\mathcal{P}_P$ is the union of the privacy constraint graphs $\mathcal{P}_L$ of each looping branch $L\subseteq P$.
\end{prop}

This is the linear time algorithm we desired. 

\begin{thm}\label{LinearTimeDecidingPrograms}
    Given a program $P$ generated by a graph $G$, we can decide if $P$ is differentially private in linear time in the size of $G$.
\end{thm}

\begin{proof}
    Constructing the privacy constraint graph $G_P$ takes linear time in the size of $P$, on which we can perform a breadth first search to check whether there is a path from $\bf 1$ to $\bf -1$. This can be done in linear time in the size of $G_P$, which is linear in the size of $P$.
\end{proof}

We observe that we match the runtime obtained by Chadha, Sistla, and Viswanathan~\cite{chadhaLinearTimeDecidability2021} for deciding the privacy of programs under this model. 


