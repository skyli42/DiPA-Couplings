
\subsection{Minimizing a privacy budget}

If we have a differentially private program, we'd also like to optimize its privacy cost. Because we must have separate coupling strategies for every looping branch of a program, we optimize the privacy cost of each looping branch independently.

\begin{prop}
    \label{prop:compute_opt_cost}
    Let $L$ be a looping branch in a program $P$ consisting of $n$ transitions. The cost $opt(L)$ of the optimal coupling strategy for $L$ can be computed by the following optimization problem: 
    \sky{should this be written without the $\Delta$? it needs to be explained at least}
    \begin{align*}
        opt(L) = \max_{\Delta \in [-1, 1]^n} &\min_{\gamma, \gamma' \in [-1, 1]^n} \sum_{i = 1}^n \left(|\Delta_i - \gamma_i| d_i + |\Delta_i - \gamma_i'|d_i' \right)\\ 
            \text{subject to }
            &\ \gamma_{at(i)} \leq \gamma_i \text{ if } c_i = \gguard, \\
            &\ \gamma_{at(i)} \geq \gamma_i \text{ if } c_i = \lguard, \\
            &\ \gamma_i = 0 \text{ if } \sigma_i = \texttt{insample}, \\
            &\ \gamma_i' = 0 \text{ if } \sigma_i = \texttt{insample}'\\
            &\ \gamma_i = \gamma_i'= \Delta_i \text{ if } t_i \text{ is in a cycle}
    \end{align*}
    If a looping branch $L$ is not differentially private, a solution to this optimization problem does not exist, and we write $opt(L) = \infty$.
\end{prop}

\begin{proof}
    Assume that $L$ is differentially private. Then there exists a valid coupling strategy for $L$ with finite cost, showing that the constraints above are feasible, and a finite solution to the optimization problem exists.  

    We will specify a valid coupling strategy $C^* = (\gamma^*, {\gamma'}^*)$ for $L$ with the cost stated above, and then show it is optimal. Define $\Gamma: [-1, 1]^n \to [-1, 1]^n \times [-1, 1]^n$ as follows, where $\Gamma(\Delta)$ is a pair $(\gamma, \gamma')$: 
    \begin{align*}
        \Gamma(\Delta) = &\argmin_{\gamma, \gamma' \in [-1, 1]^n} \sum_{i = 1}^n \left(|\Delta_i - \gamma_i| d_i + |\Delta_i - \gamma_i'|d_i' \right)\\ 
        \text{subject to }
        &\ \gamma_{at(i)} \leq \gamma_i \text{ if } c_i = \gguard, \\
        &\ \gamma_{at(i)} \geq \gamma_i \text{ if } c_i = \lguard, \\
        &\ \gamma_i = 0 \text{ if } \sigma_i = \texttt{insample}, \\
        &\ \gamma_i' = 0 \text{ if } \sigma_i = \texttt{insample}'\\
        &\ \gamma_i = \gamma_i'= \Delta_i \text{ if } t_i \text{ is in a cycle}
    \end{align*}
    Then define \[(\gamma^*(\texttt{in}\brangle{1}, \texttt{in}\brangle{2}), {\gamma'}^*(\texttt{in}\brangle{1}, \texttt{in}\brangle{2})) = \Gamma(\texttt{in}\brangle{1} - \texttt{in}\brangle{2})\]

    Notice the following: 

    \begin{enumerate}
        \item $C^*$ is a valid coupling strategy for $L$, since the privacy constraints on $\gamma^*$ and ${\gamma'}^*$ are satisfied by construction.
        \item $C^*$ has the cost given by the solution to the optimization problem, since 
        \begin{align*}
            cost(C^*) &= \max_{\texttt{in}\brangle{1}\sim \texttt{in}\brangle{2}} \sum_{i = 1}^n  |\texttt{in}_i\brangle{1} - \texttt{in}_i\brangle{2} - \gamma_i^*(\texttt{in}\brangle{1}, \texttt{in}\brangle{2})| d_i \\ 
            \phantom{cost(C^*)} &\phantom{=\max_{\texttt{in}\brangle{1}\sim \texttt{in}\brangle{2}}\qquad } + |\texttt{in}_i\brangle{1} - \texttt{in}_i\brangle{2} - {\gamma'}_i^*(\texttt{in}\brangle{1}, \texttt{in}\brangle{2})|d_i' \\
            &= \max_{\texttt{in}\brangle{1}\sim \texttt{in}\brangle{2}} \sum_{i = 1}^n  |\texttt{in}_i\brangle{1} - \texttt{in}_i\brangle{2} - \Gamma_1(\texttt{in}\brangle{1} - \texttt{in}\brangle{2})_i| d_i \\ 
            \phantom{cost(C^*)} &\phantom{=\max_{\texttt{in}\brangle{1}\sim \texttt{in}\brangle{2}}\qquad } + |\texttt{in}_i\brangle{1} - \texttt{in}_i\brangle{2} - \Gamma_2(\texttt{in}\brangle{1} - \texttt{in}\brangle{2})_i|d_i' \\
            &= \max_{\Delta \in [-1, 1]^n} \sum_{i = 1}^n  |\Delta_i - \Gamma_1(\Delta)_i| d_i + |\Delta_i - \Gamma_2(\Delta)_i|d_i' \\
            &= \max_{\Delta \in [-1, 1]^n} \min_{\gamma, \gamma' \in [-1, 1]^n} \sum_{i = 1}^n  |\Delta_i - \gamma_i| d_i + |\Delta_i - \gamma_i'|d_i'
        \end{align*}
        \item $C^*$ is optimal, since for any valid coupling strategy $C = (\delta, \delta')$ for $L$, we have
        \begin{align*}
            cost(C) &= \max_{\texttt{in}\brangle{1}\sim \texttt{in}\brangle{2}} \sum_{i = 1}^n  |\texttt{in}_i\brangle{1} - \texttt{in}_i\brangle{2} - \delta_i(\texttt{in}\brangle{1}, \texttt{in}\brangle{2})| d_i \\
            \phantom{cost(C)} &\phantom{=\max_{\texttt{in}\brangle{1}\sim \texttt{in}\brangle{2}}\qquad } + |\texttt{in}_i\brangle{1} - \texttt{in}_i\brangle{2} - \delta_i'(\texttt{in}\brangle{1}, \texttt{in}\brangle{2})|d_i' \\
            &\geq \max_{\Delta \in [-1, 1]^n} \sum_{i = 1}^n  |\Delta_i - \delta_i(0, \Delta_i)| d_i + |\Delta_i - \delta_i'(0, \Delta_i)|d_i' \\
            &\geq \max_{\Delta \in [-1, 1]^n} \min_{\gamma, \gamma' \in [-1, 1]^n} \sum_{i = 1}^n \left(|\Delta_i - \gamma_i| d_i + |\Delta_i - \gamma_i'|d_i' \right)\\
            &= cost(C^*)
        \end{align*}
    \end{enumerate}
    which shows that the optimization problem computes the optimal cost of a coupling strategy for $L$ that satisfies the privacy constraints.
\end{proof}

We observe that the inner problem is convex, and so the outer problem is that of convex maximization. 

In the absence of an immediate polynomial time algorithm for solving this optimization problem, we analyze an approximation of the problem. 

\begin{defn}
    Define the approximate privacy cost of a differentially private looping branch $L$ to be as follows. Let $I$ be the set of transitions in $L$ that do $\textit{not}$ appear in a cycle.  
    \begin{align*} 
        approx(L) = &\sum_{t_i \text{outputs \texttt{insample}}} d_i' + \min_{\gamma \in [-1, 1]^n} \sum_{i \in I} \left(1 + |\gamma_i| \right) d_i  \\
            \text{subject to } 
            &\ \gamma_{at(i)} \leq \gamma_i \text{ if } c_i = \gguard, \\
            &\ \gamma_{at(i)} \geq \gamma_i \text{ if } c_i = \lguard, \\
            &\ \gamma_i = 0 \text{ if } \sigma_i = \texttt{insample}, \\
            &\ \gamma_i = 1 \text{ if } t_i \text{ is in a cycle and has } c_i = \lguard,\\ 
            &\ \gamma_i = -1 \text{ if } t_i \text{ is in a cycle and has } c_i = \gguard
    \end{align*}
    which is the cost of a coupling strategy in which $\gamma$ and $\gamma'$ are constant with respect to $\texttt{in}\brangle{1}$ and $\texttt{in}\brangle{2}$ on non-cyclic transitions. 
\end{defn}

Solving this approximation still provides us with a valid coupling strategy.

\begin{prop}
    \label{prop:approx_exists}
    Given a differentially private looping branch $L$, there exists a valid coupling strategy $C_L$ for $L$ such that $cost(C_L) = approx(L)$.

    Moreover, if $\gamma \in [-1, 1]^n$ satisfies the approximate privacy constraints, then there is a valid coupling strategy $C_L = (\gamma^*, {\gamma'}^*)$ for $L$ such that $\gamma_i^*(\texttt{in}\brangle{1}, \texttt{in}\brangle{2}) = \gamma_i$ for all $t_i$ that do not appear in cycles. 
\end{prop}


\begin{proof}
    Let 
    \[\gamma = \argmin_{\gamma \in [-1, 1]^n} \sum_{i \in I} \left(1 + |\gamma_i| \right) d_i\]
    subject to the constraints above. Define $C_L = (\gamma^*, {\gamma'}^*)$ where
    \begin{align*}
        \gamma_i^*(\texttt{in}\brangle{1}, \texttt{in}\brangle{2}) &= \begin{cases}
            \texttt{in}\brangle{1}_i - \texttt{in}\brangle{2}_i &\text{ if } t_i \text{ is in a cycle} \\
            \gamma_i &\text{ otherwise}
        \end{cases} \\[1em]
        {\gamma'}_i^*(\texttt{in}\brangle{1}, \texttt{in}\brangle{2}) &= \begin{cases}
            0 &\text{ if } t_i \text{ outputs \texttt{insample}} \\
            \texttt{in}\brangle{1}_i - \texttt{in}\brangle{2}_i &\text{ otherwise}
        \end{cases}
    \end{align*}
    Notice the following: 

    \begin{itemize}
        \item $C_L$ satisfies the privacy constraints, and so is valid.
        
        If $t_i$ is in a cycle with $c_i = \lguard$, then the constraints on $\gamma$ require that $\gamma_i = 1$, and so $1 = \gamma_i \leq \gamma_{at(i)} = 1$. As a result, we will satisfy the privacy constraint $\gamma_{at(i)}^* \geq \gamma_i^*$: 
        \[\gamma_i^* = \texttt{in}\brangle{1}_i - \texttt{in}\brangle{2}_i \leq 1 = \gamma_{at(i)}^*\]
        A similar argument holds for if $t_i$ is in a cycle with $c_i = \gguard$.

        All other privacy constraints are satisfied by construction.

        \item $C_L$ has the cost given by the solution to the optimization problem, since
        
        \begin{align*}
            cost(C_L) &= \max_{\texttt{in}\brangle{1}\sim \texttt{in}\brangle{2}} \sum_{i = 1}^n  |\texttt{in}_i\brangle{1} - \texttt{in}_i\brangle{2} - \gamma_i^*(\texttt{in}\brangle{1}, \texttt{in}\brangle{2})| d_i \\ 
            \phantom{cost(C_L)} &\phantom{=\max_{\texttt{in}\brangle{1}\sim \texttt{in}\brangle{2}}\qquad } + |\texttt{in}_i\brangle{1} - \texttt{in}_i\brangle{2} - {\gamma'}_i^*(\texttt{in}\brangle{1}, \texttt{in}\brangle{2})|d_i' \\
            &= \max_{\texttt{in}\brangle{1}\sim \texttt{in}\brangle{2}} \left(\sum_{i \in I} |\texttt{in}_i\brangle{1} - \texttt{in}_i\brangle{2} - \gamma_i^*(\texttt{in}\brangle{1}, \texttt{in}\brangle{2})| d_i\right) \\
            \phantom{cost(C_L)} &\phantom{=\max_{\texttt{in}\brangle{1}\sim \texttt{in}\brangle{2}}\qquad } + \left(\sum_{t_i \text{outputs \texttt{insample}}}|\texttt{in}_i\brangle{1} - \texttt{in}_i\brangle{2} - {\gamma'}_i^*(\texttt{in}\brangle{1}, \texttt{in}\brangle{2})|d_i'\right)\\
            &= \max_{\texttt{in}\brangle{1}\sim \texttt{in}\brangle{2}} \left(\sum_{i \in I} |\texttt{in}_i\brangle{1} - \texttt{in}_i\brangle{2} - \gamma_i| d_i\right) \\
            \phantom{cost(C_L)} &\phantom{=\max_{\texttt{in}\brangle{1}\sim \texttt{in}\brangle{2}}\qquad } + \left(\sum_{t_i \text{outputs \texttt{insample}}}|\texttt{in}_i\brangle{1} - \texttt{in}_i\brangle{2}|d_i'\right)\\
            &= \sum_{i \in I} (1 + |\gamma_i|) d_i + \sum_{t_i \text{outputs \texttt{insample}}} d_i' \\
            &= approx(L)
        \end{align*}
    \end{itemize}
\end{proof}

Importantly, the approximate LP is solvable in polynomial time through an application of the ellipsoid method.


\begin{prop}
    The approximate privacy cost of $L$ can be computed in time polynomial in $n$, the number of transitions in $L$.
\end{prop}

\begin{proof}
    To compute the solution to the minimization problem, we can set up the following linear program: 
    \begin{align*}
        \min_{\gamma, A_i \in [-1, 1]^n} &\sum_{i = 1}^n \left(1 + A_i \right) d_i \\ 
            \text{subject to } 
            &\ \gamma_{at(i)} \leq \gamma_i \text{ if } c_i = \lguard, \\
            &\ \gamma_{at(i)} \geq \gamma_i \text{ if } c_i = \gguard, \\
            &\ \gamma_i = 0 \text{ if } \sigma_i = \texttt{insample}, \\
            &\ \gamma_i = 1 \text{ if } t_i \text{ is in a cycle and has } c_i = \lguard,\\ 
            &\ \gamma_i = -1 \text{ if } t_i \text{ is in a cycle and has } c_i = \gguard,\\
            &\ \gamma_i \leq A_i, -\gamma_i \leq A_i \text{ for all } i \in \{1, \dots, n\} 
    \end{align*}
    This program can be solved using the ellipsoid method in polynomial time.
\end{proof}

Further, we obtain an approximation factor bounded by a term linear in the number of non-cyclic transitions in $L$. 

\begin{prop}
    \label{prop:approx_opt_are_close}
    For a looping branch $L$ with $n$ distinct transitions, we have 
    \[opt(L) \leq approx(L) \leq opt(L) + \sum_{i \in I }^n d_i + \sum_{t_i \text{outputs \texttt{insample}}} d_i'\]
    where $I$ is the set of transitions in $L$ that do $\textit{not}$ appear in a cycle.
\end{prop}

\begin{proof}
    We have $opt(L) \leq approx(L)$ by Proposition \ref{prop:compute_opt_cost}. Let $I$ be the set of transitions in $L$ that do $\textit{not}$ appear in a cycle. Then we have
    \begin{align*}
        opt(L) &= \max_{\Delta \in [-1, 1]^n} \min_{\gamma, \gamma' \in [-1, 1]^n} \sum_{i = 1}^n \left(|\Delta_i - \gamma_i| d_i + |\Delta_i - \gamma_i'|d_i' \right)\\
        &= \max_{\Delta \in [-1, 1]^n} \min_{\gamma, \gamma' \in [-1, 1]^n} \sum_{i \in I} \left(|\Delta_i - \gamma_i| d_i + |\Delta_i - \gamma_i'|d_i' \right)\\
        &= \max_{\Delta \in [-1, 1]^n} \min_{\gamma, \gamma' \in [-1, 1]^n} \left(\sum_{i \in I} \left(|\Delta_i - \gamma_i| d_i \right) + \sum_{t_i \text{outputs \texttt{insample}}} |\Delta_i| d_i' \right)\\
        &\geq \max_{\Delta \in [-1, 1]^n} \min_{\gamma, \gamma' \in [-1, 1]^n} \left(\sum_{i \in I} \left(|\gamma_i| - |\Delta_i| \right) d_i  + \sum_{t_i \text{outputs \texttt{insample}}} |\Delta_i| d_i' \right)\\
        &= \max_{\Delta \in [-1, 1]^n} \left(- \sum_{i \in I} |\Delta_i| d_i + \sum_{t_i \text{outputs \texttt{insample}}} |\Delta_i| d_i' + \min_{\gamma, \gamma' \in [-1, 1]^n} \sum_{i \in I}|\gamma_i| d_i \right)\\
        &\geq \min_{\gamma, \gamma' \in [-1, 1]^n} \sum_{i \in I}|\gamma_i| d_i \\
        &= approx(L) - \sum_{t_i \text{outputs \texttt{insample}}} d_i' - \sum_{i \in I} d_i'
    \end{align*}
    showing the second inequality.
\end{proof}

We conjecture that the optimal coupling cost, as represented by the un-approximated LP, exactly matches the ``true'' privacy cost of a program. In this case, coupling proofs would truly be ``complete'' for this program model. 

\begin{conj}
    The optimal coupling cost is the ``true'' privacy cost of a looping branch. That is, 
    \begin{align*}
        opt(L) = \sup_{\rho \in L} \sup_{\texttt{in}\brangle{1} \sim \texttt{in}\brangle{2}} D_\infty(\PP[\rho, \texttt{in}\brangle{1}, o]\; ||\; \PP[\rho, \texttt{in}\brangle{2}, o])
    \end{align*}
    representing the worst-case privacy loss over all possible paths in $L$ and all possible adjacent inputs.
\end{conj}

\subsection{Deciding Privacy}

We now turn to algorithms for the decision problem of privacy.

We can check whether a looping branch is differentially private by checking whether the approximate privacy cost of the looping branch is finite. This is true if and only if the \textit{approximate privacy constraints} are feasible.

We will give an algorithm to check whether the approximate privacy constraints are feasible. Although this is equivalent to solving a 2SAT instance, we will encode these constraints in a graph which preserves some of the structure of our program. 

\begin{defn}[Privacy constraint graph]
    Let $L$ be a looping branch in a program $P$. The \textbf{privacy constraint graph} $G_L = (V, E)$ of $L$ is a directed graph where: 
    \begin{itemize}
        \item For every transition $t_i$ in $L$, there is a vertex $v_i \in V$ representing the shift $\gamma_i$ on $t_i$.
        \item For every pair of transitions $(t_i, t_j)$ which have the privacy constraint $\gamma_i \leq \gamma_j$, there is an edge $(v_i, v_j) \in E$.
        \item There are nodes ${\bf 1, -1}\ \in V$ such that: 
        \begin{itemize}
            \item The edges $({\bf -1}, v)$ exist for all $v \in V$.
            \item The edges $(v, {\bf 1})$ exist for all $v \in V$.
            \item For every transition $t_i$ with the privacy constraint $\gamma_i = 1$, there is an edge $({\bf 1}, v_i) \in E$.
            \item For every transition $t_i$ with the privacy constraint $\gamma_i = -1$, there is an edge $(v_i, {\bf -1}) \in E$. 
        \end{itemize}
    \end{itemize}

\end{defn}

\begin{prop}
    A looping branch $L$ is differentially private if and only if there does not exist a path from $\bf 1$ to $\bf -1$ in the privacy constraint graph of $L$.
\end{prop}

\begin{proof}
    $(\implies)$ Let $L$ be differentially private. Then $approx(L) < \infty$, and so there exists $\gamma \in [-1, 1]^n$ such that the approximate privacy constraints are satisfied by $\gamma$. Aiming for a contradiction, assume that there exists a path ${\bf 1} \to v_{i_1} \to \dots \to v_{i_k} \to {\bf -1}$ in the privacy constraint graph of $L$. This corresponds to the sequence of privacy constraint inequalities
    \[1 \leq \gamma_{i_1} \leq \dots \leq \gamma_{i_k} \leq -1\]
    which is a contradiction, showing that no such $\gamma$ could exist. Therefore, there is no path from $\bf 1$ to $\bf -1$ in the privacy constraint graph of $L$.

    $(\impliedby)$ Let there exist no path from $\bf 1$ to $\bf -1$ in the privacy constraint graph of $L$. Define 
    \begin{align*}
        \gamma_i = \begin{cases}
            1 &\text{ if there exists a path from } {\bf 1} \text{ to } v_i \text{ in } G_L \\
            -1 &\text{ otherwise}
        \end{cases}
    \end{align*}
    We claim that the approximate privacy constraints are satisfied by $\gamma$. \vishnu{Need to do a better job of stating what approximate privacy constraints are and why they only depend on $\gamma \in [-1, 1]^n$ and not some function of $\texttt{in}\brangle{1}$ and $\texttt{in}\brangle{2}$}.
    
    \begin{itemize}
        \item Consider the approximate privacy constraint $\gamma_i \leq \gamma_j$. This corresponds to the edge $(v_i, v_j)$ in $G_L$. If $\gamma_i = 1$, there is a path from $\bf 1$ to $v_i$, and so there is a path from $\bf 1$ to $v_j$, and so $\gamma_j = 1$, satisfying the constraint. If $\gamma_i = -1$, then any assignment of $\gamma_j$ satisfies the constraint. 
        \item Consider the constraint $\gamma_i = 1$. This corresponds to the edge $({\bf 1}, v_i)$ in $G_L$. Since there is a path from $\bf 1$ to $v_i$, we have $\gamma_i = 1$, satisfying the constraint.
        \item Consider the constraint $\gamma_i = -1$. This corresponds to the edge $(v_i, {\bf -1})$ in $G_L$. Since there is no path from $\bf 1$ to $\bf -1$ in $G_L$, there must be no path from $\bf 1$ to $v_i$. Thus, $\gamma_i = -1$, satisfying the constraint.
    \end{itemize}
    
    Thus, the approximate privacy constraints are satisfied by $\gamma$, which means that $approx(L) < \infty$ and $L$ is differentially private.
\end{proof}

We have specified a linear time algorithm only to check whether a given looping branch is private. However, we can check whether \textit{all} looping branches are private at once in linear time by combining the privacy constraint graphs for each looping branch into a single graph.

\begin{defn}
    The privacy constraint graph $G_P$ of a program $P$ is the union of the privacy constraint graphs of each looping branch in $P$.
\end{defn}

\begin{prop}
    \label{prop:paths_in_privacy_graph}
    Let $v_{i_0}, \dots, v_{i_k}$ be vertices in $G_P$ corresponding to transitions $t_{i_0}, \dots, t_{i_k}$ in $P$. If $v_{i_0} \to \dots \to v_{i_k}$ is a path in $G_P$, then there exists a path $\rho \in P$ such that 
    \begin{align*}
        t_{i_0} \cdots t_{i_k} \text{ is a subsequence of } \rho \text{ with } guard(t_{i_j}) = \gguard \text { for all } j \in \{1, \dots, k\}
    \end{align*}
    or
    \begin{align*}
        t_{i_k} \cdots t_{i_0} \text{ is a subsequence of } \rho \text{ with } guard(t_{i_j}) = \lguard \text { for all } j \in \{k - 1, \dots, 0\}
    \end{align*}
\end{prop}


\begin{proof}
    We will use induction on the length of the path $v_{i_0} \to \dots \to v_{i_k}$ in $G_P$. 

    \begin{itemize}
        \item Base Case ($k = 1$)
        
        If $k = 1$, then the path $v_{i_0} \to v_{i_1}$ comprises of a single edge $(v_{i_0}, v_{i_1})$ in $G_P$. So, there is a looping branch $L$ for which $(v_{i_0}, v_{i_1}) \in G_L$, for which there is the privacy constraint $\gamma_{i_0} \leq \gamma_{i_1}$. 
        
        We either have that $i_0 = at(i_1)$ and $c_{i_1} = \gguard$, or $i_1 = at(i_0)$ and $c_{i_0} = \lguard$. In the first case, we have that $t_{i_0} t_{i_1}$ is a subsequence of some path $\rho$ in $L$ with $guard(t_{i_1}) = \gguard$. In the second case, we have that $t_{i_1} t_{i_0}$ is a subsequence of some path $\rho$ in $L$ with $guard(t_{i_0}) = \lguard$.

        \item Inductive Step ($k > 1$)
        
        By the inductive hypothesis, we have one of the following cases: 

        \begin{enumerate}
            \item There exists a path $\rho_1 \in P$ such that $t_{i_0} \cdots t_{i_{k - 1}}$ is a subsequence of $\rho_1$ with $guard(t_{i_j}) = \gguard$ for all $j \in \{1, \dots, k - 1\}$.
            
            Since we have the edge $(v_{i_{k - 1}}, v_{i_k})$ in $G_P$, there exists a looping branch $L$ for which $(v_{i_{k - 1}}, v_{i_k}) \in G_L$, for which there is the privacy constraint $\gamma_{i_{k - 1}} \leq \gamma_{i_k}$.

            We either have that $i_{k - 1} = at(i_k)$ and $c_{i_k} = \gguard$ ($t_{i_{k - 1}}$ precedes $t_{i_k}$ in $L$), or $i_k = at(i_{k - 1})$ and $c_{i_{k - 1}} = \lguard$ ($t_{i_k}$ precedes $t_{i_{k - 1}}$ in $L$). Notice, however, that we cannot have that $t_{i_k}$ precedes $t_{i_{k - 1}}$, since we have assumed $c_{i_{k - 1}} = \gguard$. 

            So, there exists a path $\rho_2 \in L$ such that $t_{i_{k - 1}} t_{i_k}$ is a subsequence of $\rho_2$ with $guard(t_{i_k}) = \gguard$. 

            Let $j_1$ be the index at which $t_{i_{k - 1}}$ appears in $\rho_1$, and $j_2$ be the index at which it appears in $\rho_2$. Then, the path $\rho_1[:j_1] \rho_2[j_2:]$ is a path in $P$ such that $t_{i_0} \cdots t_{i_k}$ is a subsequence of $\rho_1[:j_1] \rho_2[j_2:]$ with $guard(t_{i_j}) = \gguard$ for all $j \in \{1, \dots, k\}$.

            \item There exists a path $\rho_1 \in P$ such that $t_{i_{k - 1}} \cdots t_{i_0}$ is a subsequence of $\rho_1$ with $guard(t_{i_j}) = \lguard$ for all $j \in \{k - 2, \dots, 0\}$.
            
            Similar to the argument above, we either have that $t_{i_{k - 1}}$ precedes $t_{i_k}$, or $t_{i_k}$ precedes $t_{i_{k - 1}}$ in some looping branch. We cannot have that $t_{i_{k - 1}}$ precedes $t_{i_k}$, and so $c_{i_{k - 1}}$ is forced to be $\lguard$. We can then construct a path $\rho \in P$ such that $t_{i_k} \cdots t_{i_0}$ is a subsequence of $\rho$ with $guard(t_{i_j}) = \lguard$ for all $j \in \{k - 1, \dots, 0\}$. 
        \end{enumerate}
    \end{itemize}

    This completes the proof. 
\end{proof}


\begin{cor}
    The path $v_{i_0} \to \dots \to v_{i_k}$ is in $G_P$ if and only if there is a looping branch $L$ in $P$ such that $v_{i_0} \to \dots \to v_{i_k}$ is a path in $G_L$.
\end{cor}

\begin{prop}
    A program $P$ is differentially private if and only if there does not exist a path from $\bf 1$ to $\bf -1$ in the privacy constraint graph of $P$.
\end{prop}

\begin{proof}
    There is a path from $\bf 1$ to $\bf -1$ in the privacy constraint graph of $P$ if and only if there is a path from $\bf 1$ to $\bf -1$ in the privacy constraint graph of some looping branch $L$ in $P$ by Proposition \ref{prop:paths_in_privacy_graph}. This is true if and only if there exists some $L$ which is not differentially private, which is true if and only if $P$ is not differentially private.
\end{proof}

\sky{clarify what ``size of'' $P$ is - ambiguous}

\begin{thm}
    Given a program $P$, we can check whether $P$ is differentially private in linear time in the size of $P$.
\end{thm}

\begin{proof}
    Constructing the privacy constraint graph $G_P$ takes linear time in the size of $P$, on which we can perform a breadth first search to check whether there is a path from $\bf 1$ to $\bf -1$. This can be done in linear time in the size of $G_P$, which is linear in the size of $P$.
\end{proof}

\sky{add a note about this matching DiPA bounds}


