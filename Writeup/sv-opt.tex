
Beyond solving the decision problem of privacy, we can also optimize the specific privacy cost obtained by a coupling proof of privacy for a program. 

In general, there is no way to reuse couplings for shared transitions on different paths if we are trying to find an optimal coupling strategy. In other words, one cannot choose a single coupling strategy for an arbitrary set of paths; we must choose a new coupling strategy for each path.

For example, even if two paths share a prefix, we cannot necessarily ``reuse'' the shifts from an optimal (with regards to cost) coupling strategy for the prefix in the context of the first path in the context of an optimal coupling strategy for the second path. 

\begin{prop}\label{noSharingStrategiesProp}
  There exist paths $\rho'=\pi\cdot \pi$, $\rho^*=\pi\cdot\pi^*$ that share a prefix path $\pi$ such that for all optimal coupling strategies $C'$ for $\rho'$ and $C^*$ for $\rho^*$, $C'$ and $C^*$ differ on the shift assigned to some transition in $\pi$. 
\end{prop}

A specific counterexample is provided as proposition \ref{costDependspathProp} in the appendix. 

In particular, we show that the penalty to choosing only a single coupling strategy for a set of paths increases quadratically in the size of the paths.

\begin{prop}
  For all $n\in \NN$, there exist sets of paths $B_n$ such that the cost of a coupling strategy that uses the same set of shifts for all shared transitions between paths in $B_n$ has cost that is a quadratic factor larger than the cost of an optimal coupling strategy. 
\end{prop}

Fortunately, while we cannot analyze \textit{arbitrary} sets of paths with the same coupling strategy, we can still analyze an entire \textit{looping branch} with a single coupling strategy.

We show that it is in fact optimal to only consider a single coupling strategy for an entire looping branch; not only is finding individual coupling strategies for every single path in a looping branch intractable, but it also does not lead to a better overall privacy cost. 

\begin{prop}\label{ClassCouplingStrategiesAreEnoughProp}
    If there exists a valid coupling strategy $C_\rho$ with cost $cost(C_\rho)$ for every path $\rho$ of looping branch $L$ and $\sup_{\rho\in L}cost(C_\rho)< \infty$, then there exists a valid class coupling strategy $C'$ for $L$ such that $cost(C') \leq \sup_{\rho\in L}cost(C_\rho)$. 
\end{prop}

Thus, as with the decision problem, we will optimize the cost of each looping branch of a program independently.

We first define a linear program that computes the optimal cost of any coupling strategy for a looping branch. 

\begin{prop}
    \label{prop:compute_opt_cost}
    Let $L$ be a looping branch containing $n$ distinct transitions $t_i = (c_i, \sigma_i, \tau_i)$ such that each $t_i$ has spread parameters $P(t_i) = (d_i, d'_i)$. The cost $opt(L)$ of the optimal coupling strategy for $L$ can be computed by solving the following optimization problem: 
    \begin{align*}
        opt(L) = \max_{\texttt{in}\brangle{2}-\texttt{in}\brangle{1} \in [-1, 1]^n} &\min_{\gamma, \gamma' \in [-1, 1]^n} \sum_{i = 1}^n \left(|\texttt{in}_i\brangle{2}-\texttt{in}_i\brangle{1} - \gamma_i| d_i + |\texttt{in}_i\brangle{2}-\texttt{in}_i\brangle{1} - \gamma_i'|d_i' \right)\\ 
            \text{subject to }
            &\ \gamma_{at(i)} \leq \gamma_i \text{ if } c_i = \gguard, \\
            &\ \gamma_{at(i)} \geq \gamma_i \text{ if } c_i = \lguard, \\
            &\ \gamma_i = 0 \text{ if } \sigma_i = \texttt{insample}, \\
            &\ \gamma_i' = 0 \text{ if } \sigma_i = \texttt{insample}'\\
            &\ \gamma_i = \gamma_i'= \texttt{in}_i\brangle{2}-\texttt{in}_i\brangle{1} \text{ if } t_i \text{ is in a cycle}
    \end{align*}
    If a looping branch $L$ is not differentially private, a solution to this optimization problem does not exist, and we write $opt(L) = \infty$.
\end{prop}

Observe that the inner problem is convex, and so the outer problem is that of convex maximization. 

In the absence of an immediate polynomial time algorithm \azadeh{vague} for solving this optimization problem, we consider an approximation of the problem for looping branches that are known to be private. 

\begin{defn}[Approximate Privacy Cost]
    Let $L$ be a differentially private looping branch generated by $G$ containing $n$ distinct transitions $t_i = (c_i, \sigma_i, \tau_i)$ such that each $t_i$ has spread parameters $P(t_i) = (d_i, d'_i)$.
    Let $I$ be the set of transitions in $L$ that do \textit{not} appear in a cycle in $G$. The \textbf{approximate privacy cost} $approx(L)$ of $L$ is defined as:
    \begin{align*} 
        approx(L) = &\sum_{t_i \text{outputs \texttt{insample}'}} d_i' + \min_{\gamma \in [-1, 1]^n} \sum_{t_i \in I} \left(1 + |\gamma_i| \right) d_i  \\
            \text{subject to } 
            &\ \gamma_{at(i)} \leq \gamma_i \text{ if } c_i = \gguard, \\
            &\ \gamma_{at(i)} \geq \gamma_i \text{ if } c_i = \lguard, \\
            &\ \gamma_i = 0 \text{ if } \sigma_i = \texttt{insample}, \\
            &\ \gamma_i = 1 \text{ if } t_i \text{ is in a cycle and } c_i = \lguard,\\ 
            &\ \gamma_i = -1 \text{ if } t_i \text{ is in a cycle and } c_i = \gguard
    \end{align*}
\end{defn}
Observe that this is the cost of a coupling strategy $(\gamma, \gamma')$ such that $\gamma$ and $\gamma'$ do not depend on $\texttt{in}\brangle{1}$ or $\texttt{in}\brangle{2}$ for non-cyclic transitions. 

Importantly, we show that the approximate LP is solvable in polynomial time through an application of the ellipsoid method.

\begin{prop}\label{approximateSolutionPolyTimeProp}
    The approximate privacy cost of a differentially private looping branch $L$ can be computed in time polynomial in $n$, the number of distinct transitions in $L$.
\end{prop}

As hoped, solving this approximation still provides us with a valid coupling strategy.

\begin{prop}
    \label{prop:approx_exists}
    Given a differentially private looping branch $L$, there exists a valid coupling strategy $C_L$ for $L$ such that $cost(C_L) = approx(L)$.

    Moreover, if $\gamma \in [-1, 1]^n$ satisfies the approximate privacy constraints, then there is a valid coupling strategy $C_L = (\gamma^*, {\gamma'}^*)$ for $L$ such that $\gamma_i^*(\texttt{in}\brangle{1}, \texttt{in}\brangle{2}) = \gamma_i$ for all $t_i$ that do not appear in cycles. 
\end{prop}

Further, we obtain an approximation factor bounded by a term linear in the number of non-cyclic transitions in $L$. 

\begin{prop}
    \label{prop:approx_opt_are_close}
    For a looping branch $L$ with $n$ distinct transitions, we have 
    \[opt(L) \leq approx(L) \leq opt(L) + \sum_{i \in I }^n d_i + \sum_{t_i \text{outputs \texttt{insample}}} d_i'\]
    where $I$ is the set of transitions in $L$ that do $\textit{not}$ appear in a cycle.
\end{prop}

Observe that our decision algorithm for privacy (section \ref{decisionSection}) effectively encodes the approximate privacy constraints into the privacy constraint graph; it is also possible, if less intuitive, to directly encode the approximate privacy constraints as an instance of 2SAT. 


\azadeh{check with sasho about this part}
We conjecture that the optimal coupling cost, as represented by the linear program from \ref{prop:compute_opt_cost}, exactly matches the ``true'' privacy cost of a program, as represented by the KL-divergence between the distributions of a looping branch's outputs on adjacent input sequences. In this case, coupling proofs would provide an exact bound on the privacy cost of any program that can be represented by our model. 

\begin{conj}
    For a looping branch $L$, the optimal coupling cost $opt(L)$ of $L$ matches the ``true'' privacy cost, i.e.:
    \begin{align*}
        opt(L) = \sup_{\rho \in L} \sup_{\texttt{in}\brangle{1} \sim \texttt{in}\brangle{2}} D_\infty(\PP[\rho, \texttt{in}\brangle{1}, \cdot]\; ||\; \PP[\rho, \texttt{in}\brangle{2}, \cdot])
    \end{align*}
    representing the worst-case privacy loss over all paths in $L$, all valid adjacent inputs, and all measurable output events.
\end{conj}
