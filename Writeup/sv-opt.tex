
Beyond solving the decision problem of privacy, we can also optimize the specific privacy cost obtained by a coupling proof of privacy for a program. 
In general, there is no way to reuse couplings for shared transitions between different SLPs if we are trying to find an optimal coupling strategy. In other words, one cannot choose a single coupling strategy for an arbitrary set of SLPs; we must choose a new coupling strategy for each SLP.

For example, even if two SLPs share a prefix, we cannot necessarily ``reuse'' the shifts from an optimal coupling strategy for the prefix in the context of the first SLP in the context of an optimal coupling strategy for the second SLP. 

\begin{prop}\label{noSharingStrategiesProp}
  There exist SLPs $\rho'=\pi\cdot \pi$, $\rho^*=\pi\cdot\pi^*$ that share a prefix SLP $\pi$ such that for all optimal coupling strategies $C'$ for $\rho'$ and $C^*$ for $\rho^*$, $C'$ and $C^*$ differ on the shift assigned to some transition in $\pi$. 
\end{prop}

A specific counterexample is provided as proposition \ref{costDependspathProp} in the appendix. 

In particular, we show that the penalty to choosing only a single coupling strategy for a set of SLPs increases quadratically in the size of the SLPs.

\begin{prop}\label{quadraticPenaltyProp}
  For all $n\in \NN$, there exist sets of SLPs $B_n$ such that the cost of a coupling strategy that uses the same set of shifts for all shared transitions between SLPs in $B_n$ has cost that is a quadratic factor larger than the cost of an optimal coupling strategy. 
\end{prop}

Again, we provide a specific counterexample as proposition \ref{quadraticPenaltyProp} in the appendix. 

Fortunately, while we cannot analyze \textit{arbitrary} sets of SLPs with the same coupling strategy, we can still analyze an entire \textit{periodic program} with a single coupling strategy.
We show that it is in fact optimal to only consider a single coupling strategy for an entire periodic program; not only is finding individual coupling strategies for every single SLP in a periodic program intractable, but it also does not lead to a better overall privacy cost. 

\begin{prop}\label{ClassCouplingStrategiesAreEnoughProp}
    If, for every SLP $\rho$ of a periodic program $L$, there exists a valid coupling strategy $C_\rho$ such that $\sup_{\rho\in L}cost(C_\rho)< \infty$, then there exists a valid shared coupling strategy $C_L$ for $L$ such that $cost(C_L) \leq \sup_{\rho\in L}cost(C_\rho)$. 
\end{prop}

As with the decision problem, we optimize the cost of periodic programs independently.

The privacy constraint system directly defines a linear program that computes the optimal cost of any coupling strategy for a periodic program. 

\begin{prop}
    \label{prop:compute_opt_cost}
    Let $L$ be a periodic program containing $n$ distinct transitions $t_i = (c_i, \sigma_i, \tau_i)$ such that each $t_i$ has spread parameters $P(t_i) = (d_i, d'_i)$. The cost $opt(L)$ of the optimal coupling strategy for $L$ can be computed by solving the following optimization problem: 
    \begin{align*}
        opt(L) = \max_{\texttt{in}\brangle{1}\sim\texttt{in}\brangle{2}} &\min_{\gamma, \gamma' \in [-1, 1]^n} \sum_{i = 1}^n \left(|\texttt{in}_i\brangle{2}-\texttt{in}_i\brangle{1} - \gamma_i| d_i + |\texttt{in}_i\brangle{2}-\texttt{in}_i\brangle{1} - \gamma_i'|d_i' \right)\\ 
            \text{subject to }
            &\ \gamma_{at(i)} \leq \gamma_i \text{ if } c_i = \gguard, \\
            &\ \gamma_{at(i)} \geq \gamma_i \text{ if } c_i = \lguard, \\
            &\ \gamma_i = 0 \text{ if } \sigma_i = \texttt{insample}, \\
            &\ \gamma_i' = 0 \text{ if } \sigma_i = \texttt{insample}'\\
            &\ \gamma_i = \gamma_i'= \texttt{in}_i\brangle{2}-\texttt{in}_i\brangle{1} \text{ if } t_i \text{ is in a cycle}
    \end{align*}
    If a periodic program $L$ is not differentially private, a solution to this optimization problem does not exist, and we write $opt(L) = \infty$.
\end{prop}

Observe that the inner problem is convex, and so the outer problem is that of convex maximization. However, it is unclear whether or not a polynomial time algorithm for solving this optimization problem exists; instead, we consider an approximation of the problem for periodic programs that are known to be private. 

\begin{defn}[Approximate Privacy Cost]
    Let $L$ be a differentially private periodic program generated by $G$ containing $n$ distinct transitions $t_i = (c_i, \sigma_i, \tau_i)$ such that each $t_i$ has spread parameters $P(t_i) = (d_i, d'_i)$.
    Let $I$ be the set of transitions in $L$ that do \textit{not} appear in a cycle in $G$. The \textbf{approximate privacy cost} $approx(L)$ of $L$ is defined as:
    \begin{align*} 
        approx(L) = &\sum_{t_i \text{outputs \texttt{insample}'}} d_i' + \min_{\gamma \in [-1, 1]^n} \sum_{t_i \in I} \left(1 + |\gamma_i| \right) d_i  \\
            \text{subject to } 
            &\ \gamma_{at(i)} \leq \gamma_i \text{ if } c_i = \gguard, \\
            &\ \gamma_{at(i)} \geq \gamma_i \text{ if } c_i = \lguard, \\
            &\ \gamma_i = 0 \text{ if } \sigma_i = \texttt{insample}, \\
            &\ \gamma_i = 1 \text{ if } t_i \text{ is in a cycle and } c_i = \lguard,\\ 
            &\ \gamma_i = -1 \text{ if } t_i \text{ is in a cycle and } c_i = \gguard
    \end{align*}
\end{defn}
The approximations made simplify the system from a nested optimization problem by assuming the worst case where $|\texttt{in}\brangle{1} - \texttt{in}\brangle{2}| = 1$; additionally, for \textit{non-cyclic} transitions, the approximate system assumes that $\gamma$ and $\gamma'$ do not depend on $\texttt{in}\brangle{1}$ or $\texttt{in}\brangle{2}$. 

Importantly, we show that the approximate LP is solvable in polynomial time through an application of the ellipsoid method.

\begin{prop}\label{approximateSolutionPolyTimeProp}
    The approximate privacy cost of a differentially private periodic program $L$ can be computed in time polynomial in $n$, the number of distinct transitions in $L$.
\end{prop}

As hoped, solving this approximation still provides us with a valid coupling strategy.

\begin{prop}
    \label{prop:approx_exists}
    Given a differentially private periodic program $L$, there exists a valid coupling strategy $C_L$ for $L$ such that $cost(C_L) = approx(L)$.

    Moreover, if $\gamma \in [-1, 1]^n$ satisfies the approximate privacy constraints, then there is a valid coupling strategy $C_L = (\gamma^*, {\gamma'}^*)$ for $L$ such that $\gamma_i^*(\texttt{in}\brangle{1}, \texttt{in}\brangle{2}) = \gamma_i$ for all $t_i$ that do not appear in cycles. 
\end{prop}

Further, we obtain an approximation factor bounded by a term linear in the number of non-cyclic transitions in $L$. 

\begin{prop}
    \label{prop:approx_opt_are_close}
    For a periodic program $L$ with $n$ distinct transitions, we have 
    \[opt(L) \leq approx(L) \leq opt(L) + \sum_{i \in I }^n d_i + \sum_{t_i \text{outputs \texttt{insample}}} d_i'\]
    where $I$ is the set of transitions in $L$ that do $\textit{not}$ appear in a cycle.
\end{prop}

We remark that our decision algorithm for privacy (section \ref{decisionSection}) is a simplified encoding of approximate privacy constraints into a privacy constraint graph; we also note that it is possible, if less intuitive, to directly reduce the approximate privacy constraints to an instance of 2SAT. 

\azadeh{check with sasho about this part}
We conjecture that the optimal coupling cost, as represented by the linear program from proposition \ref{prop:compute_opt_cost}, exactly matches the ``true'' privacy cost of a program, as represented by the max-divergence between the distributions of a program's outputs on adjacent input sequences. In this case, coupling proofs would provide an exact bound on the privacy cost of any program that can be represented by our model. 

\begin{conj}
    For a periodic program $L$, the optimal coupling cost $opt(L)$ of $L$ matches the ``true'' privacy cost, i.e.:
    \begin{align*}
        opt(L) = \sup_{\rho \in L} \sup_{\texttt{in}\brangle{1} \sim \texttt{in}\brangle{2}} D_\infty(\PP[\rho, \texttt{in}\brangle{1}, \cdot]\; ||\; \PP[\rho, \texttt{in}\brangle{2}, \cdot])
    \end{align*}
    representing the worst-case privacy loss over all SLPs in $L$, all valid adjacent inputs, and all measurable output events.
\end{conj}
