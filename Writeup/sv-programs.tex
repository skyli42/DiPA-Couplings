
\subsection{Programs}

We now introduce our complete program model. 

\begin{defn}
    A program $P$ is a finite union of looping branches such that $P$ is generated by some structured program graph $G_P$. 
\end{defn}

Equivalently, $P$ is a language described by a regular expression in union normal form such that each individual term describes a looping branch. 

As illustrated by branching programs, we cannot do any better with regards to coupling strategies than simply choosing one coupling strategy per looping branch. 

\begin{lemma}
    If, for every looping branch $L$ in $P$, there exists a valid coupling strategy $C_L$, then $P$ is $(\max_{L\subseteq P} cost(C_L))\varepsilon$-differentially private. 
\end{lemma}

\subsection{Deciding Privacy}

In this section, we discuss the boolean or decision problem of privacy; that is, deciding whether or not there exists \textit{any} finite $d>0$ such that a program is $d\varepsilon$-differentially private. 

Clearly, we can algorithmically show that at least some subset of differentially private programs are private through the use of couplings and coupling strategies. 

\begin{lemma}
    If, for every looping branch $L\subseteq P$ in a program $P$, there exists a coupling strategy $C_L$ that satisfies the privacy constraint system, then there exists some finite $d>0$ such that $P$ is $d\varepsilon$-differentially private.    
\end{lemma}
\begin{proof}
    Follows immediately from proposition \ref{privacyFiniteCostProp}.
\end{proof}

As previously mentioned, we will show that coupling proofs are \textbf{complete} for certain programs of this form; every differentially private program can be proved to be private using couplings. 

In particular, the key condition that a program must have is that outputs must uniquely identify paths; for any output $\sigma$ of a program $P$, there must exist exactly one path in $P$ that could have produced $\sigma$. This means that we can analyze a single path in $P$ for violations of privacy, rather than potentially having to analyze multiple paths in $P$ that could produce the same output together. 

We impose this condition through \textbf{output distinction}:

\begin{defn}[Output Distinction]
    For a program $P$ generated by $G_P = (V_P, E_P)$, we say that $P$ satisfies \textbf{output distinction} if for all locations $\ell\in V_P$, if there exist two distinct edges $(\ell, \ell')$ labeled by $t'=(c', \sigma', \tau')$ and $(\ell, \ell^*)$ labeled by $t^* = (c^*, \sigma^*, \tau^*)$, then $\sigma' \neq \sigma^*$. Additionally, at least one of $\sigma'\in \Gamma$, $\sigma^*\in \Gamma$ is true (i.e. it cannot be that both transitions output a real number).
\end{defn}

\begin{lemma}\label{ProgramCounterexampleLemma}
    If a program $P$ satisfies output distinction and, for some looping branch $L\subseteq P$, there does not exist a coupling strategy $C_L$ that satisfies the privacy constraint system, then there does not exist any finite $d>0$ such that $P$ is $d\varepsilon$-differentially private.
\end{lemma}

To complete the proof, we introduce a previously analyzed program model known as DiPA, which we claim captures the same class\sky{we could rewrite this to claim it captures some superset of our program model - our model becomes cleaner but we lose equivalence} of programs as our program model. 

\subsection{DiPA}

We now discuss a previously defined program model, DiPA, which turns out to be exactly equivalent to our own program model. The equivalence between DiPA and programs allows us to construct our completeness proof. 

\begin{defn}[\cite{chadhaLinearTimeDecidability2021}]
    A Differentially Private Automaton (DiPA) $A$ is an 8-tuple $(Q, \Sigma, C, \Gamma, q_{init}, X, P, \delta)$ where
    \begin{itemize}
        \item $Q$ is a finite set of locations partitioned into input locations $Q_{in}$ and non-input locations $Q_{non}$. 
        \item $\Sigma = \RR$ is the input alphabet
        \item $C = \{\texttt{true}, \lguard, \gguard\}$ is a set of guard conditions
        \item $\Gamma$ is a finite output alphabet
        \item $q_{init}\in Q$ is the initial location
        \item $X = \{\texttt{x}, \texttt{insample}, \texttt{insample}'\}$ is a set of variables
        \item $P: Q\to \QQ\times \QQ^{\geq 0}\times \QQ\times  \QQ^{\geq 0}$ is a parameter function that assigns sampling parameters for the Laplace distribution for each location
        \item $\delta:(Q\times C)\to (Q\times (\Gamma \cup \{\texttt{insample}, \texttt{insample}'\})\times \{\texttt{true}, \texttt{false}\})$ is a partial transition function. 
    \end{itemize}
    In addition, $\delta$ must satisfy some additional conditions:
    \begin{itemize}
        \item \textbf{Determinism:} For any location $q\in Q$, if $\delta(q,\texttt{true})$ is defined, then $\delta(q,\lguard)$ and $\delta(q,\gguard)$ are not defined. 

        \item \textbf{Output Distinction:} For any location $q\in Q$, if $\delta(q, \gguard) = (q_1, o_1, b_1)$ and $\delta(q, \lguard) = (q_2, o_2, b_2)$, then $o_1\neq o_2$ and at least one of $o_1\in \Gamma$ and $o_2\in \Gamma$ is true.

        \item \textbf{Initialization:} The initial location $q_0$ has only one outgoing transition of the form $\delta(q_0, \texttt{true}) = (q, o, \texttt{true})$.

        \item \textbf{Non-input transition:} From any $q\in Q_{non}$, if $\delta(q, c)$ is defined, then $c=\texttt{true}$.
    \end{itemize}
\end{defn}

A DiPA operates as follows: 
\begin{itemize}
    \item At each location, a real-valued input $\texttt{in}$ is read in and two variables $\texttt{insample}\sim \Lap(\texttt{in}, d\varepsilon)$ and $\texttt{insample}'\sim\Lap(\texttt{in}, d\varepsilon)$ are sampled.
    \item $\texttt{insample}$ is compared to the stored variable $\texttt{x}$, and depending on the guards of the transitions out of the current location, changes the current location and outputs a value. This value can either be $\texttt{insample}, \texttt{insample}'$, or a symbol from $\Gamma$.
    \item Finally, the  value of $\texttt{x}$ is optionally updated with the value of $\texttt{insample}$.
\end{itemize}

Just like with programs, a path $\rho$ in a DiPA $A$ reads a real-valued input sequence $\texttt{in}\in \RR^n$; the same definitions of adjacency and validity apply to DiPA input sequences.

We first establish notation for discussing the probabilities of different paths in a DiPA, which allows us to define $d\varepsilon$-differential privacy. 

\begin{defn} 
    Let $\rho$ be a path in a DiPA $A$, let $\texttt{in}$ be a valid input sequence and let $o$ be a possible output of $\rho$. In particular, if $\sigma_i\in \{\texttt{insample}, \texttt{insample}'\}$, then we require that $o_i$ is an \textit{interval} $(a, b)\subseteq \RR$, rather than simply a measurable set as before. 
    Then $\texttt{Pr}[x, \rho, \texttt{in}, o]$ is the probability of $\rho$ being taken with input sequence $\texttt{in}$ and outputting $o$. If the first location in $\rho$ is $q_{init}$, then $\texttt{Pr}[x, \rho, \texttt{in}, o]$ may be shortened to $\texttt{Pr}[\rho, \texttt{in}, o]$, since the initial value of $\texttt{x}$ is irrelevant.
\end{defn}

For a full definition of DiPA semantics, we refer back to the original work. 

\begin{defn}
    A DiPA $A$ is $d\varepsilon$-differentially private for some $d>0$ if for all paths $\rho$ in $A$, for all possible outputs $o$ of $\rho$ and valid adjacent input sequences $\texttt{in}\brangle{1}\sim \texttt{in}\brangle{2}$, \[
        \PP[\rho, \texttt{in}\brangle{1}, o]\leq e^{d\varepsilon} \PP[\rho, \texttt{in}\brangle{2}, o]
    \]
\end{defn}

Notably, we show an equivalence between programs and DiPAs.

\sky{Todo: rewrite with new program model}
\begin{prop}
    Every path $\rho$ through a DiPA $A$ is represented by a complete path $\hat{\rho}$ comprised of transitions from a valid transition alphabet $\Sigma_T$; further, the set of all possible paths through $A$ is a regular language over $\Sigma_T$.
\end{prop}
\begin{proof}
    Let $\rho = q_0\to q_1\to\ldots \to q_n$ be a path in a DiPA $A=(Q, \RR, C, \Gamma, q_0, X, P, \delta)$. 

    For all $i\in 0\ldots n-1$, there must be some $c_i$ such that $\delta(q_i, c_i) = (q_{i+1}, \sigma_i, \tau_i)$. Let $t_i = (q_i, q_{i+1}, c_i, \sigma_i, \tau_i)$ and let $\Sigma_\rho = \{t_i: i\in 0\ldots n-1\}$ be the set of all such transitions. Note that because $\delta$ satisfies the conditions of determinism, output distinction, initialization, and non-input transition, $\Sigma_\rho$ must as well. 
    Then let $\hat{\rho} = t_0\cdot t_1\cdot \ldots\cdot t_{i-1}$ be the representation of $\rho$ as a word over $\Sigma_\rho$. 

    Let $\Sigma_T = \bigcup_{\rho\in A}\Sigma_\rho$. Note that $\Sigma_T$ must have finite size because $A$ is a finite automaton and must still be a valid transition alphabet.

    Let $D = (Q, \Sigma_T, \delta_D, q_0, F=Q)$ be an NFA defined over the set of program locations $Q$ such that $\delta_D$ is defined as follows:
    Let $q\in Q$ be an arbitrary location. If $\delta(q, c) = (q', \sigma, \tau)$ is defined for some $c\in C$, let $\delta_D(q, (q, q', c, \sigma, \tau)) = q'$.

    Then clearly every path in $A$ is also a path in $D$ and vice versa; since every location in $D$ is an accepting location, $\mathcal{L}(D) = \{\hat{\rho}: \rho\in A\}$. Thus, the set of (representations of) all paths in $A$ must be a regular language. 
\end{proof}

\sky{todo: rewrite with new program model}
\begin{prop}
    For every program $P$ over a valid transition alphabet $\Sigma_T$, there exists a corresponding DiPA $A$ such that there exists a path $\rho$ in $A$ if and only if its representation is in $P$. 
\end{prop}
\begin{proof}
    The DiPA can be directly constructed from $P$.
\end{proof}

As expected, the probability of a path ``succeeding'' in a program is the same as the probability of a path being traversed in a DiPA. 

\begin{prop}
    For all paths $\rho$ in a DiPA $A$ and for all input sequences $\texttt{in}$ and possible outputs $\sigma$ of $\rho$, $\PP[\rho, \texttt{in}, \sigma] = \PP[\hat{\rho}, \texttt{in}, \sigma]$.
\end{prop}


Interestingly, the privacy of any DiPA $A$ is completely characterized by four graph-theoretic structures---we refer to the original work for full definitions.

\begin{thm}[\cite{chadhaLinearTimeDecidability2021}]\label{DiPACounterexamplesThm}
    A DiPA $A$ does not have a leaking cycle, leaking pair, disclosing cycle, or privacy violating path if and only if there exists some $d>0$ such that for all $\varepsilon>0$, $A$ is $d\varepsilon$-differentially private. 
\end{thm}

This provides us with a method of demonstrating that a program is not $d\varepsilon$-differentially private for any $d>0$.

\sky{this can just be rewritten in terms of the graph?}
\begin{cor}
    If the corresponding DiPA $A$ to a program $P$ contains a leaking cycle, leaking pair, disclosing cycle, or privacy violating path, then does not exist a finite $d>0$ such that $P$ is $d\varepsilon$-differentially private. 
\end{cor}

Our key result relates the privacy constraint system and these graph-theoretic structures.

\sky{would also have to be rewritten in terms of graph}
\begin{lemma}
    If a looping branch $L$ satisfies output distinction and there does not exist a coupling strategy $C$ that satisfies the privacy constraint system for $L$, then the corresponding DiPA to $L$ must contain either a leaking cycle, a leaking pair, a disclosing cycle, or a privacy violating path. 
\end{lemma}