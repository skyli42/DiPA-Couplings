\subsection{Decidability of Privacy}

Perhaps surprisingly, we now show that, for a subclass of programs under our program model, the privacy constraint system (definition \ref{privacyConstraintSystem}) is \textit{complete}; that is, if there exist no solutions to the privacy constraint system for a looping branch $L$, then $L$ is not private. 

The specific class of programs we show completeness for is the class of programs such that outputs to the program uniquely identify paths; for any output $\sigma$ of a program $P$, there must exist exactly one path in $P$ that could have produced $\sigma$. In particular, this means that if a path-program fails to be private, a program containing that path must also fail be to private. 

We call this condition \textbf{output distinction}:

\azadeh{move this to proper graph defn?}
\sky{the forward direction doesn't need it, so it felt more appropriate putting it here}
\begin{defn}[Output Distinction]\label{outputDistinctionDef}
    For a program $P$ generated by $G_P = (V_P, E_P)$, we say that $P$ satisfies \textbf{output distinction} if for all locations $\ell\in V_P$, if there exist two distinct edges $(\ell, \ell')$ labeled by $t'=(c', \sigma', \tau')$ and $(\ell, \ell^*)$ labeled by $t^* = (c^*, \sigma^*, \tau^*)$, then $\sigma' \neq \sigma^*$. Additionally, at least one of $\sigma'\in \Gamma$, $\sigma^*\in \Gamma$ is true (i.e. it cannot be that both transitions output a real number).
\end{defn}

\begin{prop}\sky{probably move this to appendix}
    Let $P$ be a program generated by $G_P$ that satisfies output distinction. Let $O\subseteq (\gamma\cup \RR)^*$ be the set of all possible outputs of paths in $P$. There exists an injection $f: O\to P$ from the set of all possible outputs to paths in $P$. 
\end{prop}

\azadeh{rephrase to be more impactful}
For any program $P$ that satisfies output distinction, we show that the existence of coupling proofs completely characterizes the privacy of $P$.

\begin{thm}\label{ProgramCounterexampleThm}
    If a program $P$ satisfies output distinction and, for some looping branch $L\subseteq P$, there does not exist a coupling strategy $C_L$ that satisfies the privacy constraint system, then there does not exist any finite $d>0$ such that $P$ is $d\varepsilon$-differentially private.
\end{thm}

\subsection{DiPA}

We have shown that we can decide whether or not a program is differentially private for the class of output-distinct programs under our program model. In particular, we argue that this procedure is conceptually simple (as we show later, it is also efficient) and demonstrate that it is \textit{complete} for this class of programs.

Notably, Chadha, Sistla, and Viswanathan \cite{chadhaLinearTimeDecidability2021} also have shown, through a completely different proof methodology, that there exists a complete and efficient decision procedure for determining if programs derived from an automata-theoretic model are differentially private. 

We demonstrate that these two program models are, in fact, equivalent. Additionally, we show that the privacy constraint system for looping branches can partially explain the somewhat arbitrary \sky{is this too harsh?} graph conditions checked for in the decision procedure of \cite{chadhaLinearTimeDecidability2021}.

We first introduce the program model of \cite{chadhaLinearTimeDecidability2021}, which is called DiPA:

\begin{defn}[\cite{chadhaLinearTimeDecidability2021}]
    A Differentially Private Automaton (DiPA) $A$ is an 8-tuple $(Q, \Sigma, C, \Gamma, q_{init}, X, P, \delta)$ where
    \begin{itemize}
        \item $Q$ is a finite set of locations partitioned into input locations $Q_{in}$ and non-input locations $Q_{non}$. 
        \item $\Sigma = \RR$ is the input alphabet
        \item $C = \{\texttt{true}, \lguard, \gguard\}$ is a set of guard conditions
        \item $\Gamma$ is a finite output alphabet
        \item $q_{init}\in Q$ is the initial location
        \item $X = \{\texttt{x}, \texttt{insample}, \texttt{insample}'\}$ is a set of variables
        \item $P: Q\to \QQ\times \QQ^{\geq 0}\times \QQ\times  \QQ^{\geq 0}$ is a parameter function that assigns sampling parameters for the Laplace distribution for each location
        \item $\delta:(Q\times C)\to (Q\times (\Gamma \cup \{\texttt{insample}, \texttt{insample}'\})\times \{\texttt{true}, \texttt{false}\})$ is a partial transition function. 
    \end{itemize}
    In addition, $\delta$ must satisfy the following conditions:
    \begin{itemize}
        \item \textbf{Determinism:} For any location $q\in Q$, if $\delta(q,\texttt{true})$ is defined, then $\delta(q,\lguard)$ and $\delta(q,\gguard)$ are not defined. 

        \item \textbf{Output Distinction:} For any location $q\in Q$, if $\delta(q, \gguard) = (q_1, o_1, b_1)$ and $\delta(q, \lguard) = (q_2, o_2, b_2)$, then $o_1\neq o_2$ and at least one of $o_1\in \Gamma$ and $o_2\in \Gamma$ is true.

        \item \textbf{Initialization:} The initial location $q_0$ has only one outgoing transition of the form $\delta(q_0, \texttt{true}) = (q, o, \texttt{true})$.

        \item \textbf{Non-input transition:} From any $q\in Q_{non}$, if $\delta(q, c)$ is defined, then $c=\texttt{true}$.
    \end{itemize}
\end{defn}

A DiPA operates as follows: 
\begin{itemize}
    \item At each location $q$ such that $P(q) = (a_q, d_q, a_q', d_q')$, a real-valued input $\texttt{in}$ is read in and two variables $\texttt{insample}\sim \Lap(\texttt{in}+a_q, \frac{1}{d_q\varepsilon})$ and $\texttt{insample}'\sim\Lap(\texttt{in}+a_q', \frac{1}{d_q'\varepsilon})$ are sampled. If $q\in Q_{non}$, then $\texttt{in}$ must be 0.
    \item $\texttt{insample}$ is compared to the stored variable $\texttt{x}$, and depending on the guards of the transitions out of the current location, the current location is changed and a value is output. This value can either be $\texttt{insample}, \texttt{insample}'$, or a symbol from $\Gamma$.
    \item Finally, the  value of $\texttt{x}$ is optionally updated with the value of $\texttt{insample}$.
\end{itemize}

Just like with programs, a path $\rho$ in a DiPA $A$ reads a real-valued input sequence $\texttt{in}\in \RR^n$.

We establish notation for discussing the probabilities of different paths in a DiPA, which allows us to define $d\varepsilon$-differential privacy. 

\begin{defn} 
    Let $\rho$ be a path in a DiPA $A$, let $\texttt{in}$ be a valid input sequence and let $o$ be an \textit{interval} $o\subseteq \RR$ of real numbers.
    Then $\texttt{Pr}[x, \rho, \texttt{in}, o]$ is the probability of $\rho$ being taken with input sequence $\texttt{in}$ and outputting $o$. If the first location in $\rho$ is $q_{init}$, then $\texttt{Pr}[x, \rho, \texttt{in}, o]$ may be shortened to $\texttt{Pr}[\rho, \texttt{in}, o]$, since the initial value of $\texttt{x}$ is irrelevant.
\end{defn}

For a full definition of DiPA semantics, we refer back to \cite{chadhaLinearTimeDecidability2021}. 

\begin{defn}
    A DiPA $A$ is $d\varepsilon$-differentially private for some $d>0$ if for all paths $\rho$ in $A$, for all possible outputs $o$ of $\rho$ and valid adjacent input sequences $\texttt{in}\brangle{1}\sim \texttt{in}\brangle{2}$, \[
        \PP[\rho, \texttt{in}\brangle{1}, o]\leq e^{d\varepsilon} \PP[\rho, \texttt{in}\brangle{2}, o]
    \]
\end{defn}

We argue that our program model completely encapsulates the set of programs modelable by DiPA. 
For example, observe that each condition on $\delta$ corresponds to a restriction on program-generating graphs; in particular, the non-input transition condition corresponds to public transitions. 
Indeed, the class of \textit{output distinct} programs under our program model and the class of programs modeled by DiPA are exactly equivalent. 

\begin{prop}
    If a program $P$ generated by $G_P$ satisfies output distinction, then $G_P$ represents a valid DiPA $A_P$, and, for any DiPA $A$, the automata graph of $A$ generates a program $P_A$. 
\end{prop}

We note that the probability of a run $\rho$ of a generated DiPA $G_P$ producing an output $\sigma$ is equal to the probability of the corresponding path in $P$ producing the same output, so these models are indeed equivalent as expected.

\begin{prop}
    Let $P$ be a program generated by a DiPA $G_P$. For all runs $\rho$ of $G_P$, all possible output events $o$ of $\rho$, and input sequences $\texttt{in}$, \[
        \PP[\rho, \texttt{in}, o]=\PP[\Psi(\rho), \texttt{in}, o],
    \]
    where $\Psi$ is the forgetful homomorphism that drops states from runs. 
\end{prop}

The decision procedure of \cite{chadhaLinearTimeDecidability2021} analyzes the existence of four graph-theoretic structures in the graph of a DiPA $A$, called leaking cycles, leaking pairs, disclosing cycles, and privacy violating paths. 

As an example, we define leaking pairs, and refer back to \cite{chadhaLinearTimeDecidability2021} for full definitions of the other three graph structures.

\begin{defn}
    A cycle $\rho$ of a DiPA $A$ is called an \lcycle~(respectively, \gcycle) if there is an $i< |\rho|$ such that the guard of the $i$'th transition in $\rho$ is $\lguard$ (respectively, $\gguard$).

    We say that a path $\rho$ of a DiPA $A$ is an \texttt{AL}-path (respectively, \texttt{AG}-path) if all assignment transitions on $\rho$ have guard $\lguard$ (respectively, $\gguard$).

    Observe that a cycle can be both an \lcycle~and a \gcycle.
    Further, a path with no assignment transitions (including the empty path) is simultaneously both an AL-path and an AG-path.
\end{defn} 
\begin{defn}[Leaking Pairs~\cite{chadhaLinearTimeDecidability2021}]\label{defLeakingPairs}
    A pair of cycles $(C, C')$ in a DiPA A is called a leaking pair if one of the following two conditions is satisfied:
    \begin{enumerate}
        \item $C$ is an \lcycle, $C'$ is a \gcycle, and there is an AG-path
        from a state in $C$ to a state in $C'$.
        \item $C$ is a \gcycle, $C'$ is an \lcycle, and there is an AL-path
        from a state in $C$ to a state in $C'$.
    \end{enumerate}
\end{defn} 

Specifically, the existence of \textit{any} such ``problematic'' graph structures completely characterizes the privacy of a DiPA. 

\begin{thm}[\cite{chadhaLinearTimeDecidability2021}]\label{DiPACounterexamplesThm}
    For a DiPA $A$, there exists some $d>0$ such that for all $\varepsilon>0$, $A$ is $d\varepsilon$-differentially private if and only if the graph of $A$ does not contain a leaking cycle, leaking pair, disclosing cycle, or privacy violating path.
\end{thm}

Unfortunately, the graph structures used by \cite{chadhaLinearTimeDecidability2021} to decide the privacy of a DiPA can be rather obtuse; it is hard to see why these four specific structures would characterize the privacy of a program. However, there is a direct relationship between these graph structures and the privacy constraint system of our model. 

As a simple example, a disclosing cycle can be defined in the generating graph $G$ of a program as any cycle $C$ such that $C$ contains an edge labeled by a non-public transition that outputs either $\texttt{insample}$ or $\texttt{insample}'$. 
Then observe that the existence of a disclosing cycle in a graph directly implies that constraints (3) and (5) cannot both be satisfied for the looping branch that $C$ lies in.

A similar pattern holds for the other graph structures. For example, consider a leaking pair $C, C'$ such that $C$ is an \lcycle~and $C'$ is a \gcycle. 
Through constraints (1) and (5), it can be shown that if $t_i$ is an assignment transition immediately preceding an \lcycle~in a looping branch, we must have that $\gamma_i = 1$. Symmetrically, constraints (2) and (5) require that for all assignment transitions $t_i$ immediately preceding a \gcycle~in a looping branch, we must have that $\gamma_i =-1$. 
However, for every assignment transition $t_j = (c_j, \sigma_j, \tau_j)$ on the $\texttt{AG}$-path from $C$ to $C'$, constraint (2) would thus require that $\gamma_j = 1$, since $\gamma_{at(j)} = 1$ and $c_j =\gguard[\texttt{x}]$. Thus, the assignment transition $t_i$ immediately preceding $C'$ is required to have shift $\gamma_i = 1$ \textit{and} $\gamma_i = -1$, a contradiction.

\sky{these examples could also go in the other direction (i.e. constraint violation $\implies$ graph structure), but im not sure which is more meaningful/gets the point across better}


Thus, we argue that the DiPA graph conditions are best viewed as a summary of the different ways in which the privacy constraint system can be unsatisfiable for a looping branch. In this sense, we posit that coupling proofs provide a simpler characterization and rationale for the decision problem of privacy. 


\sky{move this to appendix later? or is still worth keeping}
\begin{lemma}
    If a looping branch $L$ generated by $G_L$ satisfies output distinction and there does not exist a coupling strategy $C$ that satisfies the privacy constraint system for $L$, then $G_L$ must contain either a leaking cycle, a leaking pair, a disclosing cycle, or a privacy violating path. 
\end{lemma}