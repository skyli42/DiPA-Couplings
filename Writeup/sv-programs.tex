\subsection{Deciding Privacy}

In this section, we discuss the boolean or decision problem of privacy; that is, deciding whether or not there exists \textit{any} finite $d>0$ such that a program is $d\varepsilon$-differentially private.

Specifically, we show that coupling proofs are \textbf{complete} for a certain subclass of programs in our model.

The key condition that a program must have is that outputs must uniquely identify paths; for any output $\sigma$ of a program $P$, there must exist exactly one path in $P$ that could have produced $\sigma$. In particular, this means that if a path fails to be private on its own, any program containing the path must also fail be to private. 

We impose this condition through \textbf{output distinction}:

\begin{defn}[Output Distinction]\label{outputDistinctionDef}
    For a program $P$ generated by $G_P = (V_P, E_P)$, we say that $P$ satisfies \textbf{output distinction} if for all locations $\ell\in V_P$, if there exist two distinct edges $(\ell, \ell')$ labeled by $t'=(c', \sigma', \tau')$ and $(\ell, \ell^*)$ labeled by $t^* = (c^*, \sigma^*, \tau^*)$, then $\sigma' \neq \sigma^*$. Additionally, at least one of $\sigma'\in \Gamma$, $\sigma^*\in \Gamma$ is true (i.e. it cannot be that both transitions output a real number).
\end{defn}

\begin{prop}
    Let $P$ be a program generated by $G_P$. Let $O\subseteq (\gamma\cup \RR)^*$ be the set of all possible outputs of paths in $P$. There exists an injection $f: O\to P$ from the set of all possible outputs to paths in $P$. 
\end{prop}

For any program $P$ that satisfies output determinism, we show that the existence of coupling proofs completely characterizes the privacy of $P$.

\begin{lemma}\label{ProgramCounterexampleLemma}
    If a program $P$ satisfies output distinction and, for some looping branch $L\subseteq P$, there does not exist a coupling strategy $C_L$ that satisfies the privacy constraint system, then there does not exist any finite $d>0$ such that $P$ is $d\varepsilon$-differentially private.
\end{lemma}

\subsection{DiPA}

For the proof of lemma \ref{ProgramCounterexampleLemma}, we introduce a previously analyzed program model known as DiPA, which we show is equivalent to the current program model. The equivalence between DiPA and programs allows us to construct our completeness proof. 

\begin{defn}[\cite{chadhaLinearTimeDecidability2021}]
    A Differentially Private Automaton (DiPA) $A$ is an 8-tuple $(Q, \Sigma, C, \Gamma, q_{init}, X, P, \delta)$ where
    \begin{itemize}
        \item $Q$ is a finite set of locations partitioned into input locations $Q_{in}$ and non-input locations $Q_{non}$. 
        \item $\Sigma = \RR$ is the input alphabet
        \item $C = \{\texttt{true}, \lguard, \gguard\}$ is a set of guard conditions
        \item $\Gamma$ is a finite output alphabet
        \item $q_{init}\in Q$ is the initial location
        \item $X = \{\texttt{x}, \texttt{insample}, \texttt{insample}'\}$ is a set of variables
        \item $P: Q\to \QQ\times \QQ^{\geq 0}\times \QQ\times  \QQ^{\geq 0}$ is a parameter function that assigns sampling parameters for the Laplace distribution for each location
        \item $\delta:(Q\times C)\to (Q\times (\Gamma \cup \{\texttt{insample}, \texttt{insample}'\})\times \{\texttt{true}, \texttt{false}\})$ is a partial transition function. 
    \end{itemize}
    In addition, $\delta$ must satisfy some additional conditions:
    \begin{itemize}
        \item \textbf{Determinism:} For any location $q\in Q$, if $\delta(q,\texttt{true})$ is defined, then $\delta(q,\lguard)$ and $\delta(q,\gguard)$ are not defined. 

        \item \textbf{Output Distinction:} For any location $q\in Q$, if $\delta(q, \gguard) = (q_1, o_1, b_1)$ and $\delta(q, \lguard) = (q_2, o_2, b_2)$, then $o_1\neq o_2$ and at least one of $o_1\in \Gamma$ and $o_2\in \Gamma$ is true.

        \item \textbf{Initialization:} The initial location $q_0$ has only one outgoing transition of the form $\delta(q_0, \texttt{true}) = (q, o, \texttt{true})$.

        \item \textbf{Non-input transition:} From any $q\in Q_{non}$, if $\delta(q, c)$ is defined, then $c=\texttt{true}$.
    \end{itemize}
\end{defn}

We observe that each condition on $\delta$ corresponds to a restriction on program-generating graphs; in particular, the non-input transition condition corresponds to the definition of public transitions and the restrictions on them. 

A DiPA operates as follows: 
\begin{itemize}
    \item At each location $q$ such that $P(q) = (d_q, d_q')$, a real-valued input $\texttt{in}$ is read in and two variables $\texttt{insample}\sim \Lap(\texttt{in}, \frac{1}{d_q\varepsilon})$ and $\texttt{insample}'\sim\Lap(\texttt{in}, \frac{1}{d_q'\varepsilon})$ are sampled.
    \item $\texttt{insample}$ is compared to the stored variable $\texttt{x}$, and depending on the guards of the transitions out of the current location, the current location is changed and a value is output. This value can either be $\texttt{insample}, \texttt{insample}'$, or a symbol from $\Gamma$.
    \item Finally, the  value of $\texttt{x}$ is optionally updated with the value of $\texttt{insample}$.
\end{itemize}

Just like with programs, a path $\rho$ in a DiPA $A$ reads a real-valued input sequence $\texttt{in}\in \RR^n$; the same definitions of adjacency and validity apply to DiPA input sequences.

We first establish notation for discussing the probabilities of different paths in a DiPA, which allows us to define $d\varepsilon$-differential privacy. 

\begin{defn} 
    Let $\rho$ be a path in a DiPA $A$, let $\texttt{in}$ be a valid input sequence and let $o$ be a possible output of $\rho$. In particular, if $\sigma_i\in \{\texttt{insample}, \texttt{insample}'\}$, then we require that $o_i$ is an \textit{interval} $(a, b)\subseteq \RR$, rather than simply a measurable set as before. 
    Then $\texttt{Pr}[x, \rho, \texttt{in}, o]$ is the probability of $\rho$ being taken with input sequence $\texttt{in}$ and outputting $o$. If the first location in $\rho$ is $q_{init}$, then $\texttt{Pr}[x, \rho, \texttt{in}, o]$ may be shortened to $\texttt{Pr}[\rho, \texttt{in}, o]$, since the initial value of $\texttt{x}$ is irrelevant.
\end{defn}

For a full definition of DiPA semantics, we refer back to the original work. 

\begin{defn}
    A DiPA $A$ is $d\varepsilon$-differentially private for some $d>0$ if for all paths $\rho$ in $A$, for all possible outputs $o$ of $\rho$ and valid adjacent input sequences $\texttt{in}\brangle{1}\sim \texttt{in}\brangle{2}$, \[
        \PP[\rho, \texttt{in}\brangle{1}, o]\leq e^{d\varepsilon} \PP[\rho, \texttt{in}\brangle{2}, o]
    \]
\end{defn}

Notably, we show an equivalence between output-distinct programs and DiPAs.

\begin{prop}
    If a program $P$ generated by $G_P$ satisfies output distinction, then $G_P$ represents a valid DiPA $A_P$, and, for any DiPA $A$, the labeled graph representing $A$ generates a program $P_A$. 
\end{prop}

We observe that the probability of a run $\rho$ of a generated DiPA $G_P$ producing an output $\sigma$ is equal to the probability of the corresponding path in $P$ producing the same output. 

\begin{prop}
    Let $P$ be a program generated by a DiPA $G_P$. For all runs $\rho$ of $G_P$, all possible output events $o$ of $\rho$, and input sequences $\texttt{in}$, \[
        \PP[\rho, \texttt{in}, o]=\PP[\Psi(\rho), \texttt{in}, o],
    \]
    where $\Psi$ is the forgetful homomorphism that drops states from runs. 
\end{prop}

Interestingly, the privacy of any DiPA $A$ is completely characterized by four graph-theoretic structures---we provide an example of one such structure and refer to the original work for full definitions.

We provide some preliminary definitions. 
\begin{defn}
    A cycle $\rho$ of a DiPA $A$ is called an \lcycle (respectively, \gcycle) if there is an $i< |\rho|$ such that the guard of the $i$'th transition in $\rho$ is $\lguard$ (respectively, $\gguard$).

    We say that a path $\rho$ of a DiPA $A$ is an \texttt{AL}-path (respectively, \texttt{AG}-path) if all assignment transitions on $\rho$ have guard $\lguard$ (respectively, $\gguard$).

    Observe that a cycle can be both an \lcycle~and a \gcycle.
    Further, a path with no assignment transitions (including the empty path) is simultaneously both an AL-path and an AG-path.
\end{defn} 
\begin{defn}[Leaking Pairs~\cite{chadhaLinearTimeDecidability2021}]\label{defLeakingPairs}
    A pair of cycles $(C, C')$ in a DiPA A is called a leaking pair if one of the following two conditions is satisfied:
    \begin{enumerate}
        \item $C$ is an \lcycle, $C'$ is a \gcycle, and there is an AG-path
        from a state in $C$ to a state in $C'$.
        \item $C$ is a \gcycle, $C'$ is an \lcycle, and there is an AL-path
        from a state in $C$ to a state in $C'$.
    \end{enumerate}
\end{defn} 

\begin{thm}[\cite{chadhaLinearTimeDecidability2021}]\label{DiPACounterexamplesThm}
    A DiPA $A$ does not have a leaking cycle, leaking pair, disclosing cycle, or privacy violating path if and only if there exists some $d>0$ such that for all $\varepsilon>0$, $A$ is $d\varepsilon$-differentially private. 
\end{thm}

This provides us with a method of demonstrating that a program is not $d\varepsilon$-differentially private for any $d>0$.

\begin{cor}
    If a program $P$ is generated by $G_P$ and $G_P$ contains a leaking cycle, leaking pair, disclosing cycle, or privacy violating path, then does not exist a finite $d>0$ such that $P$ is $d\varepsilon$-differentially private. 
\end{cor}

Our key result relates the privacy constraint system and these graph-theoretic structures.

\begin{lemma}
    If a looping branch $L$ generated by $G_L$ satisfies output distinction and there does not exist a coupling strategy $C$ that satisfies the privacy constraint system for $L$, then $G_L$ must contain either a leaking cycle, a leaking pair, a disclosing cycle, or a privacy violating path. 
\end{lemma}