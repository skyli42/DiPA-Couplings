
\subsection{Individual Transitions}

Transitions act as guarded statements whose guard is dependent on a persistent real-valued variable $\texttt{x}$ and a real-valued input to which random noise is added; we can thus formally define individual transitions as follows:

\begin{defn}[Transitions]
    A transition is a tuple $t = (c, \sigma, \tau)$, where \begin{itemize}
        \item $c\in \{\texttt{true}, \lguard[\texttt{x}], \gguard[\texttt{x}]\}$ is a guard for the transition.
        \item $\sigma \in \Gamma\cup\{\texttt{insample}, \texttt{insample}'\}$ is the output of $t$.
        \item $\tau\in\{\texttt{true}, \texttt{false}\}$ is a boolean value indicating whether or not the stored value of $\texttt{x}$ will be updated.
    \end{itemize}
\end{defn}

We additionally associate two real-valued noise parameters $P(t) = (d, d')$ with each transition. 

\subsubsection{Transition Semantics}

We can think of each transition as defining an extremely small program: a transition reads a real valued input $\texttt{in}$, compares it to a threshold $\texttt{x}$, and, depending on the result of the comparison, outputs a value $\sigma$ and possibly updates the value of $\texttt{x}$.

More specifically, given some threshold value $\texttt{x}$, each transition $t = (c, \sigma, \tau)$ will first read a real number input $\texttt{in}$, sample two random variables $z\sim\Lap(0, \frac{1}{d\varepsilon})$, $z'\sim\Lap(0, \frac{1}{d'\varepsilon})$, where $P(t) = (d, d')$, and then assign two variables $\texttt{insample} = \texttt{in} + z$ and $\texttt{insample}' = \texttt{in} + z'$. 
If the guard $c$ is satisfied by comparing $\texttt{insample}$ to $\texttt{x}$, then we output $\sigma$ and, if $\tau = \texttt{true}$, reassign $\texttt{x} = \texttt{insample}$. 

We can describe the semantics of a transition as a function that maps an initial program state and a real-valued input to a subsequent program state. 

We define a program state as a tuple consisting of a distribution of possible values for the program value $\texttt{x}$, and a distribution of possible values for the current output $\sigma$. 

Let $S = dist(\RR\times (\Gamma \cup \RR)^*)$ be the set of all possible distributions over program states. 
As expected, every possible input is simply an element of $\RR$.

Then the semantics of a transition $t$ can be defined as a function $\Phi_t:dist_\downarrow(S)\times \RR \to dist_\downarrow (S)$ that maps a subdistribution of initial program states and an input to a subdistribution of subsequent program states. 

More precisely, if $t = (c, \sigma, \tau)$, let $P(t) = (d_t, d_t')$, let $\texttt{insample}$ be the Laplace distribution $\Lap(\texttt{in}, \frac{1}{d_t\varepsilon})$, and $\texttt{insample}'$ be the independent Laplace distribution $\Lap(\texttt{in}, \frac{1}{d_t'\varepsilon})$. Let $\theta$ be a subdistribution of program states. 

Then $\Phi_t(\theta, \texttt{in})$ is a subdistribution $O$ such that for all states $(\texttt{x}, \sigma_0)$, \begin{align*}
   \PP_O[(\texttt{x}, \sigma_0\cdot \sigma)]&= \begin{cases}
    \PP_\theta[(\texttt{x}, \sigma_0)]\PP[c \text{ is satisfied}]& \tau = \texttt{false}\\
    0 & \tau = \texttt{true}
  \end{cases} \\
  \PP_O[(\texttt{insample}, \sigma_0\cdot \sigma)]&= \begin{cases}
    \PP_\theta[(\texttt{x}, \sigma_0)]\PP[c \text{ is satisfied}]& \tau = \texttt{true}\\
    0 & \tau = \texttt{false}
  \end{cases} 
\end{align*}

Every other possible state is assigned probability 0 by $O$; with probability $1-\sum_{s\in S} \PP_O[s]$, we consider the transition to have halted.

We primarily are concerned with the probability that a transition outputs a specific value, that is, the probability that from location $q$, the program defined by $t$ transitions to location $q'$ and outputs a certain value $o$, where $o\subseteq (\Gamma\cup\RR)^*$ is a measurable event.

We denote this probability as $\PP[\texttt{x}, t, \texttt{in}, o]$, where $\texttt{x} \in dist(\RR)$ is the initial distribution of $\texttt{x}$, $t$ is a transition, $\texttt{in}\in \RR$ is a real-valued input, and $o\in \Gamma\cup \mathcal{P}(\RR)$ is a possible output of $t$. 
Specifically, let $O = \Phi_t((\texttt{x}, \lambda), \texttt{in})$, where $\texttt{x}$ is a distribution over real numbers and $\lambda$ represents the distribution that assigns probability 1 to the empty string, be a subdistribution.
 We abuse notation by using $(\texttt{x}, \lambda)$ to represent, depending on context, a tuple of distributions, the joint distribution of $\texttt{x}$ and $\lambda$, or the subdistribution that assigns probability 1 to the state $(\texttt{x}, \lambda)$. Then $\PP[\texttt{x}, t, \texttt{in}, o]$ is the marginal of $\Phi_t((\texttt{x}, \lambda), \texttt{in})$ on $(\cdot, o)$. \sky{come back to this phrasing}

\subsubsection{Couplings}

We will now construct couplings for transitions with the aim of using them as building blocks for proofs of privacy.

First, we need to adapt standard privacy definitions to our specific setting; recall that $\texttt{in}$, in reality, represents a \textbf{function} of some underlying dataset. This means that `closeness' in this context is defined as follows:

\begin{defn}[Adjacency]
    Two inputs $\texttt{in}\sim_{\Delta} \texttt{in}'$ are $\Delta$-adjacent if $|\texttt{in}-\texttt{in}'|\leq \Delta$. If $\Delta$ is not specified, we assume that $\Delta = 1$. 
\end{defn}

To construct approximate liftings, we will analyze the behaviour of two different \textbf{runs} of a transition $t = (c, \sigma, \tau)$, one with input $\texttt{in}\brangle{1}$ and one with input $\texttt{in}\brangle{2}$. 

Our approach to couplings will be that for every Laplace-distributed variable, we will couple the value of the variable in one run with its value in the other \textbf{shifted} by some amount. 

We differentiate between the values of variables in the first and second run by using angle brackets $\brangle{k}$, so, for example, we will take $X\brangle{1}$ to be the value of $\texttt{x}$ at location $q$ in the run of $t$ with input $\texttt{in}\brangle{1}$ and $X\brangle{2}$ to be the value of $\texttt{x}$ in the run of $t$ with input $\texttt{in}\brangle{2}$. 

We thus want to create the lifting $o\brangle{1}\{(a, b): a=\sigma\implies b=\sigma\}o\brangle{2}$, where $o\brangle{1}$ and $o\brangle{2}$ are random variables representing the possible outputs of $t\brangle{1}$ and $t\brangle{2}$, respectively.

We must guarantee two things: that if the first transition's guard is satisfied, then the second transition's guard is also satisfied and that both runs output the same value $\sigma$ when the guard is satisfied. Note that if $c = \texttt{true}$, the first condition is trivially satisfied and when $\sigma\in \Gamma$, the second condition is trivially satisfied. 

This gives us our major coupling lemma, which defines a family of couplings for privacy proofs.


\begin{lemma}[Less messy lemma \ref{indTransitionCoupling}]\label{simplifiedIndTransitionCoupling}
  For any transition $t$, any possible output event $\sigma$ of $t$ and any two adjacent inputs $\texttt{in}\brangle{1}\sim \texttt{in}\brangle{2}$, if we are given three real number ``shifts'' $\gamma_x, \gamma_t, \gamma_t'$ such that \[
    \begin{cases}
      \gamma_t\leq\gamma_x & c = \lguard[\texttt{x}]\\
      \gamma_t\geq\gamma_x & c = \gguard[\texttt{x}]\\
      \gamma_t=0 & \sigma = \texttt{insample}\\
      \gamma_t'=0 & \sigma = \texttt{insample}'
    \end{cases},
  \]
  then we can construct an approximate lifting that proves $\PP[X\brangle{1}, t, \texttt{in}\brangle{1}, \sigma]\leq e^{d\varepsilon}\PP[X\brangle{2}, t, \texttt{in}\brangle{2}, \sigma]$ for some bounded $d>0$ and initial threshold Laplace-distributed variables $X\brangle{1}$, $X\brangle{2}$.
\end{lemma}

The precise technical version of the lemma is in the appendix. 

\begin{lemma}\label{indTransitionCoupling}
    Let $X\brangle{1}\sim \Lap(\mu\brangle{1}, \frac{1}{d_x\varepsilon}), X\brangle{2}\sim\Lap(\mu\brangle{2}, \frac{1}{d_x\varepsilon})$ be random variables representing possible initial values of $\texttt{x}$ and let $t = (c, \sigma, \tau)$ be a transition from some valid transition alphabet $\Sigma_T$.
    Let $P(t) = (d_t, d_t')$.

    Let $\texttt{in}\brangle{1}\sim \texttt{in}\brangle{2}$ be an arbitrary adjacent input pair and let $o\brangle{1}$, $o\brangle{2}$ be random variables representing possible outputs of $t$ given inputs $\texttt{in}\brangle{1}$ and $\texttt{in}\brangle{2}$, respectively. 

    Then $\forall \varepsilon>0$ and for all $\gamma_x, \gamma_t, \gamma_t'\in [-1, 1]$ that satisfy the constraints \[
        \begin{cases}
          \gamma_t\leq\gamma_x & c = \lguard[\texttt{x}]\\
          \gamma_t\geq\gamma_x & c = \gguard[\texttt{x}]\\
          \gamma_t=0 & \sigma = \texttt{insample}\\
          \gamma_t'=0 & \sigma = \texttt{insample}'
        \end{cases},
      \]
      the lifting $o\brangle{1}\{(a, b): a=\sigma\implies b=\sigma\}^{\#d\varepsilon}o\brangle{2}$ is valid for $d = (|\mu\brangle{1}-\mu\brangle{2}+\gamma_x|)d_x+(|-\texttt{in}\brangle{1}+\texttt{in}\brangle{2}-\gamma_t|)d_t+(|-\texttt{in}\brangle{1}+\texttt{in}\brangle{2}-\gamma_t'|)d_t'$.
\end{lemma}

\begin{proof}
Fix $\varepsilon>0$.

We can first create the lifting $X\brangle{1}+\gamma_x (=)^{\#(|\mu\brangle{1}-\mu\brangle{2}+\gamma_x|)d_x\varepsilon}X\brangle{2}$. 

Additionally, create the lifting $z\brangle{1} (=)^{\#(|-\texttt{in}\brangle{1}+\texttt{in}\brangle{2}-\gamma_t|)d_t\varepsilon}z\brangle{2} - \texttt{in}\brangle{1}+\texttt{in}\brangle{2}-\gamma_t$, which is equivalent to creating the lifting $\texttt{insample}\brangle{1} +\gamma_t{(=)}^{\#(|-\texttt{in}\brangle{1}+\texttt{in}\brangle{2}-\gamma_t|)d_t\varepsilon}\texttt{insample}\brangle{2}$.

Finally, create the lifting $z'\brangle{1} (=)^{\#(|-\texttt{in}\brangle{1}+\texttt{in}\brangle{2}-\gamma_t'|)d_t'\varepsilon}z'\brangle{2} - \texttt{in}\brangle{1}+\texttt{in}\brangle{2}-\gamma_t'$. As before, this is equivalent to creating the lifting $\texttt{insample}'\brangle{1} +\gamma_t'{(=)}^{\#(|-\texttt{in}\brangle{1}+\texttt{in}\brangle{2}-\gamma_t'|)d_t'\varepsilon}\texttt{insample}'\brangle{2}$.

Thus, we emerge with three key statements to leverage:\begin{itemize}
    \item $X\brangle{1} + \gamma_x = X\brangle{2}$
    \item $z\brangle{1} = z\brangle{2} - \texttt{in}\brangle{1}+\texttt{in}\brangle{2}-\gamma_t$
    \item $z'\brangle{1} = z'\brangle{2} - \texttt{in}\brangle{1}+\texttt{in}\brangle{2}-\gamma_t'$
\end{itemize}

So if $c=\lguard[\texttt{x}]$ and $\gamma_t\leq \gamma_x$, then \begin{align*}
    \texttt{insample}\brangle{1}<X\brangle{1}&\implies \texttt{in}\brangle{1}+z\brangle{1}<X\brangle{1}\\
    &\implies \texttt{in}\brangle{1}+z\brangle{2}-\texttt{in}\brangle{1}+\texttt{in}\brangle{2}-\gamma_t<X\brangle{2}-\gamma_x\\
    &\implies \texttt{insample}\brangle{2}<X\brangle{2}
\end{align*}

Similarly, if $c=\gguard[\texttt{x}]$ and $\gamma_t\geq \gamma_x$, then \begin{align*}
    \texttt{insample}\brangle{1}\geq X\brangle{1}&\implies \texttt{in}\brangle{1}+z\brangle{1}\geq X\brangle{1}\\
    &\implies \texttt{in}\brangle{1}+z\brangle{2}-\texttt{in}\brangle{1}+\texttt{in}\brangle{2}-\gamma_t\geq X\brangle{2}-\gamma_x\\
    &\implies \texttt{insample}\brangle{2}\geq X\brangle{2}
\end{align*}

With these liftings, we have ensured that if the first run satisfies the guard of $t$, then the second run does as well. 

As noted, if $\sigma \in \Gamma$ and the first run taking transition $t$ implies that the second run does as well, then $o\brangle{1} = \sigma \implies o\brangle{2}=\sigma$ trivially.

Now, if $\sigma=\texttt{insample}$ and $\gamma_t=0$, then clearly we have that $\texttt{insample}\brangle{1}=\texttt{insample}\brangle{2}$, so for all $a\in \RR$, $o\brangle{1} = a\implies o\brangle{2} = a$.

Similarly, if $\sigma=\texttt{insample}'$ and $\gamma_t'=0$, we have that for all $a\in \RR$, $o\brangle{1} = a\implies o\brangle{2} = a$.

Thus, given the constraints \[
  \begin{cases}
    \gamma_t\leq\gamma_x & c = \lguard[\texttt{x}]\\
    \gamma_t\geq\gamma_x & c = \gguard[\texttt{x}]\\
    \gamma_t=0 & \sigma = \texttt{insample}\\
    \gamma_t'=0 & \sigma = \texttt{insample}'
  \end{cases},
\]
we have shown that the lifting $o\brangle{1}\{(a, b): a=\sigma\implies b=\sigma\}^{\#d\varepsilon}o\brangle{2}$ is valid, where the cost $d = (|\mu\brangle{1}-\mu\brangle{2}+\gamma_x|)d_x+(|-\texttt{in}\brangle{1}+\texttt{in}\brangle{2}-\gamma_t|)d_t+(|-\texttt{in}\brangle{1}+\texttt{in}\brangle{2}-\gamma_t'|)d_t'$. 

\end{proof}

We can thus think of couplings for a transition as being \textbf{parameterized} by $\gamma_x$, $\gamma_q$, and $\gamma_q'$. In particular, we will view choices of $\gamma_x, \gamma_q$, and $\gamma_q'$ as a \textbf{strategy} for proving that a transition is differentially private. 


\begin{defn}[Valid Coupling Strategies]
    A \textbf{valid coupling strategy} for a transition $t_i = (q_i, q_{i+1}, c_i, \sigma_i, \tau_i)$ is a tuple $C_i = (\gamma_x^{(i)}, \gamma_i, \gamma_i')\in [-1, 1]^3$ such that the constraints \[
        \begin{cases}
          \gamma_i\leq\gamma_x^{(i)} & c_i = \lguard[\texttt{x}]\\
          \gamma_i\geq\gamma_x^{(i)} & c_i = \gguard[\texttt{x}]\\
          \gamma_i=0 & \sigma_i = \texttt{insample}\\
          \gamma_i'=0 & \sigma_i = \texttt{insample}'
        \end{cases},
      \]
      are all satisfied. 
\end{defn}