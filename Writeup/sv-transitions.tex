
\subsection{Individual Transitions}\label{svTransitionsSec}

Transitions act as guarded statements whose guard is dependent on a persistent real-valued variable $\texttt{x}$ and a real-valued input to which random noise is added. Formally, we define transitions as follows: 

\begin{defn}[Transitions]\label{svTransDef}
    A transition is a tuple $t = (c, \sigma, \tau)$, where \begin{itemize}
        \item $c\in \{\texttt{true}, \lguard[\texttt{x}], \gguard[\texttt{x}]\}$ is a \textbf{guard} for the transition. We use $c(in, x)$ to denote the boolean function that corresponds to substituting $\texttt{insample} = in$ and $\texttt{x} =x $ into $c$.
        \item $\sigma \in \Gamma\cup\{\texttt{insample}, \texttt{insample}'\}$ is the \textbf{output} of $t$, for some finite alphabet of symbols $\Gamma$.
        \item $\tau\in\{\texttt{true}, \texttt{false}\}$ is an \textbf{assignment control}: a boolean value indicating whether or not the stored value of $\texttt{x}$ will be updated when $t$ is taken.
    \end{itemize}
\end{defn}

We additionally associate two positive real-valued \textbf{noise parameters} $P(t) = (d, d')$ with each transition. 

Furthermore, we introduce a special type of transition called a \textbf{public} transition, which we use to represent reading publicly-available data. Every \textbf{public} transition must be of the form $t_{pub}= (\texttt{true}, \sigma, \tau)$ for some $\sigma \in \Gamma\cup\{\texttt{insample}, \texttt{insample}'\}$ and $\tau \in \{\texttt{true}, \texttt{false}\}$. We will call all other transitions \textbf{private}. 

\subsubsection{Transition Semantics}

Each transition can be thought of as defining an atomic program step: a transition reads a real valued input $\texttt{in}$, compares a noisy version of it (\texttt{insample}) to a threshold $\texttt{x}$, and, depending on the result of the comparison, outputs a value $\sigma$ and possibly updates the value of $\texttt{x}$.

The semantics of a transition are defined by a function that maps an initial program \textbf{state} and a real-valued input to a subsequent program state. 

A program state is a tuple consisting of a distribution of possible values for the program value $\texttt{x}$, and a distribution of possible values for the current output $\sigma$; thus, $S = \RR\times (\Gamma \cup \RR)^*$ is set of all program states. 

Then we can formally define the semantics of a transition $t$ can be defined as a function $\Phi_t:dist_\downarrow(S)\times \RR \to dist_\downarrow (S)$ that maps a subdistribution\footnote{A \textbf{subdistribution} is a standard probability function except that the probability of the entire event space can be less than 1} of initial program states and an input to a subdistribution of subsequent program states. 

More precisely, given $t = (c, \sigma, \tau)$ where $t$ has noise parameters $P(t) = (d_t, d_t')$, let $\texttt{insample}\sim \Lap(\texttt{in}, \frac{1}{d_t\varepsilon})$ and $\texttt{insample}'\sim \Lap(\texttt{in}, \frac{1}{d_t'\varepsilon})$ be independent Laplace random variables. 

Given an initial program state $(x, \sigma_0)\in S$, we define the distribution of states that $t$ maps $(x, \sigma_0)$ to as follows: if $\sigma \in \Gamma$,
\[\PP_{(x, \sigma_0)}[(x', \sigma_0\cdot \sigma)] = \begin{cases}
  \mathbb{I}[x = x']\PP[c(\texttt{insample}, x)] & \tau = \texttt{false}\\
  \PP[\texttt{insample} = x' \land c(\texttt{insample}, x)] & \tau = \texttt{true}
\end{cases}\]

Otherwise, if $\sigma=\texttt{insample}$, for all $y\in \RR$,
\[\PP_{(x, \sigma_0)}[(x', \sigma_0\cdot y)] = \begin{cases}
  \mathbb{I}[x = x']\PP[\texttt{insample} = y\land c(\texttt{insample}, x)] & \tau = \texttt{false}\\
  \mathbb{I}[x'=y]\PP[\texttt{insample} = x' \land c(\texttt{insample}, x)] & \tau = \texttt{true}
\end{cases}\]

Finally, if $\sigma=\texttt{insample}'$, for all $y\in \RR$
\[\PP_{(x, \sigma_0)}[(x', \sigma_0\cdot y)] = \begin{cases}
  \mathbb{I}[x = x']\PP[c(\texttt{insample}, x)]\PP[\texttt{insample}' = y] & \tau = \texttt{false}\\
  \PP[\texttt{insample} = x' \land c(\texttt{insample}, x)]\PP[\texttt{insample}' = y] & \tau = \texttt{true}
\end{cases}\]

Here, $\mathbb{I}[p]$ is used to denote the Iverson bracket, which evaluates to 1 if the predicate $p$ is true and 0 otherwise.

Then for any initial subdistribution of states $\theta$, $\Phi_t(\theta, \texttt{in})$ can be defined by weighting the sum of all subdistributions $\PP_s$ for $s\in S$, where the weight of each $\PP_s$ is given by the probability of $s$ given by $\theta$. 

More formally, for any initial output sequence $\sigma_0\in (\Gamma\cup\RR)^*$, let $S^{(\sigma_0)}$ be the set of program states $(x, \sigma)$ such that $\sigma = \sigma_0$. Then for any subdistribution of program states $\theta$, $\Phi_t(\theta, \texttt{in})$ is a subdistribution $O$ such that, for all $x'\in \RR$ and $\sigma \in \Gamma\cup\RR$,
\[
  \PP_O[(x', \sigma_0\cdot \sigma)] = \int_{S^{(\sigma_0)}}\PP_\theta[s]\PP_s[(x', \sigma_0\cdot \sigma)]ds
\]

Every other possible state is assigned probability 0 by $O$; with probability $1-\int_S \PP_O[s]ds$, we consider the transition to have halted.

We primarily are concerned with the probability that a transition outputs a value (i.e. its guard is satisfied) and, in particular, outputs a certain value $o$, where $o\subseteq \Gamma\cup\RR$ is a measurable event.

We denote this probability as $\PP[\texttt{x}, t, \texttt{in}, o]$, where $\texttt{x} \in dist(\RR)$ is the initial distribution of $\texttt{x}$, $t$ is a transition, $\texttt{in}\in \RR$ is a real-valued input, and $o\in \Gamma\cup \mathcal{P}(\RR)$ is a possible output of $t$. 
Specifically, let $O = \Phi_t((\texttt{x}, \lambda), \texttt{in})$, where $\texttt{x}\in \RR$ and $\lambda$ represents the empty string. Then $\PP[\texttt{x}, t, \texttt{in}, o]$ is the marginal of $\Phi_t((\texttt{x}, \lambda), \texttt{in})$ on $(\cdot, o)$.

\subsubsection{Couplings}

We now construct approximate liftings for transitions with the aim of using them as building blocks for proofs of privacy.

First, we need to adapt standard privacy definitions to our specific setting; for example, recall that $\texttt{in}$, in reality, represents a \textbf{function} of some underlying dataset. This means that `closeness' in this context is defined as follows:

\begin{defn}[Adjacency and Validity]
    Two real-valued inputs $\texttt{in}\sim_{\Delta} \texttt{in}'$ are $\Delta$-adjacent if $|\texttt{in}-\texttt{in}'|\leq \Delta$. For a private transition $t$, $\texttt{in}\sim_{\Delta}\texttt{in}'$ is a \textbf{valid} pair of adjacent inputs if $\Delta = 1$ and for a public transition, $\texttt{in}\sim_{\Delta}\texttt{in}'$ is valid if $\Delta = 0$.
\end{defn}

Note that, in order to properly model public transitions (i.e. transitions whose input is public information), we require that all adjacent runs of the program provide the exact same input to a public transition. 

We show that, for any transition, we can construct approximate liftings such that each transition is proven ``private''. 

In particular, we construct liftings that are \textbf{parameterized} by three real values, $\gamma_x$, $\gamma_q$, and $\gamma_q'$. We view choices of $\gamma_x, \gamma_q$, and $\gamma_q'$ as a \textbf{strategy} for proving that a transition is differentially private.

\begin{defn}[Valid Coupling Strategies]
    A \textbf{valid coupling strategy} for a transition $t = (c, \sigma, \tau)$ is a function $C_t:\RR^2\to[-1, 1]^3$ such that for any two valid adjacent inputs $\texttt{in}\brangle{1}\sim\texttt{in}\brangle{2}$, $C_t$ maps $(\texttt{in}\brangle{1},\texttt{in}\brangle{2})$ to a tuple $(\gamma_x, \gamma_t, \gamma_t')$ such that, for any valid adjacent pair of inputs, the constraints \[
        \begin{cases}
          \gamma_t\leq\gamma_x & c = \lguard[\texttt{x}]\\
          \gamma_t\geq\gamma_x & c = \gguard[\texttt{x}]\\
          \gamma_t=0 & \sigma = \texttt{insample}\\
          \gamma_t'=0 & \sigma = \texttt{insample}'
        \end{cases}
      \]
      are all satisfied. 
\end{defn}

For any \textbf{valid} coupling strategy, we can bound the difference of any transition outputting a specific value for any two valid adjacent inputs. 

To construct approximate liftings of a transition $t = (c, \sigma, \tau)$, we will analyze the behaviour of two different \textbf{runs} of $t$, one with input $\texttt{in}\brangle{1}$ and one with input $\texttt{in}\brangle{2}$. 
We differentiate between the values of variables in the first and second run by using angle brackets $\brangle{k}$, so, for example, we will take $X\brangle{1}$ to be the value of $\texttt{x}$ at location $q$ in the run of $t$ with input $\texttt{in}\brangle{1}$ and $X\brangle{2}$ to be the value of $\texttt{x}$ in the run of $t$ with input $\texttt{in}\brangle{2}$. 

Our approach to couplings will be that for every Laplace-distributed variable, we will couple the value of the variable in one run with its value in the other \textbf{shifted} by some amount. 

We thus want to create the lifting $o\brangle{1}\{(a, b): a=\sigma\implies b=\sigma\}o\brangle{2}$, where $o\brangle{1}$ and $o\brangle{2}$ are random variables representing the possible outputs of $t$ in run $\brangle{1}$ and in run $\brangle{2}$, respectively.

We must guarantee two things: (1) that if the first transition's guard is satisfied, then the second transition's guard is also satisfied and (2) that both runs output the same value $\sigma$ when the guard is satisfied. Note that if $c = \texttt{true}$, the first condition is trivially satisfied and when $\sigma\in \Gamma$, the second condition is trivially satisfied. 

This leads to our major coupling lemma, which allows us to define a coupling proof for every valid coupling strategy.

\begin{lemma}\label{simplifiedIndTransitionCoupling}
  For any transition $t$, any possible output event $\sigma$ of $t$ and any two valid adjacent inputs $\texttt{in}\brangle{1}\sim \texttt{in}\brangle{2}$, if we are given a valid coupling strategy, i.e. three real valued ``shifts'' $\gamma_x, \gamma_t, \gamma_t'\in [-1, 1]$ such that \[
    \begin{cases}
      \gamma_t\leq\gamma_x & c = \lguard[\texttt{x}]\\
      \gamma_t\geq\gamma_x & c = \gguard[\texttt{x}]\\
      \gamma_t=0 & \sigma = \texttt{insample}\\
      \gamma_t'=0 & \sigma = \texttt{insample}'
    \end{cases},
  \]
  then we can construct an approximate lifting that proves $\PP[X\brangle{1}, t, \texttt{in}\brangle{1}, \sigma]\leq e^{d\varepsilon}\PP[X\brangle{2}, t, \texttt{in}\brangle{2}, \sigma]$ for some bounded $d>0$ and initial threshold Laplace-distributed variables $X\brangle{1}$, $X\brangle{2}$.
\end{lemma}

A precise version of this lemma can be found in the appendix as lemma \ref{indTransitionCoupling}. 

Specifically, lemma \ref{indTransitionCoupling} gives us the bound $d \leq (|\mu\brangle{1}-\mu\brangle{2}+\gamma_x|)d_x+(|-\texttt{in}\brangle{1}+\texttt{in}\brangle{2}-\gamma_t|)d_t+(|-\texttt{in}\brangle{1}+\texttt{in}\brangle{2}-\gamma_t'|)d_t'$, where $\mu\brangle{1}$ and $\mu\brangle{2}$, respectively, are the means of $X\brangle{1}$ and $X\brangle{2}$.

We call $d$ the \textbf{cost} of any coupling strategy $C = (\gamma_x, \gamma_t, \gamma_t')$.
