
\subsection{Individual Transitions}

We will define transitions over an underlying finite set of program \textbf{locations}; a transition defines how a program moves from one location to the next. 

\begin{defn}[Program locations]
    $Q$ is a finite set of program locations partitioned into input locations $Q_{in}$ and non-input locations $Q_{non}$. We will also associate each program location with two \textbf{noise parameters} using the function $P: Q\to \RR^{\geq 0}\times \RR^{\geq 0}$.
\end{defn}

Transitions act as guarded statements whose guard is dependent on a persistent real-valued variable $\texttt{x}$ and a real-valued input to which random noise is added; we can thus formally define individual transitions as follows:

\begin{defn}[Transitions]
    Given a finite set of program locations $Q$, a transition is a tuple $t = (q, q', c, \sigma, \tau)$, where \begin{itemize}
        \item $q\in Q$ is the initial location.
        \item $q'\in Q$ is the following location.
        \item $c\in \{\texttt{true}, \lguard[\texttt{x}], \gguard[\texttt{x}]\}$ is a guard for the transition.
        \item $\sigma \in \Gamma\cup\{\texttt{insample}, \texttt{insample}'\}$ is the output of $t$.
        \item $\tau\in\{\texttt{true}, \texttt{false}\}$ is a boolean value indicating whether or not the stored value of $\texttt{x}$ will be updated.
    \end{itemize}
    Depending on context, we may also notate $t$ as $q\to q'$. 
\end{defn}


\subsubsection{Constructing an alphabet}

As mentioned, we will consider individual transitions as part of an \textit{alphabet}; in particular, we will show that there is an interesting set of regular languages over an alphabet of transitions that we can apply the coupling framework to. 

However, in order to ensure that these languages do in fact correspond to semantically coherent programs, we need to restrict possible transition alphabets as follows. 

\begin{defn}[Valid Transition Alphabets]
    Let $\Sigma_T$ be a finite alphabet of transitions. We call $\Sigma_T$ \textbf{valid} if it satisfies the following conditions:
    \begin{itemize}
        \item \textbf{Initialization:} There exists some $t_{init}\in \Sigma_T$ such that $t_{init} = (q_0, q_1, \texttt{true}, \sigma, \texttt{true})$ for some $q_0, q_1\in Q$, $\sigma \in \Gamma\cup\{\texttt{insample}, \texttt{insample}'\}$. 
        \item \textbf{Determinism}: If any transition $t\in \Sigma_T$ is of the form $t=(q, q', c, \sigma, \tau)$, then no other transitions of the form $(q, q^*, c, \sigma', \tau')$ for $q, q', q^*\in Q$ exist in $\Sigma_T$. 
        Additionally, if there exists a transition $t=(q, q', \texttt{true}, \sigma, \tau)$ such that $t\in \Sigma_T$, then transitions of the form $(q, q^*, \lguard[\texttt{x}], \sigma', \tau')$ or $(q, q^*, \lguard[\texttt{x}], \sigma', \tau')$ are not in $\Sigma_T$. 
        \item \textbf{Output distinction}: If there exist some $\sigma, \sigma', \tau, \tau'$ such that $(q, q', \lguard[\texttt{x}], \sigma, \tau)\in \Sigma_T$  and $(q, q^*, \gguard[\texttt{x}], \sigma', \tau') \in \Sigma_T$, then $\sigma \neq \sigma'$. Additionally, at least one of $\sigma\in \Gamma$, $\sigma'\in \Gamma$ is true.
        \item \textbf{Non-input location condition}: For all locations $q\in Q_{non}$, if there exists a transition $t=(q, q', c, \sigma, \tau)$ such that $t\in \Sigma_T$, then $c = \texttt{true}$.
    \end{itemize}
\end{defn}

\subsubsection{Transition Semantics}

We can think of each transition as defining an extremely small program; beginning at location $q$, a transition reads a real valued input $\texttt{in}$, compares it to a threshold $\texttt{x}$, and, depending on the result of the comparison, moves to a location $q'$ while outputting a value $\sigma$ and possibly updating the value of $\texttt{x}$.

More specifically, given some threshold value $\texttt{x}$, each transition $t = (q, q', c, \sigma, \tau)$ will first read a real number input $\texttt{in}$, sample two random variables $z\sim\Lap(0, \frac{1}{d\varepsilon})$, $z'\sim\Lap(0, \frac{1}{d'\varepsilon})$, where $P(q) = (d, d')$, and then assign two variables $\texttt{insample} = \texttt{in} + z$ and $\texttt{insample}' = \texttt{in} + z'$. 
If the guard $c$ is satisfied by comparing $\texttt{insample}$ to $\texttt{x}$, then we transition to location $q'$, outputting $\sigma$ and, if $\tau = \texttt{true}$, reassigning $\texttt{x} = \texttt{insample}$. 

We can describe the semantics of a transition as a function that maps an initial program location and a real-valued input to a distribution of subsequent program locations, an output value, and a possibly new value for $\texttt{x}$. 

To be precise, we define a program state as a tuple consisting of a program location and a value for the program variable $\texttt{x}$. Let $S = Q\times\RR$ be the set of all possible program states. As expected, every possible input is simply an element of $\RR$. 
An output can either be a symbol from some finite alphabet $\Gamma$ or a real number; thus, the set of all possible output events is $\Gamma \cup \Sigma$, where $\Sigma$ is some $\sigma$-algebra over $\RR$.
In particular, we will take $\Sigma$ to be the standard $\sigma$-algebra of all Lebesgue measurable sets. 

It follows that the semantics of a transition $t$ can be defined as a function $\Phi_t: S\times \RR\to dist(S\times (\Gamma\cup\RR\cup \lambda))$ that maps an initial program location $q\in Q$ and an input $\texttt{in}\in \RR$ to a distribution of subsequent program locations and an output event following the expected semantics; $\lambda$ here denotes the empty string (i.e. no output). 

{\color{red}[If we need space, move the following section to appendix]}

Let $q\in Q$ be an initial location, $\texttt{x}\in \RR$ be an initial threshold value, and $P(q) = (d_q, d_q')$ be the distributional parameters associated with $q$. Let $t = (q, q', c, \sigma, \tau)$ be the transition whose semantics we are defining. 

Let $\texttt{in}\in \RR$ be a real-valued input and $o\in (\Gamma\cup\Sigma\cup\lambda)$ be a possible output event of $t$.

Let $I\sim \Lap(\texttt{in}, \frac{1}{d_q\varepsilon})$ and $I'\sim \Lap(\texttt{in}, \frac{1}{d_q'\varepsilon})$ be independent random variables corresponding to $\texttt{insample}$ and $\texttt{insample}'$. 

For both $I$ and $I'$, given $o$, we say that $I$ (or $I'$, respectively) matches $o$ if $o\subseteq \RR$ and $I\in o$. 

Let $T$ be the event that $c$ is satisfied given $\texttt{x}$ and $\texttt{insample}=I$ and let $O$ be the event that, if $\sigma=\texttt{insample}$, then $I$ matches $o$ and if $\sigma = \texttt{insample}'$, then $I'$ matches $o$. 

Then $\Phi_t((q, \texttt{x}), \texttt{in})$ is a distribution that assigns probabilities to output events as follows:

If $\tau = \texttt{false}$, then $\Phi_t((q, \texttt{x}), \texttt{in})$ assigns probability 0 to all events $((q', \texttt{x}'), o)$ such that $\texttt{x}'\neq \texttt{x}$. For all other events $((q', \texttt{x}), o)$. $\Phi_t((q, \texttt{x}), \texttt{in})$ assigns probability $\PP[T\land O]$ to $((q', \texttt{x}), o)$ and probability $\PP[\lnot T]$ to the event $((q_{term}, \texttt{x}), \lambda)$.


Similarly, if $\tau = \texttt{true}$, then $\Phi_t((q, \texttt{x}), \texttt{in})$ assigns probability $\PP[T\land O] $ to the event $((q', I), o)$ and assigns probability $\PP[\lnot T]$ to the event $((q_{term}, I), \lambda)$.

Here, $q_{term}$ is a sink or terminal location with no transitions allowed out of it. Equivalently, we could say that the program simply terminates and fails to transition to any new state.

We primarily are concerned with the probability that a transition ``succeeds'', that is, the probability that from location $q$, the program defined by $t$ transitions to location $q'$ and outputs a certain value. 

We denote this probability as $\PP[\texttt{x}, t, \texttt{in}, o]$, where $\texttt{x} \in \RR$ is the initial value of $\texttt{x}$, $t$ is a transition, $\texttt{in}\in \RR$ is a real-valued input, and $o\in \Gamma\cup\Sigma$ is a possible output of $t$. Specifically, $\PP[\texttt{x}, t, \texttt{in}, o] = \int_{-\infty}^\infty\PP[\texttt{x}\gets \texttt{x}']\Phi_t((q, \texttt{x}), \texttt{in})((q', \texttt{x}'), o)d\texttt{x}'$, 
where $\PP[\texttt{x}\gets \texttt{x}']$ is the probability that $\texttt{x}$ is updated to have the value $\texttt{x}'$.

Note that this is an aggregated probability over all possible final values of $\texttt{x}$---it is not particularly important what the specific final value of $\texttt{x}$ is. 



\subsubsection{Couplings}

We will now construct couplings for transitions with the aim of using them as building blocks for proofs of privacy.

First, we need to adapt standard privacy definitions to our specific setting; recall that $\texttt{in}$, in reality, represents a \textbf{function} of some underlying dataset. This means that `closeness' in this context is defined as follows:

\begin{defn}[Adjacency]
    Two inputs $\texttt{in}\sim_{\Delta} \texttt{in}'$ are $\Delta$-adjacent if $|\texttt{in}-\texttt{in}'|\leq \Delta$. If $\Delta$ is not specified, we assume that $\Delta = 1$. 
\end{defn}

Additionally, recall that some program locations ($Q_{non}$) in our model do not read in any input; to model this, we require that whenever input is passed into a non-input location, the actual input value is always 0.

\begin{defn}[Valid inputs]
    Let $t = (q, q', c, \sigma, \tau)$ be a transition over $Q$. A valid input to $t$ is a real number $\texttt{in}$ such that if $q\in Q_{non}$, then $\texttt{in} = 0$.
    In general, we will assume that all inputs are valid.
\end{defn}

To construct approximate liftings, we will analyze the behaviour of two different \textbf{runs} of a transition $t = (q, q', c, \sigma, \tau)$, one with input $\texttt{in}\brangle{1}$ and one with input $\texttt{in}\brangle{2}$. 

Our approach to couplings will be that for every Laplace-distributed variable, we will couple the value of the variable in one run with its value in the other \textbf{shifted} by some amount. 

We differentiate between the values of variables in the first and second run by using angle brackets $\brangle{k}$, so, for example, we will take $X\brangle{1}$ to be the value of $\texttt{x}$ at location $q$ in the run of $t$ with input $\texttt{in}\brangle{1}$ and $X\brangle{2}$ to be the value of $\texttt{x}$ in the run of $t$ with input $\texttt{in}\brangle{2}$. 

We thus want to create the lifting $o\brangle{1}\{(a, b): a=\sigma\implies b=\sigma\}o\brangle{2}$, where $o\brangle{1}$ and $o\brangle{2}$ are random variables representing the possible outputs of $t\brangle{1}$ and $t\brangle{2}$, respectively.

We must guarantee two things: that if the first transition is taken, then the second is also taken and that both runs output the same value $\sigma$ when taking the transition. Note that if $c = \texttt{true}$, the first condition is trivially satisfied and when $\sigma\in \Gamma$, the second condition is trivially satisfied. 

This gives us our major coupling lemma, which defines a family of couplings for privacy proofs.

\begin{lemma}\label{indTransitionCoupling}
    Let $X\brangle{1}\sim \Lap(\mu\brangle{1}, \frac{1}{d_x\varepsilon}), X\brangle{2}\sim\Lap(\mu\brangle{2}, \frac{1}{d_x\varepsilon})$ be random variables be random variables representing possible initial values of $\texttt{x}$ and let $t = (q, q^*, c, \sigma, \tau)$ be a transition from some valid transition alphabet $\Sigma_T$.
    Let $P(q) = (d_q, d_q')$.

    Let $\texttt{in}\brangle{1}\sim \texttt{in}\brangle{2}$ be an arbitrary adjacent input pair and let $o\brangle{1}$, $o\brangle{2}$ be random variables representing possible outputs of $t$ given inputs $\texttt{in}\brangle{1}$ and $\texttt{in}\brangle{2}$, respectively. 

    Then $\forall \varepsilon>0$ and for all $\gamma_x, \gamma_q, \gamma_q'\in [-1, 1]$ that satisfy the constraints \[
        \begin{cases}
          \gamma_q\leq\gamma_x & c = \lguard[\texttt{x}]\\
          \gamma_q\geq\gamma_x & c = \gguard[\texttt{x}]\\
          \gamma_q=0 & \sigma = \texttt{insample}\\
          \gamma_q'=0 & \sigma = \texttt{insample}'
        \end{cases},
      \]
      the lifting $o\brangle{1}\{(a, b): a=\sigma\implies b=\sigma\}^{\#d\varepsilon}o\brangle{2}$ is valid for $d = (|\mu\brangle{1}-\mu\brangle{2}+\gamma_x|)d_x+(|-\texttt{in}\brangle{1}+\texttt{in}\brangle{2}-\gamma_q|)d_q+(|-\texttt{in}\brangle{1}+\texttt{in}\brangle{2}-\gamma_q'|)d_q'$.
\end{lemma}

\begin{proof}
Fix $\varepsilon>0$.

We can first create the lifting $X\brangle{1}+\gamma_x (=)^{\#(|\mu\brangle{1}-\mu\brangle{2}+\gamma_x|)d_x\varepsilon}X\brangle{2}$. 

Additionally, create the lifting $z\brangle{1} (=)^{\#(|-\texttt{in}\brangle{1}+\texttt{in}\brangle{2}-\gamma_q|)d_q\varepsilon}z\brangle{2} - \texttt{in}\brangle{1}+\texttt{in}\brangle{2}-\gamma_q$, which is equivalent to creating the lifting $\texttt{insample}\brangle{1} +\gamma_q{(=)}^{\#(|-\texttt{in}\brangle{1}+\texttt{in}\brangle{2}-\gamma_q|)d_q\varepsilon}\texttt{insample}\brangle{2}$.

Finally, create the lifting $z'\brangle{1} (=)^{\#(|-\texttt{in}\brangle{1}+\texttt{in}\brangle{2}-\gamma_q'|)d_q'\varepsilon}z'\brangle{2} - \texttt{in}\brangle{1}+\texttt{in}\brangle{2}-\gamma_q'$. As before, this is equivalent to creating the lifting $\texttt{insample}'\brangle{1} +\gamma_q'{(=)}^{\#(|-\texttt{in}\brangle{1}+\texttt{in}\brangle{2}-\gamma_q'|)d_q'\varepsilon}\texttt{insample}'\brangle{2}$.

Thus, we emerge with three key statements to leverage:\begin{itemize}
    \item $X\brangle{1} + \gamma_x = X\brangle{2}$
    \item $z\brangle{1} = z\brangle{2} - \texttt{in}\brangle{1}+\texttt{in}\brangle{2}-\gamma_q$
    \item $z'\brangle{1} = z'\brangle{2} - \texttt{in}\brangle{1}+\texttt{in}\brangle{2}-\gamma_q'$
\end{itemize}

So if $c=\lguard[\texttt{x}]$ and $\gamma_q\leq \gamma_x$, then \begin{align*}
    \texttt{insample}\brangle{1}<X\brangle{1}&\implies \texttt{in}\brangle{1}+z\brangle{1}<X\brangle{1}\\
    &\implies \texttt{in}\brangle{1}+z\brangle{2}-\texttt{in}\brangle{1}+\texttt{in}\brangle{2}-\gamma_q<X\brangle{2}-\gamma_x\\
    &\implies \texttt{insample}\brangle{2}<X\brangle{2}
\end{align*}

Similarly, if $c=\gguard[\texttt{x}]$ and $\gamma_q\geq \gamma_x$, then \begin{align*}
    \texttt{insample}\brangle{1}\geq X\brangle{1}&\implies \texttt{in}\brangle{1}+z\brangle{1}\geq X\brangle{1}\\
    &\implies \texttt{in}\brangle{1}+z\brangle{2}-\texttt{in}\brangle{1}+\texttt{in}\brangle{2}-\gamma_q\geq X\brangle{2}-\gamma_x\\
    &\implies \texttt{insample}\brangle{2}\geq X\brangle{2}
\end{align*}

With these liftings, we have ensured that if the first run takes transition $t$, then the second run does as well. 

As noted, if $\sigma \in \Gamma$ and the first run taking transition $t$ implies that the second run does as well, then $o\brangle{1} = \sigma \implies o\brangle{2}=\sigma$ trivially.

Now, if $\sigma=\texttt{insample}$ and $\gamma_q=0$, then clearly we have that $\texttt{insample}\brangle{1}=\texttt{insample}\brangle{2}$, so for all $a\in \RR$, $o\brangle{1} = a\implies o\brangle{2} = a$.

Similarly, if $\sigma=\texttt{insample}'$ and $\gamma_q'=0$, we have that for all $a\in \RR$, $o\brangle{1} = a\implies o\brangle{2} = a$.

Thus, given the constraints \[
  \begin{cases}
    \gamma_q\leq\gamma_x & c = \lguard[\texttt{x}]\\
    \gamma_q\geq\gamma_x & c = \gguard[\texttt{x}]\\
    \gamma_q=0 & \sigma = \texttt{insample}\\
    \gamma_q'=0 & \sigma = \texttt{insample}'
  \end{cases},
\]
we have shown that the lifting $o\brangle{1}\{(a, b): a=\sigma\implies b=\sigma\}^{\#d\varepsilon}o\brangle{2}$ is valid, where the cost $d = (|\mu\brangle{1}-\mu\brangle{2}+\gamma_x|)d_x+(|-\texttt{in}\brangle{1}+\texttt{in}\brangle{2}-\gamma_q|)d_q+(|-\texttt{in}\brangle{1}+\texttt{in}\brangle{2}-\gamma_q'|)d_q'$. 

\end{proof}

We can thus think of couplings for a transition as being \textbf{parameterized} by $\gamma_x$, $\gamma_q$, and $\gamma_q'$. In particular, we will view choices of $\gamma_x, \gamma_q$, and $\gamma_q'$ as a \textbf{strategy} for proving that a transition is differentially private. 


\begin{defn}[Valid Coupling Strategies]
    A \textbf{valid coupling strategy} for a transition $t_i = (q_i, q_{i+1}, c_i, \sigma_i, \tau_i)$ is a tuple $C_i = (\gamma_x^{(i)}, \gamma_i, \gamma_i')\in [-1, 1]^3$ such that the constraints \[
        \begin{cases}
          \gamma_i\leq\gamma_x^{(i)} & c_i = \lguard[\texttt{x}]\\
          \gamma_i\geq\gamma_x^{(i)} & c_i = \gguard[\texttt{x}]\\
          \gamma_i=0 & \sigma_i = \texttt{insample}\\
          \gamma_i'=0 & \sigma_i = \texttt{insample}'
        \end{cases},
      \]
      are all satisfied. 
\end{defn}